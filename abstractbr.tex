Gramáticas de Análise Sintática de Expressão
(PEGs, do inglês Parsing Expression Languages) são uma classe de
gramáticas formais determinísticas originalmente descritas por Ford e
amplamente utilizadas para descrever e analisar linguagens de programação.
PEGs foram implementadas por diversos projetos. Um desses projetos é \lpeg{},
uma biblioteca Lua que compila PEGs para código otimizado que é executado
por uma máquina virtual especializada.

A implementação de \lpeg{} apresenta dois algoritmos-chave
que nunca foram publicados ou verificados formalmente.
Primeiramente, \lpeg{} possui sua própria
implementação da verificação de boa-formação introduzida por Ford, essencial
para garantir que a análise sintática termine.
Em segundo lugar, \lpeg{} implementa um algoritmo que computa
o conjunto de primeiros caracteres que
podem ser aceitos por um padrão, utilizado para gerar código de
máquina virtual mais eficiente para certos padrões.

Este trabalho formaliza esses algoritmos
e prova que estão corretos usando o
provador de teoremas Coq.
Além disso, provamos que esses algoritmos
terminam utilizando uma
abordagem baseada em consumo de gás.