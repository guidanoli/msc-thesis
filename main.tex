%% -*- coding: utf-8; -*-
% Use a opção 'digital' para ativar o retorno de referências. Essa função é indicada para a versão digital do documento
%\documentclass[phd,american]{thesispuc}%english thesis
\documentclass[msc,american]{thesispuc}%english dissertation
% \documentclass[phd,brazilian,digital]{thesispuc}%tese em portugês
% \documentclass[msc,brazilian]{thesispuc}%disseretação em portuguŝ

%%%
%%% Additional Packages
%%%
\usepackage{tabularx}
\usepackage{multirow}
\usepackage{multicol}
\usepackage{colortbl}
\usepackage[%
    dvipsnames,
    svgnames,
    x11names,
    table
]{xcolor}
\usepackage{numprint}
\usepackage{textcomp}
\usepackage{booktabs}
\usepackage{amsmath}
\usepackage{enumitem}
\usepackage{amssymb}
% Estilo de citação ABNT, O estilo atual está dinido como ordem e citação alfabetica, se você precisar
% trocar a ordenação por ordem de citação, modifique na linha abaixo álf'para 'num' e também no fim dá página
% troque bibliographystyle pela versão comentada ao lado.
\usepackage[num,bibjustif,abnt-emphasize=bf]{abntex2cite}
%\usepackage[linesnumbered, ruled, vlined]{algorithm2e}
%\usepackage{pgfplots,pgfplotstable} 
%\usepackage{array}

\usepackage{graphicx}
% \usepackage[a4paper, margin=2cm]{geometry}

\usepackage{hyperref} % links
% \usepackage{amsmath} % math utilities
% \usepackage{amssymb} % math symbols
\usepackage{amsthm} % theorems and lemmas
\usepackage{mathpartir} % type inference rules
\usepackage{cleveref} % clever referencing
\usepackage{lineno} % line numbers (good for reviewing)
\usepackage{stmaryrd} % double square brackets
% \usepackage{authblk} % author and affiliation blocks
% \usepackage[numbers]{natbib} % numeric references

\usepackage{mathtools}
\usepackage{listings}

\lstdefinestyle{myLuaStyle}
{
  language = {[5.2]Lua},
  frame = none,
  commentstyle = \color{blue},
}

\lstset{style=myLuaStyle}
% Authors

\newcommand{\guilherme}[0]{Guilherme Dantas de Oliveira}
\newcommand{\roberto}[0]{Roberto Ierusalimschy}

\newenvironment{biography}[1]{
    \begin{minipage}[c]{0.2\linewidth}
        \begin{flushleft}
            \includegraphics[width=0.8\linewidth]{#1}
        \end{flushleft}
    \end{minipage}
    \begin{minipage}[c]{0.7\linewidth}
}{
    \end{minipage}
}

% Links

\newcommand{\selfhref}[1]{\href{#1}{#1}}

% mathpartir

\newcommand{\rulename}[1]{\text{\fontfamily{cmr}\selectfont\textsc{\small(#1)}}}
\newcommand{\namedinferrule}[3]{\inferrule{#2}{#3}\enspace\rulename{#1}}

% Coq

\newcommand{\scor}[0]{\textsc{or}}
\newcommand{\scand}[0]{\textsc{and}}

\def\None{None}
\def\Some#1{Some\ #1}

\newcommand{\EmptyString}[0]{nil}
\newcommand{\String}[2]{#1 :: #2}

\DeclareMathOperator{\dplus}{+\kern -0.4em+}

\newcommand{\matchwith}[1]{\text{\textbf{match} $#1$ \textbf{with}}}
\newcommand{\matchcase}[2]{\vert\ #1 \Rightarrow #2}
\newcommand{\matchend}[0]{\text{\textbf{end}}}
\newcommand{\letin}[2]{\text{\textbf{let} $#1$ \textbf{:=} $#2$ \textbf{in}}}

\newcommand{\Suffix}[2]{#1 \preceq #2}
\newcommand{\ProperSuffix}[2]{#1 \prec #2}

% PEGs

%% Syntax

\newcommand{\PEmpty}[0]{\varepsilon}
\newcommand{\PSet}[1]{[#1]}
\newcommand{\PRange}[2]{\PSet{#1\text{--}#2}}
\newcommand{\PSequence}[2]{#1\ #2}
\newcommand{\PChoice}[2]{#1\ /\ #2}
\newcommand{\PRepetition}[1]{#1^\star}
\newcommand{\PNot}[1]{!#1}
\newcommand{\PAnd}[1]{\&#1}
\newcommand{\PNT}[1]{R_{#1}}
\newcommand{\Rule}[2]{#1 \leftarrow #2}
\newcommand{\POptional}[1]{#1?}
\newcommand{\PPlus}[1]{#1^+}
\newcommand{\PDot}[0]{.}

%% Operators

\newcommand{\length}[1]{|#1|}
\newcommand{\size}[1]{||#1||}
\newcommand{\dsqb}[1]{\llbracket #1 \rrbracket}

%% Semantics

\def\predicate#1{\xrightarrow{\mathit{#1}}}

\DeclareMathOperator{\matches}{\predicate{m}}
\def\Matches#1#2#3#4{(#1,#2,#3) \matches #4}
\DeclareMathOperator{\Failure}{\bot}
\def\DoesNotMatch#1#2#3{\Matches{#1}{#2}{#3}{\Failure}}
\DeclareMathOperator{\matchescomp}{m_{comp}}

\DeclareMathOperator{\coherent}{\predicate{c}}
\def\Coherent#1#2#3{(#1,#2) \coherent #3}
\newcommand{\coherentname}[0]{\text{coherent}}
\newcommand{\coherentfunc}[2]{\coherentname{}\ #1\ #2}
\DeclareMathOperator{\lcoherent}{\predicate{lc}}
\newcommand{\lcoherentname}[0]{\text{lcoherent}}
\newcommand{\lcoherentfunc}[2]{\lcoherentname{}\ #1\ #2}
\newcommand{\lCoherent}[3]{(#1,#2) \lcoherent #3}

\DeclareMathOperator{\verifyrule}{\predicate{vr}}
\def\VerifyRule#1#2#3#4#5#6{(#1,#2,#3,#4) \verifyrule (#5,#6)}
\newcommand{\verifyrulename}[0]{\text{verifyrule}}
\newcommand{\verifyrulecomp}[5]{\verifyrulename{}\ #1\ #2\ #3\ #4\ #5}
\DeclareMathOperator{\lverifyrule}{\predicate{lvr}}
\newcommand{\lverifyrulename}[0]{\text{lverifyrule}}
\newcommand{\lverifyrulecomp}[3]{\lverifyrulename{}\ #1\ #2\ #3}
\newcommand{\lVerifyRule}[3]{(#1,#2) \lverifyrule #3}

\DeclareMathOperator{\nullable}{\predicate{n}}
\newcommand{\Nullable}[4]{(#1,#2,#3) \nullable #4}
\newcommand{\nullablecomp}[4]{\text{nullable}\ #1\ #2\ #3\ #4}

\DeclareMathOperator{\checkloops}{\predicate{cl}}
\newcommand{\checkloopscomp}[4]{\text{checkloops}\ #1\ #2\ #3\ #4}
\newcommand{\CheckLoops}[4]{(#1,#2,#3) \checkloops #4}
\DeclareMathOperator{\lcheckloops}{\predicate{lcl}}
\newcommand{\lcheckloopsname}[0]{\text{lcheckloops}}
\newcommand{\lcheckloopscomp}[3]{\lcheckloopsname{}\ #1\ #2\ #3}
\newcommand{\lCheckLoops}[3]{(#1,#2) \lcheckloops #3}

\DeclareMathOperator{\verifygrammar}{\predicate{vg}}
\newcommand{\verifygrammarname}[0]{verifygrammar}
\newcommand{\verifygrammarcomp}[2]{\text{\verifygrammarname{}}\ #1\ #2}
\newcommand{\VerifyGrammar}[2]{#1 \verifygrammar #2}
\newcommand{\verifygrammargas}[1]{mingas_{vg}(#1)}

\newcommand{\wf}[1]{\text{wf}\ #1}

%% Optimizations

% \DeclareMathOperator{\first}{\predicate{f}}
\newcommand{\firstname}[0]{\mathcal{F}}
\newcommand{\firstcomp}[4]{\firstname{}\ #1\ #2\ #3\ #4}
% \newcommand{\First}[5]{(#1,#2,#3) \first (#4,#5)}
\newcommand{\firstgas}[2]{mingas_{f}(#1, #2)}

%% Sets

\newcommand{\EmptySet}[0]{\varnothing}
\newcommand{\Set}[1]{\{#1\}}
\newcommand{\SetUnion}[2]{#1 \cup #2}
\newcommand{\SetIntersection}[2]{#1 \cap #2}
\newcommand{\SetMinus}[2]{#1 \setminus #2}

% LPEG

\newcommand{\lpeg}[0]{LPeg}
\newtheorem{lemma}{Lemma}[chapter]
\newtheorem{theorem}[lemma]{Theorem}
\newtheorem{example}[lemma]{Example}


% Line numbers (for reviewing)
% \linenumbers
% \linenumberdisplaymath

% numprint
\npthousandsep{.}
\npdecimalsign{,}

%% ThesisPUC option
\tablesmode{none} %% [none, fig, tab ou figtab]
\algorithmsmode{none} %% [none ou use] %% Default is [use]
\codesmode{none} %% [none ou use] %% Default is [use]
\abreviationsmode{none} %% [none ou use] %% Default is [use]

%%%
%%% Counters
%%%

%% uncomment and change for other depth values
\setcounter{tocdepth}{1}
%\setcounter{lofdepth}{3}
%\setcounter{lotdepth}{3}
%\setcounter{secnumdepth}{3}

%%%
%%% Misc.
%%%

\usecolour{true}


%%%
%%% Titulos
%%%

\author{Guilherme Dantas de Oliveira}
\authorR{Dantas de Oliveira, Guilherme} % É necessário colocar o nome completo Sobrenomes, Nomes

\advisor{Roberto Ierusalimschy}{} %Name LastName
\advisorR{Ierusalimschy, Roberto} %LastName, Name
% Se o orientado for de uma instituição diferente, descomente a linh abaixo e defina a instituição
%\advisorInst{institution name}{acronym}

%\coadvisor{Otávio da Fonseca Martins Gomes}{Dr.}
%\coadvisorR{da Fonseca Martins Gomes, Otávio}
%\coadvisorInst{Centro de Tecnologia Mineral}{CETEM/MCTI}

%\coadvisor{Otávio da Fonseca Martins Gomes}{Dr.}
%\coadvisorR{da Fonseca Martins Gomes, Otávio}
%\coadvisorInst{Centro de Tecnologia Mineral}{CETEM/MCTI}

\title{Formalização de Algoritmos-Chave de \lpeg{}} %Título em portugês

\titleuk{Formalization of Key Algorithms from \lpeg{}} %Título em Inglês

%%\subtitulo{Aqui vai o subtitulo caso precise}

\day{7}
\month{April}
\year{2025}

\city{Rio de Janeiro}
\CDD{004}
\department{Informática}
\program{Informática}
\school{Pós-Graduação em Informática}
\university{Pontifícia Universidade Católica do Rio de Janeiro}
\uni{PUC-Rio}


%%%
%%% Banca
%%%

\jury{%
  \jurymember{Sérgio Queiroz de Medeiros}{}
    {Escola de Ciências e Tecnologia}{UFRN}
  \jurymember{Hugo Musso Gualandi}{}
    {Instituto de Computação}{UFRJ}
  % \jurymember{Noemi Rodriguez}{Profa.}
  %   {Departamento de Informática}{PUC-Rio}
}


%%%
%%% Resumo Pessoal
%%%

\resume{%
% Se o seu resumo ocupar menos de uma linha, utilize o seguinte comando para manter o texto justificado:
% \makebox[\textwidth][s]{Graduado em ciência da computação pela Universidade de Harvard.}
% Coso contrário, apenas digite seu resumo normalmente
    \makebox[\textwidth][s]{Bacharel em Engenharia da Computação pela PUC-Rio.}
}

%%%
%%% Agradecimentos
%%%
% Se você recebeu bolsa de insenção PUC ou bolsa CAPES você deve manter a frase abaixo em seus agradecimentos
\acknowledgment{%
\noindent To my advisor Professor Roberto Ierusalimschy for the incentive and partnership
to carry out this work.
\bigskip

\noindent To PUC-Rio, for the aids granted, without which this work
could not have been accomplished.
}


%%%
%%% Palavra chave do católogo bibliografico
%%%

\catalogprekeywords{%
  \catalogprekey{Informática}%
}

%%%
%%% Palavras chave do trabalho - Não adicione % no final da linha da definão do /key
%%%

\keywords{%
  \key{Gramáticas de Análise Sintática de Expressão}
  \key{Bem-formação}
  \key{LPeg}
}

\keywordsuk{%
  \key{Parsing Expression Grammars}
  \key{Well-formedness}
  \key{LPeg}
}

%%%
%%% Resumo e Resumo em Inglês
%%%

\abstract{%
    Gramáticas de Análise Sintática de Expressão
(PEGs, do inglês Parsing Expression Languages) são uma classe de
gramáticas formais determinísticas originalmente descritas por Ford e
amplamente utilizadas para descrever e analisar linguagens de programação.
PEGs foram implementadas por diversos projetos. Um desses projetos é \lpeg{},
uma biblioteca Lua que compila PEGs para código otimizado que é executado
por uma máquina virtual especializada.

A implementação de \lpeg{} apresenta dois algoritmos-chave
que nunca foram publicados ou verificados formalmente.
Primeiramente, \lpeg{} possui sua própria
implementação da verificação de boa-formação introduzida por Ford, essencial
para garantir que a análise sintática termine.
Em segundo lugar, \lpeg{} implementa um algoritmo que computa
o conjunto de primeiros caracteres que
podem ser aceitos por um padrão, utilizado para gerar código de
máquina virtual mais eficiente para certos padrões.

Este trabalho formaliza esses algoritmos
e prova que estão corretos usando o
provador de teoremas Coq.
Além disso, provamos que esses algoritmos
terminam utilizando uma
abordagem baseada em consumo de gás.
}

\abstractuk{%
    Parsing Expression Grammars (PEGs)
are a class of deterministic formal grammars
originally described by Ford.
They are widely used to describe and parse machine-oriented languages
and have been implemented by several projects.
One such project is \lpeg{},
a Lua library that compiles PEGs into optimized code
that is run by a specialized virtual machine.

The implementation of \lpeg{} features two key algorithms
that have never been published or verified before.
First, \lpeg{} has its own implementation of
the well-formedness check introduced by Ford,
which is crucial for ensuring that parsing terminates.
Second, \lpeg{} implements an algorithm
that computes the set of first characters
that may be accepted by a pattern,
which it uses to optimize
the virtual-machine code for certain patterns.

This work formalizes these algorithms
and proves their correctness
using the Coq proof assistant.
We also prove their termination
using a gas-based approach.
}

%%%
%%% Dedicatória
%%%

\dedication{%
    To my parents, sisters and family for their support and engouragement.
}

%%%
%%% Epigrafo
%%%

% \epigraph{%
%   My beautifull epigraph
% }
% \epigraphauthor{Wassily Kandinsky}
% \epigraphbook{Regards sur le passé}

\begin{document}
    \chapter{Introduction}


% PEGs
% LPegs
% dentro de LPeg: well-formed (tá ótimo), first-set (descreve O ALGORITMO)
% adapta first de CFG para PEGs (ler sobre first, e da sua importância para CFG)

Parsing Expression Grammars (PEGs)
are a type of formal grammar
introduced by Ford~\cite{ford_parsing_2004}
to describe and parse machine-oriented languages.
In contrast to Context-Free Grammars (CFGs),
established by Chomsky~\cite{chomsky_three_1956},
PEGs are deterministic,
which makes them particularly suitable for parsing
programming languages and structured data formats.

PEGs have been implemented by several projects
and in a variety of programming languages.
One such implementation is
\lpeg{}~\cite{ierusalimschy_text_2009},
a library for the Lua programming language.
Following the SNOBOL tradition,
patterns in \lpeg{} are first-class citizens
and can be programmatically constructed in a bottom-up fashion.
Starting from basic primitives such as character classes,
patterns can be combined through the use of operators:
\texttt{*}~for~sequences,
\texttt{+}~for~ordered choices,
\texttt{\^}~(caret)~for~repetitions,
and so on.

\lpeg{}
features several interesting algorithms in its implementation,
but two algorithms stand out for their complexity
and lack of formal documentation:
The well-formedness check
and the first-set computation.

The first algorithm, the well-formedness check,
ensures that a pattern is complete,
meaning it yields a match result for any input string.
The concept of well-formedness
was introduced by Ford
as a conservative approximation to completeness,
after proving that the problem of detecting completeness is undecidable,
and noticing that incompleteness is caused by
left-recursive rules and degenerate loops.

The well-formedness algorithm proposed by Ford
iteratively constructs a set of well-formed expressions
until a fixed-point is reached.
A grammar is then deemed well-formed if its expression set
matches the set of well-formed expressions derived
from the fixed-point iteration.
Koprowski~et~al.~\cite{koprowski_trx_2011}
formalized a similar algorithm
for an extended definition of PEGs
and proved its correctness
using the Coq proof assistant.

Meanwhile,
the well-formedness check implemented in \lpeg{}
does not use iteration, fixed-point checks,
or data structures to represent sets of expressions.
These details make the algorithm simpler to implement,
specially in programming languages
with modest standard libraries, such~as~C.
Moreover,
this alternative algorithm
has neither been published nor formally verified yet,
which makes it an interesting research topic.

The second algorithm implemented in \lpeg{}
that we highlight in this work is
the first-set algorithm.
It takes a pattern
and returns the first-set and the emptiness value
of the pattern.
The definition of first-set
is well-established in the area of CFGs,
but their application in PEGs has not yet
been documented in the literature.
Basically speaking,
the first-set of a pattern
is the set of first characters
that can be accepted by the pattern.
More precisely,
a pattern fails any string
that starts with a character
that is not in its first-set.

The first-set algorithm also returns an emptiness value,
which, when false, indicates that the pattern
fails to match the empty string.
(This Boolean value corresponds to the inclusion of
$\varepsilon$ in the first-set of some formalizations for CFGs.)
This other return value is necessary because
the first-set cannot be used to determine whether
a pattern fails the empty string,
because it has no first character.
Together, both return values
help \lpeg{} optimize certain patterns,
such as ordered choices.

Both aforementioned algorithms implemented in \lpeg{}
are complex and lack formal documentation,
which can make them difficult to maintain and reason about.
This work aims to bridge this gap.
We present and analyze these algorithms,
proving their correctness using the Coq proof assistant%
\footnote{The code is publicly available on
GitHub at \selfhref{https://github.com/guidanoli/peg-coq}}.
Moreover, we also prove their termination through a gas-based approach.
Also, during the process of proving
key properties about these algorithms,
we have noticed a few issues
that could be reviewed
in future versions of \lpeg{}.

The remainder of this work is structured as follows:
\Cref{chapter:pegs} presents the syntax and semantics of PEGs.
In \Cref{chapter:wf-algorithm}
we formalize the well-formedness algorithm
and prove its correctness and termination.
\Cref{chapter:first-set}
does the same for the first-set algorithm.
In \Cref{chapter:related-work},
we analyze how our work differs from prior research,
highlighting key differences and improvements.
Finally, \Cref{chapter:conclusion}
summarizes our findings and outlines directions
for future research.
    \chapter{Syntax and Semantics of PEGs}
\label{chapter:pegs}

When Ford~\cite{ford_parsing_2004} introduced PEGs,
he also defined a syntax
with several useful constructions
for practical language description purposes:
literal strings (e.g. ``abc''),
character classes (e.g. $\PRange{a}{z}$), any character ($\PDot{}$),
optionals $\POptional{p}$, zero-or-more repetitions $\PRepetition{p}$,
one-or-more repetitions $\PPlus{p}$, not-predicates $\PNot{p}$,
and-predicates $\PAnd{p}$, sequences $\PSequence{p_1}{p_2}$,
and ordered choices $\PChoice{p_1}{p_2}$.

However, for the purposes of formal analysis,
Ford realized it would be more
convenient to define an abstract syntax for PEGs
that represents its essential structure.
This abstract syntax does not include
several aforementioned constructions,
which are treated as \emph{syntactic sugar},
and includes two new constructions:
empty ($\PEmpty$) and single characters (e.g. `a').

With these new constructions,
Ford \emph{desugars} the PEG syntax as follows.
First, the any-character pattern ($\PDot{}$) is reduced
into a character class with all characters in the alphabet.
Then, literal strings are reduced into sequences of characters,
and character classes are reduced into choices of characters.
And-predicate patterns $\PAnd{p}$ are reduced into $\PNot{(\PNot{p})}$,
optionals $\POptional{p}$ into $\PChoice{p}{\PEmpty}$,
and one-or-more repetitions $\PPlus{p}$ into $\PSequence{p}{\PRepetition{p}}$.

This desugared syntax allowed Ford to reduce the size of proofs,
as there were fewer cases to be treated during case analysis.
However, for the purposes of formalizing the first-set algorithm,
we realized it would be beneficial for us to have our own
desugaring of the PEG syntax.
Compared to Ford's desugared syntax,
\emph{our} desugared syntax
includes and-predicates $\PAnd{p}$
and replaces characters (e.g. `a')
with character sets (e.g. $\PRange{a}{z}$).

Characters and character sets
have equivalent expressiveness,
as character sets can be reduced into choices of characters.
However, in our case, we prefer character sets as they allow us to
simplify the definition of the first-set algorithm.
Moreover, we decided to keep and-predicates $\PAnd{p}$
to stay more loyal to the implementation of \lpeg{},
which doesn't treat $\PAnd{p}$ as syntactic sugar for $\PNot{(\PNot{p})}$
and handles them differently in the first-set algorithm.
Our desugared syntax is depicted in \Cref{fig:syntax}.
\begin{figure}[t]
    \centering
    \begin{tabular}{llll|l|l|l|l|l|l|l}
    Pattern & $p$ & $:=$
    & $\PEmpty$
    & $\PSet{cs}$
    & $\PNT{i}$
    & $\PRepetition{p}$
    & $\PNot{p}$
    & $\PAnd{p}$
    & $\PSequence{p_1}{p_2}$
    & $\PChoice{p_1}{p_2}$
\end{tabular}
    \caption{Our desugared syntax of PEGs.}
    \label{fig:syntax}
\end{figure}

Having defined the syntax of PEGs,
we now define their semantics.
In the context of a grammar,
a pattern is parsed against an input string,
and the match may be either a success or a failure.
In the case of a success,
the pattern leaves the unconsumed suffix
of the input string as a result.

To be more precise,
we define a match predicate.
The predicate
takes a grammar $g$,
a pattern $p$,
and an input string $s$,
and returns a match result $res$.
It is denoted as $\Matches{g}{p}{s}{res}$.
A successful match result is represented by $s'$,
the unconsumed suffix of the input string,
while a failed match result is represented as $\Failure$.

The predicate is inductively defined in \Cref{fig:match}.
\begin{figure}
    \begin{mathpar}
    \namedinferrule{m-eps}
    { }
    {\Matches{g}{\PEmpty}{s}{s}}

    \namedinferrule{m-set-nil}
    { }
    {\DoesNotMatch{g}{\PSet{cs}}{\EmptyString}}

    \namedinferrule{m-set-cons-in}
    {a \in cs}
    {\Matches{g}{\PSet{cs}}{\String{a}{s}}{s}}

    \namedinferrule{m-set-cons-not-in}
    {a \not\in cs}
    {\DoesNotMatch{g}{\PSet{cs}}{\String{a}{s}}}

    \namedinferrule{m-nonterminal}
    {g[i] = \Some p \\ \Matches{g}{p}{s}{res}}
    {\Matches{g}{\PNT i}{s}{res}}

    \namedinferrule{m-rep-fail}
    {\DoesNotMatch{g}{p}{s}}
    {\Matches{g}{\PRepetition p}{s}{s}}

    \namedinferrule{m-rep-succ}
    {\Matches{g}{p}{s}{s'} \\ \Matches{g}{\PRepetition p}{s'}{res}}
    {\Matches{g}{\PRepetition p}{s}{res}}

    \namedinferrule{m-not-succ}
    {\Matches{g}{p}{s}{s'}}
    {\DoesNotMatch{g}{\PNot p}{s}}

    \namedinferrule{m-not-fail}
    {\DoesNotMatch{g}{p}{s}}
    {\Matches{g}{\PNot p}{s}{s}}

    \namedinferrule{m-and-succ}
    {\Matches{g}{p}{s}{s'}}
    {\Matches{g}{\PAnd{p}}{s}{s}}

    \namedinferrule{m-and-fail}
    {\DoesNotMatch{g}{p}{s}}
    {\DoesNotMatch{g}{\PAnd{p}}{s}}

    \namedinferrule{m-seq-fail}
    {\DoesNotMatch{g}{p_1}{s}}
    {\DoesNotMatch{g}{\PSequence{p_1}{p_2}}{s}}

    \namedinferrule{m-seq-succ}
    {\Matches{g}{p_1}{s}{s'} \\ \Matches{g}{p_2}{s'}{res}}
    {\Matches{g}{\PSequence{p_1}{p_2}}{s}{res}}

    \namedinferrule{m-choice-succ}
    {\Matches{g}{p_1}{s}{s'}}
    {\Matches{g}{\PChoice{p_1}{p_2}}{s}{s'}}

    \namedinferrule{m-choice-fail}
    {\DoesNotMatch{g}{p_1}{s} \\ \Matches{g}{p_2}{s}{res}}
    {\Matches{g}{\PChoice{p_1}{p_2}}{s}{res}}
\end{mathpar}
    \caption{The match predicate.}
    \label{fig:match}
\end{figure}
The empty pattern $\PEmpty$
matches any string,
without consuming anything.
A character set pattern ${\PSet{cs}}$
matches only strings that start with a character in the set $\Set{cs}$,
and consumes this character.
The nonterminal pattern ${\PNT i}$
matches the ${i^{th}}$ rule of the grammar.
The repetition pattern ${\PRepetition p}$
matches $p$ as many times as possible
in sequence.
The not-predicate pattern ${\PNot{p}}$ matches
iff $p$ does not match.
(It never consumes any input,
as either $p$ or ${\PNot{p}}$ always fail.)
The and-predicate pattern ${\PAnd{p}}$ matches
iff $p$ matches,
but without consuming any input.
A sequence pattern ${\PSequence{p_1}{p_2}}$
matches $p_1$ followed by $p_2$.
Lastly, the pattern ${\PChoice{p_1}{p_2}}$
constitutes an \emph{ordered choice}:
It first tries to match $p_1$,
and, if that fails, tries to match $p_2$.

In case the reader is not familiar with the semantics of PEGs,
it can be helpful to go through some examples.
Let us start with the pattern $\PSet{ab}$,
which matches the letters ``a'' or ``b''.
Unsurprisingly,
this pattern matches and consumes
the whole string ``a'',
leaving the empty string as a result.
It also matches the string ``baby'',
consuming just the first letter ``b''
and leaving the suffix ``aby'' unconsumed.
Meanwhile, for strings that do not start
with either the character ``a'' or ``b'',
the pattern fails.
For example, the pattern fails for the empty string
and for the string ``kaaba''.

If we wish the pattern to match zero or more letters ``a'' or ``b'',
we can use the repetition operator on the previous pattern,
resulting in the pattern $\PRepetition{\PSet{ab}}$.
This pattern matches the string ``baby'',
consuming the prefix ``bab''
and leaving the letter ``y'' unconsumed.
It also matches the empty string.

Some repetitions, however, are incomplete,
meaning they may yield no match result.
Take, for example, the pattern $\PRepetition{\PEmpty}$,
which matches $\PEmpty$ as many times as possible.
The repetition body $\PEmpty$ always matches and consumes no input,
so the match never makes any progress and gets stuck in an infinite loop.
This happens for any repetition whose body
may match a string while consuming no input.
\begin{align*}
    \forall g, \forall s, \Matches{g}{p}{s}{s} \implies
    \nexists res, \Matches{g}{\PRepetition{p}}{s}{res}
\end{align*}

Some grammar rules can also be incomplete,
which we will demonstrate by showing how repetitions
can be reduced to recursive grammar rules.
Given a repetition pattern $\PRepetition{p}$,
we can rewrite it as the non-terminal pattern $\PNT{i}$
in a grammar whose $i^{th}$ rule is defined as
$\PChoice{\PSequence{p}{\PNT{i}}}{\PEmpty}$.
If $\PRepetition{p}$ is incomplete,
then its reduction to a recursive grammar rules is also incomplete,
even if $p$ contains no repetitions.
This demonstrates that some grammar rules can be
intrinsically incomplete, even when free of repetitions.
This happens when rules reference themselves
(directly or indirectly) while consuming no input in-between.
We call these left-recursive rules.
The simplest example of such a rule is
one whose body is a non-terminal pattern that references
the rule itself.
\begin{align*}
    \forall g, \forall i, g[i] = \Some \PNT{i} \implies
    \forall s, \nexists res, \Matches{g}{\PNT{i}}{s}{res}
\end{align*}

As briefly displayed in the reduction of repetitions
into recursive grammar rules,
we can chain patterns together
with the sequence operator.
For example,
the pattern $\PSequence{\PRepetition{\PSet{ab}}}{\PSet{y}}$
matches the string ``baby'' but not the string ``babies''.

If we wish a pattern to be optional,
similar to how the $\POptional{p}$ operator from the
complete PEG syntax does,
we can make a choice between this pattern
and the empty pattern $\PEmpty$.
For example,
the pattern $\PSequence{\PRepetition{\PSet{ab}}}{(\PChoice{\PSet{y}}{\PEmpty})}$
matches both strings ``baby'' and ``babies'',
but leaves the suffix ``ies'' unconsumed.

We can also choose to fail when a given pattern matches,
through the not-predicate operator $\PNot{p}$.
Let us say, for example,
that the consumed input prefix
must not contain the substring ``aa'.
This behavior is implemented in the pattern
$\PRepetition{(\PSequence{\PNot{(\PSequence{\PSet{a}}{\PSet{a}})}}{\PSet{ab}})}$.
This pattern matches the strings ``baba'' and ``baaba'',
but leaves the suffix ``aaba'' unconsumed.

We can also check whether a pattern matches
but not consume any input.
That is possible with the and-predicate operator $\PAnd{p}$.
For example,
the pattern $\PAnd{\PSet{ab}}$
checks whether a string
starts with either the character ``a'' or ``b'',
but does not consume it.
This pattern matches the string ``baby'',
while not consuming any characters,
and fails to match the string ``kaaba''.

One aspect of PEGs that can vary
depending on the implementation
is how rules are identified and referenced,
that is, the set of nonterminals.
The original syntax proposed by Ford
uses strings~\cite{ford_parsing_2004}.
Meanwhile,
\lpeg{} allows rules to be identified 
by arbitrary Lua values~\cite{ierusalimschy_lpeg_2024}.
Meanwhile, our formalization uses natural indices
to identify the rules,
as it allows us to represent
grammars as lists of rules in Coq.

Regarding the match predicate,
we can already state two important,
yet simple, properties.
First, the predicate is deterministic:
Given a grammar, a pattern, and an input string,
there is at most one possible match result.

\begin{lemma}%[Determinism]
    If ${\Matches{g}{p}{s}{res_1}}$
    and ${\Matches{g}{p}{s}{res_2}}$,
    then ${res_1 = res_2}$.
    \label{lemma:match-determinism}
\end{lemma}

The second property states that the result string
is a suffix of the input string.
We use the symbol ``${\Suffix{}{}}$''
to denote the suffix relation.
This lemma might seem obvious,
but we assure the reader that it will be necessary later
to prove lemmas by induction on the length of the input string.

\begin{lemma}%[Result String is Suffix of Input String]
    If ${\Matches{g}{p}{s}{s'}}$,
    then ${\Suffix{s'}{s}}$.
    \label{lemma:match-suffix}
\end{lemma}

An important characteristic of this predicate,
which has profound consequences in our formalization efforts,
is that it not represent a total function:
Not all combinations of
grammars, patterns, and input strings
relate to a match result.
This may happen for several reasons,
which we will go through
in the following paragraphs.

The first reason is that
the grammar may be empty,
meaning it does not have any rule.
This is a problem,
because we use the first rule of the grammar
as the initial pattern.
If the grammar is empty,
then there is no first rule.

The second reason is that
the pattern may reference a non-existent rule.
This problem is not exclusive to our formalization,
which references rules by indices.
Ford and \lpeg{}
also suffer from this problem,
because,
in order to textually describe a recursive grammar,
you need to use some form of reference,
which may be invalid.

The third reason for
a pattern to not yield a match result
is because of so-called \emph{left-recursive} rules.
To illustrate what we mean by such,
consider the following rule.
\begin{equation}
    \Rule{\PNT{i}}{\PNT{i}}
    \label{eqn:lr-rule-simple}
\end{equation}

If we were to parse the pattern ${\PNT{i}}$,
we would resolve it to the body of the ${i^{th}}$ rule,
which is also the pattern ${\PNT{i}}$.
In doing so,
neither the pattern nor the input string would change,
which clearly leads to an infinite loop.
This is the simplest example of a left-recursive rule.

The previous example involves only one rule.
Left-recursive rules, however,
can also involve
multiple rules.
The following grammar,
for example,
has three rules,
one referencing another,
all of which are left-recursive.
\begin{equation*}
    \begin{cases}
        \Rule{\PNT{i}}{\PNT{j}} \\
        \Rule{\PNT{j}}{\PNT{k}} \\
        \Rule{\PNT{k}}{\PNT{i}}
    \end{cases}
\end{equation*}

The reader might notice that,
in both examples,
there are rules referencing one another.
However, this doesn't necessarily imply in left recursion.
The following rule, for example,
references itself, but is not left-recursive.

\begin{equation}
    \Rule{\PNT{i}}{\PChoice{\PSequence{\PSet{cs}}{\PNT{i}}}{\PEmpty}}
    \label{eqn:a-star-rule}
\end{equation}

What makes the rule from \Cref{eqn:lr-rule-simple} left-recursive,
but not the one from \Cref{eqn:a-star-rule},
is a subtle, yet important difference.
In \Cref{eqn:lr-rule-simple},
the nonterminal pattern ${\PNT{i}}$
is always visited
with the same input string
as the one provided to the rule.
Meanwhile,
in \Cref{eqn:a-star-rule},
the nonterminal pattern ${\PNT{i}}$
is always visited
with an input string
shorter than the one provided to the rule,
since the character set pattern ${\PSet{cs}}$
always consumes a character when it matches.
Given that input strings are finite,
we can prove that this rule yields a result
for any input string.
The proof is carried out by induction on the length of
the input string,
using \Cref{lemma:match-suffix}.

The fourth and last reason
for the lack of a match result
are so-called degenerate loops,
which are repetition patterns ${\PRepetition{p}}$
whose body $p$
may match without consuming any input.
If $p$ matches, but does not consume any input,
then it will do so infinitely many times.
One example of a degenerate loop is the pattern
${\PRepetition{(\PRepetition{\PSet{cs}})}}$.

In any of these cases,
a match result is not guaranteed to exist.
In practice,
we would like to ensure that
a grammar yields a match result
for any input string.
This property is called \emph{completeness}.
\begin{equation}
    \text{$g$ is complete} \iff
    \forall s, \exists res, \Matches{g}{\PNT{0}}{s}{res}
\end{equation}

Ford~\cite{ford_parsing_2004} showed that
the problem of knowing whether a grammar
is complete is undecidable.
He then presented a conservative approximation
to completeness known as \emph{well-formedness},
which is decidable.

We formalize this well-formedness check
as a computable function \textit{wf}
that takes a grammar and returns a Boolean value
indicating whether the grammar is well-formed or not.
We also prove its correctness by showing that,
for any given grammar $g$,
if $\wf{g} = true$,
then $g$ is complete.

How \lpeg{} implements this function \textit{wf}
is the focus of the next chapter,
as well as its proof of correctness.
All definitions and proofs
are publicly available on this project's GitHub repository%
\footnote{\selfhref{https://github.com/guidanoli/peg-coq}}.
    \chapter{Well-formedness Algorithm}
\label{chapter:wf-algorithm}

In this chapter,
we present the well-formedness algorithm
implemented in \lpeg{}
and prove its correctness.
We start with the overall structure
of the algorithm.
\Cref{fig:verifygrammar-function}
displays the function that implements the algorithm,
which can be seen as a sequence of four steps,
each of which we will get into later.
\begin{figure}
    \centering
    \begin{equation*}
    \begin{aligned}[t]
        & \verifygrammarcomp{g}{gas} = \\
        & \begin{aligned}[t]
            & \matchwith{\coherentfunc{g}{\PNT{0}}} \\
            & \matchcase{true}{\begin{aligned}[t]
                & \matchwith{\lcoherentfunc{g}{g}} \\
                & \matchcase{true}{\begin{aligned}[t]
                    & \matchwith{\lverifyrulecomp{g}{g}{gas}} \\
                    & \matchcase{\Some true}{\begin{aligned}[t]
                        & \matchwith{\lcheckloopscomp{g}{g}{gas}} \\
                        & \matchcase{\Some b}{\Some \neg b} \\
                        & \matchcase{\None}{\None} \\
                        & \matchend{}
                    \end{aligned}} \\
                    & \matchcase{res}{res} \\
                    & \matchend{}
                \end{aligned}} \\
                & \matchcase{false}{\Some false} \\
                & \matchend{}
            \end{aligned}} \\
            & \matchcase{false}{\Some false} \\
            & \matchend{}
        \end{aligned}
    \end{aligned}
\end{equation*}
    \caption{The well-formedness function with gas.}
    \label{fig:verifygrammar-function}
\end{figure}
The function \textit{\verifygrammarname{}}
takes a grammar and some gas as parameters,
and returns an optional Boolean value.
The gas counter serves merely to convince Coq
that the function terminates for any input.
If this counter ever reaches zero,
the function returns $\None$.
Otherwise, it returns $\Some b$,
where the Boolean value $b$
indicates whether the grammar
is well-formed or not.

We prove that,
for any input grammar $g$,
there exists a lower bound for the gas counter
for which \textit{\verifygrammarname{}} returns $\Some b$.
We do not need to assume anything
about the grammar, because the function
performs all the necessary checks
in the correct order.
This works as a proof of termination
for the algorithm.
\begin{lemma}
    \label{lemma:verifygrammar-termination}
    $\forall gas \ge \verifygrammargas{g}$,
    $\exists b\ \verifygrammarcomp{g}{gas} = \Some b$.
\end{lemma}

We define this lower bound
in \Cref{fig:verifygrammargas}.
\begin{figure}
    \centering
    \begin{equation*}
    \verifygrammargas{g} = (\length{g} + 2) \cdot \size{g}
\end{equation*}
    \caption{The well-formedness function gas lower bound.}
    \label{fig:verifygrammargas}
\end{figure}
The function takes into account
$\length{g}$, the number of rules in the grammar,
and $\size{g}$, the size of the grammar.
We define the size of a pattern $p$, also denoted as $\size{p}$,
as the number of nodes in its abstract syntax tree,
and the size of a grammar $g$
as the summation of the sizes of its rules,
that is, $\sum_{r \in g} \size{r}$.

With this lower bound,
we can define the function \textit{wf},
which takes a grammar $g$ and returns a Boolean value
indicating whether $g$ is well-formed.
\Cref{fig:wf} displays the implementation
of the function \textit{wf}.
It basically calls \textit{\verifygrammarname{}}
with $\verifygrammargas{g}$ as the gas counter.
If it returns $\Some b$, we simply return $b$.
Otherwise, we return $true$.
This default value is irrelevant,
because this case cannot happen.
Nevertheless, we return $true$
to demonstrate that the function
and gas estimation are correct.

\begin{figure}
    \centering
    \begin{equation*}
    \wf{g} = \begin{aligned}[t]
        & \matchwith{\verifygrammarcomp{g}{\verifygrammargas{g}}} \\
        & \matchcase{\Some b}{b} \\
        & \matchcase{\None}{true} \\
        & \matchend{}
    \end{aligned}
\end{equation*}
    \caption{The well-formedness function.}
    \label{fig:wf}
\end{figure}

Now that we have defined \textit{wf},
let us get into \textit{\verifygrammarname{}}.
In \Cref{fig:verifygrammar-function},
the reader may observe that
it calls four functions that we have not defined yet.
The first two functions,
\textit{\coherentname{}} and \textit{\lcoherentname{}},
do not use the gas counter,
because they operate recursively on the structure of patterns
and lists of patterns, respectively.
This type of structural recursion is enough for Coq
to determine that these functions terminate.
Meanwhile, the last two functions,
\textit{\lverifyrulename{}} and \textit{\lcheckloopsname{}},
need to visit rules recursively,
and are, therefore, defined recursively on a gas parameter.
Just like we did for
the \textit{\verifygrammarname{}} function,
we also prove that \textit{\lverifyrulename{}} and \textit{\lcheckloopsname{}}
terminate by providing a lower bound for the gas parameter.
(Without these proofs,
we would not be able to prove the termination
of \textit{\verifygrammarname{}}.)

Regarding the implementation of these functions,
we will get into each of them in the following sections.
For now, we will briefly explain what each of them does.
The first two steps are relatively simple,
while the third and fourth steps are more complex,
as they involve symbolically parsing each rule.

The first step of the algorithm is trivial.
It merely ensures that $\PNT{0}$,
the first rule of the grammar,
is defined,
given that it is used as the starting point
for parsing the grammar.
We implement this step with the function
\textit{\coherentname{}},
which takes a grammar $g$ and a pattern $p$
and returns a Boolean value
that indicates whether $p$ only references rules
that exist in $g$.

The second step is similar to the first one,
as it makes sure that every rule in the grammar
only references rules that exist in the grammar.
This ensures that we can safely
dereference any nonterminal patterns later on.
This step is implemented by the function \textit{\lcoherentname{}},
which is an extended version of \textit{\coherentname{}}
for lists of patterns.

The third step
ensures that the grammar contains no
left-recursive rules.
It does so by symbolically executing the parsing routine
for each rule in the grammar
and checking whether it can reach the same rule twice
without consuming any input.
This step is implemented by the function \textit{\lverifyrulename{}},
which takes a grammar $g$, a list of rules $rs$, and a gas counter
and returns an optional Boolean value
indicating whether all rules in the list $rs$
are not left-recursive.

The fourth and final step
looks for any degenerate loops,
which are repetitions of patterns
that may match while consuming no input.
This step is implemented by the function \textit{\lcheckloopsname{}},
and uses a simpler version
of the algorithm from the previous step.
It takes a grammar $g$, a list of rules $rs$, and a gas counter
and returns an optional Boolean value
indicating whether all rules in the list $rs$
are free of degenerate loops.

If a grammar passes all these checks,
then it is considered well-formed.
In the following sections,
we go into each of these steps
in greater detail.
We also define equivalent inductive predicates
for each step and for the \textit{\verifygrammarname{}} function,
to aid us in the proofs.
We also prove these predicates
follow the corresponding fixed-point definitions.

\section{References to nonexistent rules}
\label{section:coherent}

The verification process
starts by checking whether
every nonterminal pattern references
an existing rule in the grammar.
This process is quite simple,
but we present it here in the name of completeness.

We say a pattern is
\emph{coherent} in respect to a grammar
if all of its nonterminals reference existing rules in the grammar.
\Cref{fig:coherentfunc}
\begin{figure}
    \centering
    \begin{align*}
    \begin{aligned}[t]
        & \coherentfunc{g}{p} = \\
        & \begin{aligned}[t]
            & \matchwith{p} \\
            & \matchcase{\PEmpty}{true} \\
            & \matchcase{\PSet{cs}}{true} \\
            & \matchcase{\PNT{i}}{\begin{aligned}[t]
                & \matchwith{g[i]} \\
                & \matchcase{\Some p}{true} \\
                & \matchcase{\None}{false} \\
                & \matchend{}
            \end{aligned}} \\
            & \matchcase{\PRepetition{p}}{\coherentfunc{g}{p}} \\
            & \matchcase{\PNot{p}}{\coherentfunc{g}{p}} \\
            & \matchcase{\PAnd{p}}{\coherentfunc{g}{p}} \\
            & \matchcase{\PSequence{p_1}{p_2}}
                {\coherentfunc{g}{p_1} \wedge \coherentfunc{g}{p_2}} \\
            & \matchcase{\PChoice{p_1}{p_2}}
                {\coherentfunc{g}{p_1} \wedge \coherentfunc{g}{p_2}} \\
            & \matchend{}
        \end{aligned}
    \end{aligned}
\end{align*}
    \caption{The coherence function.}
    \label{fig:coherentfunc}
\end{figure}
defines a fixed-point function
that performs this verification.
To aid us in later induction proofs,
we also define an equivalent predicate in \Cref{fig:coherent}.
\begin{figure}
    \section{References to nonexistent rules}
\label{section:coherent}

The verification process
starts by checking whether
every nonterminal pattern references
an existing rule in the grammar.
This process is quite simple,
but we present it here in the name of completeness.

We say a pattern is
\emph{coherent} in respect to a grammar
if all of its nonterminals reference existing rules in the grammar.
\Cref{fig:coherentfunc}
\begin{figure}
    \centering
    \input{coherentfunc}
    \caption{The coherence function.}
    \label{fig:coherentfunc}
\end{figure}
defines a fixed-point function
that performs this verification.
To aid us in later induction proofs,
we also define an equivalent predicate in \Cref{fig:coherent}.
\begin{figure}
    \input{coherent}
    \caption{The coherence predicate.}
    \label{fig:coherent}
\end{figure}
\Cref{lemma:coherent-deterministic} states
that the predicate is deterministic on the result,
and \Cref{lemma:coherent-follows} states that
the predicate follows the function.
It is easy to see that
both lemmas together imply that the
predicate is equivalent to the function.

% TODO: reduce these to just one lemma as A <-> B ?

\begin{lemma}
    \label{lemma:coherent-deterministic}
    If $\Coherent{g}{p}{res_1}$ and $\Coherent{g}{p}{res_2}$,
    then $res_1 = res_2$.
\end{lemma}

\begin{lemma}
    \label{lemma:coherent-follows}
    If $\coherentfunc{g}{p} = res$, then $\Coherent{g}{p}{res}$.
\end{lemma}

\Cref{fig:lcoherent-function} trivially generalizes the coherence check for a list of patterns.
\begin{figure}
    \centering
    \include{lcoherentfunc}
    \caption{The coherence function for lists of patterns.}
    \label{fig:lcoherent-function}
\end{figure}
This function is defined over an arbitrary list of rules,
but is meant to be called for the whole grammar.
We also define, in \Cref{fig:lcoherent}, an inductive predicate
\begin{figure}
    \centering
    \input{lcoherent}
    \caption{The coherence predicate for lists of patterns.}
    \label{fig:lcoherent}
\end{figure}
equivalent to this function to be
later used in proofs by induction.
We also show that this predicate
is deterministic and follows
the original fixed-point definition.
See \Cref{lemma:lcoherent-determinism,lemma:lcoherent-follows}.

\begin{lemma}
    If $\lCoherent{g}{rs}{res_1}$,
    and $\lCoherent{g}{rs}{res_2}$,
    then $res_1 = res_2$.
    \label{lemma:lcoherent-determinism}
\end{lemma}

\begin{lemma}
    If $\lcoherentfunc{g}{rs} = res$,
    then $\lCoherent{g}{rs}{res}$.
    \label{lemma:lcoherent-follows}
\end{lemma}

Finally, we prove that
if a list of patterns passes the list-based check,
then any pattern in the list passes the individual check.

\begin{lemma}%[Coherent List Safety]
    \label{lemma:lcoherent-safety}
    If $\lCoherent{g}{rs}{true}$,
    then, $\forall r \in rs, \Coherent{g}{r}{true}$.
\end{lemma}

    \caption{The coherence predicate.}
    \label{fig:coherent}
\end{figure}
\Cref{lemma:coherent-deterministic} states
that the predicate is deterministic on the result,
and \Cref{lemma:coherent-follows} states that
the predicate follows the function.
It is easy to see that
both lemmas together imply that the
predicate is equivalent to the function.

% TODO: reduce these to just one lemma as A <-> B ?

\begin{lemma}
    \label{lemma:coherent-deterministic}
    If $\Coherent{g}{p}{res_1}$ and $\Coherent{g}{p}{res_2}$,
    then $res_1 = res_2$.
\end{lemma}

\begin{lemma}
    \label{lemma:coherent-follows}
    If $\coherentfunc{g}{p} = res$, then $\Coherent{g}{p}{res}$.
\end{lemma}

\Cref{fig:lcoherent-function} trivially generalizes the coherence check for a list of patterns.
\begin{figure}
    \centering
    \begin{equation*}
    \lcoherentfunc{g}{rs} = \begin{aligned}[t]
        & \matchwith{rs} \\
        & \matchcase{nil}{true} \\
        & \matchcase{r::rs'}{\coherentfunc{g}{r} \wedge \lcoherentfunc{g}{rs'}} \\
        & \matchend{}
    \end{aligned}
\end{equation*}
    \caption{The coherence function for lists of patterns.}
    \label{fig:lcoherent-function}
\end{figure}
This function is defined over an arbitrary list of rules,
but is meant to be called for the whole grammar.
We also define, in \Cref{fig:lcoherent}, an inductive predicate
\begin{figure}
    \centering
    \begin{mathpar}
    \namedinferrule{lc-nil}
    { }
    {\lCoherent{g}{nil}{true}}

    \namedinferrule{lc-cons}
    {\Coherent{g}{r}{b_1} \\ \lCoherent{g}{rs}{b_2}}
    {\lCoherent{g}{r::rs}{b_1 \wedge b_2}}
\end{mathpar}
    \caption{The coherence predicate for lists of patterns.}
    \label{fig:lcoherent}
\end{figure}
equivalent to this function to be
later used in proofs by induction.
We also show that this predicate
is deterministic and follows
the original fixed-point definition.
See \Cref{lemma:lcoherent-determinism,lemma:lcoherent-follows}.

\begin{lemma}
    If $\lCoherent{g}{rs}{res_1}$,
    and $\lCoherent{g}{rs}{res_2}$,
    then $res_1 = res_2$.
    \label{lemma:lcoherent-determinism}
\end{lemma}

\begin{lemma}
    If $\lcoherentfunc{g}{rs} = res$,
    then $\lCoherent{g}{rs}{res}$.
    \label{lemma:lcoherent-follows}
\end{lemma}

Finally, we prove that
if a list of patterns passes the list-based check,
then any pattern in the list passes the individual check.

\begin{lemma}%[Coherent List Safety]
    \label{lemma:lcoherent-safety}
    If $\lCoherent{g}{rs}{true}$,
    then, $\forall r \in rs, \Coherent{g}{r}{true}$.
\end{lemma}

\section{Left-recursive rules}
\label{section:lr-rules}

In general,
we consider a rule to be left-recursive
if it can wind up in itself
without consuming any input in-between.
This brings us to the heart
of the algorithm that
detects left-recursive rules.
On a high level,
it symbolically parses each rule,
until it either consumes some input,
visits some rule twice without consuming any input,
or simply finishes.

The algorithm categorizes patterns into three groups.
If a pattern can be parsed until its end without consuming any input,
it is said to be \emph{nullable}.
If, otherwise, it always consumes some input,
it is categorized as \emph{non-nullable}.
Alternatively, if it can lead to some rule twice,
without consuming any input,
it is categorized as left-recursive.

In order to check whether a pattern
is guaranteed to consume some input,
the algorithm uses a conservative approximation
proposed by Ford~\cite{ford_parsing_2004},
which makes two assumptions.
The first one is that $\PNot{p}$ may match,
and the second one is that,
in the case of $\PChoice{p_1}{p_2}$,
it may visit $p_2$,
without checking whether $p_1$ always matches.
For illustrative purposes,
we present a simple counterexample
for each assumption.

A counterexample for the first assumption
is the pattern $\PNot{\PEmpty}$,
which never matches.
Meanwhile, the second assumption doesn't hold
for the pattern $\PChoice{\PEmpty}{p_2}$,
because $\PEmpty$ always matches,
and, therefore, $p_2$ is never visited.
The reader might think that
these cases can be easily spotted,
by the simplicity of the counterexamples.
However,
Ford~\cite{ford_parsing_2004} proved that
the general case of this problem is undecidable.

One of the conditions for the algorithm to yield a result
is whenever it consumes some input.
As a result,
it exclusively visits patterns that may be reached
without consuming any input.
Therefore, if the algorithm revisits a rule,
this means a path exists in which
the parsing routine may reach the same rule
and with the same input string,
which would indicate that such rule is left-recursive.
We will now discuss possible ways to detect
when a rule has been visited twice.

One possible way to detect left-recursive rules
is through a set of visited rules,
which is checked and updated
every time a nonterminal pattern is visited.
For a grammar with $n$ rules,
this set could be implemented
as an array of $n$ Boolean values,
each representing a rule.
This method achieves
a computation and spatial
completity of $O(n)$.

Another approach,
which is simpler and takes less memory space,
uses a counter of visited rules,
which starts at zero
and gets incremented every time a rule is visited.
If this value ever surpasses the number of grammar rules,
then we know, by the pigeonhole principle,
that some rule has been visited more than once.
In the case of grammars with left-recursive rules,
we may visit more rules than necessary,
however, we are not particularly worried
about the performance of the algorithm
in the case of errors.

\lpeg{} adopts this last approach.
Our formalization follows \lpeg{},
though with a small twist:
instead of counting visited rules from zero until the limit,
we count to-be-visited rules from the limit down to zero.
This simplifies our formalization
by moving the limit calculation out of the algorithm body,
and letting the limit be passed down as a parameter instead.

At this point,
it is important to draw a distinction between
exhausting the counter of to-be-visited rules
and correctly identifying a left-recursive rule.
When the algorithm starts,
the counter is initialized with the provided limit.
It is then decremented every time a rule is visited.
If the counter ever reaches zero,
then attempting to visit any rule
will return an error.
The algorithm does not determine
whether this error indicates left recursion,
because it would require the algorithm to check
whether the limit is greater than the number of grammar rules.
Instead, we leave it to the caller
to provide a high enough limit,
in which case the algorithm indeed
correctly labels rules as left-recursive
by returning an error.

Because of this shift in responsibilities,
we adapt the nomenclature
for the counter parameter and associated error,
based on an analogy with call stacks.
If, every time a rule is visited, it were pushed onto a stack $k$,
then we could think of the counter parameter $d$ as the stack depth limit;
and surpassing it would be similar to a stack overflow error.

For the sole purpose of helping us prove certain properties about the algorithm,
we will also include this stack $k$, a list of rule indices, as an output,
though it doesn't affect the algorithm.
As we will soon see, it is either appended, passed along, or ignored.
It works as a trace of the inner workings of the function,
a high-level concept we only use for proving lemmas about this algorithm.
\lpeg{} also implements this output, but it is only used
when formatting error messages about left-recursive rules.

We now describe the algorithm
for detecting left-recursive rules,
starting with its inputs and outputs.
It receives a pattern, a grammar, and a stack depth limit,
and returns a label and a stack.
We represent labels by
optional Boolean values
$\Some true$ (nullable),
$\Some false$ (non-nullable) and
$\None$ (stack overflow error);
and stacks by either
$nil$ (an empty stack) or
$i :: k$ (a rule of index $i$ concatenated with a stack $k$).
For now, we will work with this signature,
but beware that the actual function,
displayed in \Cref{fig:vr-function},
\begin{figure}
    \centering
    \begin{align*}
    \begin{aligned}[t]
        & \verifyrulecomp{g}{p}{d}{nb}{0} = \None \\
        & \verifyrulecomp{g}{p}{d}{nb}{(1+gas)} = \\
        & \begin{aligned}[t]
            & \matchwith{p} \\
            & \matchcase{\PEmpty}{\Some(\Some true, nil)} \\
            & \matchcase{\PSet{cs}}{\Some(\Some nb, nil)} \\
            & \matchcase{\PRepetition{p'}}{\verifyrulecomp{g}{p'}{d}{true}{gas}} \\
            & \matchcase{\PNot{p'}}{\verifyrulecomp{g}{p'}{d}{true}{gas}} \\
            & \matchcase{\PAnd{p'}}{\verifyrulecomp{g}{p'}{d}{true}{gas}} \\
            & \matchcase{\PNT{i}}{\begin{aligned}[t]
                & \matchwith{g[i]} \\
                & \matchcase{\None}{\None} \\
                & \matchcase{\Some p'}{\begin{aligned}[t]
                    & \matchwith{d} \\
                    & \matchcase{0}{\Some(\None, nil)} \\
                    & \matchcase{1+d'}{\begin{aligned}[t]
                        & \matchwith{\verifyrulecomp{g}{p'}{d'}{nb}{gas}} \\
                        & \matchcase{\Some (res, k)}{\Some (res, i :: k)} \\
                        & \matchcase{\None}{\None} \\
                        & \matchend{}
                    \end{aligned}} \\
                    & \matchend{}
                \end{aligned}} \\
                & \matchend{}
            \end{aligned}} \\
            & \matchcase{\PSequence{p_1}{p_2}}{\begin{aligned}[t]
                & \matchwith{\verifyrulecomp{g}{p_1}{d}{false}{gas}} \\
                & \matchcase{\Some(\Some true, k)}{\verifyrulecomp{g}{p_2}{d}{nb}{gas}} \\
                & \matchcase{\Some(\Some false, k)}{\Some(\Some nb, k)} \\
                & \matchcase{res}{res} \\
                & \matchend{}
            \end{aligned}} \\
            & \matchcase{\PChoice{p_1}{p_2}}{\begin{aligned}[t]
                & \matchwith{\verifyrulecomp{g}{p_1}{d}{nb}{gas}} \\
                & \matchcase{\Some(\Some nb', k)}{\verifyrulecomp{g}{p_2}{d}{nb'}{gas}} \\
                & \matchcase{res}{res} \\
                & \matchend{}
            \end{aligned}} \\
            & \matchend{}
        \end{aligned}
    \end{aligned}
\end{align*}
    \caption{The left recursion detection function.}
    \label{fig:vr-function}
\end{figure}
receives an extra parameter
which we will introduce later in this section.

The function is defined recursively.
In most cases, it calls itself for each sub-pattern.
In the case of nonterminal patterns, however,
it calls itself for the referenced rule.
Furthermore, the function propagates any stack overflow errors.
This means that,
if some recursive call returns $\None$,
signaling a stack overflow,
and a stack $k$,
then the function also returns $\None$ and $k$.

For the empty pattern $\PEmpty$,
the function returns a label $\Some true$ and $nil$,
because it is nullable and doesn't visit any nonterminal.
We categorize it as nullable because it may match while consuming no input.
In particular, it always matches while consuming no input.

As for character set patterns $\PSet{cs}$,
the function returns $\Some false$ and $nil$,
because it is non-nullable,
meaning it always consumes some input when it matches.
It also doesn't visit any nonterminal.

For a nonterminal pattern $\PNT{i}$,
the function first checks the stack depth limit $d$.
If $d=0$, it returns $\None$ and $nil$,
signaling a stack overflow
and that it didn't visit any nonterminal.
Otherwise, if $d\ge1$,
then the function calls itself for the $i^{th}$ rule of the grammar,
while passing a stack depth limit of $d-1$.
If this recursive call returns a label $res$ and a stack $k$,
then the function returns $res$ and $i :: k$.
This way, the stack accumulates
the indices of the grammar rules
in the same order in which they are visited.

For a repetition pattern $\PRepetition{p}$,
the function evaluates $p$,
which returns $res$ and $k$,
to check for any stack overflow errors.
If $res\ne\None$,
then it returns $\Some true$ and $k$,
as it can match while consuming no input,
in case $p$ fails.
We assume that $p$ can fail because,
in the final verification step,
we ensure that $p$ is non-nullable,
and we know that non-nullable patterns
fail to match the empty string.

Predicate patterns $\PNot{p}$ and $\PAnd{p}$ are evaluated
in the same way as repetition patterns,
but for different reasons.
Repetition patterns are nullable
because they can always match without consuming any input.
Meanwhile, predicate patterns are nullable by approximation,
under the assumption that $p$ may match,
in the case of $\PAnd{p}$,
or fail to match,
in the case of $\PNot{p}$.

For a sequence pattern $\PSequence{p_1}{p_2}$,
the function first evaluates $p_1$.
If $p_1$ is non-nullable,
then so is the sequence $\PSequence{p_1}{p_2}$,
and the function returns the same label and stack as $p_1$.
Note that $p_2$ is not even evaluated in this case,
because it would be visited with a shorter input string
during parsing.
This is the only case in which
the nullable property comes into play
in this algorithm.
If, otherwise, $p_1$ is nullable,
then it evaluates $p_2$
and returns the same label and stack as $p_2$.

Finally, for a choice pattern $\PChoice{p_1}{p_2}$,
it first evaluates $p_1$.
If it returns $\Some b_1$,
indicating that $p_1$ did not overflow the stack,
then it evaluates $p_2$.
If it also returns $\Some b_2$,
then the function returns $\Some (b_1 \vee b_2)$
and the same stack as $p_2$.

The algorithm we've just described
is quite similar to the one implemented in \lpeg{}.
There is, however, one small difference
related to the use of tail calls
as an optimization technique.
In C, tail calls are implemented
with \texttt{goto} statements.
To apply this optimization technique,
\lpeg{} adds an extra parameter to the function
to work as an accumulator for the nullable property.
Without this accumulator parameter,
the evaluation of choice patterns $\PChoice{p_1}{p_2}$
would rely solely on recursion.
It would evaluate $p_1$ and $p_2$,
then perform a Boolean \scor{} operation on the results.

With the addition of a Boolean parameter $nb$,
we can turn the evaluation of $p_2$ into a tail call.
Instead of making the Boolean \scor{} operation explicitly,
we let the accumulator do it under-the-hood.
This works because,
in the base cases of the recursion,
in which the function would return
either $\Some true$ or $\Some false$,
we return instead $\Some (true \vee nb)$ and $\Some (false \vee nb)$,
which get simplified to $\Some true$ and $\Some nb$, respectively.
\Cref{fig:evalchoice} shows a Coq-like pseudocode of how choice patterns
\begin{figure}
    \centering
    \newcommand{\eval}[4]{eval\ #1\ #2\ #3\ #4}

\begin{align*}
    \eval{g}{\dsqb{\PChoice{p_1}{p_2}}}{d}{nb} =\
    & \matchwith{\eval{g}{p_1}{d}{nb}} \\
    & \matchcase{(\Some nb', k)}{\eval{g}{p_2}{d}{nb'}} \\
    & \matchcase{(\None, k)}{(\None, k)} \\
    & \matchend{}
\end{align*}
    \caption{Pseudocode of the evaluation of choice patterns.}
    \label{fig:evalchoice}
\end{figure}
are evaluated with the Boolean parameter $nb$.

We would also like to highlight
how this nullable accumulator allows
the evaluation of repetitions and predicates
to be rewritten as tail calls.
Previously,
we would have to check if $p$ evaluated to $\None$,
before returning $\Some true$.
Now, we can simply pass $true$ as the nullable accumulator,
which guarantees that, if $p$ does not evaluate to $\None$,
it evaluates to $\Some true$.

There are some ways in which this function
could be implemented in Coq as a fixed-point.
The classical way is to add a gas parameter,
which gets decremented in every recursive call.
We make the function return an optional value,
such that, if the gas parameter ever reaches zero,
it returns $\None$.
Other ways are providing a well-formedness proof,
or a measure function.
We choose the first strategy,
because it is the simplest to implement.

Finally,
\Cref{fig:vr-function} presents the algorithm
defined as a fixed-point function.
It returns an optional value,
because we adopted the gas strategy,
but also because it cannot evaluate
nonterminal patterns that
reference nonexistent rules.
In this case,
the function also returns $\None$.
In all other cases,
the function returns $\Some (res, k)$,
with $res$ being a label, and $k$, a stack.

At this point,
the reader should be warned that
we will not attempt to prove the correctness
of this function in isolation.
In fact, we will not even try to formally
define left-recursive rules.
This might frustrate the reader,
but we assure you that such proof will not be necessary.
Instead,
we will later prove the correctness of the whole algorithm
once we introduce all steps of the verification process.
In this section,
we will simply prove
that the label returned by the function
is monotonic and eventually constant
with respect to the gas counter and stack depth limit.

A function $f$ is said to be monotonically increasing
if, for any $x$ and $y$, such that $x \le y$,
it is always true that $f(x) \le f(y)$.
In the case of the \textit{\verifyrulename{}} function,
this will be true for the gas counter and stack depth limit parameters,
and the order between optionals is $\None < \Some res$, for any $res$.
This means that, if the function ever returns $\Some res$,
increasing the gas counter or the stack depth limit
will not alter the return value.

A function $f$ is eventually constant
if, for some $N$ and for any $x$ and $y$,
such that $x, y \ge N$,
it is true that $f(x) = f(y)$.
In the case of \textit{\verifyrulename{}} function,
this will be true for the gas counter and stack depth limit parameters
and for the label return value.
This means that both the gas counter and stack depth limit
have lower bounds for which the returned label stabilizes.

About this fixed-point definition,
we will initially prove some basic lemmas.
Starting with \Cref{lemma:vr-gas-monotonicity},
we state that, if the function returns $\Some (res, k)$,
then increasing the value of the gas parameter
will not change the result.
This is what we mean by
the function being monotonic and eventually constant
with respect to the gas counter.

\begin{lemma}%[Verify Rule Gas Eventual Constancy]
    If $\verifyrulecomp{g}{p}{d}{nb}{gas} = \Some (res, k)$, \\
    then $\forall gas' \ge gas$,
    $\verifyrulecomp{g}{p}{d}{nb}{gas'} = \Some (res, k)$.
    \label{lemma:vr-gas-monotonicity}
\end{lemma}

\Cref{lemma:vr-termination} states that,
for any coherent pattern and grammar,
there exists a lower bound for the gas parameter,
for which the function returns $\Some (res, k)$.
The lower bound
takes into account
the size of the pattern $\size{p}$,
the size of the grammar $\size{g}$,
and the stack depth limit $d$.

\begin{lemma}%[Verify Rule Termination]
    If $\Coherent{g}{p}{true}$,
    and $\lCoherent{g}{g}{true}$, \\
    then, $\forall gas \ge \size{p} + d \cdot \size{g}$,
    $\exists res\ \exists k\ \verifyrulecomp{g}{p}{d}{nb}{gas} = \Some (res, k)$.
    \label{lemma:vr-termination}
\end{lemma}

\begin{proof}
    For most patterns,
    the proof follows from induction on the pattern $p$.
    Meanwhile, for non-terminal patterns,
    the proof follows from induction on the stack depth limit $d$.
    We show below how the lower bound for a rule $r$ and stack depth limit $d$
    is derived from a non-terminal $\PNT{i}$ that references $r$ and stack depth limit $d+1$.
    We use \Cref{lemma:size-of-r-le-size-of-g} to show that $\size{g} \ge \size{r}$.
    \begin{align*}
        gas & \ge \size{\PNT{i}} + (d + 1) \cdot \size{g} \\
        & \ge 1 + (d + 1) \cdot \size{g} \\
        & \ge 1 + \size{g} + d \cdot \size{g} \\
        & \ge 1 + \size{r} + d \cdot \size{g}
    \end{align*}
\end{proof}

Now, we would like to
prove that the label returned by the \textit{\verifyrulename{}} function
is monotonic and eventually constant with respect to the
stack depth limit.
We discard the returned stack in this context because,
in the case of left-recursive rules,
the stack returned by the function will,
in fact,
diverge.
However, we are not interested in
the output stack, in this case.
What really matters to the following
steps of the verification process
is the label.
In particular, we would like to make sure
that no rule in the grammar is marked
with the label $\None$,
meaning ``stack overflow''.

In order to prove such lemma,
we realized an inductive, gasless predicate
would be better suited
than the fixed-point definition,
as it would be easier to perform proofs by induction,
and without having to deal with a gas parameter.
\Cref{fig:verifyrule}
\begin{figure}[ht!]
    \section{Left-recursive rules}
\label{section:lr-rules}

In general,
we consider a rule to be left-recursive
if it can wind up in itself
without consuming any input in-between.
This brings us to the heart
of the algorithm that
detects left-recursive rules.
On a high level,
it symbolically parses each rule,
until it either consumes some input,
visits some rule twice without consuming any input,
or simply finishes.

The algorithm categorizes patterns into three groups.
If a pattern can be parsed until its end without consuming any input,
it is said to be \emph{nullable}.
If, otherwise, it always consumes some input,
it is categorized as \emph{non-nullable}.
Alternatively, if it can lead to some rule twice,
without consuming any input,
it is categorized as left-recursive.

In order to check whether a pattern
is guaranteed to consume some input,
the algorithm uses a conservative approximation
proposed by Ford~\cite{ford_parsing_2004},
which makes two assumptions.
The first one is that $\PNot{p}$ may match,
and the second one is that,
in the case of $\PChoice{p_1}{p_2}$,
it may visit $p_2$,
without checking whether $p_1$ always matches.
For illustrative purposes,
we present a simple counterexample
for each assumption.

A counterexample for the first assumption
is the pattern $\PNot{\PEmpty}$,
which never matches.
Meanwhile, the second assumption doesn't hold
for the pattern $\PChoice{\PEmpty}{p_2}$,
because $\PEmpty$ always matches,
and, therefore, $p_2$ is never visited.
The reader might think that
these cases can be easily spotted,
by the simplicity of the counterexamples.
However,
Ford~\cite{ford_parsing_2004} proved that
the general case of this problem is undecidable.

One of the conditions for the algorithm to yield a result
is whenever it consumes some input.
As a result,
it exclusively visits patterns that may be reached
without consuming any input.
Therefore, if the algorithm revisits a rule,
this means a path exists in which
the parsing routine may reach the same rule
and with the same input string,
which would indicate that such rule is left-recursive.
We will now discuss possible ways to detect
when a rule has been visited twice.

One possible way to detect left-recursive rules
is through a set of visited rules,
which is checked and updated
every time a nonterminal pattern is visited.
For a grammar with $n$ rules,
this set could be implemented
as an array of $n$ Boolean values,
each representing a rule.
This method achieves
a computation and spatial
completity of $O(n)$.

Another approach,
which is simpler and takes less memory space,
uses a counter of visited rules,
which starts at zero
and gets incremented every time a rule is visited.
If this value ever surpasses the number of grammar rules,
then we know, by the pigeonhole principle,
that some rule has been visited more than once.
In the case of grammars with left-recursive rules,
we may visit more rules than necessary,
however, we are not particularly worried
about the performance of the algorithm
in the case of errors.

\lpeg{} adopts this last approach.
Our formalization follows \lpeg{},
though with a small twist:
instead of counting visited rules from zero until the limit,
we count to-be-visited rules from the limit down to zero.
This simplifies our formalization
by moving the limit calculation out of the algorithm body,
and letting the limit be passed down as a parameter instead.

At this point,
it is important to draw a distinction between
exhausting the counter of to-be-visited rules
and correctly identifying a left-recursive rule.
When the algorithm starts,
the counter is initialized with the provided limit.
It is then decremented every time a rule is visited.
If the counter ever reaches zero,
then attempting to visit any rule
will return an error.
The algorithm does not determine
whether this error indicates left recursion,
because it would require the algorithm to check
whether the limit is greater than the number of grammar rules.
Instead, we leave it to the caller
to provide a high enough limit,
in which case the algorithm indeed
correctly labels rules as left-recursive
by returning an error.

Because of this shift in responsibilities,
we adapt the nomenclature
for the counter parameter and associated error,
based on an analogy with call stacks.
If, every time a rule is visited, it were pushed onto a stack $k$,
then we could think of the counter parameter $d$ as the stack depth limit;
and surpassing it would be similar to a stack overflow error.

For the sole purpose of helping us prove certain properties about the algorithm,
we will also include this stack $k$, a list of rule indices, as an output,
though it doesn't affect the algorithm.
As we will soon see, it is either appended, passed along, or ignored.
It works as a trace of the inner workings of the function,
a high-level concept we only use for proving lemmas about this algorithm.
\lpeg{} also implements this output, but it is only used
when formatting error messages about left-recursive rules.

We now describe the algorithm
for detecting left-recursive rules,
starting with its inputs and outputs.
It receives a pattern, a grammar, and a stack depth limit,
and returns a label and a stack.
We represent labels by
optional Boolean values
$\Some true$ (nullable),
$\Some false$ (non-nullable) and
$\None$ (stack overflow error);
and stacks by either
$nil$ (an empty stack) or
$i :: k$ (a rule of index $i$ concatenated with a stack $k$).
For now, we will work with this signature,
but beware that the actual function,
displayed in \Cref{fig:vr-function},
\begin{figure}
    \centering
    \input{verifyrulecomp}
    \caption{The left recursion detection function.}
    \label{fig:vr-function}
\end{figure}
receives an extra parameter
which we will introduce later in this section.

The function is defined recursively.
In most cases, it calls itself for each sub-pattern.
In the case of nonterminal patterns, however,
it calls itself for the referenced rule.
Furthermore, the function propagates any stack overflow errors.
This means that,
if some recursive call returns $\None$,
signaling a stack overflow,
and a stack $k$,
then the function also returns $\None$ and $k$.

For the empty pattern $\PEmpty$,
the function returns a label $\Some true$ and $nil$,
because it is nullable and doesn't visit any nonterminal.
We categorize it as nullable because it may match while consuming no input.
In particular, it always matches while consuming no input.

As for character set patterns $\PSet{cs}$,
the function returns $\Some false$ and $nil$,
because it is non-nullable,
meaning it always consumes some input when it matches.
It also doesn't visit any nonterminal.

For a nonterminal pattern $\PNT{i}$,
the function first checks the stack depth limit $d$.
If $d=0$, it returns $\None$ and $nil$,
signaling a stack overflow
and that it didn't visit any nonterminal.
Otherwise, if $d\ge1$,
then the function calls itself for the $i^{th}$ rule of the grammar,
while passing a stack depth limit of $d-1$.
If this recursive call returns a label $res$ and a stack $k$,
then the function returns $res$ and $i :: k$.
This way, the stack accumulates
the indices of the grammar rules
in the same order in which they are visited.

For a repetition pattern $\PRepetition{p}$,
the function evaluates $p$,
which returns $res$ and $k$,
to check for any stack overflow errors.
If $res\ne\None$,
then it returns $\Some true$ and $k$,
as it can match while consuming no input,
in case $p$ fails.
We assume that $p$ can fail because,
in the final verification step,
we ensure that $p$ is non-nullable,
and we know that non-nullable patterns
fail to match the empty string.

Predicate patterns $\PNot{p}$ and $\PAnd{p}$ are evaluated
in the same way as repetition patterns,
but for different reasons.
Repetition patterns are nullable
because they can always match without consuming any input.
Meanwhile, predicate patterns are nullable by approximation,
under the assumption that $p$ may match,
in the case of $\PAnd{p}$,
or fail to match,
in the case of $\PNot{p}$.

For a sequence pattern $\PSequence{p_1}{p_2}$,
the function first evaluates $p_1$.
If $p_1$ is non-nullable,
then so is the sequence $\PSequence{p_1}{p_2}$,
and the function returns the same label and stack as $p_1$.
Note that $p_2$ is not even evaluated in this case,
because it would be visited with a shorter input string
during parsing.
This is the only case in which
the nullable property comes into play
in this algorithm.
If, otherwise, $p_1$ is nullable,
then it evaluates $p_2$
and returns the same label and stack as $p_2$.

Finally, for a choice pattern $\PChoice{p_1}{p_2}$,
it first evaluates $p_1$.
If it returns $\Some b_1$,
indicating that $p_1$ did not overflow the stack,
then it evaluates $p_2$.
If it also returns $\Some b_2$,
then the function returns $\Some (b_1 \vee b_2)$
and the same stack as $p_2$.

The algorithm we've just described
is quite similar to the one implemented in \lpeg{}.
There is, however, one small difference
related to the use of tail calls
as an optimization technique.
In C, tail calls are implemented
with \texttt{goto} statements.
To apply this optimization technique,
\lpeg{} adds an extra parameter to the function
to work as an accumulator for the nullable property.
Without this accumulator parameter,
the evaluation of choice patterns $\PChoice{p_1}{p_2}$
would rely solely on recursion.
It would evaluate $p_1$ and $p_2$,
then perform a Boolean \scor{} operation on the results.

With the addition of a Boolean parameter $nb$,
we can turn the evaluation of $p_2$ into a tail call.
Instead of making the Boolean \scor{} operation explicitly,
we let the accumulator do it under-the-hood.
This works because,
in the base cases of the recursion,
in which the function would return
either $\Some true$ or $\Some false$,
we return instead $\Some (true \vee nb)$ and $\Some (false \vee nb)$,
which get simplified to $\Some true$ and $\Some nb$, respectively.
\Cref{fig:evalchoice} shows a Coq-like pseudocode of how choice patterns
\begin{figure}
    \centering
    \input{evalchoice}
    \caption{Pseudocode of the evaluation of choice patterns.}
    \label{fig:evalchoice}
\end{figure}
are evaluated with the Boolean parameter $nb$.

We would also like to highlight
how this nullable accumulator allows
the evaluation of repetitions and predicates
to be rewritten as tail calls.
Previously,
we would have to check if $p$ evaluated to $\None$,
before returning $\Some true$.
Now, we can simply pass $true$ as the nullable accumulator,
which guarantees that, if $p$ does not evaluate to $\None$,
it evaluates to $\Some true$.

There are some ways in which this function
could be implemented in Coq as a fixed-point.
The classical way is to add a gas parameter,
which gets decremented in every recursive call.
We make the function return an optional value,
such that, if the gas parameter ever reaches zero,
it returns $\None$.
Other ways are providing a well-formedness proof,
or a measure function.
We choose the first strategy,
because it is the simplest to implement.

Finally,
\Cref{fig:vr-function} presents the algorithm
defined as a fixed-point function.
It returns an optional value,
because we adopted the gas strategy,
but also because it cannot evaluate
nonterminal patterns that
reference nonexistent rules.
In this case,
the function also returns $\None$.
In all other cases,
the function returns $\Some (res, k)$,
with $res$ being a label, and $k$, a stack.

At this point,
the reader should be warned that
we will not attempt to prove the correctness
of this function in isolation.
In fact, we will not even try to formally
define left-recursive rules.
This might frustrate the reader,
but we assure you that such proof will not be necessary.
Instead,
we will later prove the correctness of the whole algorithm
once we introduce all steps of the verification process.
In this section,
we will simply prove
that the label returned by the function
is monotonic and eventually constant
with respect to the gas counter and stack depth limit.

A function $f$ is said to be monotonically increasing
if, for any $x$ and $y$, such that $x \le y$,
it is always true that $f(x) \le f(y)$.
In the case of the \textit{\verifyrulename{}} function,
this will be true for the gas counter and stack depth limit parameters,
and the order between optionals is $\None < \Some res$, for any $res$.
This means that, if the function ever returns $\Some res$,
increasing the gas counter or the stack depth limit
will not alter the return value.

A function $f$ is eventually constant
if, for some $N$ and for any $x$ and $y$,
such that $x, y \ge N$,
it is true that $f(x) = f(y)$.
In the case of \textit{\verifyrulename{}} function,
this will be true for the gas counter and stack depth limit parameters
and for the label return value.
This means that both the gas counter and stack depth limit
have lower bounds for which the returned label stabilizes.

About this fixed-point definition,
we will initially prove some basic lemmas.
Starting with \Cref{lemma:vr-gas-monotonicity},
we state that, if the function returns $\Some (res, k)$,
then increasing the value of the gas parameter
will not change the result.
This is what we mean by
the function being monotonic and eventually constant
with respect to the gas counter.

\begin{lemma}%[Verify Rule Gas Eventual Constancy]
    If $\verifyrulecomp{g}{p}{d}{nb}{gas} = \Some (res, k)$, \\
    then $\forall gas' \ge gas$,
    $\verifyrulecomp{g}{p}{d}{nb}{gas'} = \Some (res, k)$.
    \label{lemma:vr-gas-monotonicity}
\end{lemma}

\Cref{lemma:vr-termination} states that,
for any coherent pattern and grammar,
there exists a lower bound for the gas parameter,
for which the function returns $\Some (res, k)$.
The lower bound
takes into account
the size of the pattern $\size{p}$,
the size of the grammar $\size{g}$,
and the stack depth limit $d$.

\begin{lemma}%[Verify Rule Termination]
    If $\Coherent{g}{p}{true}$,
    and $\lCoherent{g}{g}{true}$, \\
    then, $\forall gas \ge \size{p} + d \cdot \size{g}$,
    $\exists res\ \exists k\ \verifyrulecomp{g}{p}{d}{nb}{gas} = \Some (res, k)$.
    \label{lemma:vr-termination}
\end{lemma}

\begin{proof}
    For most patterns,
    the proof follows from induction on the pattern $p$.
    Meanwhile, for non-terminal patterns,
    the proof follows from induction on the stack depth limit $d$.
    We show below how the lower bound for a rule $r$ and stack depth limit $d$
    is derived from a non-terminal $\PNT{i}$ that references $r$ and stack depth limit $d+1$.
    We use \Cref{lemma:size-of-r-le-size-of-g} to show that $\size{g} \ge \size{r}$.
    \begin{align*}
        gas & \ge \size{\PNT{i}} + (d + 1) \cdot \size{g} \\
        & \ge 1 + (d + 1) \cdot \size{g} \\
        & \ge 1 + \size{g} + d \cdot \size{g} \\
        & \ge 1 + \size{r} + d \cdot \size{g}
    \end{align*}
\end{proof}

Now, we would like to
prove that the label returned by the \textit{\verifyrulename{}} function
is monotonic and eventually constant with respect to the
stack depth limit.
We discard the returned stack in this context because,
in the case of left-recursive rules,
the stack returned by the function will,
in fact,
diverge.
However, we are not interested in
the output stack, in this case.
What really matters to the following
steps of the verification process
is the label.
In particular, we would like to make sure
that no rule in the grammar is marked
with the label $\None$,
meaning ``stack overflow''.

In order to prove such lemma,
we realized an inductive, gasless predicate
would be better suited
than the fixed-point definition,
as it would be easier to perform proofs by induction,
and without having to deal with a gas parameter.
\Cref{fig:verifyrule}
\begin{figure}[ht!]
    \input{verifyrule}
    \caption{The left recursion detection predicate.}
    \label{fig:verifyrule}
\end{figure}
defines such predicate,
denoted as $\VerifyRule{g}{p}{d}{nb}{res}{k}$.
It takes a grammar $g$,
a pattern $p$,
a stack depth limit $d$,
and a nullable accumulator $nb$,
and outputs a result $res$,
and a stack trace $k$.

In order to reach our final goal
of proving that the label returned
by the \textit{\verifyrulename{}} function
is monotonic and eventually constant
with respect to the stack depth limit,
we need to first prove some intermediary lemmas.
First, we need to relate the predicate and
the fixed-point definition together,
so that we can apply the proofs about the
former to the latter.

We begin with
\Cref{lemma:vr-determinism},
which states that,
for identical input,
the predicate yields the same output.
We can therefore state
that the predicate is deterministic.

\begin{lemma}
    If $\VerifyRule{g}{p}{d}{nb}{res_1}{k_1}$,
    and $\VerifyRule{g}{p}{d}{nb}{res_2}{k_2}$, \\
    then $res_1 = res_2$ and $k_1 = k_2$.
    \label{lemma:vr-determinism}
\end{lemma}

\Cref{lemma:vr-follows} shows that
every result returned by the fixed-point definition
can be inductively constructed using the predicate definition.

\begin{lemma}
    If $\verifyrulecomp{g}{p}{d}{nb}{gas} = \Some (res, k)$, \\
    then $\VerifyRule{g}{p}{d}{nb}{res}{k}$.
    \label{lemma:vr-follows}
\end{lemma}

\Cref{lemma:stack-depth-monotonicity-not-lr-pattern} shows that,
if a pattern evaluates to either nullable or non-nullable,
then increasing the stack depth limit
doesn't affect the result.
This is expected,
because, in both cases,
it doesn't surpass the limit,
and increasing it preserves this property
by transitivity.

\begin{lemma}%[Stack Depth Limit Increase without Overflow]
    If $\VerifyRule{g}{p}{d}{nb}{\Some nb'}{k}$, \\
    then $\forall d' \ge d, \VerifyRule{g}{p}{d'}{nb}{\Some nb'}{k}$.
    \label{lemma:stack-depth-monotonicity-not-lr-pattern}
\end{lemma}

\Cref{lemma:stack-depth-lr-pattern} shows that,
on stack overflow,
the output stack $k$ has length $d$.
That is expected,
because a stack overflow happens
when the stack is full
before trying to visit a rule.

\begin{lemma}%[Stack Length on Overflow]
    If $\VerifyRule{g}{p}{d}{nb}{\None}{k}$,
    then $\length{k} = d$.
    \label{lemma:stack-depth-lr-pattern}
\end{lemma}

\Cref{lemma:coherent-stack} states
that the output stack
only contains references to
existing rules in the grammar.
Since we are identifying rules
by their indices in a list,
we prove this by showing that these indices
are less than the number of rules in the grammar,
denoted as $\length{g}$.
This lemma may seem trivial,
but it is necessary for us to later prove,
using the pigeonhole principle,
that a stack with more rules than the grammar
will have at least one repeated rule.

\begin{lemma}%[Coherent Stack]
    If $\VerifyRule{g}{p}{d}{nb}{res}{k}$,
    then $\forall i \in k, i < \length{g}$.
    \label{lemma:coherent-stack}
\end{lemma}

% Next, we prove in \Cref{lemma:true-as-nb} that,
% when provided with $true$ as the value
% for the nullable accumulator $nb$,
% the algorithm outputs as a result
% either $\None$ (which means ``left-recursive'')
% or $\Some true$ (which means ``nullable'').

% \begin{lemma}%[True As Nullable Accumulator]
%     For a grammar $g$,
%     a pattern $p$,
%     a natural number $d$,
%     an optional Boolean value $res$,
%     and a list of natural numbers $k$,
%     if $\VerifyRule{g}{p}{d}{true}{res}{k}$,
%     then $res \in \{\None,\ \Some true\}$.
%     \label{lemma:true-as-nb}
% \end{lemma}

% We also state in \Cref{lemma:nb-change-not-lr-pattern} that,
% when a pattern is identified as either nullable or non-nullable,
% then changing the nullable accumulator
% might change the result to either nullable or non-nullable,
% but the output stack $k$ will remain the same.
% Note that we're saying that it will not make
% the pattern be marked as left-recursive.

% \begin{lemma}%[Nullable Accumulator Change When Pattern Is Not Left-Recursive]
%     For a grammar $g$,
%     a pattern $p$,
%     a natural number $d$,
%     Boolean values $nb$, $nb'$ and $b$,
%     and a list of natural numbers $k$,
%     if $\VerifyRule{g}{p}{d}{nb}{\Some b}{k}$,
%     then there exists some Boolean value $b'$,
%     such that $\VerifyRule{g}{p}{d}{nb'}{\Some b'}{k}$.
%     \label{lemma:nb-change-not-lr-pattern}
% \end{lemma}

% Alternatively,
% when a pattern is identified as left-recursive,
% changing the nullable accumulator
% will change neither the result nor the stack $k$.
% This is put forth in \Cref{lemma:nb-change-lr-pattern}.

% \begin{lemma}%[Nullable Accumulator Change When Pattern Is Left-Recursive]
%     For a grammar $g$,
%     a pattern $p$,
%     a natural number $d$,
%     Boolean values $nb$ and $nb'$,
%     and a list of natural numbers $k$,
%     if $\VerifyRule{g}{p}{d}{nb}{\None}{k}$,
%     then $\VerifyRule{g}{p}{d}{nb'}{\None}{k}$.
%     \label{lemma:nb-change-lr-pattern}
% \end{lemma}

% The following lemmas will start to deal with the output stack more closely.
% It is used by \lpeg{} to format error messages,
% but we will use it to help us formalize certain properties about the algorithm.
% \Cref{lemma:existential-ff} states that,
% for any evaluation with an output stack $k_1 \dplus i :: k_2$,
% there exists a stack depth limit and a nullable accumulator,
% for which the evaluation of the nonterminal pattern $\PNT{i}$
% yields the output stack $i :: k_2$.
% We call this operation ``fast-forwarding''.

% \begin{lemma}%[Existential Fast-forward]
%     $\forall gas\ \forall p\ \forall d\ \forall nb\ \forall res\ \forall k_2\ \forall k_2\ \forall i$, \\
%     if $\VerifyRule{g}{p}{d}{nb}{res}{k_1 \dplus i :: k_2}$,
%     then $\exists d'\ \exists nb'\ \exists res'$,
%     such that $\VerifyRule{g}{\PNT i}{d'}{nb'}{res'}{i :: k_2}$.
%     \label{lemma:existential-ff}
% \end{lemma}

% \Cref{lemma:existential-ff-for-lr-patterns}
% is a particular case of \Cref{lemma:existential-ff},
% in which the evaluation of pattern $p$ overflows the stack,
% and, as a result, so does the evaluation of nonterminal pattern $\PNT{i}$.

% \begin{lemma}%[Existential Fast-forward On Overflow]
%     $\forall gas\ \forall p\ \forall d\ \forall nb\ \forall k_2\ \forall k_2\ \forall i$, \\
%     if $\VerifyRule{g}{p}{d}{nb}{\None}{k_1 \dplus i :: k_2}$,
%     then $\exists d'\ \exists nb'$,
%     such that $\VerifyRule{g}{\PNT i}{d'}{nb'}{\None}{i :: k_2}$.
%     \label{lemma:existential-ff-for-lr-patterns}
% \end{lemma}

\Cref{lemma:ff-for-lr-patterns} states that,
for any evaluation that results in a stack overflow,
we can pick any rule $i$ from the output stack $k$,
and evaluate it with a certain stack depth limit,
so that it also results in a stack overflow,
and returns a suffix of the original stack $k$,
starting from $i$.

\begin{lemma}%[Fast-forward on Overflow]
    If $\VerifyRule{g}{p}{d}{nb}{\None}{k_1 \dplus i :: k_2}$, \\
    then $\VerifyRule{g}{\PNT i}{1+\length{k_2}}{nb'}{\None}{i :: k_2}$.
    \label{lemma:ff-for-lr-patterns}
\end{lemma}

Under the same assumptions,
\Cref{lemma:d-increase-lr} shows that,
if we evaluate a rule $i$ from the stack
with an increased stack depth limit
and it still results in a stack overflow
and returns a stack $i :: k_3$,
then we can increase the stack depth limit
of the original evaluation by the same amount,
it will also result in a stack overflow,
and return a stack that ends with $i :: k_3$.

\begin{lemma}%[Increase Overflown Stack Depth Limit]
    If $\VerifyRule{g}{p}{d}{nb}{\None}{k_1 \dplus i :: k_2}$, \\
    and $\VerifyRule{g}{\PNT i}{1+\length{k_3}}{nb'}{\None}{i :: k_3}$,
    and $\length{k_2} \le \length{k_3}$, \\
    then $\VerifyRule{g}{p}{1 + \length{k_1} + \length{k_3}}{nb}{\None}{k_1 \dplus i :: k_3}$.
    \label{lemma:d-increase-lr}
\end{lemma}

\Cref{lemma:repeated-rule-in-stack} shows that,
if an evaluation results in a stack overflow,
and a rule $i$ occurs more than once in the output stack,
then we can increase the stack depth limit by a certain amount,
and both conditions will still hold true.

\begin{lemma}%[Repeated Overflow Stack Section]
    If $\VerifyRule{g}{p}{d}{nb}{\None}{k_1 \dplus i :: k_2 \dplus i :: k_3}$, \\
    then $\exists d'$,
    such that $\VerifyRule{g}{p}{d'}{nb}{\None}{k_1 \dplus i :: k_2 \dplus i :: k_2 \dplus i :: k_3}$.
    \label{lemma:repeated-rule-in-stack}
\end{lemma}

Finally, we present the main lemma
that we wanted to prove.
\Cref{lemma:stack-depth-eventual-constancy}
shows that,
if an evaluation with a stack depth limit
greater than the number of grammar rules
yields a result,
then any evaluation with an even greater stack depth limit
yields the same result.
The stacks can be different,
but they are irrelevant
for our purpose of identifying
left-recursive rules.

\begin{lemma}%[Stack Depth Limit Increase]
    If $\VerifyRule{g}{p}{d}{nb}{res}{k}$, and $d > \length{g}$, \\
    then, for any $d' \ge d$,
    $\exists k'$,
    such that $\VerifyRule{g}{p}{d'}{nb}{res}{k'}$.
    \label{lemma:stack-depth-eventual-constancy}
\end{lemma}

We now explain the proof of this lemma.
For the cases in which the evaluation
does not result in a stack overflow,
the proof follows from \Cref{lemma:stack-depth-monotonicity-not-lr-pattern}.
Now, in the case of a stack overflow,
we know from \Cref{lemma:stack-depth-lr-pattern}
that the length of the stack $k$ is equal to the stack depth limit $d$,
which, in this case, we assume to be greater than $n$, the number of grammar rules.
Therefore, $\length{k} > n$.
We know from \Cref{lemma:coherent-stack}
that the stack only contains valid grammar rule indices.
That is, $\forall i \in k, i < n$.
We use these two observations and the pigeonhole principle to conclude
that the stack must have at least one repeated rule.
From \Cref{lemma:repeated-rule-in-stack},
we show that we can increase the stack depth limit arbitrarily,
and it will still result in a stack overflow.

Having defined the algorithm
that checks if a pattern is free of left recursion,
we now use this definition to create a function
that performs this check for a list of patterns.
\Cref{fig:lverifyrule-function} defines this function,
\begin{figure}
    \centering
    \input{lverifyrulecomp}
    \caption{The left recursion detection function for lists of patterns.}
    \label{fig:lverifyrule-function}
\end{figure}
which receives a grammar, a list of patterns,
and a gas counter,
and returns an optional Boolean value
indicating whether all patterns in the grammar
are free of left recursion.

This new function provides values
for two of the parameters of the underlying function:
the stack depth limit $d$, initialized with $\length{g}+1$,
the lower bound from \Cref{lemma:stack-depth-eventual-constancy},
and the nullable accumulator $nb$, initialized with $false$.
We could have omitted the gas counter,
by providing the lower bound from \Cref{lemma:vr-termination},
but we decided to postpone
this omission to the top-most definition
of well-formedness in our formalization.

We provide a lower bound for the gas parameter,
for which the function returns some result.
Note that we're assuming that both the grammar $g$
and the list of rules $rs$ are coherent,
because they could be different.
In practice, however,
they will be the same.
In this case,
where $rs = g$,
the equation for the lower bound can be simplified
to $(\length{g} + 2) \cdot \size{g}$.
That is the origin of the lower bound of the
\textit{\verifygrammarname{}} function
as displayed in \Cref{fig:verifygrammargas}.

\begin{lemma}%[Verify Rules Termination]
    If $\lCoherent{g}{g}{true}$ and $\lCoherent{g}{rs}{true}$, \\
    then, $\forall gas \ge \size{rs} + (\length{g} + 1) \cdot \size{g}$,
    $\exists res\ \lverifyrulecomp{g}{rs}{gas} = \Some res$.
\end{lemma}

\begin{proof}
    The proof follows by induction on the list of rules $rs$,
    and from the gas lower bound for the function \textit{\verifyrulename{}}
    from \Cref{lemma:vr-termination},
    substituting the stack depth limit $d$ with $\length{g}+1$.
\end{proof}

We will use this function for
verifying that the grammar contains
no left-recursive rules,
since it's implemented as a list of rules.
Since we will also be using it in our proofs,
we will need an analogous inductive definition.
\Cref{fig:lverifyrule} defines this predicate,
\begin{figure}
    \centering
    \input{lverifyrule}
    \caption{The left recursion detection predicate for lists of patterns.}
    \label{fig:lverifyrule}
\end{figure}
which also receives a grammar and a list of patterns,
and yields a Boolean value indicating
whether all patterns in the list are free of left recursion.

This predicate differs from the function in one aspect.
While the function provides an exact value for the
stack depth limit, the predicate allows any stack depth limit
to identify a pattern as either nullable or non-nullable.
That is because, according to \Cref{lemma:stack-depth-monotonicity-not-lr-pattern},
the returned label stays constant with increasing stack depth limits in such cases.
In the general case, however,
a stack depth limit greater than
the number of rules in the grammar
is necessary.

We prove some lemmas about this predicate.
\Cref{lemma:lverifyrule-determinism}
states that this predicate is deterministic,
and \Cref{lemma:lverifyrule-follows}
states that it follows the fixed-point definition.

\begin{lemma}
    If $\lVerifyRule{g}{rs}{b_1}$
    and $\lVerifyRule{g}{rs}{b_2}$,
    then $b_1 = b_2$.
    \label{lemma:lverifyrule-determinism}
\end{lemma}

\begin{lemma}
    If $\lverifyrulecomp{g}{rs}{gas} = \Some b$,
    then $\lVerifyRule{g}{rs}{b}$.
    \label{lemma:lverifyrule-follows}
\end{lemma}

\Cref{lemma:lverifyrule-safety} states that,
if a list of patterns passes the check,
then every pattern in the list
passes the individual check,
being either nullable or non-nullable.

\begin{lemma}%[Verify Rules Safety]
    If $\lVerifyRule{g}{rs}{true}$, \\
    then, $\forall r \in rs, \exists d\ \exists b\ \exists k\ \VerifyRule{g}{r}{d}{nb}{\Some b}{k}$.
    \label{lemma:lverifyrule-safety}
\end{lemma}

Before we end this section,
there is one final lemma we would like to present,
which uses all the predicates of the verification algorithm
we have defined up until now.
\Cref{lemma:no-lr-rule-in-grammar} shows that,
if a grammar is free of incoherent and left-recursive rules,
then any coherent pattern is either nullable or non-nullable.

\begin{lemma}%[No Left-Recursive Rule in Grammar]
    If $\Coherent{g}{p}{true}$,
    and $\lCoherent{g}{g}{true}$, \\
    and $\lVerifyRule{g}{g}{true}$,
    then $\exists d\ \exists b\ \exists k$,
    such that $\VerifyRule{g}{p}{d}{nb}{\Some b}{k}$.
    \label{lemma:no-lr-rule-in-grammar}
\end{lemma}

    \caption{The left recursion detection predicate.}
    \label{fig:verifyrule}
\end{figure}
defines such predicate,
denoted as $\VerifyRule{g}{p}{d}{nb}{res}{k}$.
It takes a grammar $g$,
a pattern $p$,
a stack depth limit $d$,
and a nullable accumulator $nb$,
and outputs a result $res$,
and a stack trace $k$.

In order to reach our final goal
of proving that the label returned
by the \textit{\verifyrulename{}} function
is monotonic and eventually constant
with respect to the stack depth limit,
we need to first prove some intermediary lemmas.
First, we need to relate the predicate and
the fixed-point definition together,
so that we can apply the proofs about the
former to the latter.

We begin with
\Cref{lemma:vr-determinism},
which states that,
for identical input,
the predicate yields the same output.
We can therefore state
that the predicate is deterministic.

\begin{lemma}
    If $\VerifyRule{g}{p}{d}{nb}{res_1}{k_1}$,
    and $\VerifyRule{g}{p}{d}{nb}{res_2}{k_2}$, \\
    then $res_1 = res_2$ and $k_1 = k_2$.
    \label{lemma:vr-determinism}
\end{lemma}

\Cref{lemma:vr-follows} shows that
every result returned by the fixed-point definition
can be inductively constructed using the predicate definition.

\begin{lemma}
    If $\verifyrulecomp{g}{p}{d}{nb}{gas} = \Some (res, k)$, \\
    then $\VerifyRule{g}{p}{d}{nb}{res}{k}$.
    \label{lemma:vr-follows}
\end{lemma}

\Cref{lemma:stack-depth-monotonicity-not-lr-pattern} shows that,
if a pattern evaluates to either nullable or non-nullable,
then increasing the stack depth limit
doesn't affect the result.
This is expected,
because, in both cases,
it doesn't surpass the limit,
and increasing it preserves this property
by transitivity.

\begin{lemma}%[Stack Depth Limit Increase without Overflow]
    If $\VerifyRule{g}{p}{d}{nb}{\Some nb'}{k}$, \\
    then $\forall d' \ge d, \VerifyRule{g}{p}{d'}{nb}{\Some nb'}{k}$.
    \label{lemma:stack-depth-monotonicity-not-lr-pattern}
\end{lemma}

\Cref{lemma:stack-depth-lr-pattern} shows that,
on stack overflow,
the output stack $k$ has length $d$.
That is expected,
because a stack overflow happens
when the stack is full
before trying to visit a rule.

\begin{lemma}%[Stack Length on Overflow]
    If $\VerifyRule{g}{p}{d}{nb}{\None}{k}$,
    then $\length{k} = d$.
    \label{lemma:stack-depth-lr-pattern}
\end{lemma}

\Cref{lemma:coherent-stack} states
that the output stack
only contains references to
existing rules in the grammar.
Since we are identifying rules
by their indices in a list,
we prove this by showing that these indices
are less than the number of rules in the grammar,
denoted as $\length{g}$.
This lemma may seem trivial,
but it is necessary for us to later prove,
using the pigeonhole principle,
that a stack with more rules than the grammar
will have at least one repeated rule.

\begin{lemma}%[Coherent Stack]
    If $\VerifyRule{g}{p}{d}{nb}{res}{k}$,
    then $\forall i \in k, i < \length{g}$.
    \label{lemma:coherent-stack}
\end{lemma}

% Next, we prove in \Cref{lemma:true-as-nb} that,
% when provided with $true$ as the value
% for the nullable accumulator $nb$,
% the algorithm outputs as a result
% either $\None$ (which means ``left-recursive'')
% or $\Some true$ (which means ``nullable'').

% \begin{lemma}%[True As Nullable Accumulator]
%     For a grammar $g$,
%     a pattern $p$,
%     a natural number $d$,
%     an optional Boolean value $res$,
%     and a list of natural numbers $k$,
%     if $\VerifyRule{g}{p}{d}{true}{res}{k}$,
%     then $res \in \{\None,\ \Some true\}$.
%     \label{lemma:true-as-nb}
% \end{lemma}

% We also state in \Cref{lemma:nb-change-not-lr-pattern} that,
% when a pattern is identified as either nullable or non-nullable,
% then changing the nullable accumulator
% might change the result to either nullable or non-nullable,
% but the output stack $k$ will remain the same.
% Note that we're saying that it will not make
% the pattern be marked as left-recursive.

% \begin{lemma}%[Nullable Accumulator Change When Pattern Is Not Left-Recursive]
%     For a grammar $g$,
%     a pattern $p$,
%     a natural number $d$,
%     Boolean values $nb$, $nb'$ and $b$,
%     and a list of natural numbers $k$,
%     if $\VerifyRule{g}{p}{d}{nb}{\Some b}{k}$,
%     then there exists some Boolean value $b'$,
%     such that $\VerifyRule{g}{p}{d}{nb'}{\Some b'}{k}$.
%     \label{lemma:nb-change-not-lr-pattern}
% \end{lemma}

% Alternatively,
% when a pattern is identified as left-recursive,
% changing the nullable accumulator
% will change neither the result nor the stack $k$.
% This is put forth in \Cref{lemma:nb-change-lr-pattern}.

% \begin{lemma}%[Nullable Accumulator Change When Pattern Is Left-Recursive]
%     For a grammar $g$,
%     a pattern $p$,
%     a natural number $d$,
%     Boolean values $nb$ and $nb'$,
%     and a list of natural numbers $k$,
%     if $\VerifyRule{g}{p}{d}{nb}{\None}{k}$,
%     then $\VerifyRule{g}{p}{d}{nb'}{\None}{k}$.
%     \label{lemma:nb-change-lr-pattern}
% \end{lemma}

% The following lemmas will start to deal with the output stack more closely.
% It is used by \lpeg{} to format error messages,
% but we will use it to help us formalize certain properties about the algorithm.
% \Cref{lemma:existential-ff} states that,
% for any evaluation with an output stack $k_1 \dplus i :: k_2$,
% there exists a stack depth limit and a nullable accumulator,
% for which the evaluation of the nonterminal pattern $\PNT{i}$
% yields the output stack $i :: k_2$.
% We call this operation ``fast-forwarding''.

% \begin{lemma}%[Existential Fast-forward]
%     $\forall gas\ \forall p\ \forall d\ \forall nb\ \forall res\ \forall k_2\ \forall k_2\ \forall i$, \\
%     if $\VerifyRule{g}{p}{d}{nb}{res}{k_1 \dplus i :: k_2}$,
%     then $\exists d'\ \exists nb'\ \exists res'$,
%     such that $\VerifyRule{g}{\PNT i}{d'}{nb'}{res'}{i :: k_2}$.
%     \label{lemma:existential-ff}
% \end{lemma}

% \Cref{lemma:existential-ff-for-lr-patterns}
% is a particular case of \Cref{lemma:existential-ff},
% in which the evaluation of pattern $p$ overflows the stack,
% and, as a result, so does the evaluation of nonterminal pattern $\PNT{i}$.

% \begin{lemma}%[Existential Fast-forward On Overflow]
%     $\forall gas\ \forall p\ \forall d\ \forall nb\ \forall k_2\ \forall k_2\ \forall i$, \\
%     if $\VerifyRule{g}{p}{d}{nb}{\None}{k_1 \dplus i :: k_2}$,
%     then $\exists d'\ \exists nb'$,
%     such that $\VerifyRule{g}{\PNT i}{d'}{nb'}{\None}{i :: k_2}$.
%     \label{lemma:existential-ff-for-lr-patterns}
% \end{lemma}

\Cref{lemma:ff-for-lr-patterns} states that,
for any evaluation that results in a stack overflow,
we can pick any rule $i$ from the output stack $k$,
and evaluate it with a certain stack depth limit,
so that it also results in a stack overflow,
and returns a suffix of the original stack $k$,
starting from $i$.

\begin{lemma}%[Fast-forward on Overflow]
    If $\VerifyRule{g}{p}{d}{nb}{\None}{k_1 \dplus i :: k_2}$, \\
    then $\VerifyRule{g}{\PNT i}{1+\length{k_2}}{nb'}{\None}{i :: k_2}$.
    \label{lemma:ff-for-lr-patterns}
\end{lemma}

Under the same assumptions,
\Cref{lemma:d-increase-lr} shows that,
if we evaluate a rule $i$ from the stack
with an increased stack depth limit
and it still results in a stack overflow
and returns a stack $i :: k_3$,
then we can increase the stack depth limit
of the original evaluation by the same amount,
it will also result in a stack overflow,
and return a stack that ends with $i :: k_3$.

\begin{lemma}%[Increase Overflown Stack Depth Limit]
    If $\VerifyRule{g}{p}{d}{nb}{\None}{k_1 \dplus i :: k_2}$, \\
    and $\VerifyRule{g}{\PNT i}{1+\length{k_3}}{nb'}{\None}{i :: k_3}$,
    and $\length{k_2} \le \length{k_3}$, \\
    then $\VerifyRule{g}{p}{1 + \length{k_1} + \length{k_3}}{nb}{\None}{k_1 \dplus i :: k_3}$.
    \label{lemma:d-increase-lr}
\end{lemma}

\Cref{lemma:repeated-rule-in-stack} shows that,
if an evaluation results in a stack overflow,
and a rule $i$ occurs more than once in the output stack,
then we can increase the stack depth limit by a certain amount,
and both conditions will still hold true.

\begin{lemma}%[Repeated Overflow Stack Section]
    If $\VerifyRule{g}{p}{d}{nb}{\None}{k_1 \dplus i :: k_2 \dplus i :: k_3}$, \\
    then $\exists d'$,
    such that $\VerifyRule{g}{p}{d'}{nb}{\None}{k_1 \dplus i :: k_2 \dplus i :: k_2 \dplus i :: k_3}$.
    \label{lemma:repeated-rule-in-stack}
\end{lemma}

Finally, we present the main lemma
that we wanted to prove.
\Cref{lemma:stack-depth-eventual-constancy}
shows that,
if an evaluation with a stack depth limit
greater than the number of grammar rules
yields a result,
then any evaluation with an even greater stack depth limit
yields the same result.
The stacks can be different,
but they are irrelevant
for our purpose of identifying
left-recursive rules.

\begin{lemma}%[Stack Depth Limit Increase]
    If $\VerifyRule{g}{p}{d}{nb}{res}{k}$, and $d > \length{g}$, \\
    then, for any $d' \ge d$,
    $\exists k'$,
    such that $\VerifyRule{g}{p}{d'}{nb}{res}{k'}$.
    \label{lemma:stack-depth-eventual-constancy}
\end{lemma}

We now explain the proof of this lemma.
For the cases in which the evaluation
does not result in a stack overflow,
the proof follows from \Cref{lemma:stack-depth-monotonicity-not-lr-pattern}.
Now, in the case of a stack overflow,
we know from \Cref{lemma:stack-depth-lr-pattern}
that the length of the stack $k$ is equal to the stack depth limit $d$,
which, in this case, we assume to be greater than $n$, the number of grammar rules.
Therefore, $\length{k} > n$.
We know from \Cref{lemma:coherent-stack}
that the stack only contains valid grammar rule indices.
That is, $\forall i \in k, i < n$.
We use these two observations and the pigeonhole principle to conclude
that the stack must have at least one repeated rule.
From \Cref{lemma:repeated-rule-in-stack},
we show that we can increase the stack depth limit arbitrarily,
and it will still result in a stack overflow.

Having defined the algorithm
that checks if a pattern is free of left recursion,
we now use this definition to create a function
that performs this check for a list of patterns.
\Cref{fig:lverifyrule-function} defines this function,
\begin{figure}
    \centering
    \begin{equation*}
    \begin{aligned}[t]
        & \lverifyrulecomp{g}{rs}{gas} := \\
        & \begin{aligned}[t]
            & \matchwith{rs} \\
            & \matchcase{nil}{\Some true} \\
            & \matchcase{r::rs'}{\begin{aligned}[t]
                & \letin{d}{\length{g}+1} \\
                & \matchwith{\verifyrulecomp{g}{r}{d}{false}{gas}} \\
                & \matchcase{\Some (\Some b, k)}{\lverifyrulecomp{g}{rs'}{gas}} \\
                & \matchcase{\Some (\None, k)}{\Some false} \\
                & \matchcase{\None}{\None} \\
                & \matchend{}
            \end{aligned}} \\
            & \matchend{}
        \end{aligned}
    \end{aligned}
\end{equation*}


% \begin{equation}
%     \lverifyrulecomp(g,[\ ],gas) = \Some(true)
% \end{equation}

% \begin{equation}
%     \lverifyrulecomp(g,r::rs,gas) = \begin{cases}
%         \text{where } ores := \verifyrulecomp(g,r,1+|g|,false,gas) \\
%         \lverifyrulecomp(g,rs,gas), \text{if } ores = \Some(\Some(b),k) \\
%         \Some(false), \text{if } ores = \Some(\None, k) \\
%         \None, \text{if } ores = \None
%     \end{cases}
% \end{equation}

    \caption{The left recursion detection function for lists of patterns.}
    \label{fig:lverifyrule-function}
\end{figure}
which receives a grammar, a list of patterns,
and a gas counter,
and returns an optional Boolean value
indicating whether all patterns in the grammar
are free of left recursion.

This new function provides values
for two of the parameters of the underlying function:
the stack depth limit $d$, initialized with $\length{g}+1$,
the lower bound from \Cref{lemma:stack-depth-eventual-constancy},
and the nullable accumulator $nb$, initialized with $false$.
We could have omitted the gas counter,
by providing the lower bound from \Cref{lemma:vr-termination},
but we decided to postpone
this omission to the top-most definition
of well-formedness in our formalization.

We provide a lower bound for the gas parameter,
for which the function returns some result.
Note that we're assuming that both the grammar $g$
and the list of rules $rs$ are coherent,
because they could be different.
In practice, however,
they will be the same.
In this case,
where $rs = g$,
the equation for the lower bound can be simplified
to $(\length{g} + 2) \cdot \size{g}$.
That is the origin of the lower bound of the
\textit{\verifygrammarname{}} function
as displayed in \Cref{fig:verifygrammargas}.

\begin{lemma}%[Verify Rules Termination]
    If $\lCoherent{g}{g}{true}$ and $\lCoherent{g}{rs}{true}$, \\
    then, $\forall gas \ge \size{rs} + (\length{g} + 1) \cdot \size{g}$,
    $\exists res\ \lverifyrulecomp{g}{rs}{gas} = \Some res$.
\end{lemma}

\begin{proof}
    The proof follows by induction on the list of rules $rs$,
    and from the gas lower bound for the function \textit{\verifyrulename{}}
    from \Cref{lemma:vr-termination},
    substituting the stack depth limit $d$ with $\length{g}+1$.
\end{proof}

We will use this function for
verifying that the grammar contains
no left-recursive rules,
since it's implemented as a list of rules.
Since we will also be using it in our proofs,
we will need an analogous inductive definition.
\Cref{fig:lverifyrule} defines this predicate,
\begin{figure}
    \centering
    \begin{mathpar}
    \namedinferrule{lvr-nil}
    { }
    {\lVerifyRule{g}{nil}{true}}

    \namedinferrule{lvr-cons-some}
    {\VerifyRule{g}{r}{d}{false}{\Some nb}{k} \\ \lVerifyRule{g}{rs}{b}}
    {\lVerifyRule{g}{r::rs}{b}}

    \namedinferrule{lvr-cons-none}
    {|g| < d \\ \VerifyRule{g}{r}{d}{false}{\None}{k}}
    {\lVerifyRule{g}{r::rs}{false}}
\end{mathpar}
    \caption{The left recursion detection predicate for lists of patterns.}
    \label{fig:lverifyrule}
\end{figure}
which also receives a grammar and a list of patterns,
and yields a Boolean value indicating
whether all patterns in the list are free of left recursion.

This predicate differs from the function in one aspect.
While the function provides an exact value for the
stack depth limit, the predicate allows any stack depth limit
to identify a pattern as either nullable or non-nullable.
That is because, according to \Cref{lemma:stack-depth-monotonicity-not-lr-pattern},
the returned label stays constant with increasing stack depth limits in such cases.
In the general case, however,
a stack depth limit greater than
the number of rules in the grammar
is necessary.

We prove some lemmas about this predicate.
\Cref{lemma:lverifyrule-determinism}
states that this predicate is deterministic,
and \Cref{lemma:lverifyrule-follows}
states that it follows the fixed-point definition.

\begin{lemma}
    If $\lVerifyRule{g}{rs}{b_1}$
    and $\lVerifyRule{g}{rs}{b_2}$,
    then $b_1 = b_2$.
    \label{lemma:lverifyrule-determinism}
\end{lemma}

\begin{lemma}
    If $\lverifyrulecomp{g}{rs}{gas} = \Some b$,
    then $\lVerifyRule{g}{rs}{b}$.
    \label{lemma:lverifyrule-follows}
\end{lemma}

\Cref{lemma:lverifyrule-safety} states that,
if a list of patterns passes the check,
then every pattern in the list
passes the individual check,
being either nullable or non-nullable.

\begin{lemma}%[Verify Rules Safety]
    If $\lVerifyRule{g}{rs}{true}$, \\
    then, $\forall r \in rs, \exists d\ \exists b\ \exists k\ \VerifyRule{g}{r}{d}{nb}{\Some b}{k}$.
    \label{lemma:lverifyrule-safety}
\end{lemma}

Before we end this section,
there is one final lemma we would like to present,
which uses all the predicates of the verification algorithm
we have defined up until now.
\Cref{lemma:no-lr-rule-in-grammar} shows that,
if a grammar is free of incoherent and left-recursive rules,
then any coherent pattern is either nullable or non-nullable.

\begin{lemma}%[No Left-Recursive Rule in Grammar]
    If $\Coherent{g}{p}{true}$,
    and $\lCoherent{g}{g}{true}$, \\
    and $\lVerifyRule{g}{g}{true}$,
    then $\exists d\ \exists b\ \exists k$,
    such that $\VerifyRule{g}{p}{d}{nb}{\Some b}{k}$.
    \label{lemma:no-lr-rule-in-grammar}
\end{lemma}

\section{A simpler algorithm for detecting nullable patterns}
\label{section:nullable}

Once we have made sure a grammar is free of left-recursive rules,
the next step is to check for degenerate loops,
which involves checking if certain patterns are nullable.
To this end,
we could use the algorithm we have just described in \Cref{section:lr-rules},
but \lpeg{} implements a simpler version,
which takes advantage of the fact that
the grammar contains no left-recursive rules.

The difference between the algorithm from \Cref{section:lr-rules}
and this simpler version can be seen,
for example, in the case of choice patterns $\PChoice{p_1}{p_2}$.
More specifically, if $p_1$ is nullable,
the function from \Cref{section:lr-rules}
would still need to evaluate $p_2$,
as it could potentially lead to left recursion.
Meanwhile, if we assume that $p_2$ cannot lead to left recursion,
then if $p_1$ is nullable,
we can state that $\PChoice{p_1}{p_2}$ is nullable,
without having to visit $p_2$.
We can also avoid visiting sub-patterns in the cases of
repetition patterns and predicate patterns,
because they are all nullable.

In \Cref{fig:nullable-function},
\begin{figure}[ht!]
    \centering
    \begin{align*}
    \begin{aligned}[t]
        & \nullablecomp{g}{p}{d}{0} = \None \\
        & \nullablecomp{g}{p}{d}{(1+gas)} = \\
        & \begin{aligned}[t]
            & \matchwith{p} \\
            & \matchcase{\PEmpty}{\Some \Some true} \\
            & \matchcase{\PSet{cs}}{\Some \Some false} \\
            & \matchcase{\PRepetition{p}}{\Some \Some true} \\
            & \matchcase{\PNot{p}}{\Some \Some true} \\
            & \matchcase{\PAnd{p}}{\Some \Some true} \\
            & \matchcase{\PNT{i}}{\begin{aligned}[t]
                & \matchwith{g[i]} \\
                & \matchcase{\Some p}{\begin{aligned}[t]
                    & \matchwith{d} \\
                    & \matchcase{0}{\Some \None} \\
                    & \matchcase{1+d'}{\nullablecomp{g}{p}{d'}{gas}} \\
                    & \matchend{}
                \end{aligned}} \\
                & \matchcase{\None}{\None} \\
                & \matchend{}
            \end{aligned}} \\
            & \matchcase{\PSequence{p_1}{p_2}}{\begin{aligned}[t]
                & \matchwith{\nullablecomp{g}{p_1}{d}{gas}} \\
                & \matchcase{\Some \Some true}{\nullablecomp{g}{p_2}{d}{gas}} \\
                & \matchcase{res}{res} \\
                & \matchend{}
            \end{aligned}} \\
            & \matchcase{\PChoice{p_1}{p_2}}{\begin{aligned}[t]
                & \matchwith{\nullablecomp{g}{p_1}{d}{gas}} \\
                & \matchcase{\Some \Some false}{\nullablecomp{g}{p_2}{d}{gas}} \\
                & \matchcase{res}{res} \\
                & \matchend{}
            \end{aligned}} \\
            & \matchend{}
        \end{aligned}
    \end{aligned}
\end{align*}
    \caption{The nullable function.}
    \label{fig:nullable-function}
\end{figure}
we define this simpler version as a fixed-point,
which takes a grammar,
a pattern, a stack depth limit, and a gas counter,
and returns an optional label.
Possible return values are, therefore,
$\None$ (out-of-gas), $\Some \None$ (stack overflow),
$\Some \Some true$ (nullable), and $\Some \Some false$ (non-nullable).

The gas parameter is still necessary to convince Coq that
the function terminates,
and because it could be called
with an incoherent pattern or
with a grammar that contains an incoherent rule.
The stack depth limit is also necessary
because the function could be called
with a grammar that contains a left-recursive rule.
For these two reasons,
this function
may still return $\None$ (out-of-gas)
or $\Some \None$ (stack overflow).

In reality, however,
this function should only be called after
the grammar and pattern are guaranteed to be
coherent and free of left-recursive rules.
For this reason,
implementations of this algorithm
such as the one in \lpeg{}
are able to safely drop both parameters,
and return just a Boolean value
indicating whether the pattern is nullable or not.

Also note that, unlike the function from \Cref{fig:vr-function},
this function does not receive a nullable accumulator as parameter
and does not return a stack.
The nullable accumulator is not necessary,
thanks to the assumption that
the grammar contains no left-recursive rules,
and the output stack was only used for proofs,
which we will be able to reuse from \Cref{section:lr-rules}.

Just as we did for the function from \Cref{section:lr-rules},
we prove that this simpler version is also
monotonic
with respect to the gas counter.

\begin{lemma}%[Nullable Gas Monotonicity]
    If $\nullablecomp{g}{p}{d}{gas} = \Some res$, \\
    then, $\forall gas' \ge gas$,
    $\nullablecomp{g}{p}{d}{gas'} = \Some res$.
    \label{lemma:nullable-gas-monotonicity}
\end{lemma}

Besides that, we also prove the function
is eventually constant with respect to the gas counter
by giving a gas lower bound for which the function
returns some result for any coherent pattern and grammar.

\begin{lemma}%[Nullable Termination]
    If $\Coherent{g}{p}{true}$,
    and $\lCoherent{g}{g}{true}$, \\
    then, $\forall gas \ge \size{p} + d \cdot \size{g}$,
    $\exists res\ \nullablecomp{g}{p}{d}{gas} = \Some res$.
    \label{lemma:nullable-termination}
\end{lemma}

Furthermore, we would also like to prove that
this function is eventually constant and monotonic
with respect to the stack depth limit.
However, in order to do that,
we found it better to first define an equivalent inductive predicate.
\Cref{fig:nullable-predicate} defines the predicate
\begin{figure}
    \centering
    \begin{mathpar}
    \namedinferrule{n-eps}
    { }
    {\Nullable{g}{\PEmpty}{d}{\Some true}}

    \namedinferrule{n-set}
    { }
    {\Nullable{g}{\PSet{cs}}{d}{\Some false}}

    \namedinferrule{n-rep}
    { }
    {\Nullable{g}{\PRepetition{p}}{d}{\Some true}}

    \namedinferrule{n-not}
    { }
    {\Nullable{g}{\PNot{p}}{d}{\Some true}}

    \namedinferrule{n-and}
    { }
    {\Nullable{g}{\PAnd{p}}{d}{\Some true}}

    \namedinferrule{n-nonterminal-zero}
    {g[i] = \Some p}
    {\Nullable{g}{\PNT{i}}{0}{\None}}
    
    \namedinferrule{n-nonterminal-succ}
    {g[i] = \Some p \\ \Nullable{g}{p}{d}{res}}
    {\Nullable{g}{\PNT{i}}{1+d}{res}}

    \namedinferrule{n-seq-none}
    {\Nullable{g}{p_1}{d}{\None}}
    {\Nullable{g}{\PSequence{p_1}{p_2}}{d}{\None}}
    
    \namedinferrule{n-seq-some-false}
    {\Nullable{g}{p_1}{d}{\Some false}}
    {\Nullable{g}{\PSequence{p_1}{p_2}}{d}{\Some false}}
    
    \namedinferrule{n-seq-some-true}
    {\Nullable{g}{p_1}{d}{\Some true} \\ \Nullable{g}{p_2}{d}{res}}
    {\Nullable{g}{\PSequence{p_1}{p_2}}{d}{res}}

    \namedinferrule{n-choice-none}
    {\Nullable{g}{p_1}{d}{\None}}
    {\Nullable{g}{\PChoice{p_1}{p_2}}{d}{\None}}
    
    \namedinferrule{n-choice-some-false}
    {\Nullable{g}{p_1}{d}{\Some false} \\ \Nullable{g}{p_2}{d}{res}}
    {\Nullable{g}{\PChoice{p_1}{p_2}}{d}{res}}
    
    \namedinferrule{n-choice-some-true}
    {\Nullable{g}{p_1}{d}{\Some true}}
    {\Nullable{g}{\PChoice{p_1}{p_2}}{d}{\Some true}}
\end{mathpar}
    \caption{The nullable predicate.}
    \label{fig:nullable-predicate}
\end{figure}
$\Nullable{g}{p}{d}{res}$ which takes a grammar $g$,
a pattern $p$, a stack depth limit $d$, and returns a label $res$.

About this predicate, we proved some basic lemmas.
First, we proved that it is deterministic.

\begin{lemma}
    If $\Nullable{g}{p}{d}{res_1}$,
    and $\Nullable{g}{p}{d}{res_2}$,
    then $res_1 = res_2$.
\end{lemma}

We also proved that it follows
the fixed-point definition.

\begin{lemma}
    If $\nullablecomp{g}{p}{d}{gas} = \Some res$,
    then $\Nullable{g}{p}{d}{res}$.
\end{lemma}

We also tied this predicate to the
one from \Cref{section:lr-rules},
showing how similar they are,
when the nullable accumulator is $false$,
and the pattern is either nullable or non-nullable.

\begin{lemma}%[Verify Rule Similar to Nullable]
    If $\VerifyRule{g}{p}{d}{false}{\Some b}{k}$,
    then $\Nullable{g}{p}{d}{\Some b}$.
\end{lemma}

We also proved that if a pattern
was identified as either nullable or non-nullable,
then increasing the stack depth does not impact the result.

\begin{lemma}%[Stack Depth Limit Increase Without Overflow]
    If $\Nullable{g}{p}{d}{\Some b}$,
    then $\forall d' \ge d, \Nullable{g}{p}{d'}{\Some b}$.
\end{lemma}

Maybe the most important lemma
about the nullable predicate
relates to the match predicate.
It states that a non-nullable pattern
never matches without consuming
some part of the input string.

\begin{lemma}%[Non-nullable pattern and Match]
    If $\Nullable{g}{p}{d}{\Some false}$,
    then $\nexists s$ such that $\Matches{g}{p}{s}{s}$.
\end{lemma}

This lemma is used to prove that,
when matching a non-nullable pattern,
the output string is a proper suffix of the input string.
We use the symbol ``$\ProperSuffix{}{}$''
to denote this relation.

\begin{lemma}%[Non-nullable pattern and Proper Suffix]
    \label{lemma:non-nullable-pattern-proper-suffix}
    If $\Nullable{g}{p}{d}{\Some false}$,
    and $\Matches{g}{p}{s}{s'}$,
    then $\ProperSuffix{s'}{s}$.
\end{lemma}

Finally, we show that the predicate is
eventually constant
with respect to the stack depth limit,
past a lower bound given by the number of rules in the grammar,
denoted as $\length{g}$.

\begin{lemma}%[Nullable Stack Depth Limit Eventual Constancy]
    If $\Coherent{g}{p}{true}$,
    and $\lCoherent{g}{g}{true}$,
    and $\lVerifyRule{g}{g}{true}$, \\
    and $\Nullable{g}{p}{d}{res}$,
    where $d > \length{g}$,
    then, $\forall d' \ge d$,
    $\Nullable{g}{p}{d'}{res}$.
\end{lemma}

\section{Degenerate loops}

After making sure that
all rules are coherent,
and that the grammar is free of left-recursive rules,
the next step is to look for degenerate loops,
which are repetition patterns $\PRepetition{p}$
where $p$ is nullable.
To detect nullable patterns,
we use the algorithm from
\Cref{section:nullable}.

\Cref{fig:checkloops-function}
defines this step of the verification process
as a fixed-point,
which takes a grammar,
a pattern,
a stack depth limit,
and a gas counter,
and returns an optional label.
Possible return values are
$\None$ (out-of-gas),
$\Some \None$ (stack overflow),
$\Some \Some true$ (degenerate),
and $\Some \Some false$ (non-degenerate).

This fixed-point does not visit rules
referenced by nonterminal patterns.
Instead, each rule is checked separately.
In this case, the stack depth limit parameter
is simply passed down on to the function that
checks whether a pattern is nullable or not.
However, as we've discussed in \Cref{section:nullable},
actual implementations can safely drop this parameter,
given that the grammar has been checked
for left-recursive rules already.

\begin{figure}
    \centering
    \begin{align*}
    \begin{aligned}[t]
        & \checkloopscomp{g}{p}{d}{0} = \None \\
        & \checkloopscomp{g}{p}{d}{(1+gas)} = \\
        & \begin{aligned}[t]
            & \matchwith{p} \\
            & \matchcase{\PEmpty}{\Some \Some false} \\
            & \matchcase{\PSet{cs}}{\Some \Some false} \\
            & \matchcase{\PNT{i}}{\Some \Some false} \\
            & \matchcase{\PNot{p}}{\checkloopscomp{g}{p}{d}{gas}} \\
            & \matchcase{\PAnd{p}}{\checkloopscomp{g}{p}{d}{gas}} \\
            & \matchcase{\PRepetition{p}}{\begin{aligned}[t]
                & \matchwith{\nullablecomp{g}{p}{d}{gas}} \\
                & \matchcase{\Some \Some false}{\checkloopscomp{g}{p}{d}{gas}} \\
                & \matchcase{res}{res} \\
                & \matchend{}
            \end{aligned}} \\
            & \matchcase{\PSequence{p_1}{p_2}}{\begin{aligned}[t]
                & \matchwith{\checkloopscomp{g}{p_1}{d}{gas}} \\
                & \matchcase{\Some \Some false}{\checkloopscomp{g}{p_2}{d}{gas}} \\
                & \matchcase{res}{res} \\
                & \matchend{}
            \end{aligned}} \\
            & \matchcase{\PChoice{p_1}{p_2}}{\begin{aligned}[t]
                & \matchwith{\checkloopscomp{g}{p_1}{d}{gas}} \\
                & \matchcase{\Some \Some false}{\checkloopscomp{g}{p_2}{d}{gas}} \\
                & \matchcase{res}{res} \\
                & \matchend{}
            \end{aligned}} \\
            & \matchend{}
        \end{aligned}
    \end{aligned}
\end{align*}
    \caption{The degenerate loop detection function.}
    \label{fig:checkloops-function}
\end{figure}

Just as with the other gas-based functions,
we would like to prove that this function
converges with respect to the gas counter.
\Cref{lemma:checkloops-gas-convergence}
states that, if this function returns some label,
then increasing the gas counter
won't change the label being returned.

\begin{lemma}%[Check Loops Gas Convergence]
    If $\checkloopscomp{g}{p}{d}{gas} = \Some res$, \\
    then, $\forall gas' \ge gas$,
    $\checkloopscomp{g}{p}{d}{gas'} = \Some res$.
    \label{lemma:checkloops-gas-convergence}
\end{lemma}

Moreover, \Cref{lemma:checkloops-termination}
states that, for any coherent pattern and grammar,
there exists a lower bound for the gas parameter,
for which the function returns some result.

\begin{lemma}%[Check Loops Termination]
    If $\Coherent{g}{p}{true}$,
    and $\lCoherent{g}{g}{true}$, \\
    and $gas \ge \size{p} + d \cdot \size{g}$,
    then $\exists res\ \checkloopscomp{g}{p}{d}{gas} = \Some res$.
    \label{lemma:checkloops-termination}
\end{lemma}

We would also like to prove that
this function converges with
respect to the stack depth limit.
However, in order to do that,
it is better to work with
an inductively-defined predicate.
\Cref{fig:checkloops} shows
the predicate we have defined.
It takes a grammar,
a pattern,
and a stack depth limit,
and returns an optional Boolean value.

\begin{figure}
    \centering
    \begin{mathpar}
    \namedinferrule{cl-eps}
    { }
    {\CheckLoops{g}{\PEmpty}{d}{\Some false}}

    \namedinferrule{cl-set}
    { }
    {\CheckLoops{g}{\PSet{cs}}{d}{\Some false}}

    \namedinferrule{cl-nonterminal}
    { }
    {\CheckLoops{g}{\PNT{i}}{d}{\Some false}}

    \namedinferrule{cl-not}
    {\CheckLoops{g}{p}{d}{res}}
    {\CheckLoops{g}{\PNot{p}}{d}{res}}

    \namedinferrule{cl-and}
    {\CheckLoops{g}{p}{d}{res}}
    {\CheckLoops{g}{\PAnd{p}}{d}{res}}

    \namedinferrule{cl-rep-lr}
    {\Nullable{g}{p}{d}{\None}}
    {\CheckLoops{g}{\PRepetition{p}}{d}{\None}}

    \namedinferrule{cl-rep-nullable}
    {\Nullable{g}{p}{d}{\Some true}}
    {\CheckLoops{g}{\PRepetition{p}}{d}{\Some true}}

    \namedinferrule{cl-rep-non-nullable}
    {\Nullable{g}{p}{d}{\Some false} \\ \CheckLoops{g}{p}{d}{res}}
    {\CheckLoops{g}{\PRepetition{p}}{d}{res}}

    \namedinferrule{cl-seq-none-1}
    {\CheckLoops{g}{p_1}{d}{\None}}
    {\CheckLoops{g}{\PSequence{p_1}{p_2}}{d}{\None}}

    \namedinferrule{cl-seq-none-2}
    {\CheckLoops{g}{p_2}{d}{\None}}
    {\CheckLoops{g}{\PSequence{p_1}{p_2}}{d}{\None}}

    \namedinferrule{cl-seq-some}
    {\CheckLoops{g}{p_1}{d}{\Some b_1} \\ \CheckLoops{g}{p_2}{d}{\Some b_2}}
    {\CheckLoops{g}{\PSequence{p_1}{p_2}}{d}{\Some (b_1 \vee b_2)}}

    \namedinferrule{cl-choice-none-1}
    {\CheckLoops{g}{p_1}{d}{\None}}
    {\CheckLoops{g}{\PChoice{p_1}{p_2}}{d}{\None}}

    \namedinferrule{cl-choice-none-2}
    {\CheckLoops{g}{p_2}{d}{\None}}
    {\CheckLoops{g}{\PChoice{p_1}{p_2}}{d}{\None}}

    \namedinferrule{cl-choice-some}
    {\CheckLoops{g}{p_1}{d}{\Some b_1} \\ \CheckLoops{g}{p_2}{d}{\Some b_2}}
    {\CheckLoops{g}{\PChoice{p_1}{p_2}}{d}{\Some (b_1 \vee b_2)}}
\end{mathpar}
    \caption{The degenerate loop detection predicate.}
    \label{fig:checkloops}
\end{figure}

As usual, we first prove
some basic lemmas about the predicate.
\Cref{lemma:checkloops-determinism}
states that the predicate
is deterministic,
meaning that, for the same input,
it yields the same output.

\begin{lemma}
    If $\CheckLoops{g}{p}{d}{res_1}$,
    and $\CheckLoops{g}{p}{d}{res_2}$,
    then $res_1 = res_2$.
    \label{lemma:checkloops-determinism}
\end{lemma}

\Cref{lemma:checkloops-follows}
states that every result returned by the function
can be constructed using the predicate.
We therefore say the predicate follows
the function.

\begin{lemma}
    If $\checkloopscomp{g}{p}{d}{gas} = \Some res$,
    then $\CheckLoops{g}{p}{d}{res}$.
    \label{lemma:checkloops-follows}
\end{lemma}

\Cref{lemma:checkloops-d-increase-no-overflow}
states that, if the predicate yields some result,
then increasing the stack depth limit
will not alter the result.

\begin{lemma}%[Stack Depth Limit Increase Without Overflow]
    If $\CheckLoops{g}{p}{d}{\Some res}$,
    then, $\forall d' \ge d$,
    $\CheckLoops{g}{p}{d'}{\Some res}$.
    \label{lemma:checkloops-d-increase-no-overflow}
\end{lemma}

Finally, \Cref{lemma:checkloops-d-convergence}
states that, for any coherent pattern and grammar
without left-recursive rules,
the label returned by the predicate converges
when the stack depth limit is greater than
the number of rules in the grammar.

\begin{lemma}%[Check Loops Stack Depth Limit Convergence]
    If $\Coherent{g}{p}{true}$,
    and $\lCoherent{g}{g}{true}$,
    and $\lVerifyRule{g}{g}{true}$, \\
    and $\CheckLoops{g}{p}{d}{res}$,
    where $d > \length{g}$,
    then, $\forall d' \ge d, \CheckLoops{g}{p}{d'}{res}$.
    \label{lemma:checkloops-d-convergence}
\end{lemma}

Having defined the algorithm
that checks if a pattern contains any degenerate loops,
we now define a function
that performs this check for a list of patterns.
Naturally, we will be using this
function to check all the rules of a grammar.
\Cref{fig:lcheckloops-function}
displays this function,
which takes a grammar, a list of patterns, and a gas counter,
and returns an optional Boolean value,
indicating whether it has found any degenerate loop.
We pass $\length{g}+1$ as
the stack depth limit to the underlying function.

\begin{figure}
    \centering
    \begin{equation*}
    \lcheckloopscomp{g}{rs}{gas} = \begin{aligned}[t]
        & \matchwith{rs} \\
        & \matchcase{nil}{\Some false} \\
        & \matchcase{r::rs'}{\begin{aligned}[t]
            & \letin{d}{\length{g}+1} \\
            & \matchwith{\checkloopscomp{g}{r}{d}{gas}} \\
            & \matchcase{\Some \Some false}{\lcheckloopscomp{g}{rs'}{gas}} \\
            & \matchcase{\Some \Some true}{\Some true} \\
            & \matchcase{res}{\None} \\
            & \matchend{}
        \end{aligned}} \\
        & \matchend{}
    \end{aligned}
\end{equation*}
    \caption{The degenerate loop detection function for lists of patterns.}
    \label{fig:lcheckloops-function}
\end{figure}

We prove that there is a lower bound for the gas counter
for which this function returns some result,
assuming the grammar
contains no incoherent or left-recursive rules,
and that the list of patterns only
contains coherent patterns.
In reality, we will be calling this
function while passing $g$ as the $rs$ parameter,
so, in our case,
it would suffice to state that
$g$ contains no incoherent or left-recursive rules.

\begin{lemma}%[Coherent Loops in List Termination]
    If $\lCoherent{g}{g}{true}$,
    and $\lCoherent{g}{rs}{true}$,
    and $\lVerifyRule{g}{g}{true}$, \\
    then, $\forall gas \ge \size{rs} + (\length{g} + 1) \cdot \size{g}$,
    $\exists b\ \lcheckloopscomp{g}{rs}{gas} = \Some b$.
\end{lemma}

In order to abstract away the gas counter,
and to help us in later induction proofs,
we also define an equivalent inductive predicate
for this list-based degenerate loop checker.
\Cref{fig:lcheckloops} displays this predicate,
which takes a grammar and a list of patterns,
and returns a Boolean value, indicating
whether none of the patterns in the list
contain a degenerate loop.

\begin{figure}
    \centering
    \begin{mathpar}
    \namedinferrule{lcl-nil}
    { }
    {\lCheckLoops{g}{nil}{false}}

    \namedinferrule{lcl-cons}
    {\CheckLoops{g}{r}{d}{\Some b_1} \\ \lCheckLoops{g}{rs}{b_2}}
    {\lCheckLoops{g}{r::rs}{b_1 \vee b_2}}
\end{mathpar}
    \caption{The degenerate loop detection predicate for lists of patterns.}
    \label{fig:lcheckloops}
\end{figure}

As usual, we prove some basic lemmas
about this predicate.
\Cref{lemma:lcheckloops-determinism}
states that it is
deterministic,
and \Cref{lemma:lcheckloops-follows}
states that it follows
the fixed-point definition.

\begin{lemma}
    \label{lemma:lcheckloops-determinism}
    If $\lCheckLoops{g}{rs}{b_1}$,
    and $\lCheckLoops{g}{rs}{b_2}$,
    then $b_1=b_2$.
\end{lemma}

\begin{lemma}
    \label{lemma:lcheckloops-follows}
    If $\lcheckloopscomp{g}{rs}{gas} = \Some b$,
    then $\lCheckLoops{g}{rs}{b}$.
\end{lemma}

We also prove that if a list of patterns
passes this list-based check,
then each pattern in this list also
passes the individual check.

\begin{lemma}%[Check Loops in List Safety]
    \label{lemma:lcheckloops-safety}
    If $\lCheckLoops{g}{rs}{false}$,
    then $\forall r \in rs, \exists d\ \CheckLoops{g}{r}{d}{\Some false}$.
\end{lemma}

\section{Correctness}

Having introduced each step of the well-formedness algorithm,
we can now present the definition of the \textit{\verifygrammarname{}} function,
which implements the algorithm step-by-step.
It starts by checking whether the grammar defines a first rule,
and whether every rule in the grammar is coherent.
It then makes sure the grammar contains no left-recurive rules,
and no degenerate loops, in this order.
\Cref{fig:verifygrammar-function} displays the function.
\begin{figure}
    \centering
    \begin{equation*}
    \begin{aligned}[t]
        & \verifygrammarcomp{g}{gas} = \\
        & \begin{aligned}[t]
            & \matchwith{\coherentfunc{g}{\PNT{0}}} \\
            & \matchcase{true}{\begin{aligned}[t]
                & \matchwith{\lcoherentfunc{g}{g}} \\
                & \matchcase{true}{\begin{aligned}[t]
                    & \matchwith{\lverifyrulecomp{g}{g}{gas}} \\
                    & \matchcase{\Some true}{\begin{aligned}[t]
                        & \matchwith{\lcheckloopscomp{g}{g}{gas}} \\
                        & \matchcase{\Some b}{\Some \neg b} \\
                        & \matchcase{\None}{\None} \\
                        & \matchend{}
                    \end{aligned}} \\
                    & \matchcase{res}{res} \\
                    & \matchend{}
                \end{aligned}} \\
                & \matchcase{false}{\Some false} \\
                & \matchend{}
            \end{aligned}} \\
            & \matchcase{false}{\Some false} \\
            & \matchend{}
        \end{aligned}
    \end{aligned}
\end{equation*}
    \caption{The well-formedness function with gas.}
    \label{fig:verifygrammar-function}
\end{figure}

Now, let us prove that the well-formedness check is correct.
To do so, we first define an
inductive predicate equivalent to
the \textit{\verifygrammarname{}} function
to help us in proofs by induction.
\Cref{fig:verifygrammar}
presents this predicate,
\begin{figure}
    \centering
    \begin{mathpar}
    \namedinferrule{vg1}
    {\Coherent{g}{\PNT{0}}{false}}
    {\VerifyGrammar{g}{false}}
    
    \namedinferrule{vg2}
    {\Coherent{g}{\PNT{0}}{true} \\ \lCoherent{g}{g}{false}}
    {\VerifyGrammar{g}{false}}

    \namedinferrule{vg3}
    {\Coherent{g}{\PNT{0}}{true} \\ \lCoherent{g}{g}{true} \\ \lVerifyRule{g}{g}{false}}
    {\VerifyGrammar{g}{false}}

    \namedinferrule{vg4}
    {\Coherent{g}{\PNT{0}}{true} \\ \lCoherent{g}{g}{true} \\ \lVerifyRule{g}{g}{true} \\ \lCheckLoops{g}{g}{true}}
    {\VerifyGrammar{g}{false}}

    \namedinferrule{vg5}
    {\Coherent{g}{\PNT{0}}{true} \\ \lCoherent{g}{g}{true} \\ \lVerifyRule{g}{g}{true} \\ \lCheckLoops{g}{g}{false}}
    {\VerifyGrammar{g}{true}}
\end{mathpar}
    \caption{The well-formedness predicate.}
    \label{fig:verifygrammar}
\end{figure}
which takes a grammar
and returns a Boolean value
indicating whether the grammar
passes all the checks.
It is based on the predicates
presented in the previous sections.
\Cref{lemma:verifygrammar-determinism}
states that the predicate is deterministic,
and \Cref{lemma:verifygrammar-follows}
states that it follows the fixed-point definition.

\begin{lemma}
    \label{lemma:verifygrammar-determinism}
    If $\VerifyGrammar{g}{b_1}$,
    and $\VerifyGrammar{g}{b_2}$,
    then $b_1 = b_2$.
\end{lemma}

\begin{lemma}
    \label{lemma:verifygrammar-follows}
    If $\verifygrammarcomp{g}{gas} = \Some b$,
    then $\VerifyGrammar{g}{b}$.
\end{lemma}

In order to prove correctness,
we need to generalize the pattern from $\PNT{0}$
to any pattern $p$,
and to break down the function \textit{wf}
into its separate steps.
We need to make this generalization
because the match predicate is defined
recursively on the current pattern.
The generalized theorem we need to prove is the following:
Given a grammar $g$ and a pattern $p$,
if $g$ only contains coherent rules
that do not lead to left recursion
and that do not have any degenerate loops,
and if $p$ is coherent
and does not have degenerate loops,
then ${\forall s, \exists res\ \Matches{g}{p}{s}{res}}$.

We begin the proof by doing a strong induction on $n$,
the length of the input string $s$,
which gives us the inductive hypothesis ``IHn''.
This hypothesis states that for any input string shorter than $s$
and any pattern, we can assume that the match yields some result.
From \Cref{lemma:no-lr-rule-in-grammar}
and the assumption that the grammar contains no left-recursive rules,
we can infer that the pattern also does not lead to left recursion.
This gives us the inductive predicate
$\VerifyRule{g}{p}{d}{nb}{\Some b}{k}$,
which tells us that $p$ is either nullable or non-nullable.
We do an induction on this predicate
and handle each case separately.

Let us start with the basic patterns.
The case of the empty pattern $\PEmpty{}$ is trivial,
as it matches any input string without consuming anything.
The case of the character set pattern $\PSet{cs}$ is also simple.
If $s$ is the empty string, the pattern fails to match.
Otherwise, the string may or may not begin with the character $a \in cs$.
If it does, then it matches while consuming $a$.
Otherwise, it fails to match.

The sequence pattern $\PSequence{p_1}{p_2}$ has two cases:
one in which $p_1$ is non-nullable and $p_2$ is not visited,
and another in which $p_1$ is nullable and $p_2$ is visited.
In both cases, we have an inductive hypothesis stating that
$p_1$ has a match result for input string $s$.
If this match result is a failure,
then the whole sequence $\PSequence{p_1}{p_2}$ also fails.
If the match result is a success,
then $p_1$ leaves a suffix string $s'$ unconsumed.
Let us handle this case for both scenarios separately.

If $p_1$ is non-nullable,
then we can use \Cref{lemma:non-nullable-pattern-proper-suffix}
to state that $s'$ is a proper suffix of $s$,
and therefore shorter than $s$.
This allows us to use IHn,
and state that $p_2$ has a match result for the input string $s'$.
This implies that the sequence has this same match result.

Otherwise, if $p_1$ is nullable, then we can use
\Cref{lemma:match-suffix} to state that
$s'$ is a suffix of $s$.
This means that $s'$ is either equal to $s$, or a proper suffix of $s$.
If $s$ is equal to $s'$,
then we can use the inductive hypothesis
to state that $p_2$ yields a match result for $s'=s$.
Otherwise, then we can use IHn in the same way as the previous case,
because $s'$ would be shorter than $s$.

Now let us consider the case of the choice pattern $\PChoice{p_1}{p_2}$.
Similar to the case of the sequence pattern,
we have induction hypotheses for $p_1$ and $p_2$
yielding a match result for the input string $s$.
This case is simpler because $s$ is the same input string for both choice options,
so these induction hypotheses are enough to prove this case.

The case of the repetition pattern $\PRepetition{p}$ is the most interesting one,
as we get to use the fact that $p$ must be non-nullable in the proof.
From the induction step, we have the inductive hypothesis that
$p$ yields a match result for the input string $s$.
If $p$ fails to match, then $\PRepetition{p}$ matches without consuming anything.
If, otherwise, $p$ matches, then it leaves a string $s'$ unconsumed.
Since $p$ is non-nullable, $s'$ must be a proper suffix of $s$.
Therefore, we can use IHn to state that $\PRepetition{p}$
yields a match result $res$ for $s'$, because it is shorter than $s$.
In this case, $\PRepetition{p}$ also yields $res$ for $s$.

The case of the predicates $\PNot{p}$ and $\PAnd{p}$ are pretty straightforward.
We first use the inductive hypothesis that $p$
yields a match result for the input string $s$.
If $p$ matches,
then $\PAnd{p}$ matches without consuming anything,
and $\PNot{p}$ fails to match.
Otherwise, if $p$ fails to match,
then so does $\PAnd{p}$,
and $\PNot{p}$ matches without consuming anything.

Finally, we prove the case of the non-terminal pattern $\PNT{i}$,
which is surprisingly simple.
From the induction step,
we are given $p$, the $i^{th}$ rule of the grammar.
From the initial hypotheses,
we know that $p$ is coherent and free of degenerate loops,
because it is a grammar rule.
We can then use the inductive hypothesis
from the $\verifyrulename{}$ predicate to
state that $p$ yields a match result for the input string $s$.
\begin{theorem}
    \label{theorem:wf-correctness-generalized}
    Given a grammar $g$ and a pattern $p$,
    if $g$ only contains coherent rules
    that do not lead to left recursion
    and that free of degenerate loops,
    and if $p$ is coherent
    and free of degenerate loops,
    then $\forall s, \exists res\ \Matches{g}{p}{s}{res}$.
\end{theorem}
Having proved \Cref{theorem:wf-correctness-generalized},
we can finally prove the original theorem,
which states that, for any grammar $g$
that satisfies $\wf{g} = true$,
and for any input string,
the non-terminal pattern $\PNT{0}$
yields a match result.

\Cref{lemma:verifygrammar-termination} states that
the function \textit{\verifygrammarname{}} returns
$\Some b$ when given a gas counter greater or equal
to $\verifygrammargas{g}$.
In the implementation of the function \textit{wf},
we pass $\verifygrammargas{g}$ as the gas counter
for the function \textit{\verifygrammarname{}}.
Therefore, if \textit{wf} returns $true$,
then it must be because \textit{\verifygrammarname{}}
returned $\Some true$.

If the function \textit{\verifygrammarname{}}
returns $\Some true$,
then,
according to \Cref{lemma:verifygrammar-follows},
we can construct the predicate $\VerifyGrammar{g}{true}$.
We can see that this only happens when
the input grammar has passed
all the checks.
From \Cref{fig:verifygrammar},
we can see that this implies in
several other predicates,
many of which are necessary to use
\Cref{theorem:wf-correctness-generalized}.
There are only two missing predicates:
$\Coherent{g}{\PNT{0}}{true}$,
which states that the initial pattern is coherent,
and $\CheckLoops{g}{\PNT{0}}{d}{false}$,
which states that it does not contain any degenerate loops.

We can derive both predicates
from the fact that $\PNT{0}$ is a rule,
which we have checked already.
We use \Cref{lemma:lcoherent-safety}
to prove that $\PNT{0}$ is coherent,
and \Cref{lemma:lcheckloops-safety}
to prove that it contains no degenerate loops.
With this, we are able to prove the original theorem.

\begin{theorem}
    For any grammar $g$,
    if $\wf{g} = true$,
    then $g$ is complete.
\end{theorem}

    \chapter{First-set Algorithm}
\label{chapter:first-set}

In \Cref{chapter:wf-algorithm},
we presented the well-formedness check
implemented in \lpeg{},
which ensures that the input PEG is complete.
However, this is not the only role of this algorithm:
It also ensures that other algorithms
implemented in \lpeg{} terminate,
as they traverse patterns in similar ways.
This chapter presents one such algorithm,
which we label as the first-set algorithm.

The first-set algorithm is responsible for
computing the set of first characters that
can possibly be accepted by a pattern,
and a Boolean value
that indicates whether the pattern may
accept the empty string.
We call this Boolean value the
\emph{emptiness} value of the pattern.
The key properties of the algorithm,
which we later prove,
are the following:
Any pattern fails any string that starts
with a character that is not in the first-set
of the pattern;
and the pattern fails the empty string
if its emptiness value is $false$.

Both are conservative approximations,
which means that a full first-set
and an emptiness value of $true$ are
the safest options, yet the least useful ones.
That is because \lpeg{} uses this algorithm and its properties
to optimize certain patterns in the code generation phase.
So, ideally, we would like the first-set to be as small as possible,
and the emptiness value of $false$ as common as possible,
so that \lpeg{} is able to optimize code more often.

One of the patterns that \lpeg{} tries to optimize
using this algorithm is the ordered choice.
As a base reference, we display below the code
that \lpeg{} generates for an ordered choice $\PChoice{p_1}{p_2}$
without any optimizations.
\begin{align*}
    & Choice(L_2) \\
    & p_1 \\
    & Commit(L_{end}) \\
    L_2:\ & p_2 \\
    L_{end}:\ &
\end{align*}

This unoptimized code starts with a choice instruction,
which creates a checkpoint for the initial match state,
so that, if $p_1$ fails,
it is able to restore this state
before running $p_2$.
If, instead, $p_1$ succeeds, it runs a commit instruction,
which deletes this checkpoint and jumps to $L_{end}$.

However,
if $p_1$ and $p_2$ have disjoint first-sets,
and $p_1$ has an emptiness value of $false$,
then \lpeg{} generates the following optimized code.
\begin{align*}
    & TestSet(first(p_1), L_2) \\
    & p_1 \\
    & Jump (L_{end}) \\
    L_2:\ & p_2 \\
    L_{end}:\ &
\end{align*}

This optimized code begins with an instruction (test set)
that checks whether the input string starts with
a character in the first-set of $p_1$,
and jumps to $L_2$ if not.
In the case where the input string is empty,
we know that $p_1$ would fail,
because $p_1$ has an emptiness value of $false$.
If, otherwise, the input string starts with a character
that is not in the first-set of $p_1$,
then we know that $p_1$ would also fail,
by the key property of first-sets.
So, in either case, we know that $p_1$ would fail,
making the choice pattern equivalent to $p_2$.
That is why, in the case of failure,
the test-set instruction jumps to the code of $p_2$,
without ever running the code of $p_1$.

Meanwhile, the test-set instruction succeeds
if the input string starts with a character
that \emph{is} in the first-set of $p_1$,
and, therefore, is non-empty.
In this case, because the first-sets of $p_1$ and $p_2$ are disjoint,
we know that this character is \emph{not} in the first-set of $p_2$.
And, from the key property of first-sets,
this means that $p_2$ would fail,
making the choice pattern equivalent to $p_1$.
So, in this case, \lpeg{} simply executes the code of $p_1$.

This optimization significantly improves
the performance \lpeg{} when matching some choice patterns.
This improvement is possible
by replacing the costly creation/restoration/deletion of checkpoints
with inexpensive jumps and inspections on the input string.

In this chapter, we discuss the first-set algorithm
and prove its key properties.

\section{Building intuition}

Before we formally define the algorithm,
we would like to first develop some intuition
for the function signature.
We start with an informal definition:
the function receives a pattern
and returns the set of first characters that
can be accepted by the pattern.
Let us try to informally define the first-set
of some simple patterns
using this initial signature.

\newcommand{\firsta}[1]{\firstname{}(#1)}

Starting with $\PEmpty$ and $\PRepetition{p}$,
we know both patterns accept every input string,
so it makes sense for this function to return
the full character set, which we represent as $\Sigma$.
As for the character set pattern $\PSet{cs}$,
we know that it only accepts strings that start with
a character in the set $\Set{cs}$.
\begin{align*}
    \firsta{\PEmpty} &= \Sigma \\
    \firsta{\PRepetition{p}} &= \Sigma \\
    \firsta{\PSet{cs}} &= \Set{cs}
\end{align*}

However, when we try to define the first-set of sequence patterns,
we run into some issues.
To better comprehend them,
it can be helpful to mentally compute the first-set of
some simple sequences,
using the pattern types for which
we have defined the function so far.
This may give us some intuition as to how the first-set
of sequences may be derived from its sub-patterns.
It is worth highlighting that these definitions
are informal and derived purely from intuition.
The actual formal definitions
are presented later in this chapter.
\begin{align*}
    \firsta{\PSequence{\PEmpty}{\PEmpty}} &= \Sigma \\
    \firsta{\PSequence{\PSet{cs_1}}{\PEmpty}} &= \Set{cs_1} \\
    \firsta{\PSequence{\PEmpty}{\PSet{cs_2}}} &= \Set{cs_2} \\
    \firsta{\PSequence{\PSet{cs_1}}{\PSet{cs_2}}} &= \Set{cs_1} \\
    \firsta{\PSequence{\PSet{cs_1}}{\PRepetition{\PSet{cs_2}}}} &= \Set{cs_1} \\
    \firsta{\PSequence{\PRepetition{\PSet{cs_1}}}{\PEmpty}} &= \Sigma \\
    \firsta{\PSequence{\PRepetition{\PSet{cs_1}}}{\PSet{cs_2}}} &=
    \SetUnion{\Set{cs_1}}{\Set{cs_2}} \\
    \firsta{\PSequence{\PRepetition{\PSet{cs_1}}}{\PRepetition{\PSet{cs_2}}}} &=
    \Sigma
\end{align*}

From the examples above,
we can observe that when $p_1$ is non-nullable,
the first-set of $p_2$ doesn't seem to be relevant
to the first-set of the sequence.
That is because $p_1$, being non-nullable,
is guaranteed to consume a character.
As a result, $p_2$ is given a proper suffix of the original string,
which doesn't contain its first character.
Based on this observation,
we can define the function
for sequences $\PSequence{p_1}{p_2}$
where $p_1$ is non-nullable
as the first-set of $p_1$.
\begin{align*}
    \firsta{\PSequence{p_1}{p_2}} &= \begin{cases}
        \firsta{p_1} & \text{if $p_1$ is non-nullable} \\
        \dots ? & \text{if $p_1$ is nullable}
    \end{cases}
\end{align*}

We still need to define the function
for the case where $p_1$ is nullable.
From the patterns we have defined so far,
only the empty pattern $\PEmpty$ and
the repetition pattern $\PRepetition{\PSet{cs_1}}$ are nullable.
Let us see their behavior in sequence with
the character set pattern $\PSet{cs_2}$.
\begin{align*}
    \firsta{\PSequence{\PEmpty}{\PSet{cs_2}}} &= \Set{cs_2} \\
    \firsta{\PSequence{\PRepetition{\PSet{cs_1}}}{\PSet{cs_2}}} &= \SetUnion{\Set{cs_1}}{\Set{cs_2}}
\end{align*}

Note that, when alone, their first-sets are equal.
However, when in sequence with the same pattern,
the resulting first-sets differ.
This discrepancy indicates that we cannot define
$\firsta{\PSequence{p_1}{p_2}}$ as a pure function
of $\firsta{p_1}$ and $\firsta{p_2}$.
Instead, we would have to break the definition down
for each case of $p_1$.
\begin{align*}
    \firsta{\PSequence{\PEmpty}{p_2}} &= \firsta{p_2} \\
    \firsta{\PSequence{\PRepetition{p}}{p_2}} &=
    \SetUnion{\firsta{p}}{\firsta{p_2}} \\
    & \dots
\end{align*}

With this signature,
we would have to define the function for
every nullable pattern twice:
one when alone, and another when followed by another pattern.
However, we would like to define the function recursively
only once for each pattern type.

\newcommand{\firstb}[2]{\firstname{}(#1, #2)}

To solve this issue, \lpeg{} adds an accumulator parameter
for the first-set of the following pattern in the sequence.
When the pattern is not followed by a pattern,
\lpeg{} uses the full character set as the accumulator.
We call this accumulator the \emph{follow-set}.
Let us adapt our definitions for this new signature:
For each basic pattern $p_1$,
we can derive the definition of $\firstb{p_1}{follow}$
from the previous definition of $\firsta{\PSequence{p_1}{p_2}}$,
and by replacing $\firsta{p_2}$ with the new parameter $follow$.
\begin{align*}
    \firstb{\PEmpty}{follow} &= follow \\
    \firstb{\PRepetition{p}}{follow} &= \SetUnion{\firstb{p}{follow}}{follow} \\
    \firstb{\PSet{cs}}{follow} &= \Set{cs}
\end{align*}

The first-set of the character set pattern $\PSet{cs}$
does not use the follow parameter,
because it doesn't depend on the first-set of the following pattern.
This is because the character set pattern is non-nullable.
We later prove this property for any non-nullable pattern.

The case of the sequence pattern is interesting,
as it demonstrates the follow-set working as an accumulator:
In the case where $p_1$ is nullable,
we use the first-set of $p_2$ as the follow-set of $p_1$.
As for the case in which $p_1$ is non-nullable,
we use the first-set of $p_1$ with any follow-set.
We choose the full character set as the follow-set
to demonstrate this independence from the accumulator,
but any character set could be used.
\begin{align*}
    \firstb{\PSequence{p_1}{p_2}}{follow} &= \begin{cases}
        \firstb{p_1}{\firstb{p_2}{follow}} & \text{if $p_1$ is nullable} \\
        \firstb{p_1}{\Sigma} & \text{if $p_1$ is non-nullable}
    \end{cases}
\end{align*}

Let us now define the first-set of the other pattern types.
Starting with the choice pattern $p_1/p_2$,
we intuitively define it as the union
of the first-sets of $p_1$ and $p_2$,
passing down the follow-set parameter to each sub-call.
\begin{align*}
    \firstb{p_1/p_2}{follow} &=
    \SetUnion{\firstb{p_1}{follow}}{\firstb{p_2}{follow}}
\end{align*}

The case of the not-predicate pattern $\PNot{p}$
highlights the conservative nature of first-sets.
From the first-set of $p$,
we can only infer the set of first characters that make $p$ fail,
and, therefore, make $\PNot{p}$ succeed.
This, however, gives us no information about
first characters that make $p$ succeed,
and, therefore, make $\PNot{p}$ fail.
Therefore, in general, we cannot use the first-set of $p$
to compute the first-set of $\PNot{p}$.

There is, however, information
that we can extract from the follow-set parameter.
When $\PNot{p}$ is followed by a pattern $p_2$,
the follow-set indicates which first characters make $p_2$ fail.
Given that $\PNot{p}$ doesn't consume any input,
these first characters also make the sequence $\PSequence{\PNot{p}}{p_2}$ fail,
so they should be part of the first-set of $\PNot{p}$.
Therefore, we could, in general, use the follow-set
as the first-set of any predicate.
However, for the specific case of $\PNot{\PSet{cs}}$,
\lpeg{} instead computes the first-set
of patterns as ${\SetMinus{\Sigma}{\Set{cs}}}$.
\begin{align*}
    \firstb{\PNot{p}}{follow} &= \begin{cases}
        \SetMinus{\Sigma}{\Set{cs}} & \text{if $p = \PSet{cs}$} \\
        follow & \text{otherwise}
    \end{cases}
\end{align*}

One topic for future research is
to investigate whether it would be possible to
replace $\Sigma$ with the provided follow-set parameter in \lpeg{}.

As for the and-predicate pattern $\PAnd{p}$,
we use both the first-set of $p$ and the follow-set.
That is because in order to match the sequence $\PSequence{\PAnd{p}}{p_2}$,
the input string must match both $p$ and $p_2$.
Conversely, if the string starts with a character that is not in the
first-set of either pattern, the sequence fails.
So, in this case,
we define the first-set
as the intersection of both sets.
\begin{align*}
    \firstb{\PAnd{p}}{follow} &= \SetIntersection{\firstb{p}{\Sigma}}{follow}
\end{align*}

One subtle difference between this definition
and the actual implementation of the algorithm in \lpeg{} is
the follow-set parameter used to calculate the first-set of $p$.
While \lpeg{} simply passes along the follow-set parameter,
we provide the full character set $\Sigma$.
This change was necessary for us to prove the
key property of first-sets in Coq,
which we will show later in this chapter.
Future research may check whether
passing along the follow-set parameter
is incorrect, equivalent, or even better
than passing $\Sigma$ as the follow-set.

\newcommand{\firstc}[3]{\firstname{}(#1, #2, #3)}

Finally, in order to define the case of the non-terminal pattern,
we need to add the grammar as a parameter,
so that we can look up the referenced grammar rule.
In all cases, this grammar parameter is simply
passed along to each recursive call.
\begin{align*}
    \firstc{g}{\PNT{i}}{follow} &= \firstc{g}{p}{follow} & \text{if $g[i] = \Some p$}
\end{align*}

This case brings up the topic of termination,
as it does not define the recursion on
the structure of the pattern,
like the other cases do.
Instead, termination in this case
relies on the assumption
that the input PEG is well-formed,
and, therefore, free of left-recursive rules.

On a deeper level, termination is derived from
the way in which the well-formedness and first-set algorithms
traverse patterns similarly.
The most interesting case
is that of the sequence pattern:
when $p_1$ is non-nullable,
both algorithms do not visit $p_2$.
In the case of the well-formedness check,
visiting $p_2$ is not necessary
because the whole sequence is non-nullable,
and any rules visited in $p_2$ would be matched
against a proper suffix of the input string $s$
(avoiding infinite loops).

Meanwhile, in the case of the first-set algorithm,
visiting $p_2$ can be avoided for two reasons:
If $p_1$ is non-nullable,
then, as we later prove, its emptiness value is $false$,
which means that it fails the empty string.
If $p_1$ fails the empty string,
then so does the sequence,
which allows the emptiness value of the sequence to also be $false$,
regardless of the emptiness value of $p_2$.
The second reason is that, if $p_1$ is non-nullable,
then its first-set is independent of the follow-set parameter,
which, in the general case, would be the first-set of $p_2$.
Therefore, when $p_1$ is non-nullable,
we can provide any follow-set parameter, such as $\Sigma$,
in order to avoid making a recursive call to $p_2$.

\section{Matching the empty string}

Besides the first-set of a pattern,
the algorithm implemented in \lpeg{} also returns
a Boolean value, which indicates whether the
pattern may match the empty string,
a property we call \emph{emptiness}.
It is another conservative approximation:
the value $true$ has no meaning,
while the value $false$ indicates that the pattern
fails to match the empty string.
\lpeg{} needs this information
because it cannot use the first-set
to verify whether a pattern fails the empty string,
since the empty string has no first character.

Let us see how \lpeg{} computes
the emptiness of patterns.
The base cases are quite simple.
The empty pattern $\PEmpty$ and
the repetition pattern $\PRepetition{p}$
match every input string,
which includes the empty string.
So, for these patterns,
the function returns $true$.

The character class pattern $\PSet{cs}$,
as with any non-nullable pattern,
does not match the empty string.
Therefore, the value for this pattern is $false$.

For the not-predicate pattern $\PNot{p}$,
\lpeg{} is rather conservative,
always returning $true$.
In fact, this seems to be
the only case making the emptiness value
a conservative approximation.
If instead this function were to call itself
recursively for $p$ and negate its emptiness value,
we would effectively compute whether
the pattern matches the empty string.
However, the cases of $\PNot{p}$
where $p$ matches the empty string
are not common nor useful in practice.

The and-predicate pattern $\PAnd{p}$
and the non-terminal pattern $\PNT{i}$
simply forward the Boolean value
from the underlying pattern,
because they fail if and only if
the underlying pattern fails.

The sequence pattern $\PSequence{p_1}{p_2}$
matches the empty string if
both $p_1$ and $p_2$ do.
Intuitively, this would mean that the emptiness value of the sequence
would be the Boolean \scand{} of the emptiness values of $p_1$ and $p_2$,
but that is not exactly what is implemented in the algorithm.
As we have discussed at the end of the previous section,
when $p_1$ is non-nullable, $p_2$ is not visited,
and, therefore, the emptiness value of $p_2$ is not calculated.
However, we don't need this value,
since the emptiness value of $p_1$ is $false$ in this case,
which allows for the short-circuit evaluation
of the Boolean \scand{} expression to $false$.
Meanwhile, when $p_1$ is nullable,
the emptiness value of both $p_1$ and $p_2$ are computed,
and their Boolean \scand{} is calculated normally.

Finally, the case of the choice pattern $p_1/p_2$
is similar to that of the sequence pattern,
but instead of a Boolean \scand{} operation,
it performs a Boolean \scor{}
of the emptiness values of $p_1$ and $p_2$.
That is because the choice matches the empty string
if one of the options does.

\section{Formal definition}

\Cref{fig:firstcomp} presents
the formal definition of the first-set algorithm.
It takes a grammar, a pattern, a follow-set,
and some gas, and returns an optional tuple.
The recursion is defined on the gas parameter,
so that, if it reaches zero,
the function returns $\None$.
Otherwise, the function returns $\Some (b, first)$,
where $b$ is the emptiness value,
and $first$ is the first-set.
If $b=false$,
then the pattern fails to match the empty string;
and if a string starts with a character
$x \notin first$,
then it is guaranteed to fail to match that string.
The follow-set parameter is an
accumulator that should be initialized with
the full character set $\Sigma$.
In order to improve the legibility
of the function for sequence and choice patterns,
we also define the auxiliary functions $\otimes$ and $\oplus$,
respectively.
\begin{figure}
    \centering
    \begin{align*}
    \begin{aligned}[t]
        & \firstcomp{g}{p}{follow}{0} = \None \\
        & \firstcomp{g}{p}{follow}{(1+gas)} = \\
        & \begin{aligned}[t]
            & \matchwith{p} \\
            & \matchcase{\PEmpty}{\Some (true, follow)} \\
            & \matchcase{\PSet{cs}}{\Some (false, \Set{cs})} \\
            & \matchcase{\PRepetition{p}}{\begin{aligned}[t]
                & \matchwith{\firstcomp{g}{p}{follow}{gas}} \\
                & \matchcase{\Some (b, first)}{\Some (true, \SetUnion{first}{follow})} \\
                & \matchcase{\None}{\None} \\
                & \matchend{}
            \end{aligned}} \\
            & \matchcase{\PNot{p}}{\begin{aligned}[t]
                & \matchwith{p} \\
                & \matchcase{\PSet{cs}}{\Some (true, \SetMinus{\Sigma}{\Set{cs})}} \\
                & \matchcase{otherwise}{\Some (true, follow)} \\
                & \matchend{}
            \end{aligned}} \\
            & \matchcase{\PAnd{p}}{\begin{aligned}[t]
                & \matchwith{\firstcomp{g}{p}{\Sigma}{gas}} \\
                & \matchcase{\Some (b, first)}{\Some (b, \SetIntersection{first}{follow})} \\
                & \matchcase{\None}{\None} \\
                & \matchend{}
            \end{aligned}} \\
            & \matchcase{\PNT{i}}{\begin{aligned}[t]
                & \matchwith{g[i]} \\
                & \matchcase{\Some p}{\firstcomp{g}{p}{follow}{gas}} \\
                & \matchcase{\None}{\None} \\
                & \matchend{}
            \end{aligned}} \\
            & \matchcase{\PSequence{p_1}{p_2}}{\begin{aligned}[t]
                & \matchwith{\nullablecomp{g}{p_1}{gas}} \\
                & \matchcase{\Some false}{\firstcomp{g}{p_1}{\Sigma}{gas}} \\
                & \matchcase{\Some true}{\begin{aligned}[t]
                    & \matchwith{\firstcomp{g}{p_2}{follow}{gas}} \\
                    & \matchcase{\Some (b_2, first_2)}{b_2 \otimes (\firstcomp{g}{p_1}{first_2}{gas})} \\
                    & \matchcase{\None}{\None} \\
                    & \matchend{}
                \end{aligned}} \\
                & \matchcase{\None}{\None} \\
                & \matchend{}
            \end{aligned}} \\
            & \matchcase{\PChoice{p_1}{p_2}}{(\firstcomp{g}{p_1}{follow}{gas}) \oplus (\firstcomp{g}{p_2}{follow}{gas})} \\
            & \matchend{}
        \end{aligned}
    \end{aligned}
\end{align*}
    \caption{The first-set function.}
    \label{fig:firstcomp}
\end{figure}
\begin{figure}
    \centering
    \begin{align*}
    \begin{aligned}[t]
        & b \otimes res = \\
        & \matchwith{res} \\
        & \matchcase{\Some (b', first')}{\Some (b \wedge b', first')} \\
        & \matchcase{\None}{\None} \\
        & \matchend{}
    \end{aligned}
\end{align*}
    \begin{align*}
    \begin{aligned}[t]
        & res_1 \oplus res_2 = \\
        & \matchwith{res_1, res_2} \\
        & \matchcase{\Some (b_1, first_1), \Some (b_2, first_2)}
                    {\Some (b_1 \vee b_2, \SetUnion{first_1}{first_2})} \\
        & \matchcase{otherwise}{\None} \\
        & \matchend{}
    \end{aligned}
\end{align*}
    \caption{The auxiliary $\otimes$ and $\oplus$ functions.}
    \label{fig:firstaux}
\end{figure}

Having formally defined the first-set algorithm,
we now prove its key properties.
We begin by proving that if the function returns
some result for some gas amount,
it will return the same result if you provide
a higher gas amount.
In some sense, this means the function is stable
when you increase the gas amount.

\begin{lemma}
If $\firstcomp{g}{p}{follow}{gas} = \Some res$, \\
then, $\forall gas' \ge gas, \firstcomp{g}{p}{follow}{gas'} = \Some res$.
\end{lemma}

One natural consequence of this lemma is that,
for the same grammar, pattern, and follow-set,
the function cannot return contradicting results.

\begin{lemma}
If $\firstcomp{g}{p}{follow}{gas} = \Some res$, \\
and $\firstcomp{g}{p}{follow}{gas'} = \Some res'$, \\
then $res = res'$.
\end{lemma}

The previous two lemmas show how consistent
the return of the function is,
but both assume the existence of a gas amount
for which the function returns some result.
However, we know this is not always the case.
In fact, for ill-formed grammars, it may return $\None$
for any gas amount.
So, it is important to prove that,
for well-formed PEGs,
there exists a lower bound for the gas amount,
for which the function returns some result.
This lower bound effectively shows that
the algorithm terminates even without the gas parameter,
as it is implemented in \lpeg{}.

\begin{lemma}
If $\wf{g} = true$, \\
then $\forall gas \ge \size{p} + (1 + \length{g}) \cdot \size{g}$, \\
$\exists res, \firstcomp{g}{p}{follow}{gas} = \Some res$.
\end{lemma}

Having proved termination,
let us now focus on the key properties
of the first-set algorithm,
starting with the emptiness value.
We prove that,
for well-formed PEGs,
if $b=false$,
then the pattern fails the empty string
(denoted as $nil$).
Note that, in this case,
the follow-set parameter is irrelevant.

\begin{lemma}
If $\wf{g} = true$, \\
and $\firstcomp{g}{p}{follow}{gas} = \Some (false, first)$, \\
then $\Matches{g}{p}{nil}{\Failure}$.
\end{lemma}

Now, we prove lemmas about the
relation between the follow-set parameter
and the first-set return value.
These lemmas are necessary to prove a
more important lemma later in this section.
We start by proving that if the follow-set parameter
is incremented by an extra set (through a set union operation),
then the first-set return value is incremented by a subset
of this extra set.
Note that the emptiness value stays the same
with this follow-set increment.

\begin{lemma}
If $\firstcomp{g}{p}{follow}{gas} = \Some (b, first)$, \\
then $\forall extra, \exists extra' \subseteq extra$, \\
such that $\firstcomp{g}{p}{(\SetUnion{follow}{extra})}{gas} = \Some (b, (\SetUnion{first}{extra'}))$.
\end{lemma}

A particular case
is when this extra set is the first-set itself,
as if it were fed back into the function
through the follow-set parameter.
In this case, the first-set output by the function is the same,
since $\SetUnion{first}{extra'} \equiv first$ when $extra' \subseteq first$.
This particular lemma
is the one we actually use
to prove the more important lemma.

\begin{lemma}
If $\firstcomp{g}{p}{follow}{gas} = \Some (b, first)$, \\
then $\firstcomp{g}{p}{(\SetUnion{follow}{first})}{gas} = \Some (b, first)$.
\end{lemma}

The following lemma is the cornerstone of the key property of first-sets:
If $p$ matches some string $s$, leaving a suffix $s'$ unconsumed,
then $s$ must be either empty or start with a character
that \emph{is} in the first-set of $p$.
We also assume that the input PEG is well-formed,
and that $s'$ is either empty or starts with a character in the follow-set.
This last assumption is necessary to prove the lemma
in the case of the sequence pattern.

\begin{lemma}
If $\wf{g} = true$, \\
and $\firstcomp{g}{p}{follow}{gas} = \Some (b, first)$, \\
and $\Matches{g}{p}{s}{s'}$, \\
and $s'$ either is empty or starts with $x \in follow$, \\
then $s$ either is empty or starts with $y \in first$.
\end{lemma}

In the case of the and-predicate pattern $\PAnd{p}$,
we noticed that it would be easier to prove this lemma
if we passed $\Sigma$ as the follow-set of $p$.
That is because $\PAnd{p}$ matches when $p$ matches,
but $p$ leaves an unconsumed suffix $s'$ that is
discarded and whose starting character (if non-empty)
we know nothing about. Ultimately, we cannot say that $s'$
is either empty or starts with a character in
an arbitrary follow-set. Instead, we use $\Sigma$
as the follow-set of $p$, as this turns
this hypothesis into a tautology.

Finally, we prove the key property of first-sets:
For a well-formed PEG,
if the emptiness value is $false$,
then the pattern fails for any string
that does not start with a character in its first-set.
Note that we use the full character set $\Sigma$ as the follow-set.

\begin{lemma}
If $\wf{g} = true$, \\
and $\firstcomp{g}{p}{\Sigma}{gas} = \Some (false, first)$, \\
and $s$ either is empty or starts with $x \notin first$, \\
then $\Matches{g}{p}{s}{\Failure}$.
\end{lemma}

Besides this main property,
we also prove that, for non-nullable patterns,
the follow-set parameter does not influence the result.

\begin{lemma}
If $\nullablecomp{g}{p}{gas_n} = \Some false$, \\
and $\firstcomp{g}{p}{follow_1}{gas_1} = \Some res_1$, \\
and $\firstcomp{g}{p}{follow_2}{gas_2} = \Some res_2$, \\
then $res_1 = res_2$.
\end{lemma}

This lemma explains why, in the cases of character set patterns
and sequence patterns with non-nullable first patterns,
the follow-set parameter can be completely ignored.
We can also observe that, in the case of repetitions $\PRepetition{p}$,
\lpeg{} passes along the follow-set parameter to $p$,
but any follow-set could be provided,
given that $p$ is non-nullable from the well-formedness property.

Another fact about non-nullable patterns
is that their emptiness value is always $false$.
From the key property of emptiness values,
this indicates that non-nullable pattern
fail to match the empty string,
which we know is true.

\begin{lemma}
If $\nullablecomp{g}{p}{gas_n} = \Some false$, \\
and $\firstcomp{g}{p}{follow}{gas} = \Some (b, first)$, \\
then $b = false$.
\end{lemma}



% Nova seção


\section{Application in \lpeg{}}

Having proved the key properties of the first-set algorithm,
we would like to formalize its application in \lpeg{}.
As discussed at the beginning of this chapter,
\lpeg{} uses this algorithm
when generating code for choice patterns,
making use of test-set instructions.
Despite this optimization occurring at the virtual machine code level,
we would like to formalize it at the syntactic level.

\newcommand{\ifthenelsepat}[3]{\PChoice{\PSequence{\PAnd{#1}}{#2}}{\PSequence{\PNot{#1}}{#3}}}

The test-set instruction basically checks
whether the input string
starts with a character in a given set $\Set{cs}$,
jumping to a given label if it does not.
We can check the first character of the input string
through the character set pattern $\PSet{cs}$,
and emulate the logic of
``if $p_{cond}$ matches, then try $p_1$, otherwise try $p_2$''
through the following pattern construction.
\begin{align*}
    \ifthenelsepat{p_{cond}}{p_1}{p_2}
\end{align*}

In the optimized code of the choice pattern,
the test-instruction checks if the input string
starts with a character in the first-set of $p_1$,
and jumps to the code of $p_2$ if it does not.
This instruction is followed by the code of $p_1$,
which is executed if the check succeeds.
We can represent this optimized code
as the following pattern.
Let $\Set{first_1}$ denote the first-set of $p_1$.
\begin{align*}
    \ifthenelsepat{\PSet{first_1}}{p_1}{p_2}
\end{align*}

We now prove the correctness of this optimization.
Assuming the grammar $g$ and patterns $p_1$ and $p_2$ are well-formed,
and that the first-sets of $p_1$ and $p_2$ are disjoint,
and that the emptiness value of $p_1$ is $false$,
we first prove that if the original choice pattern matches a string $s$,
the optimized choice also matches $s$,
yielding the same unconsumed suffix $s'$.
We also need to assume that $s'$
either is empty or starts with a character in the
follow-set of $p_2$.

\begin{lemma}
If $g$, $p_1$ and $p_2$ are well-formed, \\
and $s'$ either is empty or starts with $x \in follow$, \\
and $\firstcomp{g}{p_1}{\Sigma}{gas_1} = \Some (false, first_1)$, \\
and $\firstcomp{g}{p_2}{follow}{gas_2} = \Some (b, first_2)$, \\
and $\SetIntersection{first_1}{first_2} = \EmptySet{}$, \\
and $\Matches{g}{\PChoice{p_1}{p_2}}{s}{s'}$, \\
then $\Matches{g}{\ifthenelsepat{\PSet{first_1}}{p_1}{p_2}}{s}{s'}$.
\end{lemma}

As for the case in which the choice fails,
we also show the optimized choice fails as well.
The proof follows from two facts:
The choice fails either because of $p_1$ or $p_2$;
And, for any input string,
either $\PAnd{\PSet{first_1}}$ matches
and $\PNot{\PSet{first_1}}$ fails,
or the other way around.

\begin{lemma}
If $\Matches{g}{\PChoice{p_1}{p_2}}{s}{\Failure}$, \\
then $\Matches{g}{\ifthenelsepat{\PSet{first_1}}{p_1}{p_2}}{s}{\Failure}$.
\end{lemma}
    \chapter{Related Work}
\label{chapter:related-work}

Ford~\cite{ford_parsing_2004} introduced PEGs
and provided initial theoretical results about them.
He proved that the problem of knowing whether
a PEG is complete is undecidable,
and presented \emph{well-formedness}
as a conservative approximation to completeness.
In this approximation,
he defined the conservative notion of \emph{nullable} expressions,
which can accept an input string without consuming any characters.

Medeiros et al.~\cite{medeiros_left_2014}
presented a conservative extension to the semantics of PEGs
based on bounded left recursion and proved its correctness.
They have also proved that every PEG in this extension
is complete, assuming that every non-terminal is valid.
We have chosen not to go in this direction,
as our main goal was to formalize \lpeg{},
which categorizes left-recursive rules as ill-formed.

Ribeiro et al.~\cite{ribeiro_towards_2019}
formalized the syntax and semantics of PEGs
using the Agda proof assistant.
They also formalized the well-formedness verification process
in a way similar to a typing procedure.
Expressions are typed according to the set of nonterminals that can
be reached without consuming any character
(called \emph{head-set})
and whether the expression is nullable,
following Ford's definition.

The most interesting restrictions in this typing
are those on nonterminal expressions and repetitions.
For nonterminals, the typing
prohibits the nonterminal itself from being contained in its head-set.
For repetitions $\PRepetition{p}$, it prohibits $p$ from being nullable.
Although correct, this method does not provide a direct algorithm for this verification, instead relying on a typing algorithm.

Koprowski et al.~\cite{koprowski_trx_2011} developed TRX,
a parser interpreter formalized using the Coq proof assistant.
Their work extended PEGs to support semantic values and actions,
and focused on extracting parsers from them
with proofs of termination and correctness.

They formalized a well-formedness check for these extended PEGs
that is largely based on Ford's original work:
The algorithm iteratively computes
a set of well-formed expressions
until a fixed-point is reached,
and then checks if this set
coincides with the expression set of the grammar.
Although proven correct,
this algorithm seems harder to implement
using low-level programming languages,
when compared to the algorithm implemented in \lpeg{},
which is written in~C in order to better interact with the Lua C~API.

Blaudeau et al.~\cite{blaudeau_verified_2020} specified
a verified packrat parser interpreter for PEGs in PVS,
emphasizing the formal verification of the parsing process.
In order for their algorithm to correctly detect left recursion,
they assume there exists a correct order for visiting non-terminals.
However, they do not provide an algorithm for computing such an order.

In contrast, our work presents a direct and practical algorithm
for verifying the well-formedness of PEGs.
By providing concrete implementation details,
we offer a more straightforward approach
to well-formedness verification compared
to the formal proof-based methods of the aforementioned works.

Another contribution of our work
is the formalization of the first-set algorithm implemented in \lpeg{}.
Although the concept of first-sets is well-established
in the area of context-free grammars~%
\cite{chomsky_three_1956},
to the extent of our knowledge
their application in PEGs has not been documented yet.
We proved the key properties of the first-set algorithm,
and the soundness of its application in \lpeg{},
as an optimization technique
during the code generation phase.

    \chapter{Conclusion}
\label{chapter:conclusion}

In this work,
we formalized two key algorithms implemented in \lpeg{}:
the well-formedness check and the first-set computation.
Both algorithms were defined as functions in Coq
using a fixed-point construction,
with the recursion being defined on a gas parameter.
We proved that both algorithms terminate
by providing a lower bound for the gas parameter.
While the well-formedness check guarantees
this property for any input PEG,
the first-set computation assumes
the input PEG has successfully passed the well-formedness check.

Besides proving their termination,
we have proved these algorithms are correct,
in their own respective ways.
For the well-formedness check,
we have proved that it correctly detects complete PEGs,
which, in turn, guarantees that parsing terminates.
Meanwhile, for the first-set algorithm,
we have proven that it computes the set of first characters that make a pattern fail,
and that it checks whether the pattern fails for the empty string.

Moreover,
we used the properties of the first-set algorithm
to prove that an optimization performed by \lpeg{}
on certain choice patterns is correct.
This optimization is also performed on other types
of patterns, but we leave the proof of their correctness
as a topic for future research.

Still on the topic of future research,
while formalizing these algorithms,
we identified some details that deserve future review,
as they could lead to future improvements in \lpeg{}.
In the particular case of the first-set algorithm,
we modified the definition for the and-predicate pattern $\PAnd{p}$
so that we would be able to prove the key properties of the algorithm.
Future research should investigate whether it would be possible
to prove these properties for the actual implementation in \lpeg{}.
Furthermore, we suspect the case of the pattern $\PNot{\PSet{cs}}$
could be reviewed to make the first-set even smaller,
and, therefore, more likely to be used in optimizations.

Future work may also seek to measure the
computational and space complexity
of these algorithm in terms of
some notion of grammar size,
such as the total number of nodes
in the abstract syntax tree of rules.
Such measurements may even help us
find opportunities for improvements.
    \arial
    \bibliographystyle{abnt-num} % \bibliographystyle{abnt-num}
    \bibliography{references}
\end{document}