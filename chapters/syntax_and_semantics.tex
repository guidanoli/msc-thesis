\chapter{Syntax and Semantics of PEGs}
\label{chapter:pegs}

When Ford~\cite{ford_parsing_2004} introduced PEGs,
he also defined a syntax
with several useful constructions
for practical language description purposes:
literal strings (e.g. ``abc''),
character classes (e.g. $\PRange{a}{z}$), any character ($\PDot{}$),
non-terminals,
optionals $\POptional{p}$, zero-or-more repetitions $\PRepetition{p}$,
one-or-more repetitions $\PPlus{p}$, not-predicates $\PNot{p}$,
and-predicates $\PAnd{p}$, sequences $\PSequence{p_1}{p_2}$,
and ordered choices $\PChoice{p_1}{p_2}$.

However, for the purposes of formal analysis,
Ford realized it would be more
convenient to define an abstract syntax for PEGs
that represents its essential structure.
This abstract syntax does not include
several aforementioned constructions,
which are treated as \emph{syntactic sugar},
and includes two new constructions:
empty ($\PEmpty$) and single characters (e.g. `a').

With these new constructions,
Ford \emph{desugars} the PEG syntax as follows.
First, the any-character pattern ($\PDot{}$) is reduced
into a character class with all characters in the alphabet.
Then, literal strings are reduced into sequences of characters,
and character classes are reduced into choices of characters.
And-predicate patterns $\PAnd{p}$ are reduced into $\PNot{(\PNot{p})}$,
optionals $\POptional{p}$ into $\PChoice{p}{\PEmpty}$,
and one-or-more repetitions $\PPlus{p}$ into $\PSequence{p}{\PRepetition{p}}$.

This desugared syntax allowed Ford to reduce the size of proofs,
as there were fewer cases to be treated during case analysis.
However, for the purposes of formalizing the first-set algorithm,
we realized it would be beneficial for us to have our own
desugaring of the PEG syntax.
Compared to Ford's desugared syntax,
\emph{our} desugared syntax
includes and-predicates $\PAnd{p}$
and replaces characters (e.g. `a')
with character sets (e.g. $\PRange{a}{z}$).

Characters and character sets
have equivalent expressiveness,
as character sets can be reduced into choices of characters.
However, in our case, we prefer character sets as they allow us to
simplify the definition of the first-set algorithm.
Moreover, we decided to keep and-predicates $\PAnd{p}$
to stay more loyal to the implementation of \lpeg{},
which doesn't treat $\PAnd{p}$ as syntactic sugar for $\PNot{(\PNot{p})}$
and handles them differently in the first-set algorithm.
We also denote non-terminals by $\PNT{i}$,
where $i$ is the index of the rule in the grammar.
Our desugared syntax is depicted in \Cref{fig:syntax}.
\begin{figure}[t]
    \centering
    \begin{tabular}{llll|l|l|l|l|l|l|l}
    Pattern & $p$ & $:=$
    & $\PEmpty$
    & $\PSet{cs}$
    & $\PNT{i}$
    & $\PRepetition{p}$
    & $\PNot{p}$
    & $\PAnd{p}$
    & $\PSequence{p_1}{p_2}$
    & $\PChoice{p_1}{p_2}$
\end{tabular}
    \caption{Our desugared syntax of PEGs.}
    \label{fig:syntax}
\end{figure}

Having defined the syntax of PEGs,
we now define their semantics.
In the context of a grammar,
a pattern is parsed against an input string,
and the match may be either a success or a failure.
In the case of a success,
the pattern leaves the unconsumed suffix
of the input string as a result.

To be more precise,
we define a match predicate.
The predicate
takes a grammar $g$,
a pattern $p$,
and an input string $s$,
and returns a match result $res$.
It is denoted as $\Matches{g}{p}{s}{res}$.
A successful match result is represented by $s'$,
the unconsumed suffix of the input string,
while a failed match result is represented as $\Failure$.

The predicate is inductively defined in \Cref{fig:match}.
\begin{figure}
    \begin{mathpar}
    \namedinferrule{m-eps}
    { }
    {\Matches{g}{\PEmpty}{s}{s}}

    \namedinferrule{m-set-nil}
    { }
    {\DoesNotMatch{g}{\PSet{cs}}{\EmptyString}}

    \namedinferrule{m-set-cons-in}
    {a \in cs}
    {\Matches{g}{\PSet{cs}}{\String{a}{s}}{s}}

    \namedinferrule{m-set-cons-not-in}
    {a \not\in cs}
    {\DoesNotMatch{g}{\PSet{cs}}{\String{a}{s}}}

    \namedinferrule{m-nonterminal}
    {g[i] = \Some p \\ \Matches{g}{p}{s}{res}}
    {\Matches{g}{\PNT i}{s}{res}}

    \namedinferrule{m-rep-fail}
    {\DoesNotMatch{g}{p}{s}}
    {\Matches{g}{\PRepetition p}{s}{s}}

    \namedinferrule{m-rep-succ}
    {\Matches{g}{p}{s}{s'} \\ \Matches{g}{\PRepetition p}{s'}{res}}
    {\Matches{g}{\PRepetition p}{s}{res}}

    \namedinferrule{m-not-succ}
    {\Matches{g}{p}{s}{s'}}
    {\DoesNotMatch{g}{\PNot p}{s}}

    \namedinferrule{m-not-fail}
    {\DoesNotMatch{g}{p}{s}}
    {\Matches{g}{\PNot p}{s}{s}}

    \namedinferrule{m-and-succ}
    {\Matches{g}{p}{s}{s'}}
    {\Matches{g}{\PAnd{p}}{s}{s}}

    \namedinferrule{m-and-fail}
    {\DoesNotMatch{g}{p}{s}}
    {\DoesNotMatch{g}{\PAnd{p}}{s}}

    \namedinferrule{m-seq-fail}
    {\DoesNotMatch{g}{p_1}{s}}
    {\DoesNotMatch{g}{\PSequence{p_1}{p_2}}{s}}

    \namedinferrule{m-seq-succ}
    {\Matches{g}{p_1}{s}{s'} \\ \Matches{g}{p_2}{s'}{res}}
    {\Matches{g}{\PSequence{p_1}{p_2}}{s}{res}}

    \namedinferrule{m-choice-succ}
    {\Matches{g}{p_1}{s}{s'}}
    {\Matches{g}{\PChoice{p_1}{p_2}}{s}{s'}}

    \namedinferrule{m-choice-fail}
    {\DoesNotMatch{g}{p_1}{s} \\ \Matches{g}{p_2}{s}{res}}
    {\Matches{g}{\PChoice{p_1}{p_2}}{s}{res}}
\end{mathpar}
    \caption{The match predicate.}
    \label{fig:match}
\end{figure}
The empty pattern $\PEmpty$
matches any string,
without consuming anything.
A character set pattern ${\PSet{cs}}$
matches only strings that start with a character in the set $\Set{cs}$,
and consumes this character.
The nonterminal pattern ${\PNT i}$
matches the ${i^{th}}$ rule of the grammar.
The repetition pattern ${\PRepetition p}$
matches $p$ as many times as possible
in sequence.
The not-predicate pattern ${\PNot{p}}$ matches
iff $p$ does not match.
(It never consumes any input,
as either $p$ or ${\PNot{p}}$ always fail.)
The and-predicate pattern ${\PAnd{p}}$ matches
iff $p$ matches,
but without consuming any input.
A sequence pattern ${\PSequence{p_1}{p_2}}$
matches $p_1$ followed by $p_2$.
Lastly, the pattern ${\PChoice{p_1}{p_2}}$
constitutes an \emph{ordered choice}:
It first tries to match $p_1$,
and, if that fails, tries to match $p_2$.

In case the reader is not familiar with the semantics of PEGs,
it can be helpful to go through some examples.
Let us start with the pattern $\PSet{ab}$,
which matches the letters ``a'' or ``b''.
Unsurprisingly,
this pattern matches and consumes
the whole string ``a'',
leaving the empty string as a result.
It also matches the string ``baby'',
consuming just the first letter ``b''
and leaving the suffix ``aby'' unconsumed.
Meanwhile, for strings that do not start
with either the character ``a'' or ``b'',
the pattern fails.
For example, the pattern fails for the empty string
and for the string ``kaaba''.

If we wish the pattern to match zero or more letters ``a'' or ``b'',
we can use the repetition operator on the previous pattern,
resulting in the pattern $\PRepetition{\PSet{ab}}$.
This pattern matches the string ``baby'',
consuming the prefix ``bab''
and leaving the letter ``y'' unconsumed.
It also matches the empty string.

Some repetitions, however, are incomplete,
meaning they may yield no match result.
Take, for example, the pattern $\PRepetition{\PEmpty}$,
which matches $\PEmpty$ as many times as possible.
The repetition body $\PEmpty$ always matches and consumes no input,
so the match never makes any progress and gets stuck in an infinite loop.
This happens for any repetition whose body
may match a string while consuming no input.
\begin{align*}
    \forall g, \forall s, \Matches{g}{p}{s}{s} \implies
    \nexists res, \Matches{g}{\PRepetition{p}}{s}{res}
\end{align*}

Some grammar rules can also be incomplete,
which we will demonstrate by showing how repetitions
can be reduced to recursive grammar rules.
Given a repetition pattern $\PRepetition{p}$,
we can rewrite it as the non-terminal pattern $\PNT{i}$
in a grammar whose $i^{th}$ rule is defined as
$\PChoice{\PSequence{p}{\PNT{i}}}{\PEmpty}$.
If $\PRepetition{p}$ is incomplete,
then its reduction to a recursive grammar rules is also incomplete,
even if $p$ contains no repetitions.
This demonstrates that some grammar rules can be
intrinsically incomplete, even when free of repetitions.
This happens when rules reference themselves
(directly or indirectly) while consuming no input in-between.
We call these left-recursive rules.
The simplest example of such a rule is
one whose body is a non-terminal pattern that references
the rule itself.
\begin{align*}
    \forall g, \forall i, g[i] = \Some \PNT{i} \implies
    \forall s, \nexists res, \Matches{g}{\PNT{i}}{s}{res}
\end{align*}

As briefly displayed in the reduction of repetitions
into recursive grammar rules,
we can chain patterns together
with the sequence operator.
For example,
the pattern $\PSequence{\PRepetition{\PSet{ab}}}{\PSet{y}}$
matches the string ``baby'' but not the string ``babies''.

If we wish a pattern to be optional,
similar to how the $\POptional{p}$ operator from the
complete PEG syntax does,
we can make a choice between this pattern
and the empty pattern $\PEmpty$.
For example,
the pattern $\PSequence{\PRepetition{\PSet{ab}}}{(\PChoice{\PSet{y}}{\PEmpty})}$
matches both strings ``baby'' and ``babies'',
but leaves the suffix ``ies'' unconsumed.

We can also choose to fail when a given pattern matches,
through the not-predicate operator $\PNot{p}$.
Let us say, for example,
that the consumed input prefix
must not contain the substring ``aa'.
This behavior is implemented in the pattern
$\PRepetition{(\PSequence{\PNot{(\PSequence{\PSet{a}}{\PSet{a}})}}{\PSet{ab}})}$.
This pattern matches the strings ``baba'' and ``baaba'',
but leaves the suffix ``aaba'' unconsumed.

We can also check whether a pattern matches
but not consume any input.
That is possible with the and-predicate operator $\PAnd{p}$.
For example,
the pattern $\PAnd{\PSet{ab}}$
checks whether a string
starts with either the character ``a'' or ``b'',
but does not consume it.
This pattern matches the string ``baby'',
while not consuming any characters,
and fails to match the string ``kaaba''.

One aspect of PEGs that can vary
depending on the implementation
is how rules are identified and referenced,
that is, the set of nonterminals.
The original syntax proposed by Ford
uses strings~\cite{ford_parsing_2004}.
Meanwhile,
\lpeg{} allows rules to be identified
by arbitrary Lua values~\cite{ierusalimschy_lpeg_2024}.
Meanwhile, our formalization uses natural indices
to identify the rules,
as it allows us to represent
grammars as lists of rules in Coq.

Regarding the match predicate,
we can already state two important,
yet simple, properties.
First, the predicate is deterministic:
Given a grammar, a pattern, and an input string,
there is at most one possible match result.

\begin{lemma}%[Determinism]
    If ${\Matches{g}{p}{s}{res_1}}$
    and ${\Matches{g}{p}{s}{res_2}}$,
    then ${res_1 = res_2}$.
    \label{lemma:match-determinism}
\end{lemma}

The second property states that the result string
is a suffix of the input string.
We use the symbol ``${\Suffix{}{}}$''
to denote the suffix relation.
This lemma might seem obvious,
but we assure the reader that it will be necessary later
to prove lemmas by induction on the length of the input string.

\begin{lemma}%[Result String is Suffix of Input String]
    If ${\Matches{g}{p}{s}{s'}}$,
    then ${\Suffix{s'}{s}}$.
    \label{lemma:match-suffix}
\end{lemma}

An important characteristic of this predicate,
which has profound consequences in our formalization efforts,
is that it not represent a total function:
Not all combinations of
grammars, patterns, and input strings
relate to a match result.
This may happen for several reasons,
which we will go through
in the following paragraphs.

The first reason is that
the grammar may be empty,
meaning it does not have any rule.
This is a problem,
because we use the first rule of the grammar
as the initial pattern.
If the grammar is empty,
then there is no first rule.

The second reason is that
the pattern may reference a non-existent rule.
This problem is not exclusive to our formalization,
which references rules by indices.
Ford and \lpeg{}
also suffer from this problem,
because,
in order to textually describe a recursive grammar,
you need to use some form of reference,
which may be invalid.

The third reason for
a pattern to not yield a match result
is because of so-called \emph{left-recursive} rules.
To illustrate what we mean by such,
consider the following rule.
\begin{equation}
    \Rule{\PNT{i}}{\PNT{i}}
    \label{eqn:lr-rule-simple}
\end{equation}

If we were to parse the pattern ${\PNT{i}}$,
we would resolve it to the body of the ${i^{th}}$ rule,
which is also the pattern ${\PNT{i}}$.
In doing so,
neither the pattern nor the input string would change,
which clearly leads to an infinite loop.
This is the simplest example of a left-recursive rule.

The previous example involves only one rule.
Left-recursive rules, however,
can also involve
multiple rules.
The following grammar,
for example,
has three rules,
one referencing another,
all of which are left-recursive.
\begin{equation*}
    \begin{cases}
        \Rule{\PNT{i}}{\PNT{j}} \\
        \Rule{\PNT{j}}{\PNT{k}} \\
        \Rule{\PNT{k}}{\PNT{i}}
    \end{cases}
\end{equation*}

The reader might notice that,
in both examples,
there are rules referencing one another.
However, this doesn't necessarily imply in left recursion.
The following rule, for example,
references itself, but is not left-recursive.

\begin{equation}
    \Rule{\PNT{i}}{\PChoice{\PSequence{\PSet{cs}}{\PNT{i}}}{\PEmpty}}
    \label{eqn:a-star-rule}
\end{equation}

What makes the rule from \Cref{eqn:lr-rule-simple} left-recursive,
but not the one from \Cref{eqn:a-star-rule},
is a subtle, yet important difference.
In \Cref{eqn:lr-rule-simple},
the nonterminal pattern ${\PNT{i}}$
is always visited
with the same input string
as the one provided to the rule.
Meanwhile,
in \Cref{eqn:a-star-rule},
the nonterminal pattern ${\PNT{i}}$
is always visited
with an input string
shorter than the one provided to the rule,
since the character set pattern ${\PSet{cs}}$
always consumes a character when it matches.
Given that input strings are finite,
we can prove that this rule yields a result
for any input string.
The proof is carried out by induction on the length of
the input string,
using \Cref{lemma:match-suffix}.

The fourth and last reason
for the lack of a match result
are so-called degenerate loops,
which are repetition patterns ${\PRepetition{p}}$
whose body $p$
may match without consuming any input.
If $p$ matches, but does not consume any input,
then it will do so infinitely many times.
One example of a degenerate loop is the pattern
${\PRepetition{(\PRepetition{\PSet{cs}})}}$.

In any of these cases,
a match result is not guaranteed to exist.
In practice,
we would like to ensure that
a grammar yields a match result
for any input string.
This property is called \emph{completeness}.
\begin{equation}
    \text{$g$ is complete} \iff
    \forall s, \exists res, \Matches{g}{\PNT{0}}{s}{res}
\end{equation}

Ford~\cite{ford_parsing_2004} showed that
the problem of knowing whether a grammar
is complete is undecidable.
He then presented a conservative approximation
to completeness known as \emph{well-formedness},
which is decidable.

We formalize this well-formedness check
as a computable function \textit{wf}
that takes a grammar and returns a Boolean value
indicating whether the grammar is well-formed or not.
We also prove its correctness by showing that,
for any given grammar $g$,
if $\wf{g} = true$,
then $g$ is complete.

How \lpeg{} implements this function \textit{wf}
is the focus of the next chapter,
as well as its proof of correctness.
All definitions and proofs
are publicly available on this project's GitHub repository%
\footnote{\selfhref{https://github.com/guidanoli/peg-coq}}.
