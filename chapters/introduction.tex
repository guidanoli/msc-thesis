\chapter{Introduction}


% PEGs
% LPegs
% dentro de LPeg: well-formed (tá ótimo), first-set (descreve O ALGORITMO)
% adapta first de CFG para PEGs (ler sobre first, e da sua importância para CFG)

Parsing Expression Grammars (PEGs)
are a type of formal grammar
introduced by Ford~\cite{ford_parsing_2004}
to describe and parse machine-oriented languages.
In contrast to Context-Free Grammars (CFGs),
established by Chomsky~\cite{chomsky_three_1956},
PEGs are deterministic,
which makes them particularly suitable for parsing
programming languages and structured data formats.

PEGs have been implemented by several projects
and in a variety of programming languages.
One such implementation is
\lpeg{}~\cite{ierusalimschy_text_2009},
a library for the Lua programming language.
Following the SNOBOL tradition,
patterns in \lpeg{} are first-class citizens
and can be programmatically constructed in a bottom-up fashion.
Starting from basic primitives such as character classes,
patterns can be combined through the use of operators:
\texttt{*}~for~sequences,
\texttt{+}~for~ordered choices,
\texttt{\^}~(caret)~for~repetitions,
and so on.

\lpeg{}
features several interesting algorithms in its implementation,
but two algorithms stand out for their complexity
and lack of formal documentation:
The well-formedness check
and the first-set computation.

The first algorithm, the well-formedness check,
ensures that a pattern is complete,
meaning it yields a match result for any input string.
The concept of well-formedness
was introduced by Ford
as a conservative approximation to completeness,
after proving that the problem of detecting completeness is undecidable,
and noticing that incompleteness is caused by
left-recursive rules and degenerate loops.

The well-formedness algorithm proposed by Ford
iteratively constructs a set of well-formed expressions
until a fixed-point is reached.
A grammar is then deemed well-formed if its expression set
matches the set of well-formed expressions derived
from the fixed-point iteration.
Koprowski~et~al.~\cite{koprowski_trx_2011}
formalized a similar algorithm
for an extended definition of PEGs
and proved its correctness
using the Coq proof assistant.

Meanwhile,
the well-formedness check implemented in \lpeg{}
does not use iteration, fixed-point checks,
or data structures to represent sets of expressions.
These details make the algorithm simpler to implement,
specially in programming languages
with modest standard libraries, such~as~C.
Moreover,
this alternative algorithm
has neither been published nor formally verified yet,
which makes it an interesting research topic.

The second algorithm implemented in \lpeg{}
that we highlight in this work is
the first-set algorithm.
It takes a pattern
and returns the first-set and the emptiness value
of the pattern.
The definition of first-set
is well-established in the area of CFGs,
but their application in PEGs has not yet
been documented in the literature.
Basically speaking,
the first-set of a pattern
is the set of first characters
that can be accepted by the pattern.
More precisely,
a pattern fails any string
that starts with a character
that is not in its first-set.

The first-set algorithm also returns an emptiness value,
which, when false, indicates that the pattern
fails to match the empty string.
(This Boolean value corresponds to the inclusion of
$\varepsilon$ in the first-set of some formalizations for CFGs.)
This other return value is necessary because
the first-set cannot be used to determine whether
a pattern fails the empty string,
because it has no first character.
Together, both return values
help \lpeg{} optimize certain patterns,
such as ordered choices.

Both aforementioned algorithms implemented in \lpeg{}
are complex and lack formal documentation,
which can make them difficult to maintain and reason about.
This work aims to bridge this gap.
We present and analyze these algorithms,
proving their correctness using the Coq proof assistant%
\footnote{The code is publicly available on
GitHub at \selfhref{https://github.com/guidanoli/peg-coq}}.
Moreover, we also prove their termination through a gas-based approach.
Also, during the process of proving
key properties about these algorithms,
we have noticed a few issues
that could be reviewed
in future versions of \lpeg{}.

The remainder of this work is structured as follows:
\Cref{chapter:pegs} presents the syntax and semantics of PEGs.
In \Cref{chapter:wf-algorithm}
we formalize the well-formedness algorithm
and prove its correctness and termination.
\Cref{chapter:first-set}
does the same for the first-set algorithm.
In \Cref{chapter:related-work},
we analyze how our work differs from prior research,
highlighting key differences and improvements.
Finally, \Cref{chapter:conclusion}
summarizes our findings and outlines directions
for future research.