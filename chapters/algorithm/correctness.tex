\section{Correctness}

Having introduced each step of the well-formedness algorithm,
we can now present the definition of the \textit{\verifygrammarname{}} function,
which implements the algorithm step-by-step.
It starts by checking whether the grammar defines a first rule,
and whether every rule in the grammar is coherent.
It then makes sure the grammar contains no left-recurive rules,
and no degenerate loops, in this order.
\Cref{fig:verifygrammar-function} displays the function.
\begin{figure}
    \centering
    \begin{equation*}
    \begin{aligned}[t]
        & \verifygrammarcomp{g}{gas} = \\
        & \begin{aligned}[t]
            & \matchwith{\coherentfunc{g}{\PNT{0}}} \\
            & \matchcase{true}{\begin{aligned}[t]
                & \matchwith{\lcoherentfunc{g}{g}} \\
                & \matchcase{true}{\begin{aligned}[t]
                    & \matchwith{\lverifyrulecomp{g}{g}{gas}} \\
                    & \matchcase{\Some true}{\begin{aligned}[t]
                        & \matchwith{\lcheckloopscomp{g}{g}{gas}} \\
                        & \matchcase{\Some b}{\Some \neg b} \\
                        & \matchcase{\None}{\None} \\
                        & \matchend{}
                    \end{aligned}} \\
                    & \matchcase{res}{res} \\
                    & \matchend{}
                \end{aligned}} \\
                & \matchcase{false}{\Some false} \\
                & \matchend{}
            \end{aligned}} \\
            & \matchcase{false}{\Some false} \\
            & \matchend{}
        \end{aligned}
    \end{aligned}
\end{equation*}
    \caption{The well-formedness function with gas.}
    \label{fig:verifygrammar-function}
\end{figure}

Now, let us prove that the well-formedness check is correct.
To do so, we first define an
inductive predicate equivalent to
the \textit{\verifygrammarname{}} function
to help us in proofs by induction.
\Cref{fig:verifygrammar}
\begin{figure}
    \centering
    \begin{mathpar}
    \namedinferrule{vg1}
    {\Coherent{g}{\PNT{0}}{false}}
    {\VerifyGrammar{g}{false}}
    
    \namedinferrule{vg2}
    {\Coherent{g}{\PNT{0}}{true} \\ \lCoherent{g}{g}{false}}
    {\VerifyGrammar{g}{false}}

    \namedinferrule{vg3}
    {\Coherent{g}{\PNT{0}}{true} \\ \lCoherent{g}{g}{true} \\ \lVerifyRule{g}{g}{false}}
    {\VerifyGrammar{g}{false}}

    \namedinferrule{vg4}
    {\Coherent{g}{\PNT{0}}{true} \\ \lCoherent{g}{g}{true} \\ \lVerifyRule{g}{g}{true} \\ \lCheckLoops{g}{g}{true}}
    {\VerifyGrammar{g}{false}}

    \namedinferrule{vg5}
    {\Coherent{g}{\PNT{0}}{true} \\ \lCoherent{g}{g}{true} \\ \lVerifyRule{g}{g}{true} \\ \lCheckLoops{g}{g}{false}}
    {\VerifyGrammar{g}{true}}
\end{mathpar}
    \caption{The well-formedness predicate.}
    \label{fig:verifygrammar}
\end{figure}
presents this predicate,
which takes a grammar
and returns a Boolean value
indicating whether the grammar
passes all the checks.
It is based on the predicates
presented in the previous sections.
\Cref{lemma:verifygrammar-determinism}
states that the predicate is deterministic,
and \Cref{lemma:verifygrammar-follows}
states that it follows the fixed-point definition.

\begin{lemma}
    \label{lemma:verifygrammar-determinism}
    If $\VerifyGrammar{g}{b_1}$,
    and $\VerifyGrammar{g}{b_2}$,
    then $b_1 = b_2$.
\end{lemma}

\begin{lemma}
    \label{lemma:verifygrammar-follows}
    If $\verifygrammarcomp{g}{gas} = \Some b$,
    then $\VerifyGrammar{g}{b}$.
\end{lemma}

In order to prove correctness,
we need to generalize the pattern from $\PNT{0}$
to any pattern $p$,
and to break down the function \textit{wf}
into its separate steps.
We need to make this generalization
because the match predicate is defined
recursively on the current pattern.
The generalized theorem we need to prove is the following:
Given a grammar $g$ and a pattern $p$,
if $g$ only contains coherent rules
that do not lead to left recursion
and that do not have any degenerate loops,
and if $p$ is coherent
and does not have degenerate loops,
then ${\forall s, \exists res\ \Matches{g}{p}{s}{res}}$.

We begin the proof by doing a strong induction on $n$,
the length of the input string $s$,
which gives us the inductive hypothesis ``IHn''.
This hypothesis states that for any input string shorter than $s$
and any pattern, we can assume that the match yields some result.
From \Cref{lemma:no-lr-rule-in-grammar}
and the assumption that the grammar contains no left-recursive rules,
we can infer that the pattern also does not lead to left recursion.
This gives us the inductive predicate
$\VerifyRule{g}{p}{d}{nb}{\Some b}{k}$,
which tells us that $p$ is either nullable or non-nullable.
We do an induction on this predicate
and handle each case separately.

Let us start with the basic patterns.
The case of the empty pattern $\PEmpty{}$ is trivial,
as it matches any input string without consuming anything.
The case of the character set pattern $\PSet{cs}$ is also simple.
If $s$ is the empty string, the pattern fails to match.
Otherwise, the string may or may not begin with the character $a \in cs$.
If it does, then it matches while consuming $a$.
Otherwise, it fails to match.

The sequence pattern $\PSequence{p_1}{p_2}$ has two cases:
one in which $p_1$ is non-nullable and $p_2$ is not visited,
and another in which $p_1$ is nullable and $p_2$ is visited.
In both cases, we have an inductive hypothesis stating that
$p_1$ has a match result for input string $s$.
If this match result is a failure,
then the whole sequence $\PSequence{p_1}{p_2}$ also fails.
If the match result is a success,
then $p_1$ leaves a suffix string $s'$ unconsumed.
Let us handle this case for both scenarios separately.

If $p_1$ is non-nullable,
then we can use \Cref{lemma:non-nullable-pattern-proper-suffix}
to state that $s'$ is a proper suffix of $s$,
and therefore shorter than $s$.
This allows us to use IHn,
and state that $p_2$ has a match result for the input string $s'$.
This implies that the sequence has this same match result.

Otherwise, if $p_1$ is nullable, then we can use
\Cref{lemma:match-suffix} to state that
$s'$ is a suffix of $s$.
This means that $s'$ is either equal to $s$, or a proper suffix of $s$.
If $s$ is equal to $s'$,
then we can use the inductive hypothesis
to state that $p_2$ yields a match result for $s'=s$.
Otherwise, then we can use IHn in the same way as the previous case,
because $s'$ would be shorter than $s$.

Now let us consider the case of the choice pattern $\PChoice{p_1}{p_2}$.
Similar to the case of the sequence pattern,
we have induction hypotheses for $p_1$ and $p_2$
yielding a match result for the input string $s$.
This case is simpler because $s$ is the same input string for both choice options,
so these induction hypotheses are enough to prove this case.

The case of the repetition pattern $\PRepetition{p}$ is the most interesting one,
as we get to use the fact that $p$ must be non-nullable in the proof.
From the induction step, we have the inductive hypothesis that
$p$ yields a match result for the input string $s$.
If $p$ fails to match, then $\PRepetition{p}$ matches without consuming anything.
If, otherwise, $p$ matches, then it leaves a string $s'$ unconsumed.
Since $p$ is non-nullable, $s'$ must be a proper suffix of $s$.
Therefore, we can use IHn to state that $\PRepetition{p}$
yields a match result $res$ for $s'$, because it is shorter than $s$.
In this case, $\PRepetition{p}$ also yields $res$ for $s$.

The case of the predicates $\PNot{p}$ and $\PAnd{p}$ are pretty straightforward.
We first use the inductive hypothesis that $p$
yields a match result for the input string $s$.
If $p$ matches,
then $\PAnd{p}$ matches without consuming anything,
and $\PNot{p}$ fails to match.
Otherwise, if $p$ fails to match,
then so does $\PAnd{p}$,
and $\PNot{p}$ matches without consuming anything.

Finally, we prove the case of the non-terminal pattern $\PNT{i}$,
which is surprisingly simple.
From the induction step,
we are given $p$, the $i^{th}$ rule of the grammar.
From the initial hypotheses,
we know that $p$ is coherent and free of degenerate loops,
because it is a grammar rule.
We can then use the inductive hypothesis
from the $\verifyrulename{}$ predicate to
state that $p$ yields a match result for the input string $s$.
\begin{theorem}
    \label{theorem:wf-correctness-generalized}
    Given a grammar $g$ and a pattern $p$,
    if $g$ only contains coherent rules
    that do not lead to left recursion
    and that free of degenerate loops,
    and if $p$ is coherent
    and free of degenerate loops,
    then $\forall s, \exists res\ \Matches{g}{p}{s}{res}$.
\end{theorem}
Having proved \Cref{theorem:wf-correctness-generalized},
we can finally prove the original theorem,
which states that, for any grammar $g$
that satisfies $\wf{g} = true$,
and for any input string,
the non-terminal pattern $\PNT{0}$
yields a match result.

\Cref{lemma:verifygrammar-termination} states that
the function \textit{\verifygrammarname{}} returns
$\Some b$ when given a gas counter greater or equal
to $\verifygrammargas{g}$.
In the implementation of the function \textit{wf},
we pass $\verifygrammargas{g}$ as the gas counter
for the function \textit{\verifygrammarname{}}.
Therefore, if \textit{wf} returns $true$,
then it must be because \textit{\verifygrammarname{}}
returned $\Some true$.

If the function \textit{\verifygrammarname{}}
returns $\Some true$,
then,
according to \Cref{lemma:verifygrammar-follows},
we can construct the predicate $\VerifyGrammar{g}{true}$.
We can see that this only happens when
the input grammar has passed
all the checks.
From \Cref{fig:verifygrammar},
we can see that this implies in
several other predicates,
many of which are necessary to use
\Cref{theorem:wf-correctness-generalized}.
There are only two missing predicates:
$\Coherent{g}{\PNT{0}}{true}$,
which states that the initial pattern is coherent,
and $\CheckLoops{g}{\PNT{0}}{d}{false}$,
which states that it does not contain any degenerate loops.

We can derive both predicates
from the fact that $\PNT{0}$ is a rule,
which we have checked already.
We use \Cref{lemma:lcoherent-safety}
to prove that $\PNT{0}$ is coherent,
and \Cref{lemma:lcheckloops-safety}
to prove that it contains no degenerate loops.
With this, we are able to prove the original theorem.

\begin{theorem}
    For any grammar $g$,
    if $\wf{g} = true$,
    then $g$ is complete.
\end{theorem}
