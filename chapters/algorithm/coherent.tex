\section{References to nonexistent rules}
\label{section:coherent}

The verification process
starts by checking whether
every nonterminal pattern references
an existing rule in the grammar.
This process is quite simple,
but we present it here in the name of completeness.

We say a pattern is
\emph{coherent} in respect to a grammar
if all of its nonterminals reference existing rules in the grammar.
\Cref{fig:coherentfunc}
\begin{figure}
    \centering
    \begin{align*}
    \begin{aligned}[t]
        & \coherentfunc{g}{p} = \\
        & \begin{aligned}[t]
            & \matchwith{p} \\
            & \matchcase{\PEmpty}{true} \\
            & \matchcase{\PSet{cs}}{true} \\
            & \matchcase{\PNT{i}}{\begin{aligned}[t]
                & \matchwith{g[i]} \\
                & \matchcase{\Some p}{true} \\
                & \matchcase{\None}{false} \\
                & \matchend{}
            \end{aligned}} \\
            & \matchcase{\PRepetition{p}}{\coherentfunc{g}{p}} \\
            & \matchcase{\PNot{p}}{\coherentfunc{g}{p}} \\
            & \matchcase{\PAnd{p}}{\coherentfunc{g}{p}} \\
            & \matchcase{\PSequence{p_1}{p_2}}
                {\coherentfunc{g}{p_1} \wedge \coherentfunc{g}{p_2}} \\
            & \matchcase{\PChoice{p_1}{p_2}}
                {\coherentfunc{g}{p_1} \wedge \coherentfunc{g}{p_2}} \\
            & \matchend{}
        \end{aligned}
    \end{aligned}
\end{align*}
    \caption{The coherence function.}
    \label{fig:coherentfunc}
\end{figure}
defines a fixed-point function
that performs this verification.
To aid us in later induction proofs,
we also define an equivalent predicate in \Cref{fig:coherent}.
\begin{figure}
    \section{References to nonexistent rules}
\label{section:coherent}

The verification process
starts by checking whether
every nonterminal pattern references
an existing rule in the grammar.
This process is quite simple,
but we present it here in the name of completeness.

We say a pattern is
\emph{coherent} in respect to a grammar
if all of its nonterminals reference existing rules in the grammar.
\Cref{fig:coherentfunc}
\begin{figure}
    \centering
    \begin{align*}
    \begin{aligned}[t]
        & \coherentfunc{g}{p} = \\
        & \begin{aligned}[t]
            & \matchwith{p} \\
            & \matchcase{\PEmpty}{true} \\
            & \matchcase{\PSet{cs}}{true} \\
            & \matchcase{\PNT{i}}{\begin{aligned}[t]
                & \matchwith{g[i]} \\
                & \matchcase{\Some p}{true} \\
                & \matchcase{\None}{false} \\
                & \matchend{}
            \end{aligned}} \\
            & \matchcase{\PRepetition{p}}{\coherentfunc{g}{p}} \\
            & \matchcase{\PNot{p}}{\coherentfunc{g}{p}} \\
            & \matchcase{\PAnd{p}}{\coherentfunc{g}{p}} \\
            & \matchcase{\PSequence{p_1}{p_2}}
                {\coherentfunc{g}{p_1} \wedge \coherentfunc{g}{p_2}} \\
            & \matchcase{\PChoice{p_1}{p_2}}
                {\coherentfunc{g}{p_1} \wedge \coherentfunc{g}{p_2}} \\
            & \matchend{}
        \end{aligned}
    \end{aligned}
\end{align*}
    \caption{The coherence function.}
    \label{fig:coherentfunc}
\end{figure}
defines a fixed-point function
that performs this verification.
To aid us in later induction proofs,
we also define an equivalent predicate in \Cref{fig:coherent}.
\begin{figure}
    \section{References to nonexistent rules}
\label{section:coherent}

The verification process
starts by checking whether
every nonterminal pattern references
an existing rule in the grammar.
This process is quite simple,
but we present it here in the name of completeness.

We say a pattern is
\emph{coherent} in respect to a grammar
if all of its nonterminals reference existing rules in the grammar.
\Cref{fig:coherentfunc}
\begin{figure}
    \centering
    \begin{align*}
    \begin{aligned}[t]
        & \coherentfunc{g}{p} = \\
        & \begin{aligned}[t]
            & \matchwith{p} \\
            & \matchcase{\PEmpty}{true} \\
            & \matchcase{\PSet{cs}}{true} \\
            & \matchcase{\PNT{i}}{\begin{aligned}[t]
                & \matchwith{g[i]} \\
                & \matchcase{\Some p}{true} \\
                & \matchcase{\None}{false} \\
                & \matchend{}
            \end{aligned}} \\
            & \matchcase{\PRepetition{p}}{\coherentfunc{g}{p}} \\
            & \matchcase{\PNot{p}}{\coherentfunc{g}{p}} \\
            & \matchcase{\PAnd{p}}{\coherentfunc{g}{p}} \\
            & \matchcase{\PSequence{p_1}{p_2}}
                {\coherentfunc{g}{p_1} \wedge \coherentfunc{g}{p_2}} \\
            & \matchcase{\PChoice{p_1}{p_2}}
                {\coherentfunc{g}{p_1} \wedge \coherentfunc{g}{p_2}} \\
            & \matchend{}
        \end{aligned}
    \end{aligned}
\end{align*}
    \caption{The coherence function.}
    \label{fig:coherentfunc}
\end{figure}
defines a fixed-point function
that performs this verification.
To aid us in later induction proofs,
we also define an equivalent predicate in \Cref{fig:coherent}.
\begin{figure}
    \section{References to nonexistent rules}
\label{section:coherent}

The verification process
starts by checking whether
every nonterminal pattern references
an existing rule in the grammar.
This process is quite simple,
but we present it here in the name of completeness.

We say a pattern is
\emph{coherent} in respect to a grammar
if all of its nonterminals reference existing rules in the grammar.
\Cref{fig:coherentfunc}
\begin{figure}
    \centering
    \input{coherentfunc}
    \caption{The coherence function.}
    \label{fig:coherentfunc}
\end{figure}
defines a fixed-point function
that performs this verification.
To aid us in later induction proofs,
we also define an equivalent predicate in \Cref{fig:coherent}.
\begin{figure}
    \input{coherent}
    \caption{The coherence predicate.}
    \label{fig:coherent}
\end{figure}
\Cref{lemma:coherent-deterministic} states
that the predicate is deterministic on the result,
and \Cref{lemma:coherent-follows} states that
the predicate follows the function.
It is easy to see that
both lemmas together imply that the
predicate is equivalent to the function.

% TODO: reduce these to just one lemma as A <-> B ?

\begin{lemma}
    \label{lemma:coherent-deterministic}
    If $\Coherent{g}{p}{res_1}$ and $\Coherent{g}{p}{res_2}$,
    then $res_1 = res_2$.
\end{lemma}

\begin{lemma}
    \label{lemma:coherent-follows}
    If $\coherentfunc{g}{p} = res$, then $\Coherent{g}{p}{res}$.
\end{lemma}

\Cref{fig:lcoherent-function} trivially generalizes the coherence check for a list of patterns.
\begin{figure}
    \centering
    \include{lcoherentfunc}
    \caption{The coherence function for lists of patterns.}
    \label{fig:lcoherent-function}
\end{figure}
This function is defined over an arbitrary list of rules,
but is meant to be called for the whole grammar.
We also define, in \Cref{fig:lcoherent}, an inductive predicate
\begin{figure}
    \centering
    \input{lcoherent}
    \caption{The coherence predicate for lists of patterns.}
    \label{fig:lcoherent}
\end{figure}
equivalent to this function to be
later used in proofs by induction.
We also show that this predicate
is deterministic and follows
the original fixed-point definition.
See \Cref{lemma:lcoherent-determinism,lemma:lcoherent-follows}.

\begin{lemma}
    If $\lCoherent{g}{rs}{res_1}$,
    and $\lCoherent{g}{rs}{res_2}$,
    then $res_1 = res_2$.
    \label{lemma:lcoherent-determinism}
\end{lemma}

\begin{lemma}
    If $\lcoherentfunc{g}{rs} = res$,
    then $\lCoherent{g}{rs}{res}$.
    \label{lemma:lcoherent-follows}
\end{lemma}

Finally, we prove that
if a list of patterns passes the list-based check,
then any pattern in the list passes the individual check.

\begin{lemma}%[Coherent List Safety]
    \label{lemma:lcoherent-safety}
    If $\lCoherent{g}{rs}{true}$,
    then, $\forall r \in rs, \Coherent{g}{r}{true}$.
\end{lemma}

    \caption{The coherence predicate.}
    \label{fig:coherent}
\end{figure}
\Cref{lemma:coherent-deterministic} states
that the predicate is deterministic on the result,
and \Cref{lemma:coherent-follows} states that
the predicate follows the function.
It is easy to see that
both lemmas together imply that the
predicate is equivalent to the function.

% TODO: reduce these to just one lemma as A <-> B ?

\begin{lemma}
    \label{lemma:coherent-deterministic}
    If $\Coherent{g}{p}{res_1}$ and $\Coherent{g}{p}{res_2}$,
    then $res_1 = res_2$.
\end{lemma}

\begin{lemma}
    \label{lemma:coherent-follows}
    If $\coherentfunc{g}{p} = res$, then $\Coherent{g}{p}{res}$.
\end{lemma}

\Cref{fig:lcoherent-function} trivially generalizes the coherence check for a list of patterns.
\begin{figure}
    \centering
    \begin{equation*}
    \lcoherentfunc{g}{rs} = \begin{aligned}[t]
        & \matchwith{rs} \\
        & \matchcase{nil}{true} \\
        & \matchcase{r::rs'}{\coherentfunc{g}{r} \wedge \lcoherentfunc{g}{rs'}} \\
        & \matchend{}
    \end{aligned}
\end{equation*}
    \caption{The coherence function for lists of patterns.}
    \label{fig:lcoherent-function}
\end{figure}
This function is defined over an arbitrary list of rules,
but is meant to be called for the whole grammar.
We also define, in \Cref{fig:lcoherent}, an inductive predicate
\begin{figure}
    \centering
    \begin{mathpar}
    \namedinferrule{lc-nil}
    { }
    {\lCoherent{g}{nil}{true}}

    \namedinferrule{lc-cons}
    {\Coherent{g}{r}{b_1} \\ \lCoherent{g}{rs}{b_2}}
    {\lCoherent{g}{r::rs}{b_1 \wedge b_2}}
\end{mathpar}
    \caption{The coherence predicate for lists of patterns.}
    \label{fig:lcoherent}
\end{figure}
equivalent to this function to be
later used in proofs by induction.
We also show that this predicate
is deterministic and follows
the original fixed-point definition.
See \Cref{lemma:lcoherent-determinism,lemma:lcoherent-follows}.

\begin{lemma}
    If $\lCoherent{g}{rs}{res_1}$,
    and $\lCoherent{g}{rs}{res_2}$,
    then $res_1 = res_2$.
    \label{lemma:lcoherent-determinism}
\end{lemma}

\begin{lemma}
    If $\lcoherentfunc{g}{rs} = res$,
    then $\lCoherent{g}{rs}{res}$.
    \label{lemma:lcoherent-follows}
\end{lemma}

Finally, we prove that
if a list of patterns passes the list-based check,
then any pattern in the list passes the individual check.

\begin{lemma}%[Coherent List Safety]
    \label{lemma:lcoherent-safety}
    If $\lCoherent{g}{rs}{true}$,
    then, $\forall r \in rs, \Coherent{g}{r}{true}$.
\end{lemma}

    \caption{The coherence predicate.}
    \label{fig:coherent}
\end{figure}
\Cref{lemma:coherent-deterministic} states
that the predicate is deterministic on the result,
and \Cref{lemma:coherent-follows} states that
the predicate follows the function.
It is easy to see that
both lemmas together imply that the
predicate is equivalent to the function.

% TODO: reduce these to just one lemma as A <-> B ?

\begin{lemma}
    \label{lemma:coherent-deterministic}
    If $\Coherent{g}{p}{res_1}$ and $\Coherent{g}{p}{res_2}$,
    then $res_1 = res_2$.
\end{lemma}

\begin{lemma}
    \label{lemma:coherent-follows}
    If $\coherentfunc{g}{p} = res$, then $\Coherent{g}{p}{res}$.
\end{lemma}

\Cref{fig:lcoherent-function} trivially generalizes the coherence check for a list of patterns.
\begin{figure}
    \centering
    \begin{equation*}
    \lcoherentfunc{g}{rs} = \begin{aligned}[t]
        & \matchwith{rs} \\
        & \matchcase{nil}{true} \\
        & \matchcase{r::rs'}{\coherentfunc{g}{r} \wedge \lcoherentfunc{g}{rs'}} \\
        & \matchend{}
    \end{aligned}
\end{equation*}
    \caption{The coherence function for lists of patterns.}
    \label{fig:lcoherent-function}
\end{figure}
This function is defined over an arbitrary list of rules,
but is meant to be called for the whole grammar.
We also define, in \Cref{fig:lcoherent}, an inductive predicate
\begin{figure}
    \centering
    \begin{mathpar}
    \namedinferrule{lc-nil}
    { }
    {\lCoherent{g}{nil}{true}}

    \namedinferrule{lc-cons}
    {\Coherent{g}{r}{b_1} \\ \lCoherent{g}{rs}{b_2}}
    {\lCoherent{g}{r::rs}{b_1 \wedge b_2}}
\end{mathpar}
    \caption{The coherence predicate for lists of patterns.}
    \label{fig:lcoherent}
\end{figure}
equivalent to this function to be
later used in proofs by induction.
We also show that this predicate
is deterministic and follows
the original fixed-point definition.
See \Cref{lemma:lcoherent-determinism,lemma:lcoherent-follows}.

\begin{lemma}
    If $\lCoherent{g}{rs}{res_1}$,
    and $\lCoherent{g}{rs}{res_2}$,
    then $res_1 = res_2$.
    \label{lemma:lcoherent-determinism}
\end{lemma}

\begin{lemma}
    If $\lcoherentfunc{g}{rs} = res$,
    then $\lCoherent{g}{rs}{res}$.
    \label{lemma:lcoherent-follows}
\end{lemma}

Finally, we prove that
if a list of patterns passes the list-based check,
then any pattern in the list passes the individual check.

\begin{lemma}%[Coherent List Safety]
    \label{lemma:lcoherent-safety}
    If $\lCoherent{g}{rs}{true}$,
    then, $\forall r \in rs, \Coherent{g}{r}{true}$.
\end{lemma}

    \caption{The coherence predicate.}
    \label{fig:coherent}
\end{figure}
\Cref{lemma:coherent-deterministic} states
that the predicate is deterministic on the result,
and \Cref{lemma:coherent-follows} states that
the predicate follows the function.
It is easy to see that
both lemmas together imply that the
predicate is equivalent to the function.

% TODO: reduce these to just one lemma as A <-> B ?

\begin{lemma}
    \label{lemma:coherent-deterministic}
    If $\Coherent{g}{p}{res_1}$ and $\Coherent{g}{p}{res_2}$,
    then $res_1 = res_2$.
\end{lemma}

\begin{lemma}
    \label{lemma:coherent-follows}
    If $\coherentfunc{g}{p} = res$, then $\Coherent{g}{p}{res}$.
\end{lemma}

\Cref{fig:lcoherent-function} trivially generalizes the coherence check for a list of patterns.
\begin{figure}
    \centering
    \begin{equation*}
    \lcoherentfunc{g}{rs} = \begin{aligned}[t]
        & \matchwith{rs} \\
        & \matchcase{nil}{true} \\
        & \matchcase{r::rs'}{\coherentfunc{g}{r} \wedge \lcoherentfunc{g}{rs'}} \\
        & \matchend{}
    \end{aligned}
\end{equation*}
    \caption{The coherence function for lists of patterns.}
    \label{fig:lcoherent-function}
\end{figure}
This function is defined over an arbitrary list of rules,
but is meant to be called for the whole grammar.
We also define, in \Cref{fig:lcoherent}, an inductive predicate
\begin{figure}
    \centering
    \begin{mathpar}
    \namedinferrule{lc-nil}
    { }
    {\lCoherent{g}{nil}{true}}

    \namedinferrule{lc-cons}
    {\Coherent{g}{r}{b_1} \\ \lCoherent{g}{rs}{b_2}}
    {\lCoherent{g}{r::rs}{b_1 \wedge b_2}}
\end{mathpar}
    \caption{The coherence predicate for lists of patterns.}
    \label{fig:lcoherent}
\end{figure}
equivalent to this function to be
later used in proofs by induction.
We also show that this predicate
is deterministic and follows
the original fixed-point definition.
See \Cref{lemma:lcoherent-determinism,lemma:lcoherent-follows}.

\begin{lemma}
    If $\lCoherent{g}{rs}{res_1}$,
    and $\lCoherent{g}{rs}{res_2}$,
    then $res_1 = res_2$.
    \label{lemma:lcoherent-determinism}
\end{lemma}

\begin{lemma}
    If $\lcoherentfunc{g}{rs} = res$,
    then $\lCoherent{g}{rs}{res}$.
    \label{lemma:lcoherent-follows}
\end{lemma}

Finally, we prove that
if a list of patterns passes the list-based check,
then any pattern in the list passes the individual check.

\begin{lemma}%[Coherent List Safety]
    \label{lemma:lcoherent-safety}
    If $\lCoherent{g}{rs}{true}$,
    then, $\forall r \in rs, \Coherent{g}{r}{true}$.
\end{lemma}
