\section{Left-recursive rules}
\label{section:lr-rules}

In general,
we consider a rule to be left-recursive
if it can wind up in itself
without consuming any input in-between.
This brings us to the heart
of the algorithm that
detects left-recursive rules.
On a high level,
it symbolically parses each rule,
until it either consumes some input,
visits some rule twice without consuming any input,
or simply finishes.

The algorithm categorizes patterns into three groups.
If a pattern can be parsed until its end without consuming any input,
it is said to be \emph{nullable}.
If, otherwise, it always consumes some input,
it is categorized as \emph{non-nullable}.
Alternatively, if it can lead to some rule twice,
without consuming any input,
it is categorized as left-recursive.

In order to check whether a pattern
is guaranteed to consume some input,
the algorithm uses a conservative approximation
proposed by Ford~\cite{ford_parsing_2004},
which makes two assumptions.
The first one is that $\PNot{p}$ may match,
and the second one is that,
in the case of $\PChoice{p_1}{p_2}$,
it may visit $p_2$,
without checking whether $p_1$ always matches.
For illustrative purposes,
we present a simple counterexample
for each assumption.

A counterexample for the first assumption
is the pattern $\PNot{\PEmpty}$,
which never matches.
Meanwhile, the second assumption doesn't hold
for the pattern $\PChoice{\PEmpty}{p_2}$,
because $\PEmpty$ always matches,
and, therefore, $p_2$ is never visited.
The reader might think that
these cases can be easily spotted,
by the simplicity of the counterexamples.
However,
Ford~\cite{ford_parsing_2004} proved that
the general case of this problem is undecidable.

One of the conditions for the algorithm to yield a result
is whenever it consumes some input.
As a result,
it exclusively visits patterns that may be reached
without consuming any input.
Therefore, if the algorithm revisits a rule,
this means a path exists in which
the parsing routine may reach the same rule
and with the same input string,
which would indicate that such rule is left-recursive.
We will now discuss possible ways to detect
when a rule has been visited twice.

One possible way to detect left-recursive rules
is through a set of visited rules,
which is checked and updated
every time a nonterminal pattern is visited.
Constant time complexity
can be achieved
by using data structures
such as hash tables.

Another approach,
which doesn't involve complex data structures,
uses a counter of visited rules,
which starts at zero
and gets incremented every time a rule is visited.
If this value ever surpasses the number of grammar rules,
then we know, by the pigeonhole principle,
that some rule has been visited more than once.
In the case of grammars with left-recursive rules,
we may visit more rules than necessary,
however, we are not particularly worried
about the performance of the algorithm
in the case of errors.

\lpeg{} adopts this last approach.
Our formalization follows \lpeg{},
though with a small twist:
instead of counting visited rules from zero until the limit,
we count to-be-visited rules from the limit down to zero.
This simplifies our formalization
by moving the limit calculation out of the algorithm body,
and letting the limit be passed down as a parameter instead.

At this point,
it is important to draw a distinction between
exhausting the counter of to-be-visited rules
and correctly identifying a left-recursive rule.
When the algorithm starts,
the counter is initialized with the provided limit.
It is then decremented every time a rule is visited.
If the counter ever reaches zero,
then attempting to visit any rule
will return an error.
The algorithm does not determine
whether this error indicates left recursion,
because it would require the algorithm to check
whether the limit is greater than the number of grammar rules.
Instead, we leave it to the caller
to provide a high enough limit,
in which case the algorithm indeed
correctly labels rules as left-recursive
by returning an error.

Because of this shift in responsibilities,
we adapt the nomenclature
for the counter parameter and associated error,
based on an analogy with call stacks.
If, every time a rule is visited, it were pushed onto a stack $k$,
then we could think of the counter parameter $d$ as the stack depth limit;
and surpassing it would be similar to a stack overflow error.

For the sole purpose of helping us prove certain properties about the algorithm,
we will also include this stack $k$, a list of rule indices, as an output,
though it doesn't affect the algorithm.
As we will soon see, it is either appended, passed along, or ignored.
It works as a trace of the inner workings of the function,
a high-level concept we only use for proving lemmas about this algorithm.
\lpeg{} also implements this output, but it is only used in error messages.

We now describe the algorithm,
starting with its inputs and outputs.
It receives a pattern, a grammar, and a stack depth limit,
and returns a label and a stack.
We represent labels by
optional Boolean values
$\Some true$ (nullable),
$\Some false$ (non-nullable) and
$\None$ (stack overflow error);
and stacks by either
$nil$ (an empty stack) or
$i :: k$ (a rule of index $i$ concatenated with a stack $k$).

The function is defined recursively.
In most cases, it calls itself for each sub-pattern.
In the case of nonterminal patterns, however,
it calls itself for the referenced rule.
Furthermore, the function propagates any stack overflow errors.
This means that,
if some recursive call returns $\None$,
signaling a stack overflow,
and a stack $k$,
then the function also returns $\None$ and $k$.

For the empty pattern $\PEmpty$,
the function returns a label $\Some true$ and $nil$,
because it is nullable and doesn't visit any nonterminal.
We categorize it as nullable because it may match while consuming no input.
In particular, it always matches while consuming no input.

As for character set patterns $\PSet{cs}$,
the function returns $\Some false$ and $nil$,
because it is non-nullable,
meaning it always consumes some input when it matches.
It also doesn't visit any nonterminal.

For a nonterminal pattern $\PNT{i}$,
the function first checks the stack depth limit $d$.
If $d=0$, it returns $\None$ and $nil$,
signaling a stack overflow
and that it didn't visit any nonterminal.
Otherwise, if $d\ge1$,
then the function calls itself for the $i^{th}$ rule of the grammar,
while passing a stack depth limit of $d-1$.
If this recursive call returns a label $res$ and a stack $k$,
then the function returns $res$ and $i :: k$.
This way, the stack accumulates
the indices of the grammar rules
in the same order in which they are visited.

For a repetition pattern $\PRepetition{p}$,
the function evaluates $p$,
which returns $res$ and $k$,
to check for any stack overflow errors.
If $res\ne\None$,
then it returns $\Some true$ and $k$,
as it can match while consuming no input,
in case $p$ fails.
We assume that $p$ can fail because,
in the final verification step,
we ensure that $p$ is non-nullable,
and we know that non-nullable patterns
fail to match the empty string.

Predicate patterns $\PNot{p}$ and $\PAnd{p}$ are evaluated
in the same way as repetition patterns,
but for different reasons.
Repetition patterns are nullable
because they can always match without consuming any input.
Meanwhile, predicate patterns are nullable by approximation,
under the assumption that $p$ may match,
in the case of $\PAnd{p}$,
or fail to match,
in the case of $\PNot{p}$.

For a sequence pattern $\PSequence{p_1}{p_2}$,
the function first evaluates $p_1$.
If $p_1$ is non-nullable,
then so is the sequence $\PSequence{p_1}{p_2}$,
and the function returns the same label and stack as $p_1$.
Note that $p_2$ is not even evaluated in this case,
because it would be visited with a shorter input string
during parsing.
This is the only case in which
the nullable property comes into play
in this algorithm.
If, otherwise, $p_1$ is nullable,
then it evaluates $p_2$
and returns the same label and stack as $p_2$.

Finally, for a choice pattern $\PChoice{p_1}{p_2}$,
it first evaluates $p_1$.
If it returns $\Some b_1$,
indicating that $p_1$ did not overflow the stack,
then it evaluates $p_2$.
If it also returns $\Some b_2$,
then the function returns $\Some (b_1 \vee b_2)$
and the same stack as $p_2$.

The algorithm we've just described
is quite similar to the one implemented in \lpeg{}.
There is, however, one small difference
related to the use of tail calls
as an optimization technique.
In C, tail calls are implemented
with \texttt{goto} statements.
To apply this optimization technique,
\lpeg{} adds an extra parameter to the function
to work as an accumulator for the nullable property.
Without this accumulator parameter,
the evaluation of choice patterns $\PChoice{p_1}{p_2}$
would rely solely on recursion.
It would evaluate $p_1$ and $p_2$,
then perform a Boolean \scor{} operation on the results.

With the addition of a Boolean parameter $nb$,
we can turn the evaluation of $p_2$ into a tail call.
Instead of making the Boolean \scor{} operation explicitly,
we let the accumulator do it under-the-hood.
This works because,
in the base cases of the recursion,
in which the function would return
either $\Some true$ or $\Some false$,
we return instead $\Some (true \vee nb)$ and $\Some (false \vee nb)$,
which get simplified to $\Some true$ and $\Some nb$, respectively.
\Cref{fig:evalchoice} shows a Coq-like pseudocode of how choice patterns
are evaluated with the Boolean parameter $nb$.

\begin{figure}
    \centering
    \newcommand{\eval}[4]{eval\ #1\ #2\ #3\ #4}

\begin{align*}
    \eval{g}{\dsqb{\PChoice{p_1}{p_2}}}{d}{nb} =\
    & \matchwith{\eval{g}{p_1}{d}{nb}} \\
    & \matchcase{(\Some nb', k)}{\eval{g}{p_2}{d}{nb'}} \\
    & \matchcase{(\None, k)}{(\None, k)} \\
    & \matchend{}
\end{align*}
    \caption{Pseudocode of the evaluation of choice patterns.}
    \label{fig:evalchoice}
\end{figure}

We would also like to highlight
how this nullable accumulator allows
the evaluation of repetitions and predicates
to be rewritten as tail calls.
Previously,
we would have to check if $p$ evaluated to $\None$,
before returning $\Some true$.
Now, we can simply pass $true$ as the nullable accumulator,
which guarantees that, if $p$ does not evaluate to $\None$,
it evaluates to $\Some true$.

There are some ways in which this function
could be implemented in Coq as a fixed-point.
The classical way is to add a gas parameter,
which gets decremented in every recursive call.
We make the function return an optional value,
such that, if the gas parameter ever reaches zero,
it returns $\None$.
Other ways are providing a well-formedness proof,
or a measure function.
We choose the first strategy,
because it is the simplest to implement.

Finally,
\Cref{fig:vr-function} presents the algorithm
defined as a fixed-point function.
It returns an optional value,
because we adopted the gas strategy,
but also because it cannot evaluate
nonterminal patterns that
reference nonexistent rules.
In this case,
the function also returns $\None$.
In all other cases,
the function returns $\Some (res, k)$,
with $res$ being a label, and $k$, a stack.
\begin{figure}
    \centering
    \begin{align*}
    \begin{aligned}[t]
        & \verifyrulecomp{g}{p}{d}{nb}{0} = \None \\
        & \verifyrulecomp{g}{p}{d}{nb}{(1+gas)} = \\
        & \begin{aligned}[t]
            & \matchwith{p} \\
            & \matchcase{\PEmpty}{\Some(\Some true, nil)} \\
            & \matchcase{\PSet{cs}}{\Some(\Some nb, nil)} \\
            & \matchcase{\PRepetition{p'}}{\verifyrulecomp{g}{p'}{d}{true}{gas}} \\
            & \matchcase{\PNot{p'}}{\verifyrulecomp{g}{p'}{d}{true}{gas}} \\
            & \matchcase{\PAnd{p'}}{\verifyrulecomp{g}{p'}{d}{true}{gas}} \\
            & \matchcase{\PNT{i}}{\begin{aligned}[t]
                & \matchwith{g[i]} \\
                & \matchcase{\None}{\None} \\
                & \matchcase{\Some p'}{\begin{aligned}[t]
                    & \matchwith{d} \\
                    & \matchcase{0}{\Some(\None, nil)} \\
                    & \matchcase{1+d'}{\begin{aligned}[t]
                        & \matchwith{\verifyrulecomp{g}{p'}{d'}{nb}{gas}} \\
                        & \matchcase{\Some (res, k)}{\Some (res, i :: k)} \\
                        & \matchcase{\None}{\None} \\
                        & \matchend{}
                    \end{aligned}} \\
                    & \matchend{}
                \end{aligned}} \\
                & \matchend{}
            \end{aligned}} \\
            & \matchcase{\PSequence{p_1}{p_2}}{\begin{aligned}[t]
                & \matchwith{\verifyrulecomp{g}{p_1}{d}{false}{gas}} \\
                & \matchcase{\Some(\Some true, k)}{\verifyrulecomp{g}{p_2}{d}{nb}{gas}} \\
                & \matchcase{\Some(\Some false, k)}{\Some(\Some nb, k)} \\
                & \matchcase{res}{res} \\
                & \matchend{}
            \end{aligned}} \\
            & \matchcase{\PChoice{p_1}{p_2}}{\begin{aligned}[t]
                & \matchwith{\verifyrulecomp{g}{p_1}{d}{nb}{gas}} \\
                & \matchcase{\Some(\Some nb', k)}{\verifyrulecomp{g}{p_2}{d}{nb'}{gas}} \\
                & \matchcase{res}{res} \\
                & \matchend{}
            \end{aligned}} \\
            & \matchend{}
        \end{aligned}
    \end{aligned}
\end{align*}
    \caption{The left recursion detection function.}
    \label{fig:vr-function}
\end{figure}

At this point,
the reader should be warned that
we will not attempt to prove the correctness
of this function in isolation.
In fact, we will not even try to formally
define left-recursive rules.
This might frustrate the reader,
but we assure you that such proof will not be necessary.
Instead,
we will later prove the correctness of the whole algorithm
once we introduce all steps of the verification process.
In this section,
we will simply prove
that the label returned by the function
converges with respect to the gas counter and stack depth limit.

About this fixed-point definition,
we will initially prove some basic lemmas.
Starting with \Cref{lemma:vr-gas-convergence},
we state that, if the function returns $\Some (res, k)$,
then increasing the value of the gas parameter
will not change the result.
This is what we mean by
the function converging
with respect to the gas counter.

\begin{lemma}%[Verify Rule Gas Convergence]
    If $\verifyrulecomp{g}{p}{d}{nb}{gas} = \Some (res, k)$, \\
    then $\forall gas' \ge gas$,
    $\verifyrulecomp{g}{p}{d}{nb}{gas'} = \Some (res, k)$.
    \label{lemma:vr-gas-convergence}
\end{lemma}

\Cref{lemma:vr-termination} states that,
for any coherent pattern and grammar,
there exists a lower bound for the gas parameter,
for which the function returns $\Some (res, k)$.
The lower bound
takes into account
the size of the pattern,
the size of the grammar,
and the stack depth limit.

\begin{lemma}%[Verify Rule Termination]
    If $\Coherent{g}{p}{true}$,
    and $\lCoherent{g}{g}{true}$, \\
    then, $\forall gas \ge \size{p} + d \cdot \size{g}$,
    $\exists res\ \exists k\ \verifyrulecomp{g}{p}{d}{nb}{gas} = \Some (res, k)$.
    \label{lemma:vr-termination}
\end{lemma}

Now, we would like to
prove that the function
converges with respect to the
stack depth limit,
although in a somewhat unconventional
sense of the word.
That is because,
in the case of left-recursive rules,
the stack returned by the function will,
in fact,
diverge.
However, we are not interested in
the output stack, in this case.
What really matters to the following
steps of the verification process
is that the label returned by the algorithm converges
with respect to the stack depth limit.

In order to prove such lemma,
we realized an inductive, gasless predicate
would be better suited
than the fixed-point definition,
as it would be easier to perform proofs by induction,
and without having to deal with a gas parameter.
\Cref{fig:verifyrule}
defines such predicate,
denoted as $\VerifyRule{g}{p}{d}{nb}{res}{k}$.
It takes a grammar $g$,
a pattern $p$,
a stack depth limit $d$,
and a nullable accumulator $nb$,
and outputs a result $res$,
and a stack trace $k$.

\begin{figure}
    \section{Left-recursive rules}
\label{section:lr-rules}

In general,
we consider a rule to be left-recursive
if it can wind up in itself
without consuming any input in-between.
This brings us to the heart
of the algorithm that
detects left-recursive rules.
On a high level,
it symbolically parses each rule,
until it either consumes some input,
visits some rule twice without consuming any input,
or simply finishes.

The algorithm categorizes patterns into three groups.
If a pattern can be parsed until its end without consuming any input,
it is said to be \emph{nullable}.
If, otherwise, it always consumes some input,
it is categorized as \emph{non-nullable}.
Alternatively, if it can lead to some rule twice,
without consuming any input,
it is categorized as left-recursive.

In order to check whether a pattern
is guaranteed to consume some input,
the algorithm uses a conservative approximation
proposed by Ford~\cite{ford_parsing_2004},
which makes two assumptions.
The first one is that $\PNot{p}$ may match,
and the second one is that,
in the case of $\PChoice{p_1}{p_2}$,
it may visit $p_2$,
without checking whether $p_1$ always matches.
For illustrative purposes,
we present a simple counterexample
for each assumption.

A counterexample for the first assumption
is the pattern $\PNot{\PEmpty}$,
which never matches.
Meanwhile, the second assumption doesn't hold
for the pattern $\PChoice{\PEmpty}{p_2}$,
because $\PEmpty$ always matches,
and, therefore, $p_2$ is never visited.
The reader might think that
these cases can be easily spotted,
by the simplicity of the counterexamples.
However,
Ford~\cite{ford_parsing_2004} proved that
the general case of this problem is undecidable.

One of the conditions for the algorithm to yield a result
is whenever it consumes some input.
As a result,
it exclusively visits patterns that may be reached
without consuming any input.
Therefore, if the algorithm revisits a rule,
this means a path exists in which
the parsing routine may reach the same rule
and with the same input string,
which would indicate that such rule is left-recursive.
We will now discuss possible ways to detect
when a rule has been visited twice.

One possible way to detect left-recursive rules
is through a set of visited rules,
which is checked and updated
every time a nonterminal pattern is visited.
For a grammar with $n$ rules,
this set could be implemented
as an array of $n$ Boolean values,
each representing a rule.
This method achieves
a computation and spatial
completity of $O(n)$.

Another approach,
which is simpler and takes less memory space,
uses a counter of visited rules,
which starts at zero
and gets incremented every time a rule is visited.
If this value ever surpasses the number of grammar rules,
then we know, by the pigeonhole principle,
that some rule has been visited more than once.
In the case of grammars with left-recursive rules,
we may visit more rules than necessary,
however, we are not particularly worried
about the performance of the algorithm
in the case of errors.

\lpeg{} adopts this last approach.
Our formalization follows \lpeg{},
though with a small twist:
instead of counting visited rules from zero until the limit,
we count to-be-visited rules from the limit down to zero.
This simplifies our formalization
by moving the limit calculation out of the algorithm body,
and letting the limit be passed down as a parameter instead.

At this point,
it is important to draw a distinction between
exhausting the counter of to-be-visited rules
and correctly identifying a left-recursive rule.
When the algorithm starts,
the counter is initialized with the provided limit.
It is then decremented every time a rule is visited.
If the counter ever reaches zero,
then attempting to visit any rule
will return an error.
The algorithm does not determine
whether this error indicates left recursion,
because it would require the algorithm to check
whether the limit is greater than the number of grammar rules.
Instead, we leave it to the caller
to provide a high enough limit,
in which case the algorithm indeed
correctly labels rules as left-recursive
by returning an error.

Because of this shift in responsibilities,
we adapt the nomenclature
for the counter parameter and associated error,
based on an analogy with call stacks.
If, every time a rule is visited, it were pushed onto a stack $k$,
then we could think of the counter parameter $d$ as the stack depth limit;
and surpassing it would be similar to a stack overflow error.

For the sole purpose of helping us prove certain properties about the algorithm,
we will also include this stack $k$, a list of rule indices, as an output,
though it doesn't affect the algorithm.
As we will soon see, it is either appended, passed along, or ignored.
It works as a trace of the inner workings of the function,
a high-level concept we only use for proving lemmas about this algorithm.
\lpeg{} also implements this output, but it is only used
when formatting error messages about left-recursive rules.

We now describe the algorithm
for detecting left-recursive rules,
starting with its inputs and outputs.
It receives a pattern, a grammar, and a stack depth limit,
and returns a label and a stack.
We represent labels by
optional Boolean values
$\Some true$ (nullable),
$\Some false$ (non-nullable) and
$\None$ (stack overflow error);
and stacks by either
$nil$ (an empty stack) or
$i :: k$ (a rule of index $i$ concatenated with a stack $k$).
For now, we will work with this signature,
but beware that the actual function,
displayed in \Cref{fig:vr-function},
\begin{figure}
    \centering
    \begin{align*}
    \begin{aligned}[t]
        & \verifyrulecomp{g}{p}{d}{nb}{0} = \None \\
        & \verifyrulecomp{g}{p}{d}{nb}{(1+gas)} = \\
        & \begin{aligned}[t]
            & \matchwith{p} \\
            & \matchcase{\PEmpty}{\Some(\Some true, nil)} \\
            & \matchcase{\PSet{cs}}{\Some(\Some nb, nil)} \\
            & \matchcase{\PRepetition{p'}}{\verifyrulecomp{g}{p'}{d}{true}{gas}} \\
            & \matchcase{\PNot{p'}}{\verifyrulecomp{g}{p'}{d}{true}{gas}} \\
            & \matchcase{\PAnd{p'}}{\verifyrulecomp{g}{p'}{d}{true}{gas}} \\
            & \matchcase{\PNT{i}}{\begin{aligned}[t]
                & \matchwith{g[i]} \\
                & \matchcase{\None}{\None} \\
                & \matchcase{\Some p'}{\begin{aligned}[t]
                    & \matchwith{d} \\
                    & \matchcase{0}{\Some(\None, nil)} \\
                    & \matchcase{1+d'}{\begin{aligned}[t]
                        & \matchwith{\verifyrulecomp{g}{p'}{d'}{nb}{gas}} \\
                        & \matchcase{\Some (res, k)}{\Some (res, i :: k)} \\
                        & \matchcase{\None}{\None} \\
                        & \matchend{}
                    \end{aligned}} \\
                    & \matchend{}
                \end{aligned}} \\
                & \matchend{}
            \end{aligned}} \\
            & \matchcase{\PSequence{p_1}{p_2}}{\begin{aligned}[t]
                & \matchwith{\verifyrulecomp{g}{p_1}{d}{false}{gas}} \\
                & \matchcase{\Some(\Some true, k)}{\verifyrulecomp{g}{p_2}{d}{nb}{gas}} \\
                & \matchcase{\Some(\Some false, k)}{\Some(\Some nb, k)} \\
                & \matchcase{res}{res} \\
                & \matchend{}
            \end{aligned}} \\
            & \matchcase{\PChoice{p_1}{p_2}}{\begin{aligned}[t]
                & \matchwith{\verifyrulecomp{g}{p_1}{d}{nb}{gas}} \\
                & \matchcase{\Some(\Some nb', k)}{\verifyrulecomp{g}{p_2}{d}{nb'}{gas}} \\
                & \matchcase{res}{res} \\
                & \matchend{}
            \end{aligned}} \\
            & \matchend{}
        \end{aligned}
    \end{aligned}
\end{align*}
    \caption{The left recursion detection function.}
    \label{fig:vr-function}
\end{figure}
receives an extra parameter
which we will introduce later in this section.

The function is defined recursively.
In most cases, it calls itself for each sub-pattern.
In the case of nonterminal patterns, however,
it calls itself for the referenced rule.
Furthermore, the function propagates any stack overflow errors.
This means that,
if some recursive call returns $\None$,
signaling a stack overflow,
and a stack $k$,
then the function also returns $\None$ and $k$.

For the empty pattern $\PEmpty$,
the function returns a label $\Some true$ and $nil$,
because it is nullable and doesn't visit any nonterminal.
We categorize it as nullable because it may match while consuming no input.
In particular, it always matches while consuming no input.

As for character set patterns $\PSet{cs}$,
the function returns $\Some false$ and $nil$,
because it is non-nullable,
meaning it always consumes some input when it matches.
It also doesn't visit any nonterminal.

For a nonterminal pattern $\PNT{i}$,
the function first checks the stack depth limit $d$.
If $d=0$, it returns $\None$ and $nil$,
signaling a stack overflow
and that it didn't visit any nonterminal.
Otherwise, if $d\ge1$,
then the function calls itself for the $i^{th}$ rule of the grammar,
while passing a stack depth limit of $d-1$.
If this recursive call returns a label $res$ and a stack $k$,
then the function returns $res$ and $i :: k$.
This way, the stack accumulates
the indices of the grammar rules
in the same order in which they are visited.

For a repetition pattern $\PRepetition{p}$,
the function evaluates $p$,
which returns $res$ and $k$,
to check for any stack overflow errors.
If $res\ne\None$,
then it returns $\Some true$ and $k$,
as it can match while consuming no input,
in case $p$ fails.
We assume that $p$ can fail because,
in the final verification step,
we ensure that $p$ is non-nullable,
and we know that non-nullable patterns
fail to match the empty string.

Predicate patterns $\PNot{p}$ and $\PAnd{p}$ are evaluated
in the same way as repetition patterns,
but for different reasons.
Repetition patterns are nullable
because they can always match without consuming any input.
Meanwhile, predicate patterns are nullable by approximation,
under the assumption that $p$ may match,
in the case of $\PAnd{p}$,
or fail to match,
in the case of $\PNot{p}$.

For a sequence pattern $\PSequence{p_1}{p_2}$,
the function first evaluates $p_1$.
If $p_1$ is non-nullable,
then so is the sequence $\PSequence{p_1}{p_2}$,
and the function returns the same label and stack as $p_1$.
Note that $p_2$ is not even evaluated in this case,
because it would be visited with a shorter input string
during parsing.
This is the only case in which
the nullable property comes into play
in this algorithm.
If, otherwise, $p_1$ is nullable,
then it evaluates $p_2$
and returns the same label and stack as $p_2$.

Finally, for a choice pattern $\PChoice{p_1}{p_2}$,
it first evaluates $p_1$.
If it returns $\Some b_1$,
indicating that $p_1$ did not overflow the stack,
then it evaluates $p_2$.
If it also returns $\Some b_2$,
then the function returns $\Some (b_1 \vee b_2)$
and the same stack as $p_2$.

The algorithm we've just described
is quite similar to the one implemented in \lpeg{}.
There is, however, one small difference
related to the use of tail calls
as an optimization technique.
In C, tail calls are implemented
with \texttt{goto} statements.
To apply this optimization technique,
\lpeg{} adds an extra parameter to the function
to work as an accumulator for the nullable property.
Without this accumulator parameter,
the evaluation of choice patterns $\PChoice{p_1}{p_2}$
would rely solely on recursion.
It would evaluate $p_1$ and $p_2$,
then perform a Boolean \scor{} operation on the results.

With the addition of a Boolean parameter $nb$,
we can turn the evaluation of $p_2$ into a tail call.
Instead of making the Boolean \scor{} operation explicitly,
we let the accumulator do it under-the-hood.
This works because,
in the base cases of the recursion,
in which the function would return
either $\Some true$ or $\Some false$,
we return instead $\Some (true \vee nb)$ and $\Some (false \vee nb)$,
which get simplified to $\Some true$ and $\Some nb$, respectively.
\Cref{fig:evalchoice} shows a Coq-like pseudocode of how choice patterns
\begin{figure}
    \centering
    \newcommand{\eval}[4]{eval\ #1\ #2\ #3\ #4}

\begin{align*}
    \eval{g}{\dsqb{\PChoice{p_1}{p_2}}}{d}{nb} =\
    & \matchwith{\eval{g}{p_1}{d}{nb}} \\
    & \matchcase{(\Some nb', k)}{\eval{g}{p_2}{d}{nb'}} \\
    & \matchcase{(\None, k)}{(\None, k)} \\
    & \matchend{}
\end{align*}
    \caption{Pseudocode of the evaluation of choice patterns.}
    \label{fig:evalchoice}
\end{figure}
are evaluated with the Boolean parameter $nb$.

We would also like to highlight
how this nullable accumulator allows
the evaluation of repetitions and predicates
to be rewritten as tail calls.
Previously,
we would have to check if $p$ evaluated to $\None$,
before returning $\Some true$.
Now, we can simply pass $true$ as the nullable accumulator,
which guarantees that, if $p$ does not evaluate to $\None$,
it evaluates to $\Some true$.

There are some ways in which this function
could be implemented in Coq as a fixed-point.
The classical way is to add a gas parameter,
which gets decremented in every recursive call.
We make the function return an optional value,
such that, if the gas parameter ever reaches zero,
it returns $\None$.
Other ways are providing a well-formedness proof,
or a measure function.
We choose the first strategy,
because it is the simplest to implement.

Finally,
\Cref{fig:vr-function} presents the algorithm
defined as a fixed-point function.
It returns an optional value,
because we adopted the gas strategy,
but also because it cannot evaluate
nonterminal patterns that
reference nonexistent rules.
In this case,
the function also returns $\None$.
In all other cases,
the function returns $\Some (res, k)$,
with $res$ being a label, and $k$, a stack.

At this point,
the reader should be warned that
we will not attempt to prove the correctness
of this function in isolation.
In fact, we will not even try to formally
define left-recursive rules.
This might frustrate the reader,
but we assure you that such proof will not be necessary.
Instead,
we will later prove the correctness of the whole algorithm
once we introduce all steps of the verification process.
In this section,
we will simply prove
that the label returned by the function
is monotonic and eventually constant
with respect to the gas counter and stack depth limit.

A function $f$ is said to be monotonically increasing
if, for any $x$ and $y$, such that $x \le y$,
it is always true that $f(x) \le f(y)$.
In the case of the \textit{\verifyrulename{}} function,
this will be true for the gas counter and stack depth limit parameters,
and the order between optionals is $\None < \Some res$, for any $res$.
This means that, if the function ever returns $\Some res$,
increasing the gas counter or the stack depth limit
will not alter the return value.

A function $f$ is eventually constant
if, for some $N$ and for any $x$ and $y$,
such that $x, y \ge N$,
it is true that $f(x) = f(y)$.
In the case of \textit{\verifyrulename{}} function,
this will be true for the gas counter and stack depth limit parameters
and for the label return value.
This means that both the gas counter and stack depth limit
have lower bounds for which the returned label stabilizes.

About this fixed-point definition,
we will initially prove some basic lemmas.
Starting with \Cref{lemma:vr-gas-monotonicity},
we state that, if the function returns $\Some (res, k)$,
then increasing the value of the gas parameter
will not change the result.
This is what we mean by
the function being monotonic and eventually constant
with respect to the gas counter.

\begin{lemma}%[Verify Rule Gas Eventual Constancy]
    If $\verifyrulecomp{g}{p}{d}{nb}{gas} = \Some (res, k)$, \\
    then $\forall gas' \ge gas$,
    $\verifyrulecomp{g}{p}{d}{nb}{gas'} = \Some (res, k)$.
    \label{lemma:vr-gas-monotonicity}
\end{lemma}

\Cref{lemma:vr-termination} states that,
for any coherent pattern and grammar,
there exists a lower bound for the gas parameter,
for which the function returns $\Some (res, k)$.
The lower bound
takes into account
the size of the pattern $\size{p}$,
the size of the grammar $\size{g}$,
and the stack depth limit $d$.

\begin{lemma}%[Verify Rule Termination]
    If $\Coherent{g}{p}{true}$,
    and $\lCoherent{g}{g}{true}$, \\
    then, $\forall gas \ge \size{p} + d \cdot \size{g}$,
    $\exists res\ \exists k\ \verifyrulecomp{g}{p}{d}{nb}{gas} = \Some (res, k)$.
    \label{lemma:vr-termination}
\end{lemma}

\begin{proof}
    For most patterns,
    the proof follows from induction on the pattern $p$.
    Meanwhile, for non-terminal patterns,
    the proof follows from induction on the stack depth limit $d$.
    We show below how the lower bound for a rule $r$ and stack depth limit $d$
    is derived from a non-terminal $\PNT{i}$ that references $r$ and stack depth limit $d+1$.
    We use \Cref{lemma:size-of-r-le-size-of-g} to show that $\size{g} \ge \size{r}$.
    \begin{align*}
        gas & \ge \size{\PNT{i}} + (d + 1) \cdot \size{g} \\
        & \ge 1 + (d + 1) \cdot \size{g} \\
        & \ge 1 + \size{g} + d \cdot \size{g} \\
        & \ge 1 + \size{r} + d \cdot \size{g}
    \end{align*}
\end{proof}

Now, we would like to
prove that the label returned by the \textit{\verifyrulename{}} function
is monotonic and eventually constant with respect to the
stack depth limit.
We discard the returned stack in this context because,
in the case of left-recursive rules,
the stack returned by the function will,
in fact,
diverge.
However, we are not interested in
the output stack, in this case.
What really matters to the following
steps of the verification process
is the label.
In particular, we would like to make sure
that no rule in the grammar is marked
with the label $\None$,
meaning ``stack overflow''.

In order to prove such lemma,
we realized an inductive, gasless predicate
would be better suited
than the fixed-point definition,
as it would be easier to perform proofs by induction,
and without having to deal with a gas parameter.
\Cref{fig:verifyrule}
\begin{figure}[ht!]
    \section{Left-recursive rules}
\label{section:lr-rules}

In general,
we consider a rule to be left-recursive
if it can wind up in itself
without consuming any input in-between.
This brings us to the heart
of the algorithm that
detects left-recursive rules.
On a high level,
it symbolically parses each rule,
until it either consumes some input,
visits some rule twice without consuming any input,
or simply finishes.

The algorithm categorizes patterns into three groups.
If a pattern can be parsed until its end without consuming any input,
it is said to be \emph{nullable}.
If, otherwise, it always consumes some input,
it is categorized as \emph{non-nullable}.
Alternatively, if it can lead to some rule twice,
without consuming any input,
it is categorized as left-recursive.

In order to check whether a pattern
is guaranteed to consume some input,
the algorithm uses a conservative approximation
proposed by Ford~\cite{ford_parsing_2004},
which makes two assumptions.
The first one is that $\PNot{p}$ may match,
and the second one is that,
in the case of $\PChoice{p_1}{p_2}$,
it may visit $p_2$,
without checking whether $p_1$ always matches.
For illustrative purposes,
we present a simple counterexample
for each assumption.

A counterexample for the first assumption
is the pattern $\PNot{\PEmpty}$,
which never matches.
Meanwhile, the second assumption doesn't hold
for the pattern $\PChoice{\PEmpty}{p_2}$,
because $\PEmpty$ always matches,
and, therefore, $p_2$ is never visited.
The reader might think that
these cases can be easily spotted,
by the simplicity of the counterexamples.
However,
Ford~\cite{ford_parsing_2004} proved that
the general case of this problem is undecidable.

One of the conditions for the algorithm to yield a result
is whenever it consumes some input.
As a result,
it exclusively visits patterns that may be reached
without consuming any input.
Therefore, if the algorithm revisits a rule,
this means a path exists in which
the parsing routine may reach the same rule
and with the same input string,
which would indicate that such rule is left-recursive.
We will now discuss possible ways to detect
when a rule has been visited twice.

One possible way to detect left-recursive rules
is through a set of visited rules,
which is checked and updated
every time a nonterminal pattern is visited.
For a grammar with $n$ rules,
this set could be implemented
as an array of $n$ Boolean values,
each representing a rule.
This method achieves
a computation and spatial
completity of $O(n)$.

Another approach,
which is simpler and takes less memory space,
uses a counter of visited rules,
which starts at zero
and gets incremented every time a rule is visited.
If this value ever surpasses the number of grammar rules,
then we know, by the pigeonhole principle,
that some rule has been visited more than once.
In the case of grammars with left-recursive rules,
we may visit more rules than necessary,
however, we are not particularly worried
about the performance of the algorithm
in the case of errors.

\lpeg{} adopts this last approach.
Our formalization follows \lpeg{},
though with a small twist:
instead of counting visited rules from zero until the limit,
we count to-be-visited rules from the limit down to zero.
This simplifies our formalization
by moving the limit calculation out of the algorithm body,
and letting the limit be passed down as a parameter instead.

At this point,
it is important to draw a distinction between
exhausting the counter of to-be-visited rules
and correctly identifying a left-recursive rule.
When the algorithm starts,
the counter is initialized with the provided limit.
It is then decremented every time a rule is visited.
If the counter ever reaches zero,
then attempting to visit any rule
will return an error.
The algorithm does not determine
whether this error indicates left recursion,
because it would require the algorithm to check
whether the limit is greater than the number of grammar rules.
Instead, we leave it to the caller
to provide a high enough limit,
in which case the algorithm indeed
correctly labels rules as left-recursive
by returning an error.

Because of this shift in responsibilities,
we adapt the nomenclature
for the counter parameter and associated error,
based on an analogy with call stacks.
If, every time a rule is visited, it were pushed onto a stack $k$,
then we could think of the counter parameter $d$ as the stack depth limit;
and surpassing it would be similar to a stack overflow error.

For the sole purpose of helping us prove certain properties about the algorithm,
we will also include this stack $k$, a list of rule indices, as an output,
though it doesn't affect the algorithm.
As we will soon see, it is either appended, passed along, or ignored.
It works as a trace of the inner workings of the function,
a high-level concept we only use for proving lemmas about this algorithm.
\lpeg{} also implements this output, but it is only used
when formatting error messages about left-recursive rules.

We now describe the algorithm
for detecting left-recursive rules,
starting with its inputs and outputs.
It receives a pattern, a grammar, and a stack depth limit,
and returns a label and a stack.
We represent labels by
optional Boolean values
$\Some true$ (nullable),
$\Some false$ (non-nullable) and
$\None$ (stack overflow error);
and stacks by either
$nil$ (an empty stack) or
$i :: k$ (a rule of index $i$ concatenated with a stack $k$).
For now, we will work with this signature,
but beware that the actual function,
displayed in \Cref{fig:vr-function},
\begin{figure}
    \centering
    \begin{align*}
    \begin{aligned}[t]
        & \verifyrulecomp{g}{p}{d}{nb}{0} = \None \\
        & \verifyrulecomp{g}{p}{d}{nb}{(1+gas)} = \\
        & \begin{aligned}[t]
            & \matchwith{p} \\
            & \matchcase{\PEmpty}{\Some(\Some true, nil)} \\
            & \matchcase{\PSet{cs}}{\Some(\Some nb, nil)} \\
            & \matchcase{\PRepetition{p'}}{\verifyrulecomp{g}{p'}{d}{true}{gas}} \\
            & \matchcase{\PNot{p'}}{\verifyrulecomp{g}{p'}{d}{true}{gas}} \\
            & \matchcase{\PAnd{p'}}{\verifyrulecomp{g}{p'}{d}{true}{gas}} \\
            & \matchcase{\PNT{i}}{\begin{aligned}[t]
                & \matchwith{g[i]} \\
                & \matchcase{\None}{\None} \\
                & \matchcase{\Some p'}{\begin{aligned}[t]
                    & \matchwith{d} \\
                    & \matchcase{0}{\Some(\None, nil)} \\
                    & \matchcase{1+d'}{\begin{aligned}[t]
                        & \matchwith{\verifyrulecomp{g}{p'}{d'}{nb}{gas}} \\
                        & \matchcase{\Some (res, k)}{\Some (res, i :: k)} \\
                        & \matchcase{\None}{\None} \\
                        & \matchend{}
                    \end{aligned}} \\
                    & \matchend{}
                \end{aligned}} \\
                & \matchend{}
            \end{aligned}} \\
            & \matchcase{\PSequence{p_1}{p_2}}{\begin{aligned}[t]
                & \matchwith{\verifyrulecomp{g}{p_1}{d}{false}{gas}} \\
                & \matchcase{\Some(\Some true, k)}{\verifyrulecomp{g}{p_2}{d}{nb}{gas}} \\
                & \matchcase{\Some(\Some false, k)}{\Some(\Some nb, k)} \\
                & \matchcase{res}{res} \\
                & \matchend{}
            \end{aligned}} \\
            & \matchcase{\PChoice{p_1}{p_2}}{\begin{aligned}[t]
                & \matchwith{\verifyrulecomp{g}{p_1}{d}{nb}{gas}} \\
                & \matchcase{\Some(\Some nb', k)}{\verifyrulecomp{g}{p_2}{d}{nb'}{gas}} \\
                & \matchcase{res}{res} \\
                & \matchend{}
            \end{aligned}} \\
            & \matchend{}
        \end{aligned}
    \end{aligned}
\end{align*}
    \caption{The left recursion detection function.}
    \label{fig:vr-function}
\end{figure}
receives an extra parameter
which we will introduce later in this section.

The function is defined recursively.
In most cases, it calls itself for each sub-pattern.
In the case of nonterminal patterns, however,
it calls itself for the referenced rule.
Furthermore, the function propagates any stack overflow errors.
This means that,
if some recursive call returns $\None$,
signaling a stack overflow,
and a stack $k$,
then the function also returns $\None$ and $k$.

For the empty pattern $\PEmpty$,
the function returns a label $\Some true$ and $nil$,
because it is nullable and doesn't visit any nonterminal.
We categorize it as nullable because it may match while consuming no input.
In particular, it always matches while consuming no input.

As for character set patterns $\PSet{cs}$,
the function returns $\Some false$ and $nil$,
because it is non-nullable,
meaning it always consumes some input when it matches.
It also doesn't visit any nonterminal.

For a nonterminal pattern $\PNT{i}$,
the function first checks the stack depth limit $d$.
If $d=0$, it returns $\None$ and $nil$,
signaling a stack overflow
and that it didn't visit any nonterminal.
Otherwise, if $d\ge1$,
then the function calls itself for the $i^{th}$ rule of the grammar,
while passing a stack depth limit of $d-1$.
If this recursive call returns a label $res$ and a stack $k$,
then the function returns $res$ and $i :: k$.
This way, the stack accumulates
the indices of the grammar rules
in the same order in which they are visited.

For a repetition pattern $\PRepetition{p}$,
the function evaluates $p$,
which returns $res$ and $k$,
to check for any stack overflow errors.
If $res\ne\None$,
then it returns $\Some true$ and $k$,
as it can match while consuming no input,
in case $p$ fails.
We assume that $p$ can fail because,
in the final verification step,
we ensure that $p$ is non-nullable,
and we know that non-nullable patterns
fail to match the empty string.

Predicate patterns $\PNot{p}$ and $\PAnd{p}$ are evaluated
in the same way as repetition patterns,
but for different reasons.
Repetition patterns are nullable
because they can always match without consuming any input.
Meanwhile, predicate patterns are nullable by approximation,
under the assumption that $p$ may match,
in the case of $\PAnd{p}$,
or fail to match,
in the case of $\PNot{p}$.

For a sequence pattern $\PSequence{p_1}{p_2}$,
the function first evaluates $p_1$.
If $p_1$ is non-nullable,
then so is the sequence $\PSequence{p_1}{p_2}$,
and the function returns the same label and stack as $p_1$.
Note that $p_2$ is not even evaluated in this case,
because it would be visited with a shorter input string
during parsing.
This is the only case in which
the nullable property comes into play
in this algorithm.
If, otherwise, $p_1$ is nullable,
then it evaluates $p_2$
and returns the same label and stack as $p_2$.

Finally, for a choice pattern $\PChoice{p_1}{p_2}$,
it first evaluates $p_1$.
If it returns $\Some b_1$,
indicating that $p_1$ did not overflow the stack,
then it evaluates $p_2$.
If it also returns $\Some b_2$,
then the function returns $\Some (b_1 \vee b_2)$
and the same stack as $p_2$.

The algorithm we've just described
is quite similar to the one implemented in \lpeg{}.
There is, however, one small difference
related to the use of tail calls
as an optimization technique.
In C, tail calls are implemented
with \texttt{goto} statements.
To apply this optimization technique,
\lpeg{} adds an extra parameter to the function
to work as an accumulator for the nullable property.
Without this accumulator parameter,
the evaluation of choice patterns $\PChoice{p_1}{p_2}$
would rely solely on recursion.
It would evaluate $p_1$ and $p_2$,
then perform a Boolean \scor{} operation on the results.

With the addition of a Boolean parameter $nb$,
we can turn the evaluation of $p_2$ into a tail call.
Instead of making the Boolean \scor{} operation explicitly,
we let the accumulator do it under-the-hood.
This works because,
in the base cases of the recursion,
in which the function would return
either $\Some true$ or $\Some false$,
we return instead $\Some (true \vee nb)$ and $\Some (false \vee nb)$,
which get simplified to $\Some true$ and $\Some nb$, respectively.
\Cref{fig:evalchoice} shows a Coq-like pseudocode of how choice patterns
\begin{figure}
    \centering
    \newcommand{\eval}[4]{eval\ #1\ #2\ #3\ #4}

\begin{align*}
    \eval{g}{\dsqb{\PChoice{p_1}{p_2}}}{d}{nb} =\
    & \matchwith{\eval{g}{p_1}{d}{nb}} \\
    & \matchcase{(\Some nb', k)}{\eval{g}{p_2}{d}{nb'}} \\
    & \matchcase{(\None, k)}{(\None, k)} \\
    & \matchend{}
\end{align*}
    \caption{Pseudocode of the evaluation of choice patterns.}
    \label{fig:evalchoice}
\end{figure}
are evaluated with the Boolean parameter $nb$.

We would also like to highlight
how this nullable accumulator allows
the evaluation of repetitions and predicates
to be rewritten as tail calls.
Previously,
we would have to check if $p$ evaluated to $\None$,
before returning $\Some true$.
Now, we can simply pass $true$ as the nullable accumulator,
which guarantees that, if $p$ does not evaluate to $\None$,
it evaluates to $\Some true$.

There are some ways in which this function
could be implemented in Coq as a fixed-point.
The classical way is to add a gas parameter,
which gets decremented in every recursive call.
We make the function return an optional value,
such that, if the gas parameter ever reaches zero,
it returns $\None$.
Other ways are providing a well-formedness proof,
or a measure function.
We choose the first strategy,
because it is the simplest to implement.

Finally,
\Cref{fig:vr-function} presents the algorithm
defined as a fixed-point function.
It returns an optional value,
because we adopted the gas strategy,
but also because it cannot evaluate
nonterminal patterns that
reference nonexistent rules.
In this case,
the function also returns $\None$.
In all other cases,
the function returns $\Some (res, k)$,
with $res$ being a label, and $k$, a stack.

At this point,
the reader should be warned that
we will not attempt to prove the correctness
of this function in isolation.
In fact, we will not even try to formally
define left-recursive rules.
This might frustrate the reader,
but we assure you that such proof will not be necessary.
Instead,
we will later prove the correctness of the whole algorithm
once we introduce all steps of the verification process.
In this section,
we will simply prove
that the label returned by the function
is monotonic and eventually constant
with respect to the gas counter and stack depth limit.

A function $f$ is said to be monotonically increasing
if, for any $x$ and $y$, such that $x \le y$,
it is always true that $f(x) \le f(y)$.
In the case of the \textit{\verifyrulename{}} function,
this will be true for the gas counter and stack depth limit parameters,
and the order between optionals is $\None < \Some res$, for any $res$.
This means that, if the function ever returns $\Some res$,
increasing the gas counter or the stack depth limit
will not alter the return value.

A function $f$ is eventually constant
if, for some $N$ and for any $x$ and $y$,
such that $x, y \ge N$,
it is true that $f(x) = f(y)$.
In the case of \textit{\verifyrulename{}} function,
this will be true for the gas counter and stack depth limit parameters
and for the label return value.
This means that both the gas counter and stack depth limit
have lower bounds for which the returned label stabilizes.

About this fixed-point definition,
we will initially prove some basic lemmas.
Starting with \Cref{lemma:vr-gas-monotonicity},
we state that, if the function returns $\Some (res, k)$,
then increasing the value of the gas parameter
will not change the result.
This is what we mean by
the function being monotonic and eventually constant
with respect to the gas counter.

\begin{lemma}%[Verify Rule Gas Eventual Constancy]
    If $\verifyrulecomp{g}{p}{d}{nb}{gas} = \Some (res, k)$, \\
    then $\forall gas' \ge gas$,
    $\verifyrulecomp{g}{p}{d}{nb}{gas'} = \Some (res, k)$.
    \label{lemma:vr-gas-monotonicity}
\end{lemma}

\Cref{lemma:vr-termination} states that,
for any coherent pattern and grammar,
there exists a lower bound for the gas parameter,
for which the function returns $\Some (res, k)$.
The lower bound
takes into account
the size of the pattern $\size{p}$,
the size of the grammar $\size{g}$,
and the stack depth limit $d$.

\begin{lemma}%[Verify Rule Termination]
    If $\Coherent{g}{p}{true}$,
    and $\lCoherent{g}{g}{true}$, \\
    then, $\forall gas \ge \size{p} + d \cdot \size{g}$,
    $\exists res\ \exists k\ \verifyrulecomp{g}{p}{d}{nb}{gas} = \Some (res, k)$.
    \label{lemma:vr-termination}
\end{lemma}

\begin{proof}
    For most patterns,
    the proof follows from induction on the pattern $p$.
    Meanwhile, for non-terminal patterns,
    the proof follows from induction on the stack depth limit $d$.
    We show below how the lower bound for a rule $r$ and stack depth limit $d$
    is derived from a non-terminal $\PNT{i}$ that references $r$ and stack depth limit $d+1$.
    We use \Cref{lemma:size-of-r-le-size-of-g} to show that $\size{g} \ge \size{r}$.
    \begin{align*}
        gas & \ge \size{\PNT{i}} + (d + 1) \cdot \size{g} \\
        & \ge 1 + (d + 1) \cdot \size{g} \\
        & \ge 1 + \size{g} + d \cdot \size{g} \\
        & \ge 1 + \size{r} + d \cdot \size{g}
    \end{align*}
\end{proof}

Now, we would like to
prove that the label returned by the \textit{\verifyrulename{}} function
is monotonic and eventually constant with respect to the
stack depth limit.
We discard the returned stack in this context because,
in the case of left-recursive rules,
the stack returned by the function will,
in fact,
diverge.
However, we are not interested in
the output stack, in this case.
What really matters to the following
steps of the verification process
is the label.
In particular, we would like to make sure
that no rule in the grammar is marked
with the label $\None$,
meaning ``stack overflow''.

In order to prove such lemma,
we realized an inductive, gasless predicate
would be better suited
than the fixed-point definition,
as it would be easier to perform proofs by induction,
and without having to deal with a gas parameter.
\Cref{fig:verifyrule}
\begin{figure}[ht!]
    \section{Left-recursive rules}
\label{section:lr-rules}

In general,
we consider a rule to be left-recursive
if it can wind up in itself
without consuming any input in-between.
This brings us to the heart
of the algorithm that
detects left-recursive rules.
On a high level,
it symbolically parses each rule,
until it either consumes some input,
visits some rule twice without consuming any input,
or simply finishes.

The algorithm categorizes patterns into three groups.
If a pattern can be parsed until its end without consuming any input,
it is said to be \emph{nullable}.
If, otherwise, it always consumes some input,
it is categorized as \emph{non-nullable}.
Alternatively, if it can lead to some rule twice,
without consuming any input,
it is categorized as left-recursive.

In order to check whether a pattern
is guaranteed to consume some input,
the algorithm uses a conservative approximation
proposed by Ford~\cite{ford_parsing_2004},
which makes two assumptions.
The first one is that $\PNot{p}$ may match,
and the second one is that,
in the case of $\PChoice{p_1}{p_2}$,
it may visit $p_2$,
without checking whether $p_1$ always matches.
For illustrative purposes,
we present a simple counterexample
for each assumption.

A counterexample for the first assumption
is the pattern $\PNot{\PEmpty}$,
which never matches.
Meanwhile, the second assumption doesn't hold
for the pattern $\PChoice{\PEmpty}{p_2}$,
because $\PEmpty$ always matches,
and, therefore, $p_2$ is never visited.
The reader might think that
these cases can be easily spotted,
by the simplicity of the counterexamples.
However,
Ford~\cite{ford_parsing_2004} proved that
the general case of this problem is undecidable.

One of the conditions for the algorithm to yield a result
is whenever it consumes some input.
As a result,
it exclusively visits patterns that may be reached
without consuming any input.
Therefore, if the algorithm revisits a rule,
this means a path exists in which
the parsing routine may reach the same rule
and with the same input string,
which would indicate that such rule is left-recursive.
We will now discuss possible ways to detect
when a rule has been visited twice.

One possible way to detect left-recursive rules
is through a set of visited rules,
which is checked and updated
every time a nonterminal pattern is visited.
For a grammar with $n$ rules,
this set could be implemented
as an array of $n$ Boolean values,
each representing a rule.
This method achieves
a computation and spatial
completity of $O(n)$.

Another approach,
which is simpler and takes less memory space,
uses a counter of visited rules,
which starts at zero
and gets incremented every time a rule is visited.
If this value ever surpasses the number of grammar rules,
then we know, by the pigeonhole principle,
that some rule has been visited more than once.
In the case of grammars with left-recursive rules,
we may visit more rules than necessary,
however, we are not particularly worried
about the performance of the algorithm
in the case of errors.

\lpeg{} adopts this last approach.
Our formalization follows \lpeg{},
though with a small twist:
instead of counting visited rules from zero until the limit,
we count to-be-visited rules from the limit down to zero.
This simplifies our formalization
by moving the limit calculation out of the algorithm body,
and letting the limit be passed down as a parameter instead.

At this point,
it is important to draw a distinction between
exhausting the counter of to-be-visited rules
and correctly identifying a left-recursive rule.
When the algorithm starts,
the counter is initialized with the provided limit.
It is then decremented every time a rule is visited.
If the counter ever reaches zero,
then attempting to visit any rule
will return an error.
The algorithm does not determine
whether this error indicates left recursion,
because it would require the algorithm to check
whether the limit is greater than the number of grammar rules.
Instead, we leave it to the caller
to provide a high enough limit,
in which case the algorithm indeed
correctly labels rules as left-recursive
by returning an error.

Because of this shift in responsibilities,
we adapt the nomenclature
for the counter parameter and associated error,
based on an analogy with call stacks.
If, every time a rule is visited, it were pushed onto a stack $k$,
then we could think of the counter parameter $d$ as the stack depth limit;
and surpassing it would be similar to a stack overflow error.

For the sole purpose of helping us prove certain properties about the algorithm,
we will also include this stack $k$, a list of rule indices, as an output,
though it doesn't affect the algorithm.
As we will soon see, it is either appended, passed along, or ignored.
It works as a trace of the inner workings of the function,
a high-level concept we only use for proving lemmas about this algorithm.
\lpeg{} also implements this output, but it is only used
when formatting error messages about left-recursive rules.

We now describe the algorithm
for detecting left-recursive rules,
starting with its inputs and outputs.
It receives a pattern, a grammar, and a stack depth limit,
and returns a label and a stack.
We represent labels by
optional Boolean values
$\Some true$ (nullable),
$\Some false$ (non-nullable) and
$\None$ (stack overflow error);
and stacks by either
$nil$ (an empty stack) or
$i :: k$ (a rule of index $i$ concatenated with a stack $k$).
For now, we will work with this signature,
but beware that the actual function,
displayed in \Cref{fig:vr-function},
\begin{figure}
    \centering
    \input{verifyrulecomp}
    \caption{The left recursion detection function.}
    \label{fig:vr-function}
\end{figure}
receives an extra parameter
which we will introduce later in this section.

The function is defined recursively.
In most cases, it calls itself for each sub-pattern.
In the case of nonterminal patterns, however,
it calls itself for the referenced rule.
Furthermore, the function propagates any stack overflow errors.
This means that,
if some recursive call returns $\None$,
signaling a stack overflow,
and a stack $k$,
then the function also returns $\None$ and $k$.

For the empty pattern $\PEmpty$,
the function returns a label $\Some true$ and $nil$,
because it is nullable and doesn't visit any nonterminal.
We categorize it as nullable because it may match while consuming no input.
In particular, it always matches while consuming no input.

As for character set patterns $\PSet{cs}$,
the function returns $\Some false$ and $nil$,
because it is non-nullable,
meaning it always consumes some input when it matches.
It also doesn't visit any nonterminal.

For a nonterminal pattern $\PNT{i}$,
the function first checks the stack depth limit $d$.
If $d=0$, it returns $\None$ and $nil$,
signaling a stack overflow
and that it didn't visit any nonterminal.
Otherwise, if $d\ge1$,
then the function calls itself for the $i^{th}$ rule of the grammar,
while passing a stack depth limit of $d-1$.
If this recursive call returns a label $res$ and a stack $k$,
then the function returns $res$ and $i :: k$.
This way, the stack accumulates
the indices of the grammar rules
in the same order in which they are visited.

For a repetition pattern $\PRepetition{p}$,
the function evaluates $p$,
which returns $res$ and $k$,
to check for any stack overflow errors.
If $res\ne\None$,
then it returns $\Some true$ and $k$,
as it can match while consuming no input,
in case $p$ fails.
We assume that $p$ can fail because,
in the final verification step,
we ensure that $p$ is non-nullable,
and we know that non-nullable patterns
fail to match the empty string.

Predicate patterns $\PNot{p}$ and $\PAnd{p}$ are evaluated
in the same way as repetition patterns,
but for different reasons.
Repetition patterns are nullable
because they can always match without consuming any input.
Meanwhile, predicate patterns are nullable by approximation,
under the assumption that $p$ may match,
in the case of $\PAnd{p}$,
or fail to match,
in the case of $\PNot{p}$.

For a sequence pattern $\PSequence{p_1}{p_2}$,
the function first evaluates $p_1$.
If $p_1$ is non-nullable,
then so is the sequence $\PSequence{p_1}{p_2}$,
and the function returns the same label and stack as $p_1$.
Note that $p_2$ is not even evaluated in this case,
because it would be visited with a shorter input string
during parsing.
This is the only case in which
the nullable property comes into play
in this algorithm.
If, otherwise, $p_1$ is nullable,
then it evaluates $p_2$
and returns the same label and stack as $p_2$.

Finally, for a choice pattern $\PChoice{p_1}{p_2}$,
it first evaluates $p_1$.
If it returns $\Some b_1$,
indicating that $p_1$ did not overflow the stack,
then it evaluates $p_2$.
If it also returns $\Some b_2$,
then the function returns $\Some (b_1 \vee b_2)$
and the same stack as $p_2$.

The algorithm we've just described
is quite similar to the one implemented in \lpeg{}.
There is, however, one small difference
related to the use of tail calls
as an optimization technique.
In C, tail calls are implemented
with \texttt{goto} statements.
To apply this optimization technique,
\lpeg{} adds an extra parameter to the function
to work as an accumulator for the nullable property.
Without this accumulator parameter,
the evaluation of choice patterns $\PChoice{p_1}{p_2}$
would rely solely on recursion.
It would evaluate $p_1$ and $p_2$,
then perform a Boolean \scor{} operation on the results.

With the addition of a Boolean parameter $nb$,
we can turn the evaluation of $p_2$ into a tail call.
Instead of making the Boolean \scor{} operation explicitly,
we let the accumulator do it under-the-hood.
This works because,
in the base cases of the recursion,
in which the function would return
either $\Some true$ or $\Some false$,
we return instead $\Some (true \vee nb)$ and $\Some (false \vee nb)$,
which get simplified to $\Some true$ and $\Some nb$, respectively.
\Cref{fig:evalchoice} shows a Coq-like pseudocode of how choice patterns
\begin{figure}
    \centering
    \input{evalchoice}
    \caption{Pseudocode of the evaluation of choice patterns.}
    \label{fig:evalchoice}
\end{figure}
are evaluated with the Boolean parameter $nb$.

We would also like to highlight
how this nullable accumulator allows
the evaluation of repetitions and predicates
to be rewritten as tail calls.
Previously,
we would have to check if $p$ evaluated to $\None$,
before returning $\Some true$.
Now, we can simply pass $true$ as the nullable accumulator,
which guarantees that, if $p$ does not evaluate to $\None$,
it evaluates to $\Some true$.

There are some ways in which this function
could be implemented in Coq as a fixed-point.
The classical way is to add a gas parameter,
which gets decremented in every recursive call.
We make the function return an optional value,
such that, if the gas parameter ever reaches zero,
it returns $\None$.
Other ways are providing a well-formedness proof,
or a measure function.
We choose the first strategy,
because it is the simplest to implement.

Finally,
\Cref{fig:vr-function} presents the algorithm
defined as a fixed-point function.
It returns an optional value,
because we adopted the gas strategy,
but also because it cannot evaluate
nonterminal patterns that
reference nonexistent rules.
In this case,
the function also returns $\None$.
In all other cases,
the function returns $\Some (res, k)$,
with $res$ being a label, and $k$, a stack.

At this point,
the reader should be warned that
we will not attempt to prove the correctness
of this function in isolation.
In fact, we will not even try to formally
define left-recursive rules.
This might frustrate the reader,
but we assure you that such proof will not be necessary.
Instead,
we will later prove the correctness of the whole algorithm
once we introduce all steps of the verification process.
In this section,
we will simply prove
that the label returned by the function
is monotonic and eventually constant
with respect to the gas counter and stack depth limit.

A function $f$ is said to be monotonically increasing
if, for any $x$ and $y$, such that $x \le y$,
it is always true that $f(x) \le f(y)$.
In the case of the \textit{\verifyrulename{}} function,
this will be true for the gas counter and stack depth limit parameters,
and the order between optionals is $\None < \Some res$, for any $res$.
This means that, if the function ever returns $\Some res$,
increasing the gas counter or the stack depth limit
will not alter the return value.

A function $f$ is eventually constant
if, for some $N$ and for any $x$ and $y$,
such that $x, y \ge N$,
it is true that $f(x) = f(y)$.
In the case of \textit{\verifyrulename{}} function,
this will be true for the gas counter and stack depth limit parameters
and for the label return value.
This means that both the gas counter and stack depth limit
have lower bounds for which the returned label stabilizes.

About this fixed-point definition,
we will initially prove some basic lemmas.
Starting with \Cref{lemma:vr-gas-monotonicity},
we state that, if the function returns $\Some (res, k)$,
then increasing the value of the gas parameter
will not change the result.
This is what we mean by
the function being monotonic and eventually constant
with respect to the gas counter.

\begin{lemma}%[Verify Rule Gas Eventual Constancy]
    If $\verifyrulecomp{g}{p}{d}{nb}{gas} = \Some (res, k)$, \\
    then $\forall gas' \ge gas$,
    $\verifyrulecomp{g}{p}{d}{nb}{gas'} = \Some (res, k)$.
    \label{lemma:vr-gas-monotonicity}
\end{lemma}

\Cref{lemma:vr-termination} states that,
for any coherent pattern and grammar,
there exists a lower bound for the gas parameter,
for which the function returns $\Some (res, k)$.
The lower bound
takes into account
the size of the pattern $\size{p}$,
the size of the grammar $\size{g}$,
and the stack depth limit $d$.

\begin{lemma}%[Verify Rule Termination]
    If $\Coherent{g}{p}{true}$,
    and $\lCoherent{g}{g}{true}$, \\
    then, $\forall gas \ge \size{p} + d \cdot \size{g}$,
    $\exists res\ \exists k\ \verifyrulecomp{g}{p}{d}{nb}{gas} = \Some (res, k)$.
    \label{lemma:vr-termination}
\end{lemma}

\begin{proof}
    For most patterns,
    the proof follows from induction on the pattern $p$.
    Meanwhile, for non-terminal patterns,
    the proof follows from induction on the stack depth limit $d$.
    We show below how the lower bound for a rule $r$ and stack depth limit $d$
    is derived from a non-terminal $\PNT{i}$ that references $r$ and stack depth limit $d+1$.
    We use \Cref{lemma:size-of-r-le-size-of-g} to show that $\size{g} \ge \size{r}$.
    \begin{align*}
        gas & \ge \size{\PNT{i}} + (d + 1) \cdot \size{g} \\
        & \ge 1 + (d + 1) \cdot \size{g} \\
        & \ge 1 + \size{g} + d \cdot \size{g} \\
        & \ge 1 + \size{r} + d \cdot \size{g}
    \end{align*}
\end{proof}

Now, we would like to
prove that the label returned by the \textit{\verifyrulename{}} function
is monotonic and eventually constant with respect to the
stack depth limit.
We discard the returned stack in this context because,
in the case of left-recursive rules,
the stack returned by the function will,
in fact,
diverge.
However, we are not interested in
the output stack, in this case.
What really matters to the following
steps of the verification process
is the label.
In particular, we would like to make sure
that no rule in the grammar is marked
with the label $\None$,
meaning ``stack overflow''.

In order to prove such lemma,
we realized an inductive, gasless predicate
would be better suited
than the fixed-point definition,
as it would be easier to perform proofs by induction,
and without having to deal with a gas parameter.
\Cref{fig:verifyrule}
\begin{figure}[ht!]
    \input{verifyrule}
    \caption{The left recursion detection predicate.}
    \label{fig:verifyrule}
\end{figure}
defines such predicate,
denoted as $\VerifyRule{g}{p}{d}{nb}{res}{k}$.
It takes a grammar $g$,
a pattern $p$,
a stack depth limit $d$,
and a nullable accumulator $nb$,
and outputs a result $res$,
and a stack trace $k$.

In order to reach our final goal
of proving that the label returned
by the \textit{\verifyrulename{}} function
is monotonic and eventually constant
with respect to the stack depth limit,
we need to first prove some intermediary lemmas.
First, we need to relate the predicate and
the fixed-point definition together,
so that we can apply the proofs about the
former to the latter.

We begin with
\Cref{lemma:vr-determinism},
which states that,
for identical input,
the predicate yields the same output.
We can therefore state
that the predicate is deterministic.

\begin{lemma}
    If $\VerifyRule{g}{p}{d}{nb}{res_1}{k_1}$,
    and $\VerifyRule{g}{p}{d}{nb}{res_2}{k_2}$, \\
    then $res_1 = res_2$ and $k_1 = k_2$.
    \label{lemma:vr-determinism}
\end{lemma}

\Cref{lemma:vr-follows} shows that
every result returned by the fixed-point definition
can be inductively constructed using the predicate definition.

\begin{lemma}
    If $\verifyrulecomp{g}{p}{d}{nb}{gas} = \Some (res, k)$, \\
    then $\VerifyRule{g}{p}{d}{nb}{res}{k}$.
    \label{lemma:vr-follows}
\end{lemma}

\Cref{lemma:stack-depth-monotonicity-not-lr-pattern} shows that,
if a pattern evaluates to either nullable or non-nullable,
then increasing the stack depth limit
doesn't affect the result.
This is expected,
because, in both cases,
it doesn't surpass the limit,
and increasing it preserves this property
by transitivity.

\begin{lemma}%[Stack Depth Limit Increase without Overflow]
    If $\VerifyRule{g}{p}{d}{nb}{\Some nb'}{k}$, \\
    then $\forall d' \ge d, \VerifyRule{g}{p}{d'}{nb}{\Some nb'}{k}$.
    \label{lemma:stack-depth-monotonicity-not-lr-pattern}
\end{lemma}

\Cref{lemma:stack-depth-lr-pattern} shows that,
on stack overflow,
the output stack $k$ has length $d$.
That is expected,
because a stack overflow happens
when the stack is full
before trying to visit a rule.

\begin{lemma}%[Stack Length on Overflow]
    If $\VerifyRule{g}{p}{d}{nb}{\None}{k}$,
    then $\length{k} = d$.
    \label{lemma:stack-depth-lr-pattern}
\end{lemma}

\Cref{lemma:coherent-stack} states
that the output stack
only contains references to
existing rules in the grammar.
Since we are identifying rules
by their indices in a list,
we prove this by showing that these indices
are less than the number of rules in the grammar,
denoted as $\length{g}$.
This lemma may seem trivial,
but it is necessary for us to later prove,
using the pigeonhole principle,
that a stack with more rules than the grammar
will have at least one repeated rule.

\begin{lemma}%[Coherent Stack]
    If $\VerifyRule{g}{p}{d}{nb}{res}{k}$,
    then $\forall i \in k, i < \length{g}$.
    \label{lemma:coherent-stack}
\end{lemma}

% Next, we prove in \Cref{lemma:true-as-nb} that,
% when provided with $true$ as the value
% for the nullable accumulator $nb$,
% the algorithm outputs as a result
% either $\None$ (which means ``left-recursive'')
% or $\Some true$ (which means ``nullable'').

% \begin{lemma}%[True As Nullable Accumulator]
%     For a grammar $g$,
%     a pattern $p$,
%     a natural number $d$,
%     an optional Boolean value $res$,
%     and a list of natural numbers $k$,
%     if $\VerifyRule{g}{p}{d}{true}{res}{k}$,
%     then $res \in \{\None,\ \Some true\}$.
%     \label{lemma:true-as-nb}
% \end{lemma}

% We also state in \Cref{lemma:nb-change-not-lr-pattern} that,
% when a pattern is identified as either nullable or non-nullable,
% then changing the nullable accumulator
% might change the result to either nullable or non-nullable,
% but the output stack $k$ will remain the same.
% Note that we're saying that it will not make
% the pattern be marked as left-recursive.

% \begin{lemma}%[Nullable Accumulator Change When Pattern Is Not Left-Recursive]
%     For a grammar $g$,
%     a pattern $p$,
%     a natural number $d$,
%     Boolean values $nb$, $nb'$ and $b$,
%     and a list of natural numbers $k$,
%     if $\VerifyRule{g}{p}{d}{nb}{\Some b}{k}$,
%     then there exists some Boolean value $b'$,
%     such that $\VerifyRule{g}{p}{d}{nb'}{\Some b'}{k}$.
%     \label{lemma:nb-change-not-lr-pattern}
% \end{lemma}

% Alternatively,
% when a pattern is identified as left-recursive,
% changing the nullable accumulator
% will change neither the result nor the stack $k$.
% This is put forth in \Cref{lemma:nb-change-lr-pattern}.

% \begin{lemma}%[Nullable Accumulator Change When Pattern Is Left-Recursive]
%     For a grammar $g$,
%     a pattern $p$,
%     a natural number $d$,
%     Boolean values $nb$ and $nb'$,
%     and a list of natural numbers $k$,
%     if $\VerifyRule{g}{p}{d}{nb}{\None}{k}$,
%     then $\VerifyRule{g}{p}{d}{nb'}{\None}{k}$.
%     \label{lemma:nb-change-lr-pattern}
% \end{lemma}

% The following lemmas will start to deal with the output stack more closely.
% It is used by \lpeg{} to format error messages,
% but we will use it to help us formalize certain properties about the algorithm.
% \Cref{lemma:existential-ff} states that,
% for any evaluation with an output stack $k_1 \dplus i :: k_2$,
% there exists a stack depth limit and a nullable accumulator,
% for which the evaluation of the nonterminal pattern $\PNT{i}$
% yields the output stack $i :: k_2$.
% We call this operation ``fast-forwarding''.

% \begin{lemma}%[Existential Fast-forward]
%     $\forall gas\ \forall p\ \forall d\ \forall nb\ \forall res\ \forall k_2\ \forall k_2\ \forall i$, \\
%     if $\VerifyRule{g}{p}{d}{nb}{res}{k_1 \dplus i :: k_2}$,
%     then $\exists d'\ \exists nb'\ \exists res'$,
%     such that $\VerifyRule{g}{\PNT i}{d'}{nb'}{res'}{i :: k_2}$.
%     \label{lemma:existential-ff}
% \end{lemma}

% \Cref{lemma:existential-ff-for-lr-patterns}
% is a particular case of \Cref{lemma:existential-ff},
% in which the evaluation of pattern $p$ overflows the stack,
% and, as a result, so does the evaluation of nonterminal pattern $\PNT{i}$.

% \begin{lemma}%[Existential Fast-forward On Overflow]
%     $\forall gas\ \forall p\ \forall d\ \forall nb\ \forall k_2\ \forall k_2\ \forall i$, \\
%     if $\VerifyRule{g}{p}{d}{nb}{\None}{k_1 \dplus i :: k_2}$,
%     then $\exists d'\ \exists nb'$,
%     such that $\VerifyRule{g}{\PNT i}{d'}{nb'}{\None}{i :: k_2}$.
%     \label{lemma:existential-ff-for-lr-patterns}
% \end{lemma}

\Cref{lemma:ff-for-lr-patterns} states that,
for any evaluation that results in a stack overflow,
we can pick any rule $i$ from the output stack $k$,
and evaluate it with a certain stack depth limit,
so that it also results in a stack overflow,
and returns a suffix of the original stack $k$,
starting from $i$.

\begin{lemma}%[Fast-forward on Overflow]
    If $\VerifyRule{g}{p}{d}{nb}{\None}{k_1 \dplus i :: k_2}$, \\
    then $\VerifyRule{g}{\PNT i}{1+\length{k_2}}{nb'}{\None}{i :: k_2}$.
    \label{lemma:ff-for-lr-patterns}
\end{lemma}

Under the same assumptions,
\Cref{lemma:d-increase-lr} shows that,
if we evaluate a rule $i$ from the stack
with an increased stack depth limit
and it still results in a stack overflow
and returns a stack $i :: k_3$,
then we can increase the stack depth limit
of the original evaluation by the same amount,
it will also result in a stack overflow,
and return a stack that ends with $i :: k_3$.

\begin{lemma}%[Increase Overflown Stack Depth Limit]
    If $\VerifyRule{g}{p}{d}{nb}{\None}{k_1 \dplus i :: k_2}$, \\
    and $\VerifyRule{g}{\PNT i}{1+\length{k_3}}{nb'}{\None}{i :: k_3}$,
    and $\length{k_2} \le \length{k_3}$, \\
    then $\VerifyRule{g}{p}{1 + \length{k_1} + \length{k_3}}{nb}{\None}{k_1 \dplus i :: k_3}$.
    \label{lemma:d-increase-lr}
\end{lemma}

\Cref{lemma:repeated-rule-in-stack} shows that,
if an evaluation results in a stack overflow,
and a rule $i$ occurs more than once in the output stack,
then we can increase the stack depth limit by a certain amount,
and both conditions will still hold true.

\begin{lemma}%[Repeated Overflow Stack Section]
    If $\VerifyRule{g}{p}{d}{nb}{\None}{k_1 \dplus i :: k_2 \dplus i :: k_3}$, \\
    then $\exists d'$,
    such that $\VerifyRule{g}{p}{d'}{nb}{\None}{k_1 \dplus i :: k_2 \dplus i :: k_2 \dplus i :: k_3}$.
    \label{lemma:repeated-rule-in-stack}
\end{lemma}

Finally, we present the main lemma
that we wanted to prove.
\Cref{lemma:stack-depth-eventual-constancy}
shows that,
if an evaluation with a stack depth limit
greater than the number of grammar rules
yields a result,
then any evaluation with an even greater stack depth limit
yields the same result.
The stacks can be different,
but they are irrelevant
for our purpose of identifying
left-recursive rules.

\begin{lemma}%[Stack Depth Limit Increase]
    If $\VerifyRule{g}{p}{d}{nb}{res}{k}$, and $d > \length{g}$, \\
    then, for any $d' \ge d$,
    $\exists k'$,
    such that $\VerifyRule{g}{p}{d'}{nb}{res}{k'}$.
    \label{lemma:stack-depth-eventual-constancy}
\end{lemma}

We now explain the proof of this lemma.
For the cases in which the evaluation
does not result in a stack overflow,
the proof follows from \Cref{lemma:stack-depth-monotonicity-not-lr-pattern}.
Now, in the case of a stack overflow,
we know from \Cref{lemma:stack-depth-lr-pattern}
that the length of the stack $k$ is equal to the stack depth limit $d$,
which, in this case, we assume to be greater than $n$, the number of grammar rules.
Therefore, $\length{k} > n$.
We know from \Cref{lemma:coherent-stack}
that the stack only contains valid grammar rule indices.
That is, $\forall i \in k, i < n$.
We use these two observations and the pigeonhole principle to conclude
that the stack must have at least one repeated rule.
From \Cref{lemma:repeated-rule-in-stack},
we show that we can increase the stack depth limit arbitrarily,
and it will still result in a stack overflow.

Having defined the algorithm
that checks if a pattern is free of left recursion,
we now use this definition to create a function
that performs this check for a list of patterns.
\Cref{fig:lverifyrule-function} defines this function,
\begin{figure}
    \centering
    \input{lverifyrulecomp}
    \caption{The left recursion detection function for lists of patterns.}
    \label{fig:lverifyrule-function}
\end{figure}
which receives a grammar, a list of patterns,
and a gas counter,
and returns an optional Boolean value
indicating whether all patterns in the grammar
are free of left recursion.

This new function provides values
for two of the parameters of the underlying function:
the stack depth limit $d$, initialized with $\length{g}+1$,
the lower bound from \Cref{lemma:stack-depth-eventual-constancy},
and the nullable accumulator $nb$, initialized with $false$.
We could have omitted the gas counter,
by providing the lower bound from \Cref{lemma:vr-termination},
but we decided to postpone
this omission to the top-most definition
of well-formedness in our formalization.

We provide a lower bound for the gas parameter,
for which the function returns some result.
Note that we're assuming that both the grammar $g$
and the list of rules $rs$ are coherent,
because they could be different.
In practice, however,
they will be the same.
In this case,
where $rs = g$,
the equation for the lower bound can be simplified
to $(\length{g} + 2) \cdot \size{g}$.
That is the origin of the lower bound of the
\textit{\verifygrammarname{}} function
as displayed in \Cref{fig:verifygrammargas}.

\begin{lemma}%[Verify Rules Termination]
    If $\lCoherent{g}{g}{true}$ and $\lCoherent{g}{rs}{true}$, \\
    then, $\forall gas \ge \size{rs} + (\length{g} + 1) \cdot \size{g}$,
    $\exists res\ \lverifyrulecomp{g}{rs}{gas} = \Some res$.
\end{lemma}

\begin{proof}
    The proof follows by induction on the list of rules $rs$,
    and from the gas lower bound for the function \textit{\verifyrulename{}}
    from \Cref{lemma:vr-termination},
    substituting the stack depth limit $d$ with $\length{g}+1$.
\end{proof}

We will use this function for
verifying that the grammar contains
no left-recursive rules,
since it's implemented as a list of rules.
Since we will also be using it in our proofs,
we will need an analogous inductive definition.
\Cref{fig:lverifyrule} defines this predicate,
\begin{figure}
    \centering
    \input{lverifyrule}
    \caption{The left recursion detection predicate for lists of patterns.}
    \label{fig:lverifyrule}
\end{figure}
which also receives a grammar and a list of patterns,
and yields a Boolean value indicating
whether all patterns in the list are free of left recursion.

This predicate differs from the function in one aspect.
While the function provides an exact value for the
stack depth limit, the predicate allows any stack depth limit
to identify a pattern as either nullable or non-nullable.
That is because, according to \Cref{lemma:stack-depth-monotonicity-not-lr-pattern},
the returned label stays constant with increasing stack depth limits in such cases.
In the general case, however,
a stack depth limit greater than
the number of rules in the grammar
is necessary.

We prove some lemmas about this predicate.
\Cref{lemma:lverifyrule-determinism}
states that this predicate is deterministic,
and \Cref{lemma:lverifyrule-follows}
states that it follows the fixed-point definition.

\begin{lemma}
    If $\lVerifyRule{g}{rs}{b_1}$
    and $\lVerifyRule{g}{rs}{b_2}$,
    then $b_1 = b_2$.
    \label{lemma:lverifyrule-determinism}
\end{lemma}

\begin{lemma}
    If $\lverifyrulecomp{g}{rs}{gas} = \Some b$,
    then $\lVerifyRule{g}{rs}{b}$.
    \label{lemma:lverifyrule-follows}
\end{lemma}

\Cref{lemma:lverifyrule-safety} states that,
if a list of patterns passes the check,
then every pattern in the list
passes the individual check,
being either nullable or non-nullable.

\begin{lemma}%[Verify Rules Safety]
    If $\lVerifyRule{g}{rs}{true}$, \\
    then, $\forall r \in rs, \exists d\ \exists b\ \exists k\ \VerifyRule{g}{r}{d}{nb}{\Some b}{k}$.
    \label{lemma:lverifyrule-safety}
\end{lemma}

Before we end this section,
there is one final lemma we would like to present,
which uses all the predicates of the verification algorithm
we have defined up until now.
\Cref{lemma:no-lr-rule-in-grammar} shows that,
if a grammar is free of incoherent and left-recursive rules,
then any coherent pattern is either nullable or non-nullable.

\begin{lemma}%[No Left-Recursive Rule in Grammar]
    If $\Coherent{g}{p}{true}$,
    and $\lCoherent{g}{g}{true}$, \\
    and $\lVerifyRule{g}{g}{true}$,
    then $\exists d\ \exists b\ \exists k$,
    such that $\VerifyRule{g}{p}{d}{nb}{\Some b}{k}$.
    \label{lemma:no-lr-rule-in-grammar}
\end{lemma}

    \caption{The left recursion detection predicate.}
    \label{fig:verifyrule}
\end{figure}
defines such predicate,
denoted as $\VerifyRule{g}{p}{d}{nb}{res}{k}$.
It takes a grammar $g$,
a pattern $p$,
a stack depth limit $d$,
and a nullable accumulator $nb$,
and outputs a result $res$,
and a stack trace $k$.

In order to reach our final goal
of proving that the label returned
by the \textit{\verifyrulename{}} function
is monotonic and eventually constant
with respect to the stack depth limit,
we need to first prove some intermediary lemmas.
First, we need to relate the predicate and
the fixed-point definition together,
so that we can apply the proofs about the
former to the latter.

We begin with
\Cref{lemma:vr-determinism},
which states that,
for identical input,
the predicate yields the same output.
We can therefore state
that the predicate is deterministic.

\begin{lemma}
    If $\VerifyRule{g}{p}{d}{nb}{res_1}{k_1}$,
    and $\VerifyRule{g}{p}{d}{nb}{res_2}{k_2}$, \\
    then $res_1 = res_2$ and $k_1 = k_2$.
    \label{lemma:vr-determinism}
\end{lemma}

\Cref{lemma:vr-follows} shows that
every result returned by the fixed-point definition
can be inductively constructed using the predicate definition.

\begin{lemma}
    If $\verifyrulecomp{g}{p}{d}{nb}{gas} = \Some (res, k)$, \\
    then $\VerifyRule{g}{p}{d}{nb}{res}{k}$.
    \label{lemma:vr-follows}
\end{lemma}

\Cref{lemma:stack-depth-monotonicity-not-lr-pattern} shows that,
if a pattern evaluates to either nullable or non-nullable,
then increasing the stack depth limit
doesn't affect the result.
This is expected,
because, in both cases,
it doesn't surpass the limit,
and increasing it preserves this property
by transitivity.

\begin{lemma}%[Stack Depth Limit Increase without Overflow]
    If $\VerifyRule{g}{p}{d}{nb}{\Some nb'}{k}$, \\
    then $\forall d' \ge d, \VerifyRule{g}{p}{d'}{nb}{\Some nb'}{k}$.
    \label{lemma:stack-depth-monotonicity-not-lr-pattern}
\end{lemma}

\Cref{lemma:stack-depth-lr-pattern} shows that,
on stack overflow,
the output stack $k$ has length $d$.
That is expected,
because a stack overflow happens
when the stack is full
before trying to visit a rule.

\begin{lemma}%[Stack Length on Overflow]
    If $\VerifyRule{g}{p}{d}{nb}{\None}{k}$,
    then $\length{k} = d$.
    \label{lemma:stack-depth-lr-pattern}
\end{lemma}

\Cref{lemma:coherent-stack} states
that the output stack
only contains references to
existing rules in the grammar.
Since we are identifying rules
by their indices in a list,
we prove this by showing that these indices
are less than the number of rules in the grammar,
denoted as $\length{g}$.
This lemma may seem trivial,
but it is necessary for us to later prove,
using the pigeonhole principle,
that a stack with more rules than the grammar
will have at least one repeated rule.

\begin{lemma}%[Coherent Stack]
    If $\VerifyRule{g}{p}{d}{nb}{res}{k}$,
    then $\forall i \in k, i < \length{g}$.
    \label{lemma:coherent-stack}
\end{lemma}

% Next, we prove in \Cref{lemma:true-as-nb} that,
% when provided with $true$ as the value
% for the nullable accumulator $nb$,
% the algorithm outputs as a result
% either $\None$ (which means ``left-recursive'')
% or $\Some true$ (which means ``nullable'').

% \begin{lemma}%[True As Nullable Accumulator]
%     For a grammar $g$,
%     a pattern $p$,
%     a natural number $d$,
%     an optional Boolean value $res$,
%     and a list of natural numbers $k$,
%     if $\VerifyRule{g}{p}{d}{true}{res}{k}$,
%     then $res \in \{\None,\ \Some true\}$.
%     \label{lemma:true-as-nb}
% \end{lemma}

% We also state in \Cref{lemma:nb-change-not-lr-pattern} that,
% when a pattern is identified as either nullable or non-nullable,
% then changing the nullable accumulator
% might change the result to either nullable or non-nullable,
% but the output stack $k$ will remain the same.
% Note that we're saying that it will not make
% the pattern be marked as left-recursive.

% \begin{lemma}%[Nullable Accumulator Change When Pattern Is Not Left-Recursive]
%     For a grammar $g$,
%     a pattern $p$,
%     a natural number $d$,
%     Boolean values $nb$, $nb'$ and $b$,
%     and a list of natural numbers $k$,
%     if $\VerifyRule{g}{p}{d}{nb}{\Some b}{k}$,
%     then there exists some Boolean value $b'$,
%     such that $\VerifyRule{g}{p}{d}{nb'}{\Some b'}{k}$.
%     \label{lemma:nb-change-not-lr-pattern}
% \end{lemma}

% Alternatively,
% when a pattern is identified as left-recursive,
% changing the nullable accumulator
% will change neither the result nor the stack $k$.
% This is put forth in \Cref{lemma:nb-change-lr-pattern}.

% \begin{lemma}%[Nullable Accumulator Change When Pattern Is Left-Recursive]
%     For a grammar $g$,
%     a pattern $p$,
%     a natural number $d$,
%     Boolean values $nb$ and $nb'$,
%     and a list of natural numbers $k$,
%     if $\VerifyRule{g}{p}{d}{nb}{\None}{k}$,
%     then $\VerifyRule{g}{p}{d}{nb'}{\None}{k}$.
%     \label{lemma:nb-change-lr-pattern}
% \end{lemma}

% The following lemmas will start to deal with the output stack more closely.
% It is used by \lpeg{} to format error messages,
% but we will use it to help us formalize certain properties about the algorithm.
% \Cref{lemma:existential-ff} states that,
% for any evaluation with an output stack $k_1 \dplus i :: k_2$,
% there exists a stack depth limit and a nullable accumulator,
% for which the evaluation of the nonterminal pattern $\PNT{i}$
% yields the output stack $i :: k_2$.
% We call this operation ``fast-forwarding''.

% \begin{lemma}%[Existential Fast-forward]
%     $\forall gas\ \forall p\ \forall d\ \forall nb\ \forall res\ \forall k_2\ \forall k_2\ \forall i$, \\
%     if $\VerifyRule{g}{p}{d}{nb}{res}{k_1 \dplus i :: k_2}$,
%     then $\exists d'\ \exists nb'\ \exists res'$,
%     such that $\VerifyRule{g}{\PNT i}{d'}{nb'}{res'}{i :: k_2}$.
%     \label{lemma:existential-ff}
% \end{lemma}

% \Cref{lemma:existential-ff-for-lr-patterns}
% is a particular case of \Cref{lemma:existential-ff},
% in which the evaluation of pattern $p$ overflows the stack,
% and, as a result, so does the evaluation of nonterminal pattern $\PNT{i}$.

% \begin{lemma}%[Existential Fast-forward On Overflow]
%     $\forall gas\ \forall p\ \forall d\ \forall nb\ \forall k_2\ \forall k_2\ \forall i$, \\
%     if $\VerifyRule{g}{p}{d}{nb}{\None}{k_1 \dplus i :: k_2}$,
%     then $\exists d'\ \exists nb'$,
%     such that $\VerifyRule{g}{\PNT i}{d'}{nb'}{\None}{i :: k_2}$.
%     \label{lemma:existential-ff-for-lr-patterns}
% \end{lemma}

\Cref{lemma:ff-for-lr-patterns} states that,
for any evaluation that results in a stack overflow,
we can pick any rule $i$ from the output stack $k$,
and evaluate it with a certain stack depth limit,
so that it also results in a stack overflow,
and returns a suffix of the original stack $k$,
starting from $i$.

\begin{lemma}%[Fast-forward on Overflow]
    If $\VerifyRule{g}{p}{d}{nb}{\None}{k_1 \dplus i :: k_2}$, \\
    then $\VerifyRule{g}{\PNT i}{1+\length{k_2}}{nb'}{\None}{i :: k_2}$.
    \label{lemma:ff-for-lr-patterns}
\end{lemma}

Under the same assumptions,
\Cref{lemma:d-increase-lr} shows that,
if we evaluate a rule $i$ from the stack
with an increased stack depth limit
and it still results in a stack overflow
and returns a stack $i :: k_3$,
then we can increase the stack depth limit
of the original evaluation by the same amount,
it will also result in a stack overflow,
and return a stack that ends with $i :: k_3$.

\begin{lemma}%[Increase Overflown Stack Depth Limit]
    If $\VerifyRule{g}{p}{d}{nb}{\None}{k_1 \dplus i :: k_2}$, \\
    and $\VerifyRule{g}{\PNT i}{1+\length{k_3}}{nb'}{\None}{i :: k_3}$,
    and $\length{k_2} \le \length{k_3}$, \\
    then $\VerifyRule{g}{p}{1 + \length{k_1} + \length{k_3}}{nb}{\None}{k_1 \dplus i :: k_3}$.
    \label{lemma:d-increase-lr}
\end{lemma}

\Cref{lemma:repeated-rule-in-stack} shows that,
if an evaluation results in a stack overflow,
and a rule $i$ occurs more than once in the output stack,
then we can increase the stack depth limit by a certain amount,
and both conditions will still hold true.

\begin{lemma}%[Repeated Overflow Stack Section]
    If $\VerifyRule{g}{p}{d}{nb}{\None}{k_1 \dplus i :: k_2 \dplus i :: k_3}$, \\
    then $\exists d'$,
    such that $\VerifyRule{g}{p}{d'}{nb}{\None}{k_1 \dplus i :: k_2 \dplus i :: k_2 \dplus i :: k_3}$.
    \label{lemma:repeated-rule-in-stack}
\end{lemma}

Finally, we present the main lemma
that we wanted to prove.
\Cref{lemma:stack-depth-eventual-constancy}
shows that,
if an evaluation with a stack depth limit
greater than the number of grammar rules
yields a result,
then any evaluation with an even greater stack depth limit
yields the same result.
The stacks can be different,
but they are irrelevant
for our purpose of identifying
left-recursive rules.

\begin{lemma}%[Stack Depth Limit Increase]
    If $\VerifyRule{g}{p}{d}{nb}{res}{k}$, and $d > \length{g}$, \\
    then, for any $d' \ge d$,
    $\exists k'$,
    such that $\VerifyRule{g}{p}{d'}{nb}{res}{k'}$.
    \label{lemma:stack-depth-eventual-constancy}
\end{lemma}

We now explain the proof of this lemma.
For the cases in which the evaluation
does not result in a stack overflow,
the proof follows from \Cref{lemma:stack-depth-monotonicity-not-lr-pattern}.
Now, in the case of a stack overflow,
we know from \Cref{lemma:stack-depth-lr-pattern}
that the length of the stack $k$ is equal to the stack depth limit $d$,
which, in this case, we assume to be greater than $n$, the number of grammar rules.
Therefore, $\length{k} > n$.
We know from \Cref{lemma:coherent-stack}
that the stack only contains valid grammar rule indices.
That is, $\forall i \in k, i < n$.
We use these two observations and the pigeonhole principle to conclude
that the stack must have at least one repeated rule.
From \Cref{lemma:repeated-rule-in-stack},
we show that we can increase the stack depth limit arbitrarily,
and it will still result in a stack overflow.

Having defined the algorithm
that checks if a pattern is free of left recursion,
we now use this definition to create a function
that performs this check for a list of patterns.
\Cref{fig:lverifyrule-function} defines this function,
\begin{figure}
    \centering
    \begin{equation*}
    \begin{aligned}[t]
        & \lverifyrulecomp{g}{rs}{gas} := \\
        & \begin{aligned}[t]
            & \matchwith{rs} \\
            & \matchcase{nil}{\Some true} \\
            & \matchcase{r::rs'}{\begin{aligned}[t]
                & \letin{d}{\length{g}+1} \\
                & \matchwith{\verifyrulecomp{g}{r}{d}{false}{gas}} \\
                & \matchcase{\Some (\Some b, k)}{\lverifyrulecomp{g}{rs'}{gas}} \\
                & \matchcase{\Some (\None, k)}{\Some false} \\
                & \matchcase{\None}{\None} \\
                & \matchend{}
            \end{aligned}} \\
            & \matchend{}
        \end{aligned}
    \end{aligned}
\end{equation*}


% \begin{equation}
%     \lverifyrulecomp(g,[\ ],gas) = \Some(true)
% \end{equation}

% \begin{equation}
%     \lverifyrulecomp(g,r::rs,gas) = \begin{cases}
%         \text{where } ores := \verifyrulecomp(g,r,1+|g|,false,gas) \\
%         \lverifyrulecomp(g,rs,gas), \text{if } ores = \Some(\Some(b),k) \\
%         \Some(false), \text{if } ores = \Some(\None, k) \\
%         \None, \text{if } ores = \None
%     \end{cases}
% \end{equation}

    \caption{The left recursion detection function for lists of patterns.}
    \label{fig:lverifyrule-function}
\end{figure}
which receives a grammar, a list of patterns,
and a gas counter,
and returns an optional Boolean value
indicating whether all patterns in the grammar
are free of left recursion.

This new function provides values
for two of the parameters of the underlying function:
the stack depth limit $d$, initialized with $\length{g}+1$,
the lower bound from \Cref{lemma:stack-depth-eventual-constancy},
and the nullable accumulator $nb$, initialized with $false$.
We could have omitted the gas counter,
by providing the lower bound from \Cref{lemma:vr-termination},
but we decided to postpone
this omission to the top-most definition
of well-formedness in our formalization.

We provide a lower bound for the gas parameter,
for which the function returns some result.
Note that we're assuming that both the grammar $g$
and the list of rules $rs$ are coherent,
because they could be different.
In practice, however,
they will be the same.
In this case,
where $rs = g$,
the equation for the lower bound can be simplified
to $(\length{g} + 2) \cdot \size{g}$.
That is the origin of the lower bound of the
\textit{\verifygrammarname{}} function
as displayed in \Cref{fig:verifygrammargas}.

\begin{lemma}%[Verify Rules Termination]
    If $\lCoherent{g}{g}{true}$ and $\lCoherent{g}{rs}{true}$, \\
    then, $\forall gas \ge \size{rs} + (\length{g} + 1) \cdot \size{g}$,
    $\exists res\ \lverifyrulecomp{g}{rs}{gas} = \Some res$.
\end{lemma}

\begin{proof}
    The proof follows by induction on the list of rules $rs$,
    and from the gas lower bound for the function \textit{\verifyrulename{}}
    from \Cref{lemma:vr-termination},
    substituting the stack depth limit $d$ with $\length{g}+1$.
\end{proof}

We will use this function for
verifying that the grammar contains
no left-recursive rules,
since it's implemented as a list of rules.
Since we will also be using it in our proofs,
we will need an analogous inductive definition.
\Cref{fig:lverifyrule} defines this predicate,
\begin{figure}
    \centering
    \begin{mathpar}
    \namedinferrule{lvr-nil}
    { }
    {\lVerifyRule{g}{nil}{true}}

    \namedinferrule{lvr-cons-some}
    {\VerifyRule{g}{r}{d}{false}{\Some nb}{k} \\ \lVerifyRule{g}{rs}{b}}
    {\lVerifyRule{g}{r::rs}{b}}

    \namedinferrule{lvr-cons-none}
    {|g| < d \\ \VerifyRule{g}{r}{d}{false}{\None}{k}}
    {\lVerifyRule{g}{r::rs}{false}}
\end{mathpar}
    \caption{The left recursion detection predicate for lists of patterns.}
    \label{fig:lverifyrule}
\end{figure}
which also receives a grammar and a list of patterns,
and yields a Boolean value indicating
whether all patterns in the list are free of left recursion.

This predicate differs from the function in one aspect.
While the function provides an exact value for the
stack depth limit, the predicate allows any stack depth limit
to identify a pattern as either nullable or non-nullable.
That is because, according to \Cref{lemma:stack-depth-monotonicity-not-lr-pattern},
the returned label stays constant with increasing stack depth limits in such cases.
In the general case, however,
a stack depth limit greater than
the number of rules in the grammar
is necessary.

We prove some lemmas about this predicate.
\Cref{lemma:lverifyrule-determinism}
states that this predicate is deterministic,
and \Cref{lemma:lverifyrule-follows}
states that it follows the fixed-point definition.

\begin{lemma}
    If $\lVerifyRule{g}{rs}{b_1}$
    and $\lVerifyRule{g}{rs}{b_2}$,
    then $b_1 = b_2$.
    \label{lemma:lverifyrule-determinism}
\end{lemma}

\begin{lemma}
    If $\lverifyrulecomp{g}{rs}{gas} = \Some b$,
    then $\lVerifyRule{g}{rs}{b}$.
    \label{lemma:lverifyrule-follows}
\end{lemma}

\Cref{lemma:lverifyrule-safety} states that,
if a list of patterns passes the check,
then every pattern in the list
passes the individual check,
being either nullable or non-nullable.

\begin{lemma}%[Verify Rules Safety]
    If $\lVerifyRule{g}{rs}{true}$, \\
    then, $\forall r \in rs, \exists d\ \exists b\ \exists k\ \VerifyRule{g}{r}{d}{nb}{\Some b}{k}$.
    \label{lemma:lverifyrule-safety}
\end{lemma}

Before we end this section,
there is one final lemma we would like to present,
which uses all the predicates of the verification algorithm
we have defined up until now.
\Cref{lemma:no-lr-rule-in-grammar} shows that,
if a grammar is free of incoherent and left-recursive rules,
then any coherent pattern is either nullable or non-nullable.

\begin{lemma}%[No Left-Recursive Rule in Grammar]
    If $\Coherent{g}{p}{true}$,
    and $\lCoherent{g}{g}{true}$, \\
    and $\lVerifyRule{g}{g}{true}$,
    then $\exists d\ \exists b\ \exists k$,
    such that $\VerifyRule{g}{p}{d}{nb}{\Some b}{k}$.
    \label{lemma:no-lr-rule-in-grammar}
\end{lemma}

    \caption{The left recursion detection predicate.}
    \label{fig:verifyrule}
\end{figure}
defines such predicate,
denoted as $\VerifyRule{g}{p}{d}{nb}{res}{k}$.
It takes a grammar $g$,
a pattern $p$,
a stack depth limit $d$,
and a nullable accumulator $nb$,
and outputs a result $res$,
and a stack trace $k$.

In order to reach our final goal
of proving that the label returned
by the \textit{\verifyrulename{}} function
is monotonic and eventually constant
with respect to the stack depth limit,
we need to first prove some intermediary lemmas.
First, we need to relate the predicate and
the fixed-point definition together,
so that we can apply the proofs about the
former to the latter.

We begin with
\Cref{lemma:vr-determinism},
which states that,
for identical input,
the predicate yields the same output.
We can therefore state
that the predicate is deterministic.

\begin{lemma}
    If $\VerifyRule{g}{p}{d}{nb}{res_1}{k_1}$,
    and $\VerifyRule{g}{p}{d}{nb}{res_2}{k_2}$, \\
    then $res_1 = res_2$ and $k_1 = k_2$.
    \label{lemma:vr-determinism}
\end{lemma}

\Cref{lemma:vr-follows} shows that
every result returned by the fixed-point definition
can be inductively constructed using the predicate definition.

\begin{lemma}
    If $\verifyrulecomp{g}{p}{d}{nb}{gas} = \Some (res, k)$, \\
    then $\VerifyRule{g}{p}{d}{nb}{res}{k}$.
    \label{lemma:vr-follows}
\end{lemma}

\Cref{lemma:stack-depth-monotonicity-not-lr-pattern} shows that,
if a pattern evaluates to either nullable or non-nullable,
then increasing the stack depth limit
doesn't affect the result.
This is expected,
because, in both cases,
it doesn't surpass the limit,
and increasing it preserves this property
by transitivity.

\begin{lemma}%[Stack Depth Limit Increase without Overflow]
    If $\VerifyRule{g}{p}{d}{nb}{\Some nb'}{k}$, \\
    then $\forall d' \ge d, \VerifyRule{g}{p}{d'}{nb}{\Some nb'}{k}$.
    \label{lemma:stack-depth-monotonicity-not-lr-pattern}
\end{lemma}

\Cref{lemma:stack-depth-lr-pattern} shows that,
on stack overflow,
the output stack $k$ has length $d$.
That is expected,
because a stack overflow happens
when the stack is full
before trying to visit a rule.

\begin{lemma}%[Stack Length on Overflow]
    If $\VerifyRule{g}{p}{d}{nb}{\None}{k}$,
    then $\length{k} = d$.
    \label{lemma:stack-depth-lr-pattern}
\end{lemma}

\Cref{lemma:coherent-stack} states
that the output stack
only contains references to
existing rules in the grammar.
Since we are identifying rules
by their indices in a list,
we prove this by showing that these indices
are less than the number of rules in the grammar,
denoted as $\length{g}$.
This lemma may seem trivial,
but it is necessary for us to later prove,
using the pigeonhole principle,
that a stack with more rules than the grammar
will have at least one repeated rule.

\begin{lemma}%[Coherent Stack]
    If $\VerifyRule{g}{p}{d}{nb}{res}{k}$,
    then $\forall i \in k, i < \length{g}$.
    \label{lemma:coherent-stack}
\end{lemma}

% Next, we prove in \Cref{lemma:true-as-nb} that,
% when provided with $true$ as the value
% for the nullable accumulator $nb$,
% the algorithm outputs as a result
% either $\None$ (which means ``left-recursive'')
% or $\Some true$ (which means ``nullable'').

% \begin{lemma}%[True As Nullable Accumulator]
%     For a grammar $g$,
%     a pattern $p$,
%     a natural number $d$,
%     an optional Boolean value $res$,
%     and a list of natural numbers $k$,
%     if $\VerifyRule{g}{p}{d}{true}{res}{k}$,
%     then $res \in \{\None,\ \Some true\}$.
%     \label{lemma:true-as-nb}
% \end{lemma}

% We also state in \Cref{lemma:nb-change-not-lr-pattern} that,
% when a pattern is identified as either nullable or non-nullable,
% then changing the nullable accumulator
% might change the result to either nullable or non-nullable,
% but the output stack $k$ will remain the same.
% Note that we're saying that it will not make
% the pattern be marked as left-recursive.

% \begin{lemma}%[Nullable Accumulator Change When Pattern Is Not Left-Recursive]
%     For a grammar $g$,
%     a pattern $p$,
%     a natural number $d$,
%     Boolean values $nb$, $nb'$ and $b$,
%     and a list of natural numbers $k$,
%     if $\VerifyRule{g}{p}{d}{nb}{\Some b}{k}$,
%     then there exists some Boolean value $b'$,
%     such that $\VerifyRule{g}{p}{d}{nb'}{\Some b'}{k}$.
%     \label{lemma:nb-change-not-lr-pattern}
% \end{lemma}

% Alternatively,
% when a pattern is identified as left-recursive,
% changing the nullable accumulator
% will change neither the result nor the stack $k$.
% This is put forth in \Cref{lemma:nb-change-lr-pattern}.

% \begin{lemma}%[Nullable Accumulator Change When Pattern Is Left-Recursive]
%     For a grammar $g$,
%     a pattern $p$,
%     a natural number $d$,
%     Boolean values $nb$ and $nb'$,
%     and a list of natural numbers $k$,
%     if $\VerifyRule{g}{p}{d}{nb}{\None}{k}$,
%     then $\VerifyRule{g}{p}{d}{nb'}{\None}{k}$.
%     \label{lemma:nb-change-lr-pattern}
% \end{lemma}

% The following lemmas will start to deal with the output stack more closely.
% It is used by \lpeg{} to format error messages,
% but we will use it to help us formalize certain properties about the algorithm.
% \Cref{lemma:existential-ff} states that,
% for any evaluation with an output stack $k_1 \dplus i :: k_2$,
% there exists a stack depth limit and a nullable accumulator,
% for which the evaluation of the nonterminal pattern $\PNT{i}$
% yields the output stack $i :: k_2$.
% We call this operation ``fast-forwarding''.

% \begin{lemma}%[Existential Fast-forward]
%     $\forall gas\ \forall p\ \forall d\ \forall nb\ \forall res\ \forall k_2\ \forall k_2\ \forall i$, \\
%     if $\VerifyRule{g}{p}{d}{nb}{res}{k_1 \dplus i :: k_2}$,
%     then $\exists d'\ \exists nb'\ \exists res'$,
%     such that $\VerifyRule{g}{\PNT i}{d'}{nb'}{res'}{i :: k_2}$.
%     \label{lemma:existential-ff}
% \end{lemma}

% \Cref{lemma:existential-ff-for-lr-patterns}
% is a particular case of \Cref{lemma:existential-ff},
% in which the evaluation of pattern $p$ overflows the stack,
% and, as a result, so does the evaluation of nonterminal pattern $\PNT{i}$.

% \begin{lemma}%[Existential Fast-forward On Overflow]
%     $\forall gas\ \forall p\ \forall d\ \forall nb\ \forall k_2\ \forall k_2\ \forall i$, \\
%     if $\VerifyRule{g}{p}{d}{nb}{\None}{k_1 \dplus i :: k_2}$,
%     then $\exists d'\ \exists nb'$,
%     such that $\VerifyRule{g}{\PNT i}{d'}{nb'}{\None}{i :: k_2}$.
%     \label{lemma:existential-ff-for-lr-patterns}
% \end{lemma}

\Cref{lemma:ff-for-lr-patterns} states that,
for any evaluation that results in a stack overflow,
we can pick any rule $i$ from the output stack $k$,
and evaluate it with a certain stack depth limit,
so that it also results in a stack overflow,
and returns a suffix of the original stack $k$,
starting from $i$.

\begin{lemma}%[Fast-forward on Overflow]
    If $\VerifyRule{g}{p}{d}{nb}{\None}{k_1 \dplus i :: k_2}$, \\
    then $\VerifyRule{g}{\PNT i}{1+\length{k_2}}{nb'}{\None}{i :: k_2}$.
    \label{lemma:ff-for-lr-patterns}
\end{lemma}

Under the same assumptions,
\Cref{lemma:d-increase-lr} shows that,
if we evaluate a rule $i$ from the stack
with an increased stack depth limit
and it still results in a stack overflow
and returns a stack $i :: k_3$,
then we can increase the stack depth limit
of the original evaluation by the same amount,
it will also result in a stack overflow,
and return a stack that ends with $i :: k_3$.

\begin{lemma}%[Increase Overflown Stack Depth Limit]
    If $\VerifyRule{g}{p}{d}{nb}{\None}{k_1 \dplus i :: k_2}$, \\
    and $\VerifyRule{g}{\PNT i}{1+\length{k_3}}{nb'}{\None}{i :: k_3}$,
    and $\length{k_2} \le \length{k_3}$, \\
    then $\VerifyRule{g}{p}{1 + \length{k_1} + \length{k_3}}{nb}{\None}{k_1 \dplus i :: k_3}$.
    \label{lemma:d-increase-lr}
\end{lemma}

\Cref{lemma:repeated-rule-in-stack} shows that,
if an evaluation results in a stack overflow,
and a rule $i$ occurs more than once in the output stack,
then we can increase the stack depth limit by a certain amount,
and both conditions will still hold true.

\begin{lemma}%[Repeated Overflow Stack Section]
    If $\VerifyRule{g}{p}{d}{nb}{\None}{k_1 \dplus i :: k_2 \dplus i :: k_3}$, \\
    then $\exists d'$,
    such that $\VerifyRule{g}{p}{d'}{nb}{\None}{k_1 \dplus i :: k_2 \dplus i :: k_2 \dplus i :: k_3}$.
    \label{lemma:repeated-rule-in-stack}
\end{lemma}

Finally, we present the main lemma
that we wanted to prove.
\Cref{lemma:stack-depth-eventual-constancy}
shows that,
if an evaluation with a stack depth limit
greater than the number of grammar rules
yields a result,
then any evaluation with an even greater stack depth limit
yields the same result.
The stacks can be different,
but they are irrelevant
for our purpose of identifying
left-recursive rules.

\begin{lemma}%[Stack Depth Limit Increase]
    If $\VerifyRule{g}{p}{d}{nb}{res}{k}$, and $d > \length{g}$, \\
    then, for any $d' \ge d$,
    $\exists k'$,
    such that $\VerifyRule{g}{p}{d'}{nb}{res}{k'}$.
    \label{lemma:stack-depth-eventual-constancy}
\end{lemma}

We now explain the proof of this lemma.
For the cases in which the evaluation
does not result in a stack overflow,
the proof follows from \Cref{lemma:stack-depth-monotonicity-not-lr-pattern}.
Now, in the case of a stack overflow,
we know from \Cref{lemma:stack-depth-lr-pattern}
that the length of the stack $k$ is equal to the stack depth limit $d$,
which, in this case, we assume to be greater than $n$, the number of grammar rules.
Therefore, $\length{k} > n$.
We know from \Cref{lemma:coherent-stack}
that the stack only contains valid grammar rule indices.
That is, $\forall i \in k, i < n$.
We use these two observations and the pigeonhole principle to conclude
that the stack must have at least one repeated rule.
From \Cref{lemma:repeated-rule-in-stack},
we show that we can increase the stack depth limit arbitrarily,
and it will still result in a stack overflow.

Having defined the algorithm
that checks if a pattern is free of left recursion,
we now use this definition to create a function
that performs this check for a list of patterns.
\Cref{fig:lverifyrule-function} defines this function,
\begin{figure}
    \centering
    \begin{equation*}
    \begin{aligned}[t]
        & \lverifyrulecomp{g}{rs}{gas} := \\
        & \begin{aligned}[t]
            & \matchwith{rs} \\
            & \matchcase{nil}{\Some true} \\
            & \matchcase{r::rs'}{\begin{aligned}[t]
                & \letin{d}{\length{g}+1} \\
                & \matchwith{\verifyrulecomp{g}{r}{d}{false}{gas}} \\
                & \matchcase{\Some (\Some b, k)}{\lverifyrulecomp{g}{rs'}{gas}} \\
                & \matchcase{\Some (\None, k)}{\Some false} \\
                & \matchcase{\None}{\None} \\
                & \matchend{}
            \end{aligned}} \\
            & \matchend{}
        \end{aligned}
    \end{aligned}
\end{equation*}


% \begin{equation}
%     \lverifyrulecomp(g,[\ ],gas) = \Some(true)
% \end{equation}

% \begin{equation}
%     \lverifyrulecomp(g,r::rs,gas) = \begin{cases}
%         \text{where } ores := \verifyrulecomp(g,r,1+|g|,false,gas) \\
%         \lverifyrulecomp(g,rs,gas), \text{if } ores = \Some(\Some(b),k) \\
%         \Some(false), \text{if } ores = \Some(\None, k) \\
%         \None, \text{if } ores = \None
%     \end{cases}
% \end{equation}

    \caption{The left recursion detection function for lists of patterns.}
    \label{fig:lverifyrule-function}
\end{figure}
which receives a grammar, a list of patterns,
and a gas counter,
and returns an optional Boolean value
indicating whether all patterns in the grammar
are free of left recursion.

This new function provides values
for two of the parameters of the underlying function:
the stack depth limit $d$, initialized with $\length{g}+1$,
the lower bound from \Cref{lemma:stack-depth-eventual-constancy},
and the nullable accumulator $nb$, initialized with $false$.
We could have omitted the gas counter,
by providing the lower bound from \Cref{lemma:vr-termination},
but we decided to postpone
this omission to the top-most definition
of well-formedness in our formalization.

We provide a lower bound for the gas parameter,
for which the function returns some result.
Note that we're assuming that both the grammar $g$
and the list of rules $rs$ are coherent,
because they could be different.
In practice, however,
they will be the same.
In this case,
where $rs = g$,
the equation for the lower bound can be simplified
to $(\length{g} + 2) \cdot \size{g}$.
That is the origin of the lower bound of the
\textit{\verifygrammarname{}} function
as displayed in \Cref{fig:verifygrammargas}.

\begin{lemma}%[Verify Rules Termination]
    If $\lCoherent{g}{g}{true}$ and $\lCoherent{g}{rs}{true}$, \\
    then, $\forall gas \ge \size{rs} + (\length{g} + 1) \cdot \size{g}$,
    $\exists res\ \lverifyrulecomp{g}{rs}{gas} = \Some res$.
\end{lemma}

\begin{proof}
    The proof follows by induction on the list of rules $rs$,
    and from the gas lower bound for the function \textit{\verifyrulename{}}
    from \Cref{lemma:vr-termination},
    substituting the stack depth limit $d$ with $\length{g}+1$.
\end{proof}

We will use this function for
verifying that the grammar contains
no left-recursive rules,
since it's implemented as a list of rules.
Since we will also be using it in our proofs,
we will need an analogous inductive definition.
\Cref{fig:lverifyrule} defines this predicate,
\begin{figure}
    \centering
    \begin{mathpar}
    \namedinferrule{lvr-nil}
    { }
    {\lVerifyRule{g}{nil}{true}}

    \namedinferrule{lvr-cons-some}
    {\VerifyRule{g}{r}{d}{false}{\Some nb}{k} \\ \lVerifyRule{g}{rs}{b}}
    {\lVerifyRule{g}{r::rs}{b}}

    \namedinferrule{lvr-cons-none}
    {|g| < d \\ \VerifyRule{g}{r}{d}{false}{\None}{k}}
    {\lVerifyRule{g}{r::rs}{false}}
\end{mathpar}
    \caption{The left recursion detection predicate for lists of patterns.}
    \label{fig:lverifyrule}
\end{figure}
which also receives a grammar and a list of patterns,
and yields a Boolean value indicating
whether all patterns in the list are free of left recursion.

This predicate differs from the function in one aspect.
While the function provides an exact value for the
stack depth limit, the predicate allows any stack depth limit
to identify a pattern as either nullable or non-nullable.
That is because, according to \Cref{lemma:stack-depth-monotonicity-not-lr-pattern},
the returned label stays constant with increasing stack depth limits in such cases.
In the general case, however,
a stack depth limit greater than
the number of rules in the grammar
is necessary.

We prove some lemmas about this predicate.
\Cref{lemma:lverifyrule-determinism}
states that this predicate is deterministic,
and \Cref{lemma:lverifyrule-follows}
states that it follows the fixed-point definition.

\begin{lemma}
    If $\lVerifyRule{g}{rs}{b_1}$
    and $\lVerifyRule{g}{rs}{b_2}$,
    then $b_1 = b_2$.
    \label{lemma:lverifyrule-determinism}
\end{lemma}

\begin{lemma}
    If $\lverifyrulecomp{g}{rs}{gas} = \Some b$,
    then $\lVerifyRule{g}{rs}{b}$.
    \label{lemma:lverifyrule-follows}
\end{lemma}

\Cref{lemma:lverifyrule-safety} states that,
if a list of patterns passes the check,
then every pattern in the list
passes the individual check,
being either nullable or non-nullable.

\begin{lemma}%[Verify Rules Safety]
    If $\lVerifyRule{g}{rs}{true}$, \\
    then, $\forall r \in rs, \exists d\ \exists b\ \exists k\ \VerifyRule{g}{r}{d}{nb}{\Some b}{k}$.
    \label{lemma:lverifyrule-safety}
\end{lemma}

Before we end this section,
there is one final lemma we would like to present,
which uses all the predicates of the verification algorithm
we have defined up until now.
\Cref{lemma:no-lr-rule-in-grammar} shows that,
if a grammar is free of incoherent and left-recursive rules,
then any coherent pattern is either nullable or non-nullable.

\begin{lemma}%[No Left-Recursive Rule in Grammar]
    If $\Coherent{g}{p}{true}$,
    and $\lCoherent{g}{g}{true}$, \\
    and $\lVerifyRule{g}{g}{true}$,
    then $\exists d\ \exists b\ \exists k$,
    such that $\VerifyRule{g}{p}{d}{nb}{\Some b}{k}$.
    \label{lemma:no-lr-rule-in-grammar}
\end{lemma}

    \caption{The left recursion detection predicate.}
    \label{fig:verifyrule}
\end{figure}

In order to reach our final goal
of proving that the function converges
with respect to the stack depth limit,
we need to first prove some intermediary lemmas.
First, we need to relate the predicate and
the fixed-point definition together,
so that we can apply the proofs about the
former to the latter.

We begin with
\Cref{lemma:vr-determinism},
which states that,
for identical input,
the predicate yields the same output.
We can therefore state
that the predicate is deterministic.

\begin{lemma}
    If $\VerifyRule{g}{p}{d}{nb}{res_1}{k_1}$,
    and $\VerifyRule{g}{p}{d}{nb}{res_2}{k_2}$, \\
    then $res_1 = res_2$ and $k_1 = k_2$.
    \label{lemma:vr-determinism}
\end{lemma}

\Cref{lemma:vr-follows} shows that
every result returned by the fixed-point definition
can be inductively constructed using the predicate definition.

\begin{lemma}
    If $\verifyrulecomp{g}{p}{d}{nb}{gas} = \Some (res, k)$, \\
    then $\VerifyRule{g}{p}{d}{nb}{res}{k}$.
    \label{lemma:vr-follows}
\end{lemma}

\Cref{lemma:stack-depth-convergence-not-lr-pattern} shows that,
if a pattern evaluates to either nullable or non-nullable,
then increasing the stack depth limit
doesn't affect the result.
This is expected,
because, in both cases,
it doesn't surpass the limit,
and increasing it preserves this property
by transitivity.

\begin{lemma}%[Stack Depth Limit Increase without Overflow]
    If $\VerifyRule{g}{p}{d}{nb}{\Some nb'}{k}$, \\
    then $\forall d' \ge d, \VerifyRule{g}{p}{d'}{nb}{\Some nb'}{k}$.
    \label{lemma:stack-depth-convergence-not-lr-pattern}
\end{lemma}

\Cref{lemma:stack-depth-lr-pattern} shows that,
on stack overflow,
the output stack $k$ has length $d$.
That is expected,
because a stack overflow happens
when the stack is full
before trying to visit a rule.

\begin{lemma}%[Stack Length on Overflow]
    If $\VerifyRule{g}{p}{d}{nb}{\None}{k}$,
    then $\length{k} = d$.
    \label{lemma:stack-depth-lr-pattern}
\end{lemma}

\Cref{lemma:coherent-stack} states
that the output stack
only contains references to
existing rules in the grammar.
Since we are identifying rules
by their indices in a list,
we prove this by showing that these indices
are less than the number of rules in the grammar,
denoted as $\length{g}$.
This lemma may seem trivial,
but it is necessary for us to later prove,
using the pigeonhole principle,
that a stack with more rules than the grammar
will have at least one repeated rule.

\begin{lemma}%[Coherent Stack]
    If $\VerifyRule{g}{p}{d}{nb}{res}{k}$,
    then $\forall i \in k, i < \length{g}$.
    \label{lemma:coherent-stack}
\end{lemma}

% Next, we prove in \Cref{lemma:true-as-nb} that,
% when provided with $true$ as the value
% for the nullable accumulator $nb$,
% the algorithm outputs as a result
% either $\None$ (which means ``left-recursive'')
% or $\Some true$ (which means ``nullable'').

% \begin{lemma}%[True As Nullable Accumulator]
%     For a grammar $g$,
%     a pattern $p$,
%     a natural number $d$,
%     an optional Boolean value $res$,
%     and a list of natural numbers $k$,
%     if $\VerifyRule{g}{p}{d}{true}{res}{k}$,
%     then $res \in \{\None,\ \Some true\}$.
%     \label{lemma:true-as-nb}
% \end{lemma}

% We also state in \Cref{lemma:nb-change-not-lr-pattern} that,
% when a pattern is identified as either nullable or non-nullable,
% then changing the nullable accumulator
% might change the result to either nullable or non-nullable,
% but the output stack $k$ will remain the same.
% Note that we're saying that it will not make
% the pattern be marked as left-recursive.

% \begin{lemma}%[Nullable Accumulator Change When Pattern Is Not Left-Recursive]
%     For a grammar $g$,
%     a pattern $p$,
%     a natural number $d$,
%     Boolean values $nb$, $nb'$ and $b$,
%     and a list of natural numbers $k$,
%     if $\VerifyRule{g}{p}{d}{nb}{\Some b}{k}$,
%     then there exists some Boolean value $b'$,
%     such that $\VerifyRule{g}{p}{d}{nb'}{\Some b'}{k}$.
%     \label{lemma:nb-change-not-lr-pattern}
% \end{lemma}

% Alternatively,
% when a pattern is identified as left-recursive,
% changing the nullable accumulator
% will change neither the result nor the stack $k$.
% This is put forth in \Cref{lemma:nb-change-lr-pattern}.

% \begin{lemma}%[Nullable Accumulator Change When Pattern Is Left-Recursive]
%     For a grammar $g$,
%     a pattern $p$,
%     a natural number $d$,
%     Boolean values $nb$ and $nb'$,
%     and a list of natural numbers $k$,
%     if $\VerifyRule{g}{p}{d}{nb}{\None}{k}$,
%     then $\VerifyRule{g}{p}{d}{nb'}{\None}{k}$.
%     \label{lemma:nb-change-lr-pattern}
% \end{lemma}

% The following lemmas will start to deal with the output stack more closely.
% It is used by \lpeg{} to format error messages,
% but we will use it to help us formalize certain properties about the algorithm.
% \Cref{lemma:existential-ff} states that,
% for any evaluation with an output stack $k_1 \dplus i :: k_2$,
% there exists a stack depth limit and a nullable accumulator,
% for which the evaluation of the nonterminal pattern $\PNT{i}$
% yields the output stack $i :: k_2$.
% We call this operation ``fast-forwarding''.

% \begin{lemma}%[Existential Fast-forward]
%     $\forall g\ \forall p\ \forall d\ \forall nb\ \forall res\ \forall k_2\ \forall k_2\ \forall i$, \\
%     if $\VerifyRule{g}{p}{d}{nb}{res}{k_1 \dplus i :: k_2}$,
%     then $\exists d'\ \exists nb'\ \exists res'$,
%     such that $\VerifyRule{g}{\PNT i}{d'}{nb'}{res'}{i :: k_2}$.
%     \label{lemma:existential-ff}
% \end{lemma}

% \Cref{lemma:existential-ff-for-lr-patterns}
% is a particular case of \Cref{lemma:existential-ff},
% in which the evaluation of pattern $p$ overflows the stack,
% and, as a result, so does the evaluation of nonterminal pattern $\PNT{i}$.

% \begin{lemma}%[Existential Fast-forward On Overflow]
%     $\forall g\ \forall p\ \forall d\ \forall nb\ \forall k_2\ \forall k_2\ \forall i$, \\
%     if $\VerifyRule{g}{p}{d}{nb}{\None}{k_1 \dplus i :: k_2}$,
%     then $\exists d'\ \exists nb'$,
%     such that $\VerifyRule{g}{\PNT i}{d'}{nb'}{\None}{i :: k_2}$.
%     \label{lemma:existential-ff-for-lr-patterns}
% \end{lemma}

\Cref{lemma:ff-for-lr-patterns} states that,
for any evaluation that results in a stack overflow,
we can pick any rule $i$ from the output stack $k$,
and evaluate it with a certain stack depth limit,
so that it also results in a stack overflow,
and returns a suffix of the original stack $k$,
starting from $i$.

\begin{lemma}%[Fast-forward on Overflow]
    If $\VerifyRule{g}{p}{d}{nb}{\None}{k_1 \dplus i :: k_2}$, \\
    then $\VerifyRule{g}{\PNT i}{1+\length{k_2}}{nb'}{\None}{i :: k_2}$.
    \label{lemma:ff-for-lr-patterns}
\end{lemma}

Under the same assumptions,
\Cref{lemma:d-increase-lr} shows that,
if we evaluate a rule $i$ from the stack
with an increased stack depth limit
and it still results in a stack overflow
and returns a stack $i :: k_3$,
then we can increase the stack depth limit
of the original evaluation by the same amount,
it will also result in a stack overflow,
and return a stack that ends with $i :: k_3$.

\begin{lemma}%[Increase Overflown Stack Depth Limit]
    If $\VerifyRule{g}{p}{d}{nb}{\None}{k_1 \dplus i :: k_2}$, \\
    and $\VerifyRule{g}{\PNT i}{1+\length{k_3}}{nb'}{\None}{i :: k_3}$,
    and $\length{k_2} \le \length{k_3}$, \\
    then $\VerifyRule{g}{p}{1 + \length{k_1} + \length{k_3}}{nb}{\None}{k_1 \dplus i :: k_3}$.
    \label{lemma:d-increase-lr}
\end{lemma}

\Cref{lemma:repeated-rule-in-stack} shows that,
if an evaluation results in a stack overflow,
and a rule $i$ occurs more than once in the output stack,
then we can increase the stack depth limit by a certain amount,
and both conditions will still hold true.

\begin{lemma}%[Repeated Overflow Stack Section]
    If $\VerifyRule{g}{p}{d}{nb}{\None}{k_1 \dplus i :: k_2 \dplus i :: k_3}$, \\
    then $\exists d'$,
    such that $\VerifyRule{g}{p}{d'}{nb}{\None}{k_1 \dplus i :: k_2 \dplus i :: k_2 \dplus i :: k_3}$.
    \label{lemma:repeated-rule-in-stack}
\end{lemma}

Finally, we present the main lemma
that we wanted to prove.
\Cref{lemma:stack-depth-convergence}
shows that,
if an evaluation with a stack depth limit
greater than the number of grammar rules
yields a result,
then any evaluation with an even greater stack depth limit
yields the same result.
The stacks can be different,
but they are irrelevant
for our purpose of identifying
left-recursive rules.

\begin{lemma}%[Stack Depth Limit Increase]
    If $\VerifyRule{g}{p}{d}{nb}{res}{k}$, and $d > \length{g}$, \\
    then, for any $d' \ge d$,
    $\exists k'$,
    such that $\VerifyRule{g}{p}{d'}{nb}{res}{k'}$.
    \label{lemma:stack-depth-convergence}
\end{lemma}

We now explain the proof of this lemma.
For the cases in which the evaluation
does not result in a stack overflow,
the proof follows from \Cref{lemma:stack-depth-convergence-not-lr-pattern}.
Now, in the case of a stack overflow,
we know from \Cref{lemma:stack-depth-lr-pattern}
that the length of the stack $k$ is equal to the stack depth limit $d$,
which, in this case, we assume to be greater than $n$, the number of grammar rules.
Therefore, $\length{k} > n$.
We know from \Cref{lemma:coherent-stack}
that the stack only contains valid grammar rule indices.
That is, $\forall i \in k, i < n$.
We use these two observations and the pigeonhole principle to conclude
that the stack must have at least one repeated rule.
From \Cref{lemma:repeated-rule-in-stack},
we show that we can increase the stack depth limit arbitrarily,
and it will still result in a stack overflow.

Having defined the algorithm
that checks if a pattern is free of left recursion,
we now use this definition to create a function
that performs this check for a list of patterns.
\Cref{fig:lverifyrule-function} defines this function,
which receives a grammar, a list of patterns,
and a gas counter,
and returns an optional Boolean value
indicating whether all patterns in the grammar
are free of left recursion.

\begin{figure}
    \centering
    \begin{equation*}
    \begin{aligned}[t]
        & \lverifyrulecomp{g}{rs}{gas} := \\
        & \begin{aligned}[t]
            & \matchwith{rs} \\
            & \matchcase{nil}{\Some true} \\
            & \matchcase{r::rs'}{\begin{aligned}[t]
                & \letin{d}{\length{g}+1} \\
                & \matchwith{\verifyrulecomp{g}{r}{d}{false}{gas}} \\
                & \matchcase{\Some (\Some b, k)}{\lverifyrulecomp{g}{rs'}{gas}} \\
                & \matchcase{\Some (\None, k)}{\Some false} \\
                & \matchcase{\None}{\None} \\
                & \matchend{}
            \end{aligned}} \\
            & \matchend{}
        \end{aligned}
    \end{aligned}
\end{equation*}


% \begin{equation}
%     \lverifyrulecomp(g,[\ ],gas) = \Some(true)
% \end{equation}

% \begin{equation}
%     \lverifyrulecomp(g,r::rs,gas) = \begin{cases}
%         \text{where } ores := \verifyrulecomp(g,r,1+|g|,false,gas) \\
%         \lverifyrulecomp(g,rs,gas), \text{if } ores = \Some(\Some(b),k) \\
%         \Some(false), \text{if } ores = \Some(\None, k) \\
%         \None, \text{if } ores = \None
%     \end{cases}
% \end{equation}

    \caption{The left recursion detection function for lists of patterns.}
    \label{fig:lverifyrule-function}
\end{figure}

This new function provides values
for two of the parameters of the underlying function:
the stack depth limit, with the lower bound from \Cref{lemma:stack-depth-convergence},
and the nullable accumulator, initialized with $false$.
We could have omitted the gas counter,
by providing the lower bound from \Cref{lemma:vr-termination},
but we decided to postpone
this omission to the top-most definition
of well-formedness in our formalization.

We provide a lower bound for the gas parameter,
for which the function returns some result.
Note that we're assuming that both the grammar $g$
and the list of rules $rs$ are coherent,
because they could be different.
In practice, however,
they will be the same.

\begin{lemma}%[Verify Rules Termination]
    If $\lCoherent{g}{g}{true}$ and $\lCoherent{g}{rs}{true}$, \\
    then, $\forall g \ge \size{rs} + (\length{g} + 1) \cdot \size{g}$,
    $\exists res\ \lverifyrulecomp{g}{rs}{gas} = \Some res$.
\end{lemma}

We will use this function for
verifying that the grammar contains
no left-recursive rules,
since it's implemented as a list of rules.
Since we will also be using it in our proofs,
we will need an analogous inductive definition.
\Cref{fig:lverifyrule} defines this predicate,
which also receives a grammar and a list of patterns,
and yields a Boolean value indicating
whether all patterns in the list are free of left recursion.

\begin{figure}
    \centering
    \begin{mathpar}
    \namedinferrule{lvr-nil}
    { }
    {\lVerifyRule{g}{nil}{true}}

    \namedinferrule{lvr-cons-some}
    {\VerifyRule{g}{r}{d}{false}{\Some nb}{k} \\ \lVerifyRule{g}{rs}{b}}
    {\lVerifyRule{g}{r::rs}{b}}

    \namedinferrule{lvr-cons-none}
    {|g| < d \\ \VerifyRule{g}{r}{d}{false}{\None}{k}}
    {\lVerifyRule{g}{r::rs}{false}}
\end{mathpar}
    \caption{The left recursion detection predicate for lists of patterns.}
    \label{fig:lverifyrule}
\end{figure}

This predicate differs from the function in one aspect.
While the function provides an exact value for the
stack depth limit, the predicate allows any stack depth limit
to identify a pattern as either nullable or non-nullable.
That is because, according to \Cref{lemma:stack-depth-convergence-not-lr-pattern},
the predicate instantly converges with the stack depth limit in such cases.
In the general case, however,
a stack depth limit greater than
the number of rules in the grammar
is necessary.

We prove some lemmas about this predicate.
\Cref{lemma:lverifyrule-determinism}
states that this predicate is deterministic,
and \Cref{lemma:lverifyrule-follows}
states that it follows the fixed-point definition.

\begin{lemma}
    If $\lVerifyRule{g}{rs}{b_1}$
    and $\lVerifyRule{g}{rs}{b_2}$,
    then $b_1 = b_2$.
    \label{lemma:lverifyrule-determinism}
\end{lemma}

\begin{lemma}
    If $\lverifyrulecomp{g}{rs}{gas} = \Some b$,
    then $\lVerifyRule{g}{rs}{b}$.
    \label{lemma:lverifyrule-follows}
\end{lemma}

\Cref{lemma:lverifyrule-safety} states that,
if a list of patterns passes the check,
then every pattern in the list
passes the individual check,
being either nullable or non-nullable.

\begin{lemma}%[Verify Rules Safety]
    If $\lVerifyRule{g}{rs}{true}$, \\
    then, $\forall r \in rs, \exists d\ \exists b\ \exists k\ \VerifyRule{g}{r}{d}{nb}{\Some b}{k}$.
    \label{lemma:lverifyrule-safety}
\end{lemma}

Before we end this section,
there is one final lemma we would like to present,
which uses all the predicates of the verification algorithm
we have defined up until now.
\Cref{lemma:no-lr-rule-in-grammar} shows that,
if a grammar is free of incoherent and left-recursive rules,
then any coherent pattern is either nullable or non-nullable.

\begin{lemma}%[No Left-Recursive Rule in Grammar]
    If $\Coherent{g}{p}{true}$,
    and $\lCoherent{g}{g}{true}$, \\
    and $\lVerifyRule{g}{g}{true}$,
    then $\exists d\ \exists b\ \exists k$,
    such that $\VerifyRule{g}{p}{d}{nb}{\Some b}{k}$.
    \label{lemma:no-lr-rule-in-grammar}
\end{lemma}
