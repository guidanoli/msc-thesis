\chapter{Well-formedness Algorithm}
\label{chapter:wf-algorithm}

In this chapter,
we present the well-formedness algorithm
implemented in \lpeg{}
and prove its correctness.
We start with the overall structure
of the algorithm.
\Cref{fig:verifygrammar-function}
displays the function that implements the algorithm,
which can be seen as a sequence of four steps,
each of which we will get into later.
\begin{figure}
    \centering
    \begin{equation*}
    \begin{aligned}[t]
        & \verifygrammarcomp{g}{gas} = \\
        & \begin{aligned}[t]
            & \matchwith{\coherentfunc{g}{\PNT{0}}} \\
            & \matchcase{true}{\begin{aligned}[t]
                & \matchwith{\lcoherentfunc{g}{g}} \\
                & \matchcase{true}{\begin{aligned}[t]
                    & \matchwith{\lverifyrulecomp{g}{g}{gas}} \\
                    & \matchcase{\Some true}{\begin{aligned}[t]
                        & \matchwith{\lcheckloopscomp{g}{g}{gas}} \\
                        & \matchcase{\Some b}{\Some \neg b} \\
                        & \matchcase{\None}{\None} \\
                        & \matchend{}
                    \end{aligned}} \\
                    & \matchcase{res}{res} \\
                    & \matchend{}
                \end{aligned}} \\
                & \matchcase{false}{\Some false} \\
                & \matchend{}
            \end{aligned}} \\
            & \matchcase{false}{\Some false} \\
            & \matchend{}
        \end{aligned}
    \end{aligned}
\end{equation*}
    \caption{The well-formedness function with gas.}
    \label{fig:verifygrammar-function}
\end{figure}
The function \textit{\verifygrammarname{}}
takes a grammar and some gas as parameters,
and returns an optional Boolean value.
The gas counter serves merely to convince Coq
that the function terminates for any input.
If this counter ever reaches zero,
the function returns $\None$.
Otherwise, it returns $\Some b$,
where the Boolean value $b$
indicates whether the grammar
is well-formed or not.

We prove that,
for any input grammar $g$,
there exists a lower bound for the gas counter
for which \textit{\verifygrammarname{}} returns $\Some b$.
We do not need to assume anything
about the grammar, because the function
performs all the necessary checks
in the correct order.
This works as a proof of termination
for the algorithm.
\begin{lemma}
    \label{lemma:verifygrammar-termination}
    $\forall gas \ge \verifygrammargas{g}$,
    $\exists b\ \verifygrammarcomp{g}{gas} = \Some b$.
\end{lemma}

We define this lower bound
in \Cref{fig:verifygrammargas}.
\begin{figure}
    \centering
    \begin{equation*}
    \verifygrammargas{g} = (\length{g} + 2) \cdot \size{g}
\end{equation*}
    \caption{The well-formedness function gas lower bound.}
    \label{fig:verifygrammargas}
\end{figure}
The function takes into account
$\length{g}$, the number of rules in the grammar,
and $\size{g}$, the size of the grammar.
We define the size of a pattern $p$, also denoted as $\size{p}$,
as the number of nodes in its abstract syntax tree,
and the size of a grammar $g$
as the summation of the sizes of its rules,
that is, $\sum_{r \in g} \size{r}$.

With this lower bound,
we can define the function \textit{wf},
which takes a grammar $g$ and returns a Boolean value
indicating whether $g$ is well-formed.
\Cref{fig:wf} displays the implementation
of the function \textit{wf}.
It basically calls \textit{\verifygrammarname{}}
with $\verifygrammargas{g}$ as the gas counter.
If it returns $\Some b$, we simply return $b$.
Otherwise, we return $true$.
This default value is irrelevant,
because this case cannot happen.
Nevertheless, we return $true$
to demonstrate that the function
and gas estimation are correct.

\begin{figure}
    \centering
    \begin{equation*}
    \wf{g} = \begin{aligned}[t]
        & \matchwith{\verifygrammarcomp{g}{\verifygrammargas{g}}} \\
        & \matchcase{\Some b}{b} \\
        & \matchcase{\None}{true} \\
        & \matchend{}
    \end{aligned}
\end{equation*}
    \caption{The well-formedness function.}
    \label{fig:wf}
\end{figure}

Now that we have defined \textit{wf},
let us get into \textit{\verifygrammarname{}}.
In \Cref{fig:verifygrammar-function},
the reader may observe that
it calls four functions that we have not defined yet.
The first two functions,
\textit{\coherentname{}} and \textit{\lcoherentname{}},
do not use the gas counter,
because they operate recursively on the structure of patterns
and lists of patterns, respectively.
This type of structural recursion is enough for Coq
to determine that these functions terminate.
Meanwhile, the last two functions,
\textit{\lverifyrulename{}} and \textit{\lcheckloopsname{}},
need to visit rules recursively,
and are, therefore, defined recursively on a gas parameter.
Just like we did for
the \textit{\verifygrammarname{}} function,
we also prove that \textit{\lverifyrulename{}} and \textit{\lcheckloopsname{}}
terminate by providing a lower bound for the gas parameter.
(Without these proofs,
we would not be able to prove the termination
of \textit{\verifygrammarname{}}.)

Regarding the implementation of these functions,
we will get into each of them in the following sections.
For now, we will briefly explain what each of them does.
The first two steps are relatively simple,
while the third and fourth steps are more complex,
as they involve symbolically parsing each rule.

The first step of the algorithm is trivial.
It merely ensures that $\PNT{0}$,
the first rule of the grammar,
is defined,
given that it is used as the starting point
for parsing the grammar.
We implement this step with the function
\textit{\coherentname{}},
which takes a grammar $g$ and a pattern $p$
and returns a Boolean value
that indicates whether $p$ only references rules
that exist in $g$.

The second step is similar to the first one,
as it makes sure that every rule in the grammar
only references rules that exist in the grammar.
This ensures that we can safely
dereference any nonterminal patterns later on.
This step is implemented by the function \textit{\lcoherentname{}},
which is an extended version of \textit{\coherentname{}}
for lists of patterns.

The third step
ensures that the grammar contains no
left-recursive rules.
It does so by symbolically executing the parsing routine
for each rule in the grammar
and checking whether it can reach the same rule twice
without consuming any input.
This step is implemented by the function \textit{\lverifyrulename{}},
which takes a grammar $g$, a list of rules $rs$, and a gas counter
and returns an optional Boolean value
indicating whether all rules in the list $rs$
are not left-recursive.

The fourth and final step
looks for any degenerate loops,
which are repetitions of patterns
that may match while consuming no input.
This step is implemented by the function \textit{\lcheckloopsname{}},
and uses a simpler version
of the algorithm from the previous step.
It takes a grammar $g$, a list of rules $rs$, and a gas counter
and returns an optional Boolean value
indicating whether all rules in the list $rs$
are free of degenerate loops.

If a grammar passes all these checks,
then it is considered well-formed.
In the following sections,
we go into each of these steps
in greater detail.
We also define equivalent inductive predicates
for each step and for the \textit{\verifygrammarname{}} function,
to aid us in the proofs.
We also prove these predicates
follow the corresponding fixed-point definitions.

\section{References to nonexistent rules}
\label{section:coherent}

The verification process
starts by checking whether
every nonterminal pattern references
an existing rule in the grammar.
This process is quite simple,
but we present it here in the name of completeness.

We say a pattern is
\emph{coherent} in respect to a grammar
if all of its nonterminals reference existing rules in the grammar.
\Cref{fig:coherentfunc}
\begin{figure}
    \centering
    \begin{align*}
    \begin{aligned}[t]
        & \coherentfunc{g}{p} = \\
        & \begin{aligned}[t]
            & \matchwith{p} \\
            & \matchcase{\PEmpty}{true} \\
            & \matchcase{\PSet{cs}}{true} \\
            & \matchcase{\PNT{i}}{\begin{aligned}[t]
                & \matchwith{g[i]} \\
                & \matchcase{\Some p}{true} \\
                & \matchcase{\None}{false} \\
                & \matchend{}
            \end{aligned}} \\
            & \matchcase{\PRepetition{p}}{\coherentfunc{g}{p}} \\
            & \matchcase{\PNot{p}}{\coherentfunc{g}{p}} \\
            & \matchcase{\PAnd{p}}{\coherentfunc{g}{p}} \\
            & \matchcase{\PSequence{p_1}{p_2}}
                {\coherentfunc{g}{p_1} \wedge \coherentfunc{g}{p_2}} \\
            & \matchcase{\PChoice{p_1}{p_2}}
                {\coherentfunc{g}{p_1} \wedge \coherentfunc{g}{p_2}} \\
            & \matchend{}
        \end{aligned}
    \end{aligned}
\end{align*}
    \caption{The coherence function.}
    \label{fig:coherentfunc}
\end{figure}
defines a fixed-point function
that performs this verification.
To aid us in later induction proofs,
we also define an equivalent predicate in \Cref{fig:coherent}.
\begin{figure}
    \section{References to nonexistent rules}
\label{section:coherent}

The verification process
starts by checking whether
every nonterminal pattern references
an existing rule in the grammar.
This process is quite simple,
but we present it here in the name of completeness.

We say a pattern is
\emph{coherent} in respect to a grammar
if all of its nonterminals reference existing rules in the grammar.
\Cref{fig:coherentfunc}
\begin{figure}
    \centering
    \begin{align*}
    \begin{aligned}[t]
        & \coherentfunc{g}{p} = \\
        & \begin{aligned}[t]
            & \matchwith{p} \\
            & \matchcase{\PEmpty}{true} \\
            & \matchcase{\PSet{cs}}{true} \\
            & \matchcase{\PNT{i}}{\begin{aligned}[t]
                & \matchwith{g[i]} \\
                & \matchcase{\Some p}{true} \\
                & \matchcase{\None}{false} \\
                & \matchend{}
            \end{aligned}} \\
            & \matchcase{\PRepetition{p}}{\coherentfunc{g}{p}} \\
            & \matchcase{\PNot{p}}{\coherentfunc{g}{p}} \\
            & \matchcase{\PAnd{p}}{\coherentfunc{g}{p}} \\
            & \matchcase{\PSequence{p_1}{p_2}}
                {\coherentfunc{g}{p_1} \wedge \coherentfunc{g}{p_2}} \\
            & \matchcase{\PChoice{p_1}{p_2}}
                {\coherentfunc{g}{p_1} \wedge \coherentfunc{g}{p_2}} \\
            & \matchend{}
        \end{aligned}
    \end{aligned}
\end{align*}
    \caption{The coherence function.}
    \label{fig:coherentfunc}
\end{figure}
defines a fixed-point function
that performs this verification.
To aid us in later induction proofs,
we also define an equivalent predicate in \Cref{fig:coherent}.
\begin{figure}
    \section{References to nonexistent rules}
\label{section:coherent}

The verification process
starts by checking whether
every nonterminal pattern references
an existing rule in the grammar.
This process is quite simple,
but we present it here in the name of completeness.

We say a pattern is
\emph{coherent} in respect to a grammar
if all of its nonterminals reference existing rules in the grammar.
\Cref{fig:coherentfunc}
\begin{figure}
    \centering
    \input{coherentfunc}
    \caption{The coherence function.}
    \label{fig:coherentfunc}
\end{figure}
defines a fixed-point function
that performs this verification.
To aid us in later induction proofs,
we also define an equivalent predicate in \Cref{fig:coherent}.
\begin{figure}
    \input{coherent}
    \caption{The coherence predicate.}
    \label{fig:coherent}
\end{figure}
\Cref{lemma:coherent-deterministic} states
that the predicate is deterministic on the result,
and \Cref{lemma:coherent-follows} states that
the predicate follows the function.
It is easy to see that
both lemmas together imply that the
predicate is equivalent to the function.

% TODO: reduce these to just one lemma as A <-> B ?

\begin{lemma}
    \label{lemma:coherent-deterministic}
    If $\Coherent{g}{p}{res_1}$ and $\Coherent{g}{p}{res_2}$,
    then $res_1 = res_2$.
\end{lemma}

\begin{lemma}
    \label{lemma:coherent-follows}
    If $\coherentfunc{g}{p} = res$, then $\Coherent{g}{p}{res}$.
\end{lemma}

\Cref{fig:lcoherent-function} trivially generalizes the coherence check for a list of patterns.
\begin{figure}
    \centering
    \include{lcoherentfunc}
    \caption{The coherence function for lists of patterns.}
    \label{fig:lcoherent-function}
\end{figure}
This function is defined over an arbitrary list of rules,
but is meant to be called for the whole grammar.
We also define, in \Cref{fig:lcoherent}, an inductive predicate
\begin{figure}
    \centering
    \input{lcoherent}
    \caption{The coherence predicate for lists of patterns.}
    \label{fig:lcoherent}
\end{figure}
equivalent to this function to be
later used in proofs by induction.
We also show that this predicate
is deterministic and follows
the original fixed-point definition.
See \Cref{lemma:lcoherent-determinism,lemma:lcoherent-follows}.

\begin{lemma}
    If $\lCoherent{g}{rs}{res_1}$,
    and $\lCoherent{g}{rs}{res_2}$,
    then $res_1 = res_2$.
    \label{lemma:lcoherent-determinism}
\end{lemma}

\begin{lemma}
    If $\lcoherentfunc{g}{rs} = res$,
    then $\lCoherent{g}{rs}{res}$.
    \label{lemma:lcoherent-follows}
\end{lemma}

Finally, we prove that
if a list of patterns passes the list-based check,
then any pattern in the list passes the individual check.

\begin{lemma}%[Coherent List Safety]
    \label{lemma:lcoherent-safety}
    If $\lCoherent{g}{rs}{true}$,
    then, $\forall r \in rs, \Coherent{g}{r}{true}$.
\end{lemma}

    \caption{The coherence predicate.}
    \label{fig:coherent}
\end{figure}
\Cref{lemma:coherent-deterministic} states
that the predicate is deterministic on the result,
and \Cref{lemma:coherent-follows} states that
the predicate follows the function.
It is easy to see that
both lemmas together imply that the
predicate is equivalent to the function.

% TODO: reduce these to just one lemma as A <-> B ?

\begin{lemma}
    \label{lemma:coherent-deterministic}
    If $\Coherent{g}{p}{res_1}$ and $\Coherent{g}{p}{res_2}$,
    then $res_1 = res_2$.
\end{lemma}

\begin{lemma}
    \label{lemma:coherent-follows}
    If $\coherentfunc{g}{p} = res$, then $\Coherent{g}{p}{res}$.
\end{lemma}

\Cref{fig:lcoherent-function} trivially generalizes the coherence check for a list of patterns.
\begin{figure}
    \centering
    \begin{equation*}
    \lcoherentfunc{g}{rs} = \begin{aligned}[t]
        & \matchwith{rs} \\
        & \matchcase{nil}{true} \\
        & \matchcase{r::rs'}{\coherentfunc{g}{r} \wedge \lcoherentfunc{g}{rs'}} \\
        & \matchend{}
    \end{aligned}
\end{equation*}
    \caption{The coherence function for lists of patterns.}
    \label{fig:lcoherent-function}
\end{figure}
This function is defined over an arbitrary list of rules,
but is meant to be called for the whole grammar.
We also define, in \Cref{fig:lcoherent}, an inductive predicate
\begin{figure}
    \centering
    \begin{mathpar}
    \namedinferrule{lc-nil}
    { }
    {\lCoherent{g}{nil}{true}}

    \namedinferrule{lc-cons}
    {\Coherent{g}{r}{b_1} \\ \lCoherent{g}{rs}{b_2}}
    {\lCoherent{g}{r::rs}{b_1 \wedge b_2}}
\end{mathpar}
    \caption{The coherence predicate for lists of patterns.}
    \label{fig:lcoherent}
\end{figure}
equivalent to this function to be
later used in proofs by induction.
We also show that this predicate
is deterministic and follows
the original fixed-point definition.
See \Cref{lemma:lcoherent-determinism,lemma:lcoherent-follows}.

\begin{lemma}
    If $\lCoherent{g}{rs}{res_1}$,
    and $\lCoherent{g}{rs}{res_2}$,
    then $res_1 = res_2$.
    \label{lemma:lcoherent-determinism}
\end{lemma}

\begin{lemma}
    If $\lcoherentfunc{g}{rs} = res$,
    then $\lCoherent{g}{rs}{res}$.
    \label{lemma:lcoherent-follows}
\end{lemma}

Finally, we prove that
if a list of patterns passes the list-based check,
then any pattern in the list passes the individual check.

\begin{lemma}%[Coherent List Safety]
    \label{lemma:lcoherent-safety}
    If $\lCoherent{g}{rs}{true}$,
    then, $\forall r \in rs, \Coherent{g}{r}{true}$.
\end{lemma}

    \caption{The coherence predicate.}
    \label{fig:coherent}
\end{figure}
\Cref{lemma:coherent-deterministic} states
that the predicate is deterministic on the result,
and \Cref{lemma:coherent-follows} states that
the predicate follows the function.
It is easy to see that
both lemmas together imply that the
predicate is equivalent to the function.

% TODO: reduce these to just one lemma as A <-> B ?

\begin{lemma}
    \label{lemma:coherent-deterministic}
    If $\Coherent{g}{p}{res_1}$ and $\Coherent{g}{p}{res_2}$,
    then $res_1 = res_2$.
\end{lemma}

\begin{lemma}
    \label{lemma:coherent-follows}
    If $\coherentfunc{g}{p} = res$, then $\Coherent{g}{p}{res}$.
\end{lemma}

\Cref{fig:lcoherent-function} trivially generalizes the coherence check for a list of patterns.
\begin{figure}
    \centering
    \begin{equation*}
    \lcoherentfunc{g}{rs} = \begin{aligned}[t]
        & \matchwith{rs} \\
        & \matchcase{nil}{true} \\
        & \matchcase{r::rs'}{\coherentfunc{g}{r} \wedge \lcoherentfunc{g}{rs'}} \\
        & \matchend{}
    \end{aligned}
\end{equation*}
    \caption{The coherence function for lists of patterns.}
    \label{fig:lcoherent-function}
\end{figure}
This function is defined over an arbitrary list of rules,
but is meant to be called for the whole grammar.
We also define, in \Cref{fig:lcoherent}, an inductive predicate
\begin{figure}
    \centering
    \begin{mathpar}
    \namedinferrule{lc-nil}
    { }
    {\lCoherent{g}{nil}{true}}

    \namedinferrule{lc-cons}
    {\Coherent{g}{r}{b_1} \\ \lCoherent{g}{rs}{b_2}}
    {\lCoherent{g}{r::rs}{b_1 \wedge b_2}}
\end{mathpar}
    \caption{The coherence predicate for lists of patterns.}
    \label{fig:lcoherent}
\end{figure}
equivalent to this function to be
later used in proofs by induction.
We also show that this predicate
is deterministic and follows
the original fixed-point definition.
See \Cref{lemma:lcoherent-determinism,lemma:lcoherent-follows}.

\begin{lemma}
    If $\lCoherent{g}{rs}{res_1}$,
    and $\lCoherent{g}{rs}{res_2}$,
    then $res_1 = res_2$.
    \label{lemma:lcoherent-determinism}
\end{lemma}

\begin{lemma}
    If $\lcoherentfunc{g}{rs} = res$,
    then $\lCoherent{g}{rs}{res}$.
    \label{lemma:lcoherent-follows}
\end{lemma}

Finally, we prove that
if a list of patterns passes the list-based check,
then any pattern in the list passes the individual check.

\begin{lemma}%[Coherent List Safety]
    \label{lemma:lcoherent-safety}
    If $\lCoherent{g}{rs}{true}$,
    then, $\forall r \in rs, \Coherent{g}{r}{true}$.
\end{lemma}

\section{Left-recursive rules}
\label{section:lr-rules}

In general,
we consider a rule to be left-recursive
if it can wind up in itself
without consuming any input in-between.
This brings us to the heart
of the algorithm that
detects left-recursive rules.
On a high level,
it symbolically parses each rule,
until it either consumes some input,
visits some rule twice without consuming any input,
or simply finishes.

The algorithm categorizes patterns into three groups.
If a pattern can be parsed until its end without consuming any input,
it is said to be \emph{nullable}.
If, otherwise, it always consumes some input,
it is categorized as \emph{non-nullable}.
Alternatively, if it can lead to some rule twice,
without consuming any input,
it is categorized as left-recursive.

In order to check whether a pattern
is guaranteed to consume some input,
the algorithm uses a conservative approximation
proposed by Ford~\cite{ford_parsing_2004},
which makes two assumptions.
The first one is that $\PNot{p}$ may match,
and the second one is that,
in the case of $\PChoice{p_1}{p_2}$,
it may visit $p_2$,
without checking whether $p_1$ always matches.
For illustrative purposes,
we present a simple counterexample
for each assumption.

A counterexample for the first assumption
is the pattern $\PNot{\PEmpty}$,
which never matches.
Meanwhile, the second assumption doesn't hold
for the pattern $\PChoice{\PEmpty}{p_2}$,
because $\PEmpty$ always matches,
and, therefore, $p_2$ is never visited.
The reader might think that
these cases can be easily spotted,
by the simplicity of the counterexamples.
However,
Ford~\cite{ford_parsing_2004} proved that
the general case of this problem is undecidable.

One of the conditions for the algorithm to yield a result
is whenever it consumes some input.
As a result,
it exclusively visits patterns that may be reached
without consuming any input.
Therefore, if the algorithm revisits a rule,
this means a path exists in which
the parsing routine may reach the same rule
and with the same input string,
which would indicate that such rule is left-recursive.
We will now discuss possible ways to detect
when a rule has been visited twice.

One possible way to detect left-recursive rules
is through a set of visited rules,
which is checked and updated
every time a nonterminal pattern is visited.
For a grammar with $n$ rules,
this set could be implemented
as an array of $n$ Boolean values,
each representing a rule.
This method achieves
a computation and spatial
completity of $O(n)$.

Another approach,
which is simpler and takes less memory space,
uses a counter of visited rules,
which starts at zero
and gets incremented every time a rule is visited.
If this value ever surpasses the number of grammar rules,
then we know, by the pigeonhole principle,
that some rule has been visited more than once.
In the case of grammars with left-recursive rules,
we may visit more rules than necessary,
however, we are not particularly worried
about the performance of the algorithm
in the case of errors.

\lpeg{} adopts this last approach.
Our formalization follows \lpeg{},
though with a small twist:
instead of counting visited rules from zero until the limit,
we count to-be-visited rules from the limit down to zero.
This simplifies our formalization
by moving the limit calculation out of the algorithm body,
and letting the limit be passed down as a parameter instead.

At this point,
it is important to draw a distinction between
exhausting the counter of to-be-visited rules
and correctly identifying a left-recursive rule.
When the algorithm starts,
the counter is initialized with the provided limit.
It is then decremented every time a rule is visited.
If the counter ever reaches zero,
then attempting to visit any rule
will return an error.
The algorithm does not determine
whether this error indicates left recursion,
because it would require the algorithm to check
whether the limit is greater than the number of grammar rules.
Instead, we leave it to the caller
to provide a high enough limit,
in which case the algorithm indeed
correctly labels rules as left-recursive
by returning an error.

Because of this shift in responsibilities,
we adapt the nomenclature
for the counter parameter and associated error,
based on an analogy with call stacks.
If, every time a rule is visited, it were pushed onto a stack $k$,
then we could think of the counter parameter $d$ as the stack depth limit;
and surpassing it would be similar to a stack overflow error.

For the sole purpose of helping us prove certain properties about the algorithm,
we will also include this stack $k$, a list of rule indices, as an output,
though it doesn't affect the algorithm.
As we will soon see, it is either appended, passed along, or ignored.
It works as a trace of the inner workings of the function,
a high-level concept we only use for proving lemmas about this algorithm.
\lpeg{} also implements this output, but it is only used
when formatting error messages about left-recursive rules.

We now describe the algorithm
for detecting left-recursive rules,
starting with its inputs and outputs.
It receives a pattern, a grammar, and a stack depth limit,
and returns a label and a stack.
We represent labels by
optional Boolean values
$\Some true$ (nullable),
$\Some false$ (non-nullable) and
$\None$ (stack overflow error);
and stacks by either
$nil$ (an empty stack) or
$i :: k$ (a rule of index $i$ concatenated with a stack $k$).
For now, we will work with this signature,
but beware that the actual function,
displayed in \Cref{fig:vr-function},
\begin{figure}
    \centering
    \begin{align*}
    \begin{aligned}[t]
        & \verifyrulecomp{g}{p}{d}{nb}{0} = \None \\
        & \verifyrulecomp{g}{p}{d}{nb}{(1+gas)} = \\
        & \begin{aligned}[t]
            & \matchwith{p} \\
            & \matchcase{\PEmpty}{\Some(\Some true, nil)} \\
            & \matchcase{\PSet{cs}}{\Some(\Some nb, nil)} \\
            & \matchcase{\PRepetition{p'}}{\verifyrulecomp{g}{p'}{d}{true}{gas}} \\
            & \matchcase{\PNot{p'}}{\verifyrulecomp{g}{p'}{d}{true}{gas}} \\
            & \matchcase{\PAnd{p'}}{\verifyrulecomp{g}{p'}{d}{true}{gas}} \\
            & \matchcase{\PNT{i}}{\begin{aligned}[t]
                & \matchwith{g[i]} \\
                & \matchcase{\None}{\None} \\
                & \matchcase{\Some p'}{\begin{aligned}[t]
                    & \matchwith{d} \\
                    & \matchcase{0}{\Some(\None, nil)} \\
                    & \matchcase{1+d'}{\begin{aligned}[t]
                        & \matchwith{\verifyrulecomp{g}{p'}{d'}{nb}{gas}} \\
                        & \matchcase{\Some (res, k)}{\Some (res, i :: k)} \\
                        & \matchcase{\None}{\None} \\
                        & \matchend{}
                    \end{aligned}} \\
                    & \matchend{}
                \end{aligned}} \\
                & \matchend{}
            \end{aligned}} \\
            & \matchcase{\PSequence{p_1}{p_2}}{\begin{aligned}[t]
                & \matchwith{\verifyrulecomp{g}{p_1}{d}{false}{gas}} \\
                & \matchcase{\Some(\Some true, k)}{\verifyrulecomp{g}{p_2}{d}{nb}{gas}} \\
                & \matchcase{\Some(\Some false, k)}{\Some(\Some nb, k)} \\
                & \matchcase{res}{res} \\
                & \matchend{}
            \end{aligned}} \\
            & \matchcase{\PChoice{p_1}{p_2}}{\begin{aligned}[t]
                & \matchwith{\verifyrulecomp{g}{p_1}{d}{nb}{gas}} \\
                & \matchcase{\Some(\Some nb', k)}{\verifyrulecomp{g}{p_2}{d}{nb'}{gas}} \\
                & \matchcase{res}{res} \\
                & \matchend{}
            \end{aligned}} \\
            & \matchend{}
        \end{aligned}
    \end{aligned}
\end{align*}
    \caption{The left recursion detection function.}
    \label{fig:vr-function}
\end{figure}
receives an extra parameter
which we will introduce later in this section.

The function is defined recursively.
In most cases, it calls itself for each sub-pattern.
In the case of nonterminal patterns, however,
it calls itself for the referenced rule.
Furthermore, the function propagates any stack overflow errors.
This means that,
if some recursive call returns $\None$,
signaling a stack overflow,
and a stack $k$,
then the function also returns $\None$ and $k$.

For the empty pattern $\PEmpty$,
the function returns a label $\Some true$ and $nil$,
because it is nullable and doesn't visit any nonterminal.
We categorize it as nullable because it may match while consuming no input.
In particular, it always matches while consuming no input.

As for character set patterns $\PSet{cs}$,
the function returns $\Some false$ and $nil$,
because it is non-nullable,
meaning it always consumes some input when it matches.
It also doesn't visit any nonterminal.

For a nonterminal pattern $\PNT{i}$,
the function first checks the stack depth limit $d$.
If $d=0$, it returns $\None$ and $nil$,
signaling a stack overflow
and that it didn't visit any nonterminal.
Otherwise, if $d\ge1$,
then the function calls itself for the $i^{th}$ rule of the grammar,
while passing a stack depth limit of $d-1$.
If this recursive call returns a label $res$ and a stack $k$,
then the function returns $res$ and $i :: k$.
This way, the stack accumulates
the indices of the grammar rules
in the same order in which they are visited.

For a repetition pattern $\PRepetition{p}$,
the function evaluates $p$,
which returns $res$ and $k$,
to check for any stack overflow errors.
If $res\ne\None$,
then it returns $\Some true$ and $k$,
as it can match while consuming no input,
in case $p$ fails.
We assume that $p$ can fail because,
in the final verification step,
we ensure that $p$ is non-nullable,
and we know that non-nullable patterns
fail to match the empty string.

Predicate patterns $\PNot{p}$ and $\PAnd{p}$ are evaluated
in the same way as repetition patterns,
but for different reasons.
Repetition patterns are nullable
because they can always match without consuming any input.
Meanwhile, predicate patterns are nullable by approximation,
under the assumption that $p$ may match,
in the case of $\PAnd{p}$,
or fail to match,
in the case of $\PNot{p}$.

For a sequence pattern $\PSequence{p_1}{p_2}$,
the function first evaluates $p_1$.
If $p_1$ is non-nullable,
then so is the sequence $\PSequence{p_1}{p_2}$,
and the function returns the same label and stack as $p_1$.
Note that $p_2$ is not even evaluated in this case,
because it would be visited with a shorter input string
during parsing.
This is the only case in which
the nullable property comes into play
in this algorithm.
If, otherwise, $p_1$ is nullable,
then it evaluates $p_2$
and returns the same label and stack as $p_2$.

Finally, for a choice pattern $\PChoice{p_1}{p_2}$,
it first evaluates $p_1$.
If it returns $\Some b_1$,
indicating that $p_1$ did not overflow the stack,
then it evaluates $p_2$.
If it also returns $\Some b_2$,
then the function returns $\Some (b_1 \vee b_2)$
and the same stack as $p_2$.

The algorithm we've just described
is quite similar to the one implemented in \lpeg{}.
There is, however, one small difference
related to the use of tail calls
as an optimization technique.
In C, tail calls are implemented
with \texttt{goto} statements.
To apply this optimization technique,
\lpeg{} adds an extra parameter to the function
to work as an accumulator for the nullable property.
Without this accumulator parameter,
the evaluation of choice patterns $\PChoice{p_1}{p_2}$
would rely solely on recursion.
It would evaluate $p_1$ and $p_2$,
then perform a Boolean \scor{} operation on the results.

With the addition of a Boolean parameter $nb$,
we can turn the evaluation of $p_2$ into a tail call.
Instead of making the Boolean \scor{} operation explicitly,
we let the accumulator do it under-the-hood.
This works because,
in the base cases of the recursion,
in which the function would return
either $\Some true$ or $\Some false$,
we return instead $\Some (true \vee nb)$ and $\Some (false \vee nb)$,
which get simplified to $\Some true$ and $\Some nb$, respectively.
\Cref{fig:evalchoice} shows a Coq-like pseudocode of how choice patterns
\begin{figure}
    \centering
    \newcommand{\eval}[4]{eval\ #1\ #2\ #3\ #4}

\begin{align*}
    \eval{g}{\dsqb{\PChoice{p_1}{p_2}}}{d}{nb} =\
    & \matchwith{\eval{g}{p_1}{d}{nb}} \\
    & \matchcase{(\Some nb', k)}{\eval{g}{p_2}{d}{nb'}} \\
    & \matchcase{(\None, k)}{(\None, k)} \\
    & \matchend{}
\end{align*}
    \caption{Pseudocode of the evaluation of choice patterns.}
    \label{fig:evalchoice}
\end{figure}
are evaluated with the Boolean parameter $nb$.

We would also like to highlight
how this nullable accumulator allows
the evaluation of repetitions and predicates
to be rewritten as tail calls.
Previously,
we would have to check if $p$ evaluated to $\None$,
before returning $\Some true$.
Now, we can simply pass $true$ as the nullable accumulator,
which guarantees that, if $p$ does not evaluate to $\None$,
it evaluates to $\Some true$.

There are some ways in which this function
could be implemented in Coq as a fixed-point.
The classical way is to add a gas parameter,
which gets decremented in every recursive call.
We make the function return an optional value,
such that, if the gas parameter ever reaches zero,
it returns $\None$.
Other ways are providing a well-formedness proof,
or a measure function.
We choose the first strategy,
because it is the simplest to implement.

Finally,
\Cref{fig:vr-function} presents the algorithm
defined as a fixed-point function.
It returns an optional value,
because we adopted the gas strategy,
but also because it cannot evaluate
nonterminal patterns that
reference nonexistent rules.
In this case,
the function also returns $\None$.
In all other cases,
the function returns $\Some (res, k)$,
with $res$ being a label, and $k$, a stack.

At this point,
the reader should be warned that
we will not attempt to prove the correctness
of this function in isolation.
In fact, we will not even try to formally
define left-recursive rules.
This might frustrate the reader,
but we assure you that such proof will not be necessary.
Instead,
we will later prove the correctness of the whole algorithm
once we introduce all steps of the verification process.
In this section,
we will simply prove
that the label returned by the function
is monotonic and eventually constant
with respect to the gas counter and stack depth limit.

A function $f$ is said to be monotonically increasing
if, for any $x$ and $y$, such that $x \le y$,
it is always true that $f(x) \le f(y)$.
In the case of the \textit{\verifyrulename{}} function,
this will be true for the gas counter and stack depth limit parameters,
and the order between optionals is $\None < \Some res$, for any $res$.
This means that, if the function ever returns $\Some res$,
increasing the gas counter or the stack depth limit
will not alter the return value.

A function $f$ is eventually constant
if, for some $N$ and for any $x$ and $y$,
such that $x, y \ge N$,
it is true that $f(x) = f(y)$.
In the case of \textit{\verifyrulename{}} function,
this will be true for the gas counter and stack depth limit parameters
and for the label return value.
This means that both the gas counter and stack depth limit
have lower bounds for which the returned label stabilizes.

About this fixed-point definition,
we will initially prove some basic lemmas.
Starting with \Cref{lemma:vr-gas-monotonicity},
we state that, if the function returns $\Some (res, k)$,
then increasing the value of the gas parameter
will not change the result.
This is what we mean by
the function being monotonic and eventually constant
with respect to the gas counter.

\begin{lemma}%[Verify Rule Gas Eventual Constancy]
    If $\verifyrulecomp{g}{p}{d}{nb}{gas} = \Some (res, k)$, \\
    then $\forall gas' \ge gas$,
    $\verifyrulecomp{g}{p}{d}{nb}{gas'} = \Some (res, k)$.
    \label{lemma:vr-gas-monotonicity}
\end{lemma}

\Cref{lemma:vr-termination} states that,
for any coherent pattern and grammar,
there exists a lower bound for the gas parameter,
for which the function returns $\Some (res, k)$.
The lower bound
takes into account
the size of the pattern $\size{p}$,
the size of the grammar $\size{g}$,
and the stack depth limit $d$.

\begin{lemma}%[Verify Rule Termination]
    If $\Coherent{g}{p}{true}$,
    and $\lCoherent{g}{g}{true}$, \\
    then, $\forall gas \ge \size{p} + d \cdot \size{g}$,
    $\exists res\ \exists k\ \verifyrulecomp{g}{p}{d}{nb}{gas} = \Some (res, k)$.
    \label{lemma:vr-termination}
\end{lemma}

\begin{proof}
    For most patterns,
    the proof follows from induction on the pattern $p$.
    Meanwhile, for non-terminal patterns,
    the proof follows from induction on the stack depth limit $d$.
    We show below how the lower bound for a rule $r$ and stack depth limit $d$
    is derived from a non-terminal $\PNT{i}$ that references $r$ and stack depth limit $d+1$.
    We use \Cref{lemma:size-of-r-le-size-of-g} to show that $\size{g} \ge \size{r}$.
    \begin{align*}
        gas & \ge \size{\PNT{i}} + (d + 1) \cdot \size{g} \\
        & \ge 1 + (d + 1) \cdot \size{g} \\
        & \ge 1 + \size{g} + d \cdot \size{g} \\
        & \ge 1 + \size{r} + d \cdot \size{g}
    \end{align*}
\end{proof}

Now, we would like to
prove that the label returned by the \textit{\verifyrulename{}} function
is monotonic and eventually constant with respect to the
stack depth limit.
We discard the returned stack in this context because,
in the case of left-recursive rules,
the stack returned by the function will,
in fact,
diverge.
However, we are not interested in
the output stack, in this case.
What really matters to the following
steps of the verification process
is the label.
In particular, we would like to make sure
that no rule in the grammar is marked
with the label $\None$,
meaning ``stack overflow''.

In order to prove such lemma,
we realized an inductive, gasless predicate
would be better suited
than the fixed-point definition,
as it would be easier to perform proofs by induction,
and without having to deal with a gas parameter.
\Cref{fig:verifyrule}
\begin{figure}[ht!]
    \section{Left-recursive rules}
\label{section:lr-rules}

In general,
we consider a rule to be left-recursive
if it can wind up in itself
without consuming any input in-between.
This brings us to the heart
of the algorithm that
detects left-recursive rules.
On a high level,
it symbolically parses each rule,
until it either consumes some input,
visits some rule twice without consuming any input,
or simply finishes.

The algorithm categorizes patterns into three groups.
If a pattern can be parsed until its end without consuming any input,
it is said to be \emph{nullable}.
If, otherwise, it always consumes some input,
it is categorized as \emph{non-nullable}.
Alternatively, if it can lead to some rule twice,
without consuming any input,
it is categorized as left-recursive.

In order to check whether a pattern
is guaranteed to consume some input,
the algorithm uses a conservative approximation
proposed by Ford~\cite{ford_parsing_2004},
which makes two assumptions.
The first one is that $\PNot{p}$ may match,
and the second one is that,
in the case of $\PChoice{p_1}{p_2}$,
it may visit $p_2$,
without checking whether $p_1$ always matches.
For illustrative purposes,
we present a simple counterexample
for each assumption.

A counterexample for the first assumption
is the pattern $\PNot{\PEmpty}$,
which never matches.
Meanwhile, the second assumption doesn't hold
for the pattern $\PChoice{\PEmpty}{p_2}$,
because $\PEmpty$ always matches,
and, therefore, $p_2$ is never visited.
The reader might think that
these cases can be easily spotted,
by the simplicity of the counterexamples.
However,
Ford~\cite{ford_parsing_2004} proved that
the general case of this problem is undecidable.

One of the conditions for the algorithm to yield a result
is whenever it consumes some input.
As a result,
it exclusively visits patterns that may be reached
without consuming any input.
Therefore, if the algorithm revisits a rule,
this means a path exists in which
the parsing routine may reach the same rule
and with the same input string,
which would indicate that such rule is left-recursive.
We will now discuss possible ways to detect
when a rule has been visited twice.

One possible way to detect left-recursive rules
is through a set of visited rules,
which is checked and updated
every time a nonterminal pattern is visited.
For a grammar with $n$ rules,
this set could be implemented
as an array of $n$ Boolean values,
each representing a rule.
This method achieves
a computation and spatial
completity of $O(n)$.

Another approach,
which is simpler and takes less memory space,
uses a counter of visited rules,
which starts at zero
and gets incremented every time a rule is visited.
If this value ever surpasses the number of grammar rules,
then we know, by the pigeonhole principle,
that some rule has been visited more than once.
In the case of grammars with left-recursive rules,
we may visit more rules than necessary,
however, we are not particularly worried
about the performance of the algorithm
in the case of errors.

\lpeg{} adopts this last approach.
Our formalization follows \lpeg{},
though with a small twist:
instead of counting visited rules from zero until the limit,
we count to-be-visited rules from the limit down to zero.
This simplifies our formalization
by moving the limit calculation out of the algorithm body,
and letting the limit be passed down as a parameter instead.

At this point,
it is important to draw a distinction between
exhausting the counter of to-be-visited rules
and correctly identifying a left-recursive rule.
When the algorithm starts,
the counter is initialized with the provided limit.
It is then decremented every time a rule is visited.
If the counter ever reaches zero,
then attempting to visit any rule
will return an error.
The algorithm does not determine
whether this error indicates left recursion,
because it would require the algorithm to check
whether the limit is greater than the number of grammar rules.
Instead, we leave it to the caller
to provide a high enough limit,
in which case the algorithm indeed
correctly labels rules as left-recursive
by returning an error.

Because of this shift in responsibilities,
we adapt the nomenclature
for the counter parameter and associated error,
based on an analogy with call stacks.
If, every time a rule is visited, it were pushed onto a stack $k$,
then we could think of the counter parameter $d$ as the stack depth limit;
and surpassing it would be similar to a stack overflow error.

For the sole purpose of helping us prove certain properties about the algorithm,
we will also include this stack $k$, a list of rule indices, as an output,
though it doesn't affect the algorithm.
As we will soon see, it is either appended, passed along, or ignored.
It works as a trace of the inner workings of the function,
a high-level concept we only use for proving lemmas about this algorithm.
\lpeg{} also implements this output, but it is only used
when formatting error messages about left-recursive rules.

We now describe the algorithm
for detecting left-recursive rules,
starting with its inputs and outputs.
It receives a pattern, a grammar, and a stack depth limit,
and returns a label and a stack.
We represent labels by
optional Boolean values
$\Some true$ (nullable),
$\Some false$ (non-nullable) and
$\None$ (stack overflow error);
and stacks by either
$nil$ (an empty stack) or
$i :: k$ (a rule of index $i$ concatenated with a stack $k$).
For now, we will work with this signature,
but beware that the actual function,
displayed in \Cref{fig:vr-function},
\begin{figure}
    \centering
    \begin{align*}
    \begin{aligned}[t]
        & \verifyrulecomp{g}{p}{d}{nb}{0} = \None \\
        & \verifyrulecomp{g}{p}{d}{nb}{(1+gas)} = \\
        & \begin{aligned}[t]
            & \matchwith{p} \\
            & \matchcase{\PEmpty}{\Some(\Some true, nil)} \\
            & \matchcase{\PSet{cs}}{\Some(\Some nb, nil)} \\
            & \matchcase{\PRepetition{p'}}{\verifyrulecomp{g}{p'}{d}{true}{gas}} \\
            & \matchcase{\PNot{p'}}{\verifyrulecomp{g}{p'}{d}{true}{gas}} \\
            & \matchcase{\PAnd{p'}}{\verifyrulecomp{g}{p'}{d}{true}{gas}} \\
            & \matchcase{\PNT{i}}{\begin{aligned}[t]
                & \matchwith{g[i]} \\
                & \matchcase{\None}{\None} \\
                & \matchcase{\Some p'}{\begin{aligned}[t]
                    & \matchwith{d} \\
                    & \matchcase{0}{\Some(\None, nil)} \\
                    & \matchcase{1+d'}{\begin{aligned}[t]
                        & \matchwith{\verifyrulecomp{g}{p'}{d'}{nb}{gas}} \\
                        & \matchcase{\Some (res, k)}{\Some (res, i :: k)} \\
                        & \matchcase{\None}{\None} \\
                        & \matchend{}
                    \end{aligned}} \\
                    & \matchend{}
                \end{aligned}} \\
                & \matchend{}
            \end{aligned}} \\
            & \matchcase{\PSequence{p_1}{p_2}}{\begin{aligned}[t]
                & \matchwith{\verifyrulecomp{g}{p_1}{d}{false}{gas}} \\
                & \matchcase{\Some(\Some true, k)}{\verifyrulecomp{g}{p_2}{d}{nb}{gas}} \\
                & \matchcase{\Some(\Some false, k)}{\Some(\Some nb, k)} \\
                & \matchcase{res}{res} \\
                & \matchend{}
            \end{aligned}} \\
            & \matchcase{\PChoice{p_1}{p_2}}{\begin{aligned}[t]
                & \matchwith{\verifyrulecomp{g}{p_1}{d}{nb}{gas}} \\
                & \matchcase{\Some(\Some nb', k)}{\verifyrulecomp{g}{p_2}{d}{nb'}{gas}} \\
                & \matchcase{res}{res} \\
                & \matchend{}
            \end{aligned}} \\
            & \matchend{}
        \end{aligned}
    \end{aligned}
\end{align*}
    \caption{The left recursion detection function.}
    \label{fig:vr-function}
\end{figure}
receives an extra parameter
which we will introduce later in this section.

The function is defined recursively.
In most cases, it calls itself for each sub-pattern.
In the case of nonterminal patterns, however,
it calls itself for the referenced rule.
Furthermore, the function propagates any stack overflow errors.
This means that,
if some recursive call returns $\None$,
signaling a stack overflow,
and a stack $k$,
then the function also returns $\None$ and $k$.

For the empty pattern $\PEmpty$,
the function returns a label $\Some true$ and $nil$,
because it is nullable and doesn't visit any nonterminal.
We categorize it as nullable because it may match while consuming no input.
In particular, it always matches while consuming no input.

As for character set patterns $\PSet{cs}$,
the function returns $\Some false$ and $nil$,
because it is non-nullable,
meaning it always consumes some input when it matches.
It also doesn't visit any nonterminal.

For a nonterminal pattern $\PNT{i}$,
the function first checks the stack depth limit $d$.
If $d=0$, it returns $\None$ and $nil$,
signaling a stack overflow
and that it didn't visit any nonterminal.
Otherwise, if $d\ge1$,
then the function calls itself for the $i^{th}$ rule of the grammar,
while passing a stack depth limit of $d-1$.
If this recursive call returns a label $res$ and a stack $k$,
then the function returns $res$ and $i :: k$.
This way, the stack accumulates
the indices of the grammar rules
in the same order in which they are visited.

For a repetition pattern $\PRepetition{p}$,
the function evaluates $p$,
which returns $res$ and $k$,
to check for any stack overflow errors.
If $res\ne\None$,
then it returns $\Some true$ and $k$,
as it can match while consuming no input,
in case $p$ fails.
We assume that $p$ can fail because,
in the final verification step,
we ensure that $p$ is non-nullable,
and we know that non-nullable patterns
fail to match the empty string.

Predicate patterns $\PNot{p}$ and $\PAnd{p}$ are evaluated
in the same way as repetition patterns,
but for different reasons.
Repetition patterns are nullable
because they can always match without consuming any input.
Meanwhile, predicate patterns are nullable by approximation,
under the assumption that $p$ may match,
in the case of $\PAnd{p}$,
or fail to match,
in the case of $\PNot{p}$.

For a sequence pattern $\PSequence{p_1}{p_2}$,
the function first evaluates $p_1$.
If $p_1$ is non-nullable,
then so is the sequence $\PSequence{p_1}{p_2}$,
and the function returns the same label and stack as $p_1$.
Note that $p_2$ is not even evaluated in this case,
because it would be visited with a shorter input string
during parsing.
This is the only case in which
the nullable property comes into play
in this algorithm.
If, otherwise, $p_1$ is nullable,
then it evaluates $p_2$
and returns the same label and stack as $p_2$.

Finally, for a choice pattern $\PChoice{p_1}{p_2}$,
it first evaluates $p_1$.
If it returns $\Some b_1$,
indicating that $p_1$ did not overflow the stack,
then it evaluates $p_2$.
If it also returns $\Some b_2$,
then the function returns $\Some (b_1 \vee b_2)$
and the same stack as $p_2$.

The algorithm we've just described
is quite similar to the one implemented in \lpeg{}.
There is, however, one small difference
related to the use of tail calls
as an optimization technique.
In C, tail calls are implemented
with \texttt{goto} statements.
To apply this optimization technique,
\lpeg{} adds an extra parameter to the function
to work as an accumulator for the nullable property.
Without this accumulator parameter,
the evaluation of choice patterns $\PChoice{p_1}{p_2}$
would rely solely on recursion.
It would evaluate $p_1$ and $p_2$,
then perform a Boolean \scor{} operation on the results.

With the addition of a Boolean parameter $nb$,
we can turn the evaluation of $p_2$ into a tail call.
Instead of making the Boolean \scor{} operation explicitly,
we let the accumulator do it under-the-hood.
This works because,
in the base cases of the recursion,
in which the function would return
either $\Some true$ or $\Some false$,
we return instead $\Some (true \vee nb)$ and $\Some (false \vee nb)$,
which get simplified to $\Some true$ and $\Some nb$, respectively.
\Cref{fig:evalchoice} shows a Coq-like pseudocode of how choice patterns
\begin{figure}
    \centering
    \newcommand{\eval}[4]{eval\ #1\ #2\ #3\ #4}

\begin{align*}
    \eval{g}{\dsqb{\PChoice{p_1}{p_2}}}{d}{nb} =\
    & \matchwith{\eval{g}{p_1}{d}{nb}} \\
    & \matchcase{(\Some nb', k)}{\eval{g}{p_2}{d}{nb'}} \\
    & \matchcase{(\None, k)}{(\None, k)} \\
    & \matchend{}
\end{align*}
    \caption{Pseudocode of the evaluation of choice patterns.}
    \label{fig:evalchoice}
\end{figure}
are evaluated with the Boolean parameter $nb$.

We would also like to highlight
how this nullable accumulator allows
the evaluation of repetitions and predicates
to be rewritten as tail calls.
Previously,
we would have to check if $p$ evaluated to $\None$,
before returning $\Some true$.
Now, we can simply pass $true$ as the nullable accumulator,
which guarantees that, if $p$ does not evaluate to $\None$,
it evaluates to $\Some true$.

There are some ways in which this function
could be implemented in Coq as a fixed-point.
The classical way is to add a gas parameter,
which gets decremented in every recursive call.
We make the function return an optional value,
such that, if the gas parameter ever reaches zero,
it returns $\None$.
Other ways are providing a well-formedness proof,
or a measure function.
We choose the first strategy,
because it is the simplest to implement.

Finally,
\Cref{fig:vr-function} presents the algorithm
defined as a fixed-point function.
It returns an optional value,
because we adopted the gas strategy,
but also because it cannot evaluate
nonterminal patterns that
reference nonexistent rules.
In this case,
the function also returns $\None$.
In all other cases,
the function returns $\Some (res, k)$,
with $res$ being a label, and $k$, a stack.

At this point,
the reader should be warned that
we will not attempt to prove the correctness
of this function in isolation.
In fact, we will not even try to formally
define left-recursive rules.
This might frustrate the reader,
but we assure you that such proof will not be necessary.
Instead,
we will later prove the correctness of the whole algorithm
once we introduce all steps of the verification process.
In this section,
we will simply prove
that the label returned by the function
is monotonic and eventually constant
with respect to the gas counter and stack depth limit.

A function $f$ is said to be monotonically increasing
if, for any $x$ and $y$, such that $x \le y$,
it is always true that $f(x) \le f(y)$.
In the case of the \textit{\verifyrulename{}} function,
this will be true for the gas counter and stack depth limit parameters,
and the order between optionals is $\None < \Some res$, for any $res$.
This means that, if the function ever returns $\Some res$,
increasing the gas counter or the stack depth limit
will not alter the return value.

A function $f$ is eventually constant
if, for some $N$ and for any $x$ and $y$,
such that $x, y \ge N$,
it is true that $f(x) = f(y)$.
In the case of \textit{\verifyrulename{}} function,
this will be true for the gas counter and stack depth limit parameters
and for the label return value.
This means that both the gas counter and stack depth limit
have lower bounds for which the returned label stabilizes.

About this fixed-point definition,
we will initially prove some basic lemmas.
Starting with \Cref{lemma:vr-gas-monotonicity},
we state that, if the function returns $\Some (res, k)$,
then increasing the value of the gas parameter
will not change the result.
This is what we mean by
the function being monotonic and eventually constant
with respect to the gas counter.

\begin{lemma}%[Verify Rule Gas Eventual Constancy]
    If $\verifyrulecomp{g}{p}{d}{nb}{gas} = \Some (res, k)$, \\
    then $\forall gas' \ge gas$,
    $\verifyrulecomp{g}{p}{d}{nb}{gas'} = \Some (res, k)$.
    \label{lemma:vr-gas-monotonicity}
\end{lemma}

\Cref{lemma:vr-termination} states that,
for any coherent pattern and grammar,
there exists a lower bound for the gas parameter,
for which the function returns $\Some (res, k)$.
The lower bound
takes into account
the size of the pattern $\size{p}$,
the size of the grammar $\size{g}$,
and the stack depth limit $d$.

\begin{lemma}%[Verify Rule Termination]
    If $\Coherent{g}{p}{true}$,
    and $\lCoherent{g}{g}{true}$, \\
    then, $\forall gas \ge \size{p} + d \cdot \size{g}$,
    $\exists res\ \exists k\ \verifyrulecomp{g}{p}{d}{nb}{gas} = \Some (res, k)$.
    \label{lemma:vr-termination}
\end{lemma}

\begin{proof}
    For most patterns,
    the proof follows from induction on the pattern $p$.
    Meanwhile, for non-terminal patterns,
    the proof follows from induction on the stack depth limit $d$.
    We show below how the lower bound for a rule $r$ and stack depth limit $d$
    is derived from a non-terminal $\PNT{i}$ that references $r$ and stack depth limit $d+1$.
    We use \Cref{lemma:size-of-r-le-size-of-g} to show that $\size{g} \ge \size{r}$.
    \begin{align*}
        gas & \ge \size{\PNT{i}} + (d + 1) \cdot \size{g} \\
        & \ge 1 + (d + 1) \cdot \size{g} \\
        & \ge 1 + \size{g} + d \cdot \size{g} \\
        & \ge 1 + \size{r} + d \cdot \size{g}
    \end{align*}
\end{proof}

Now, we would like to
prove that the label returned by the \textit{\verifyrulename{}} function
is monotonic and eventually constant with respect to the
stack depth limit.
We discard the returned stack in this context because,
in the case of left-recursive rules,
the stack returned by the function will,
in fact,
diverge.
However, we are not interested in
the output stack, in this case.
What really matters to the following
steps of the verification process
is the label.
In particular, we would like to make sure
that no rule in the grammar is marked
with the label $\None$,
meaning ``stack overflow''.

In order to prove such lemma,
we realized an inductive, gasless predicate
would be better suited
than the fixed-point definition,
as it would be easier to perform proofs by induction,
and without having to deal with a gas parameter.
\Cref{fig:verifyrule}
\begin{figure}[ht!]
    \section{Left-recursive rules}
\label{section:lr-rules}

In general,
we consider a rule to be left-recursive
if it can wind up in itself
without consuming any input in-between.
This brings us to the heart
of the algorithm that
detects left-recursive rules.
On a high level,
it symbolically parses each rule,
until it either consumes some input,
visits some rule twice without consuming any input,
or simply finishes.

The algorithm categorizes patterns into three groups.
If a pattern can be parsed until its end without consuming any input,
it is said to be \emph{nullable}.
If, otherwise, it always consumes some input,
it is categorized as \emph{non-nullable}.
Alternatively, if it can lead to some rule twice,
without consuming any input,
it is categorized as left-recursive.

In order to check whether a pattern
is guaranteed to consume some input,
the algorithm uses a conservative approximation
proposed by Ford~\cite{ford_parsing_2004},
which makes two assumptions.
The first one is that $\PNot{p}$ may match,
and the second one is that,
in the case of $\PChoice{p_1}{p_2}$,
it may visit $p_2$,
without checking whether $p_1$ always matches.
For illustrative purposes,
we present a simple counterexample
for each assumption.

A counterexample for the first assumption
is the pattern $\PNot{\PEmpty}$,
which never matches.
Meanwhile, the second assumption doesn't hold
for the pattern $\PChoice{\PEmpty}{p_2}$,
because $\PEmpty$ always matches,
and, therefore, $p_2$ is never visited.
The reader might think that
these cases can be easily spotted,
by the simplicity of the counterexamples.
However,
Ford~\cite{ford_parsing_2004} proved that
the general case of this problem is undecidable.

One of the conditions for the algorithm to yield a result
is whenever it consumes some input.
As a result,
it exclusively visits patterns that may be reached
without consuming any input.
Therefore, if the algorithm revisits a rule,
this means a path exists in which
the parsing routine may reach the same rule
and with the same input string,
which would indicate that such rule is left-recursive.
We will now discuss possible ways to detect
when a rule has been visited twice.

One possible way to detect left-recursive rules
is through a set of visited rules,
which is checked and updated
every time a nonterminal pattern is visited.
For a grammar with $n$ rules,
this set could be implemented
as an array of $n$ Boolean values,
each representing a rule.
This method achieves
a computation and spatial
completity of $O(n)$.

Another approach,
which is simpler and takes less memory space,
uses a counter of visited rules,
which starts at zero
and gets incremented every time a rule is visited.
If this value ever surpasses the number of grammar rules,
then we know, by the pigeonhole principle,
that some rule has been visited more than once.
In the case of grammars with left-recursive rules,
we may visit more rules than necessary,
however, we are not particularly worried
about the performance of the algorithm
in the case of errors.

\lpeg{} adopts this last approach.
Our formalization follows \lpeg{},
though with a small twist:
instead of counting visited rules from zero until the limit,
we count to-be-visited rules from the limit down to zero.
This simplifies our formalization
by moving the limit calculation out of the algorithm body,
and letting the limit be passed down as a parameter instead.

At this point,
it is important to draw a distinction between
exhausting the counter of to-be-visited rules
and correctly identifying a left-recursive rule.
When the algorithm starts,
the counter is initialized with the provided limit.
It is then decremented every time a rule is visited.
If the counter ever reaches zero,
then attempting to visit any rule
will return an error.
The algorithm does not determine
whether this error indicates left recursion,
because it would require the algorithm to check
whether the limit is greater than the number of grammar rules.
Instead, we leave it to the caller
to provide a high enough limit,
in which case the algorithm indeed
correctly labels rules as left-recursive
by returning an error.

Because of this shift in responsibilities,
we adapt the nomenclature
for the counter parameter and associated error,
based on an analogy with call stacks.
If, every time a rule is visited, it were pushed onto a stack $k$,
then we could think of the counter parameter $d$ as the stack depth limit;
and surpassing it would be similar to a stack overflow error.

For the sole purpose of helping us prove certain properties about the algorithm,
we will also include this stack $k$, a list of rule indices, as an output,
though it doesn't affect the algorithm.
As we will soon see, it is either appended, passed along, or ignored.
It works as a trace of the inner workings of the function,
a high-level concept we only use for proving lemmas about this algorithm.
\lpeg{} also implements this output, but it is only used
when formatting error messages about left-recursive rules.

We now describe the algorithm
for detecting left-recursive rules,
starting with its inputs and outputs.
It receives a pattern, a grammar, and a stack depth limit,
and returns a label and a stack.
We represent labels by
optional Boolean values
$\Some true$ (nullable),
$\Some false$ (non-nullable) and
$\None$ (stack overflow error);
and stacks by either
$nil$ (an empty stack) or
$i :: k$ (a rule of index $i$ concatenated with a stack $k$).
For now, we will work with this signature,
but beware that the actual function,
displayed in \Cref{fig:vr-function},
\begin{figure}
    \centering
    \input{verifyrulecomp}
    \caption{The left recursion detection function.}
    \label{fig:vr-function}
\end{figure}
receives an extra parameter
which we will introduce later in this section.

The function is defined recursively.
In most cases, it calls itself for each sub-pattern.
In the case of nonterminal patterns, however,
it calls itself for the referenced rule.
Furthermore, the function propagates any stack overflow errors.
This means that,
if some recursive call returns $\None$,
signaling a stack overflow,
and a stack $k$,
then the function also returns $\None$ and $k$.

For the empty pattern $\PEmpty$,
the function returns a label $\Some true$ and $nil$,
because it is nullable and doesn't visit any nonterminal.
We categorize it as nullable because it may match while consuming no input.
In particular, it always matches while consuming no input.

As for character set patterns $\PSet{cs}$,
the function returns $\Some false$ and $nil$,
because it is non-nullable,
meaning it always consumes some input when it matches.
It also doesn't visit any nonterminal.

For a nonterminal pattern $\PNT{i}$,
the function first checks the stack depth limit $d$.
If $d=0$, it returns $\None$ and $nil$,
signaling a stack overflow
and that it didn't visit any nonterminal.
Otherwise, if $d\ge1$,
then the function calls itself for the $i^{th}$ rule of the grammar,
while passing a stack depth limit of $d-1$.
If this recursive call returns a label $res$ and a stack $k$,
then the function returns $res$ and $i :: k$.
This way, the stack accumulates
the indices of the grammar rules
in the same order in which they are visited.

For a repetition pattern $\PRepetition{p}$,
the function evaluates $p$,
which returns $res$ and $k$,
to check for any stack overflow errors.
If $res\ne\None$,
then it returns $\Some true$ and $k$,
as it can match while consuming no input,
in case $p$ fails.
We assume that $p$ can fail because,
in the final verification step,
we ensure that $p$ is non-nullable,
and we know that non-nullable patterns
fail to match the empty string.

Predicate patterns $\PNot{p}$ and $\PAnd{p}$ are evaluated
in the same way as repetition patterns,
but for different reasons.
Repetition patterns are nullable
because they can always match without consuming any input.
Meanwhile, predicate patterns are nullable by approximation,
under the assumption that $p$ may match,
in the case of $\PAnd{p}$,
or fail to match,
in the case of $\PNot{p}$.

For a sequence pattern $\PSequence{p_1}{p_2}$,
the function first evaluates $p_1$.
If $p_1$ is non-nullable,
then so is the sequence $\PSequence{p_1}{p_2}$,
and the function returns the same label and stack as $p_1$.
Note that $p_2$ is not even evaluated in this case,
because it would be visited with a shorter input string
during parsing.
This is the only case in which
the nullable property comes into play
in this algorithm.
If, otherwise, $p_1$ is nullable,
then it evaluates $p_2$
and returns the same label and stack as $p_2$.

Finally, for a choice pattern $\PChoice{p_1}{p_2}$,
it first evaluates $p_1$.
If it returns $\Some b_1$,
indicating that $p_1$ did not overflow the stack,
then it evaluates $p_2$.
If it also returns $\Some b_2$,
then the function returns $\Some (b_1 \vee b_2)$
and the same stack as $p_2$.

The algorithm we've just described
is quite similar to the one implemented in \lpeg{}.
There is, however, one small difference
related to the use of tail calls
as an optimization technique.
In C, tail calls are implemented
with \texttt{goto} statements.
To apply this optimization technique,
\lpeg{} adds an extra parameter to the function
to work as an accumulator for the nullable property.
Without this accumulator parameter,
the evaluation of choice patterns $\PChoice{p_1}{p_2}$
would rely solely on recursion.
It would evaluate $p_1$ and $p_2$,
then perform a Boolean \scor{} operation on the results.

With the addition of a Boolean parameter $nb$,
we can turn the evaluation of $p_2$ into a tail call.
Instead of making the Boolean \scor{} operation explicitly,
we let the accumulator do it under-the-hood.
This works because,
in the base cases of the recursion,
in which the function would return
either $\Some true$ or $\Some false$,
we return instead $\Some (true \vee nb)$ and $\Some (false \vee nb)$,
which get simplified to $\Some true$ and $\Some nb$, respectively.
\Cref{fig:evalchoice} shows a Coq-like pseudocode of how choice patterns
\begin{figure}
    \centering
    \input{evalchoice}
    \caption{Pseudocode of the evaluation of choice patterns.}
    \label{fig:evalchoice}
\end{figure}
are evaluated with the Boolean parameter $nb$.

We would also like to highlight
how this nullable accumulator allows
the evaluation of repetitions and predicates
to be rewritten as tail calls.
Previously,
we would have to check if $p$ evaluated to $\None$,
before returning $\Some true$.
Now, we can simply pass $true$ as the nullable accumulator,
which guarantees that, if $p$ does not evaluate to $\None$,
it evaluates to $\Some true$.

There are some ways in which this function
could be implemented in Coq as a fixed-point.
The classical way is to add a gas parameter,
which gets decremented in every recursive call.
We make the function return an optional value,
such that, if the gas parameter ever reaches zero,
it returns $\None$.
Other ways are providing a well-formedness proof,
or a measure function.
We choose the first strategy,
because it is the simplest to implement.

Finally,
\Cref{fig:vr-function} presents the algorithm
defined as a fixed-point function.
It returns an optional value,
because we adopted the gas strategy,
but also because it cannot evaluate
nonterminal patterns that
reference nonexistent rules.
In this case,
the function also returns $\None$.
In all other cases,
the function returns $\Some (res, k)$,
with $res$ being a label, and $k$, a stack.

At this point,
the reader should be warned that
we will not attempt to prove the correctness
of this function in isolation.
In fact, we will not even try to formally
define left-recursive rules.
This might frustrate the reader,
but we assure you that such proof will not be necessary.
Instead,
we will later prove the correctness of the whole algorithm
once we introduce all steps of the verification process.
In this section,
we will simply prove
that the label returned by the function
is monotonic and eventually constant
with respect to the gas counter and stack depth limit.

A function $f$ is said to be monotonically increasing
if, for any $x$ and $y$, such that $x \le y$,
it is always true that $f(x) \le f(y)$.
In the case of the \textit{\verifyrulename{}} function,
this will be true for the gas counter and stack depth limit parameters,
and the order between optionals is $\None < \Some res$, for any $res$.
This means that, if the function ever returns $\Some res$,
increasing the gas counter or the stack depth limit
will not alter the return value.

A function $f$ is eventually constant
if, for some $N$ and for any $x$ and $y$,
such that $x, y \ge N$,
it is true that $f(x) = f(y)$.
In the case of \textit{\verifyrulename{}} function,
this will be true for the gas counter and stack depth limit parameters
and for the label return value.
This means that both the gas counter and stack depth limit
have lower bounds for which the returned label stabilizes.

About this fixed-point definition,
we will initially prove some basic lemmas.
Starting with \Cref{lemma:vr-gas-monotonicity},
we state that, if the function returns $\Some (res, k)$,
then increasing the value of the gas parameter
will not change the result.
This is what we mean by
the function being monotonic and eventually constant
with respect to the gas counter.

\begin{lemma}%[Verify Rule Gas Eventual Constancy]
    If $\verifyrulecomp{g}{p}{d}{nb}{gas} = \Some (res, k)$, \\
    then $\forall gas' \ge gas$,
    $\verifyrulecomp{g}{p}{d}{nb}{gas'} = \Some (res, k)$.
    \label{lemma:vr-gas-monotonicity}
\end{lemma}

\Cref{lemma:vr-termination} states that,
for any coherent pattern and grammar,
there exists a lower bound for the gas parameter,
for which the function returns $\Some (res, k)$.
The lower bound
takes into account
the size of the pattern $\size{p}$,
the size of the grammar $\size{g}$,
and the stack depth limit $d$.

\begin{lemma}%[Verify Rule Termination]
    If $\Coherent{g}{p}{true}$,
    and $\lCoherent{g}{g}{true}$, \\
    then, $\forall gas \ge \size{p} + d \cdot \size{g}$,
    $\exists res\ \exists k\ \verifyrulecomp{g}{p}{d}{nb}{gas} = \Some (res, k)$.
    \label{lemma:vr-termination}
\end{lemma}

\begin{proof}
    For most patterns,
    the proof follows from induction on the pattern $p$.
    Meanwhile, for non-terminal patterns,
    the proof follows from induction on the stack depth limit $d$.
    We show below how the lower bound for a rule $r$ and stack depth limit $d$
    is derived from a non-terminal $\PNT{i}$ that references $r$ and stack depth limit $d+1$.
    We use \Cref{lemma:size-of-r-le-size-of-g} to show that $\size{g} \ge \size{r}$.
    \begin{align*}
        gas & \ge \size{\PNT{i}} + (d + 1) \cdot \size{g} \\
        & \ge 1 + (d + 1) \cdot \size{g} \\
        & \ge 1 + \size{g} + d \cdot \size{g} \\
        & \ge 1 + \size{r} + d \cdot \size{g}
    \end{align*}
\end{proof}

Now, we would like to
prove that the label returned by the \textit{\verifyrulename{}} function
is monotonic and eventually constant with respect to the
stack depth limit.
We discard the returned stack in this context because,
in the case of left-recursive rules,
the stack returned by the function will,
in fact,
diverge.
However, we are not interested in
the output stack, in this case.
What really matters to the following
steps of the verification process
is the label.
In particular, we would like to make sure
that no rule in the grammar is marked
with the label $\None$,
meaning ``stack overflow''.

In order to prove such lemma,
we realized an inductive, gasless predicate
would be better suited
than the fixed-point definition,
as it would be easier to perform proofs by induction,
and without having to deal with a gas parameter.
\Cref{fig:verifyrule}
\begin{figure}[ht!]
    \input{verifyrule}
    \caption{The left recursion detection predicate.}
    \label{fig:verifyrule}
\end{figure}
defines such predicate,
denoted as $\VerifyRule{g}{p}{d}{nb}{res}{k}$.
It takes a grammar $g$,
a pattern $p$,
a stack depth limit $d$,
and a nullable accumulator $nb$,
and outputs a result $res$,
and a stack trace $k$.

In order to reach our final goal
of proving that the label returned
by the \textit{\verifyrulename{}} function
is monotonic and eventually constant
with respect to the stack depth limit,
we need to first prove some intermediary lemmas.
First, we need to relate the predicate and
the fixed-point definition together,
so that we can apply the proofs about the
former to the latter.

We begin with
\Cref{lemma:vr-determinism},
which states that,
for identical input,
the predicate yields the same output.
We can therefore state
that the predicate is deterministic.

\begin{lemma}
    If $\VerifyRule{g}{p}{d}{nb}{res_1}{k_1}$,
    and $\VerifyRule{g}{p}{d}{nb}{res_2}{k_2}$, \\
    then $res_1 = res_2$ and $k_1 = k_2$.
    \label{lemma:vr-determinism}
\end{lemma}

\Cref{lemma:vr-follows} shows that
every result returned by the fixed-point definition
can be inductively constructed using the predicate definition.

\begin{lemma}
    If $\verifyrulecomp{g}{p}{d}{nb}{gas} = \Some (res, k)$, \\
    then $\VerifyRule{g}{p}{d}{nb}{res}{k}$.
    \label{lemma:vr-follows}
\end{lemma}

\Cref{lemma:stack-depth-monotonicity-not-lr-pattern} shows that,
if a pattern evaluates to either nullable or non-nullable,
then increasing the stack depth limit
doesn't affect the result.
This is expected,
because, in both cases,
it doesn't surpass the limit,
and increasing it preserves this property
by transitivity.

\begin{lemma}%[Stack Depth Limit Increase without Overflow]
    If $\VerifyRule{g}{p}{d}{nb}{\Some nb'}{k}$, \\
    then $\forall d' \ge d, \VerifyRule{g}{p}{d'}{nb}{\Some nb'}{k}$.
    \label{lemma:stack-depth-monotonicity-not-lr-pattern}
\end{lemma}

\Cref{lemma:stack-depth-lr-pattern} shows that,
on stack overflow,
the output stack $k$ has length $d$.
That is expected,
because a stack overflow happens
when the stack is full
before trying to visit a rule.

\begin{lemma}%[Stack Length on Overflow]
    If $\VerifyRule{g}{p}{d}{nb}{\None}{k}$,
    then $\length{k} = d$.
    \label{lemma:stack-depth-lr-pattern}
\end{lemma}

\Cref{lemma:coherent-stack} states
that the output stack
only contains references to
existing rules in the grammar.
Since we are identifying rules
by their indices in a list,
we prove this by showing that these indices
are less than the number of rules in the grammar,
denoted as $\length{g}$.
This lemma may seem trivial,
but it is necessary for us to later prove,
using the pigeonhole principle,
that a stack with more rules than the grammar
will have at least one repeated rule.

\begin{lemma}%[Coherent Stack]
    If $\VerifyRule{g}{p}{d}{nb}{res}{k}$,
    then $\forall i \in k, i < \length{g}$.
    \label{lemma:coherent-stack}
\end{lemma}

% Next, we prove in \Cref{lemma:true-as-nb} that,
% when provided with $true$ as the value
% for the nullable accumulator $nb$,
% the algorithm outputs as a result
% either $\None$ (which means ``left-recursive'')
% or $\Some true$ (which means ``nullable'').

% \begin{lemma}%[True As Nullable Accumulator]
%     For a grammar $g$,
%     a pattern $p$,
%     a natural number $d$,
%     an optional Boolean value $res$,
%     and a list of natural numbers $k$,
%     if $\VerifyRule{g}{p}{d}{true}{res}{k}$,
%     then $res \in \{\None,\ \Some true\}$.
%     \label{lemma:true-as-nb}
% \end{lemma}

% We also state in \Cref{lemma:nb-change-not-lr-pattern} that,
% when a pattern is identified as either nullable or non-nullable,
% then changing the nullable accumulator
% might change the result to either nullable or non-nullable,
% but the output stack $k$ will remain the same.
% Note that we're saying that it will not make
% the pattern be marked as left-recursive.

% \begin{lemma}%[Nullable Accumulator Change When Pattern Is Not Left-Recursive]
%     For a grammar $g$,
%     a pattern $p$,
%     a natural number $d$,
%     Boolean values $nb$, $nb'$ and $b$,
%     and a list of natural numbers $k$,
%     if $\VerifyRule{g}{p}{d}{nb}{\Some b}{k}$,
%     then there exists some Boolean value $b'$,
%     such that $\VerifyRule{g}{p}{d}{nb'}{\Some b'}{k}$.
%     \label{lemma:nb-change-not-lr-pattern}
% \end{lemma}

% Alternatively,
% when a pattern is identified as left-recursive,
% changing the nullable accumulator
% will change neither the result nor the stack $k$.
% This is put forth in \Cref{lemma:nb-change-lr-pattern}.

% \begin{lemma}%[Nullable Accumulator Change When Pattern Is Left-Recursive]
%     For a grammar $g$,
%     a pattern $p$,
%     a natural number $d$,
%     Boolean values $nb$ and $nb'$,
%     and a list of natural numbers $k$,
%     if $\VerifyRule{g}{p}{d}{nb}{\None}{k}$,
%     then $\VerifyRule{g}{p}{d}{nb'}{\None}{k}$.
%     \label{lemma:nb-change-lr-pattern}
% \end{lemma}

% The following lemmas will start to deal with the output stack more closely.
% It is used by \lpeg{} to format error messages,
% but we will use it to help us formalize certain properties about the algorithm.
% \Cref{lemma:existential-ff} states that,
% for any evaluation with an output stack $k_1 \dplus i :: k_2$,
% there exists a stack depth limit and a nullable accumulator,
% for which the evaluation of the nonterminal pattern $\PNT{i}$
% yields the output stack $i :: k_2$.
% We call this operation ``fast-forwarding''.

% \begin{lemma}%[Existential Fast-forward]
%     $\forall gas\ \forall p\ \forall d\ \forall nb\ \forall res\ \forall k_2\ \forall k_2\ \forall i$, \\
%     if $\VerifyRule{g}{p}{d}{nb}{res}{k_1 \dplus i :: k_2}$,
%     then $\exists d'\ \exists nb'\ \exists res'$,
%     such that $\VerifyRule{g}{\PNT i}{d'}{nb'}{res'}{i :: k_2}$.
%     \label{lemma:existential-ff}
% \end{lemma}

% \Cref{lemma:existential-ff-for-lr-patterns}
% is a particular case of \Cref{lemma:existential-ff},
% in which the evaluation of pattern $p$ overflows the stack,
% and, as a result, so does the evaluation of nonterminal pattern $\PNT{i}$.

% \begin{lemma}%[Existential Fast-forward On Overflow]
%     $\forall gas\ \forall p\ \forall d\ \forall nb\ \forall k_2\ \forall k_2\ \forall i$, \\
%     if $\VerifyRule{g}{p}{d}{nb}{\None}{k_1 \dplus i :: k_2}$,
%     then $\exists d'\ \exists nb'$,
%     such that $\VerifyRule{g}{\PNT i}{d'}{nb'}{\None}{i :: k_2}$.
%     \label{lemma:existential-ff-for-lr-patterns}
% \end{lemma}

\Cref{lemma:ff-for-lr-patterns} states that,
for any evaluation that results in a stack overflow,
we can pick any rule $i$ from the output stack $k$,
and evaluate it with a certain stack depth limit,
so that it also results in a stack overflow,
and returns a suffix of the original stack $k$,
starting from $i$.

\begin{lemma}%[Fast-forward on Overflow]
    If $\VerifyRule{g}{p}{d}{nb}{\None}{k_1 \dplus i :: k_2}$, \\
    then $\VerifyRule{g}{\PNT i}{1+\length{k_2}}{nb'}{\None}{i :: k_2}$.
    \label{lemma:ff-for-lr-patterns}
\end{lemma}

Under the same assumptions,
\Cref{lemma:d-increase-lr} shows that,
if we evaluate a rule $i$ from the stack
with an increased stack depth limit
and it still results in a stack overflow
and returns a stack $i :: k_3$,
then we can increase the stack depth limit
of the original evaluation by the same amount,
it will also result in a stack overflow,
and return a stack that ends with $i :: k_3$.

\begin{lemma}%[Increase Overflown Stack Depth Limit]
    If $\VerifyRule{g}{p}{d}{nb}{\None}{k_1 \dplus i :: k_2}$, \\
    and $\VerifyRule{g}{\PNT i}{1+\length{k_3}}{nb'}{\None}{i :: k_3}$,
    and $\length{k_2} \le \length{k_3}$, \\
    then $\VerifyRule{g}{p}{1 + \length{k_1} + \length{k_3}}{nb}{\None}{k_1 \dplus i :: k_3}$.
    \label{lemma:d-increase-lr}
\end{lemma}

\Cref{lemma:repeated-rule-in-stack} shows that,
if an evaluation results in a stack overflow,
and a rule $i$ occurs more than once in the output stack,
then we can increase the stack depth limit by a certain amount,
and both conditions will still hold true.

\begin{lemma}%[Repeated Overflow Stack Section]
    If $\VerifyRule{g}{p}{d}{nb}{\None}{k_1 \dplus i :: k_2 \dplus i :: k_3}$, \\
    then $\exists d'$,
    such that $\VerifyRule{g}{p}{d'}{nb}{\None}{k_1 \dplus i :: k_2 \dplus i :: k_2 \dplus i :: k_3}$.
    \label{lemma:repeated-rule-in-stack}
\end{lemma}

Finally, we present the main lemma
that we wanted to prove.
\Cref{lemma:stack-depth-eventual-constancy}
shows that,
if an evaluation with a stack depth limit
greater than the number of grammar rules
yields a result,
then any evaluation with an even greater stack depth limit
yields the same result.
The stacks can be different,
but they are irrelevant
for our purpose of identifying
left-recursive rules.

\begin{lemma}%[Stack Depth Limit Increase]
    If $\VerifyRule{g}{p}{d}{nb}{res}{k}$, and $d > \length{g}$, \\
    then, for any $d' \ge d$,
    $\exists k'$,
    such that $\VerifyRule{g}{p}{d'}{nb}{res}{k'}$.
    \label{lemma:stack-depth-eventual-constancy}
\end{lemma}

We now explain the proof of this lemma.
For the cases in which the evaluation
does not result in a stack overflow,
the proof follows from \Cref{lemma:stack-depth-monotonicity-not-lr-pattern}.
Now, in the case of a stack overflow,
we know from \Cref{lemma:stack-depth-lr-pattern}
that the length of the stack $k$ is equal to the stack depth limit $d$,
which, in this case, we assume to be greater than $n$, the number of grammar rules.
Therefore, $\length{k} > n$.
We know from \Cref{lemma:coherent-stack}
that the stack only contains valid grammar rule indices.
That is, $\forall i \in k, i < n$.
We use these two observations and the pigeonhole principle to conclude
that the stack must have at least one repeated rule.
From \Cref{lemma:repeated-rule-in-stack},
we show that we can increase the stack depth limit arbitrarily,
and it will still result in a stack overflow.

Having defined the algorithm
that checks if a pattern is free of left recursion,
we now use this definition to create a function
that performs this check for a list of patterns.
\Cref{fig:lverifyrule-function} defines this function,
\begin{figure}
    \centering
    \input{lverifyrulecomp}
    \caption{The left recursion detection function for lists of patterns.}
    \label{fig:lverifyrule-function}
\end{figure}
which receives a grammar, a list of patterns,
and a gas counter,
and returns an optional Boolean value
indicating whether all patterns in the grammar
are free of left recursion.

This new function provides values
for two of the parameters of the underlying function:
the stack depth limit $d$, initialized with $\length{g}+1$,
the lower bound from \Cref{lemma:stack-depth-eventual-constancy},
and the nullable accumulator $nb$, initialized with $false$.
We could have omitted the gas counter,
by providing the lower bound from \Cref{lemma:vr-termination},
but we decided to postpone
this omission to the top-most definition
of well-formedness in our formalization.

We provide a lower bound for the gas parameter,
for which the function returns some result.
Note that we're assuming that both the grammar $g$
and the list of rules $rs$ are coherent,
because they could be different.
In practice, however,
they will be the same.
In this case,
where $rs = g$,
the equation for the lower bound can be simplified
to $(\length{g} + 2) \cdot \size{g}$.
That is the origin of the lower bound of the
\textit{\verifygrammarname{}} function
as displayed in \Cref{fig:verifygrammargas}.

\begin{lemma}%[Verify Rules Termination]
    If $\lCoherent{g}{g}{true}$ and $\lCoherent{g}{rs}{true}$, \\
    then, $\forall gas \ge \size{rs} + (\length{g} + 1) \cdot \size{g}$,
    $\exists res\ \lverifyrulecomp{g}{rs}{gas} = \Some res$.
\end{lemma}

\begin{proof}
    The proof follows by induction on the list of rules $rs$,
    and from the gas lower bound for the function \textit{\verifyrulename{}}
    from \Cref{lemma:vr-termination},
    substituting the stack depth limit $d$ with $\length{g}+1$.
\end{proof}

We will use this function for
verifying that the grammar contains
no left-recursive rules,
since it's implemented as a list of rules.
Since we will also be using it in our proofs,
we will need an analogous inductive definition.
\Cref{fig:lverifyrule} defines this predicate,
\begin{figure}
    \centering
    \input{lverifyrule}
    \caption{The left recursion detection predicate for lists of patterns.}
    \label{fig:lverifyrule}
\end{figure}
which also receives a grammar and a list of patterns,
and yields a Boolean value indicating
whether all patterns in the list are free of left recursion.

This predicate differs from the function in one aspect.
While the function provides an exact value for the
stack depth limit, the predicate allows any stack depth limit
to identify a pattern as either nullable or non-nullable.
That is because, according to \Cref{lemma:stack-depth-monotonicity-not-lr-pattern},
the returned label stays constant with increasing stack depth limits in such cases.
In the general case, however,
a stack depth limit greater than
the number of rules in the grammar
is necessary.

We prove some lemmas about this predicate.
\Cref{lemma:lverifyrule-determinism}
states that this predicate is deterministic,
and \Cref{lemma:lverifyrule-follows}
states that it follows the fixed-point definition.

\begin{lemma}
    If $\lVerifyRule{g}{rs}{b_1}$
    and $\lVerifyRule{g}{rs}{b_2}$,
    then $b_1 = b_2$.
    \label{lemma:lverifyrule-determinism}
\end{lemma}

\begin{lemma}
    If $\lverifyrulecomp{g}{rs}{gas} = \Some b$,
    then $\lVerifyRule{g}{rs}{b}$.
    \label{lemma:lverifyrule-follows}
\end{lemma}

\Cref{lemma:lverifyrule-safety} states that,
if a list of patterns passes the check,
then every pattern in the list
passes the individual check,
being either nullable or non-nullable.

\begin{lemma}%[Verify Rules Safety]
    If $\lVerifyRule{g}{rs}{true}$, \\
    then, $\forall r \in rs, \exists d\ \exists b\ \exists k\ \VerifyRule{g}{r}{d}{nb}{\Some b}{k}$.
    \label{lemma:lverifyrule-safety}
\end{lemma}

Before we end this section,
there is one final lemma we would like to present,
which uses all the predicates of the verification algorithm
we have defined up until now.
\Cref{lemma:no-lr-rule-in-grammar} shows that,
if a grammar is free of incoherent and left-recursive rules,
then any coherent pattern is either nullable or non-nullable.

\begin{lemma}%[No Left-Recursive Rule in Grammar]
    If $\Coherent{g}{p}{true}$,
    and $\lCoherent{g}{g}{true}$, \\
    and $\lVerifyRule{g}{g}{true}$,
    then $\exists d\ \exists b\ \exists k$,
    such that $\VerifyRule{g}{p}{d}{nb}{\Some b}{k}$.
    \label{lemma:no-lr-rule-in-grammar}
\end{lemma}

    \caption{The left recursion detection predicate.}
    \label{fig:verifyrule}
\end{figure}
defines such predicate,
denoted as $\VerifyRule{g}{p}{d}{nb}{res}{k}$.
It takes a grammar $g$,
a pattern $p$,
a stack depth limit $d$,
and a nullable accumulator $nb$,
and outputs a result $res$,
and a stack trace $k$.

In order to reach our final goal
of proving that the label returned
by the \textit{\verifyrulename{}} function
is monotonic and eventually constant
with respect to the stack depth limit,
we need to first prove some intermediary lemmas.
First, we need to relate the predicate and
the fixed-point definition together,
so that we can apply the proofs about the
former to the latter.

We begin with
\Cref{lemma:vr-determinism},
which states that,
for identical input,
the predicate yields the same output.
We can therefore state
that the predicate is deterministic.

\begin{lemma}
    If $\VerifyRule{g}{p}{d}{nb}{res_1}{k_1}$,
    and $\VerifyRule{g}{p}{d}{nb}{res_2}{k_2}$, \\
    then $res_1 = res_2$ and $k_1 = k_2$.
    \label{lemma:vr-determinism}
\end{lemma}

\Cref{lemma:vr-follows} shows that
every result returned by the fixed-point definition
can be inductively constructed using the predicate definition.

\begin{lemma}
    If $\verifyrulecomp{g}{p}{d}{nb}{gas} = \Some (res, k)$, \\
    then $\VerifyRule{g}{p}{d}{nb}{res}{k}$.
    \label{lemma:vr-follows}
\end{lemma}

\Cref{lemma:stack-depth-monotonicity-not-lr-pattern} shows that,
if a pattern evaluates to either nullable or non-nullable,
then increasing the stack depth limit
doesn't affect the result.
This is expected,
because, in both cases,
it doesn't surpass the limit,
and increasing it preserves this property
by transitivity.

\begin{lemma}%[Stack Depth Limit Increase without Overflow]
    If $\VerifyRule{g}{p}{d}{nb}{\Some nb'}{k}$, \\
    then $\forall d' \ge d, \VerifyRule{g}{p}{d'}{nb}{\Some nb'}{k}$.
    \label{lemma:stack-depth-monotonicity-not-lr-pattern}
\end{lemma}

\Cref{lemma:stack-depth-lr-pattern} shows that,
on stack overflow,
the output stack $k$ has length $d$.
That is expected,
because a stack overflow happens
when the stack is full
before trying to visit a rule.

\begin{lemma}%[Stack Length on Overflow]
    If $\VerifyRule{g}{p}{d}{nb}{\None}{k}$,
    then $\length{k} = d$.
    \label{lemma:stack-depth-lr-pattern}
\end{lemma}

\Cref{lemma:coherent-stack} states
that the output stack
only contains references to
existing rules in the grammar.
Since we are identifying rules
by their indices in a list,
we prove this by showing that these indices
are less than the number of rules in the grammar,
denoted as $\length{g}$.
This lemma may seem trivial,
but it is necessary for us to later prove,
using the pigeonhole principle,
that a stack with more rules than the grammar
will have at least one repeated rule.

\begin{lemma}%[Coherent Stack]
    If $\VerifyRule{g}{p}{d}{nb}{res}{k}$,
    then $\forall i \in k, i < \length{g}$.
    \label{lemma:coherent-stack}
\end{lemma}

% Next, we prove in \Cref{lemma:true-as-nb} that,
% when provided with $true$ as the value
% for the nullable accumulator $nb$,
% the algorithm outputs as a result
% either $\None$ (which means ``left-recursive'')
% or $\Some true$ (which means ``nullable'').

% \begin{lemma}%[True As Nullable Accumulator]
%     For a grammar $g$,
%     a pattern $p$,
%     a natural number $d$,
%     an optional Boolean value $res$,
%     and a list of natural numbers $k$,
%     if $\VerifyRule{g}{p}{d}{true}{res}{k}$,
%     then $res \in \{\None,\ \Some true\}$.
%     \label{lemma:true-as-nb}
% \end{lemma}

% We also state in \Cref{lemma:nb-change-not-lr-pattern} that,
% when a pattern is identified as either nullable or non-nullable,
% then changing the nullable accumulator
% might change the result to either nullable or non-nullable,
% but the output stack $k$ will remain the same.
% Note that we're saying that it will not make
% the pattern be marked as left-recursive.

% \begin{lemma}%[Nullable Accumulator Change When Pattern Is Not Left-Recursive]
%     For a grammar $g$,
%     a pattern $p$,
%     a natural number $d$,
%     Boolean values $nb$, $nb'$ and $b$,
%     and a list of natural numbers $k$,
%     if $\VerifyRule{g}{p}{d}{nb}{\Some b}{k}$,
%     then there exists some Boolean value $b'$,
%     such that $\VerifyRule{g}{p}{d}{nb'}{\Some b'}{k}$.
%     \label{lemma:nb-change-not-lr-pattern}
% \end{lemma}

% Alternatively,
% when a pattern is identified as left-recursive,
% changing the nullable accumulator
% will change neither the result nor the stack $k$.
% This is put forth in \Cref{lemma:nb-change-lr-pattern}.

% \begin{lemma}%[Nullable Accumulator Change When Pattern Is Left-Recursive]
%     For a grammar $g$,
%     a pattern $p$,
%     a natural number $d$,
%     Boolean values $nb$ and $nb'$,
%     and a list of natural numbers $k$,
%     if $\VerifyRule{g}{p}{d}{nb}{\None}{k}$,
%     then $\VerifyRule{g}{p}{d}{nb'}{\None}{k}$.
%     \label{lemma:nb-change-lr-pattern}
% \end{lemma}

% The following lemmas will start to deal with the output stack more closely.
% It is used by \lpeg{} to format error messages,
% but we will use it to help us formalize certain properties about the algorithm.
% \Cref{lemma:existential-ff} states that,
% for any evaluation with an output stack $k_1 \dplus i :: k_2$,
% there exists a stack depth limit and a nullable accumulator,
% for which the evaluation of the nonterminal pattern $\PNT{i}$
% yields the output stack $i :: k_2$.
% We call this operation ``fast-forwarding''.

% \begin{lemma}%[Existential Fast-forward]
%     $\forall gas\ \forall p\ \forall d\ \forall nb\ \forall res\ \forall k_2\ \forall k_2\ \forall i$, \\
%     if $\VerifyRule{g}{p}{d}{nb}{res}{k_1 \dplus i :: k_2}$,
%     then $\exists d'\ \exists nb'\ \exists res'$,
%     such that $\VerifyRule{g}{\PNT i}{d'}{nb'}{res'}{i :: k_2}$.
%     \label{lemma:existential-ff}
% \end{lemma}

% \Cref{lemma:existential-ff-for-lr-patterns}
% is a particular case of \Cref{lemma:existential-ff},
% in which the evaluation of pattern $p$ overflows the stack,
% and, as a result, so does the evaluation of nonterminal pattern $\PNT{i}$.

% \begin{lemma}%[Existential Fast-forward On Overflow]
%     $\forall gas\ \forall p\ \forall d\ \forall nb\ \forall k_2\ \forall k_2\ \forall i$, \\
%     if $\VerifyRule{g}{p}{d}{nb}{\None}{k_1 \dplus i :: k_2}$,
%     then $\exists d'\ \exists nb'$,
%     such that $\VerifyRule{g}{\PNT i}{d'}{nb'}{\None}{i :: k_2}$.
%     \label{lemma:existential-ff-for-lr-patterns}
% \end{lemma}

\Cref{lemma:ff-for-lr-patterns} states that,
for any evaluation that results in a stack overflow,
we can pick any rule $i$ from the output stack $k$,
and evaluate it with a certain stack depth limit,
so that it also results in a stack overflow,
and returns a suffix of the original stack $k$,
starting from $i$.

\begin{lemma}%[Fast-forward on Overflow]
    If $\VerifyRule{g}{p}{d}{nb}{\None}{k_1 \dplus i :: k_2}$, \\
    then $\VerifyRule{g}{\PNT i}{1+\length{k_2}}{nb'}{\None}{i :: k_2}$.
    \label{lemma:ff-for-lr-patterns}
\end{lemma}

Under the same assumptions,
\Cref{lemma:d-increase-lr} shows that,
if we evaluate a rule $i$ from the stack
with an increased stack depth limit
and it still results in a stack overflow
and returns a stack $i :: k_3$,
then we can increase the stack depth limit
of the original evaluation by the same amount,
it will also result in a stack overflow,
and return a stack that ends with $i :: k_3$.

\begin{lemma}%[Increase Overflown Stack Depth Limit]
    If $\VerifyRule{g}{p}{d}{nb}{\None}{k_1 \dplus i :: k_2}$, \\
    and $\VerifyRule{g}{\PNT i}{1+\length{k_3}}{nb'}{\None}{i :: k_3}$,
    and $\length{k_2} \le \length{k_3}$, \\
    then $\VerifyRule{g}{p}{1 + \length{k_1} + \length{k_3}}{nb}{\None}{k_1 \dplus i :: k_3}$.
    \label{lemma:d-increase-lr}
\end{lemma}

\Cref{lemma:repeated-rule-in-stack} shows that,
if an evaluation results in a stack overflow,
and a rule $i$ occurs more than once in the output stack,
then we can increase the stack depth limit by a certain amount,
and both conditions will still hold true.

\begin{lemma}%[Repeated Overflow Stack Section]
    If $\VerifyRule{g}{p}{d}{nb}{\None}{k_1 \dplus i :: k_2 \dplus i :: k_3}$, \\
    then $\exists d'$,
    such that $\VerifyRule{g}{p}{d'}{nb}{\None}{k_1 \dplus i :: k_2 \dplus i :: k_2 \dplus i :: k_3}$.
    \label{lemma:repeated-rule-in-stack}
\end{lemma}

Finally, we present the main lemma
that we wanted to prove.
\Cref{lemma:stack-depth-eventual-constancy}
shows that,
if an evaluation with a stack depth limit
greater than the number of grammar rules
yields a result,
then any evaluation with an even greater stack depth limit
yields the same result.
The stacks can be different,
but they are irrelevant
for our purpose of identifying
left-recursive rules.

\begin{lemma}%[Stack Depth Limit Increase]
    If $\VerifyRule{g}{p}{d}{nb}{res}{k}$, and $d > \length{g}$, \\
    then, for any $d' \ge d$,
    $\exists k'$,
    such that $\VerifyRule{g}{p}{d'}{nb}{res}{k'}$.
    \label{lemma:stack-depth-eventual-constancy}
\end{lemma}

We now explain the proof of this lemma.
For the cases in which the evaluation
does not result in a stack overflow,
the proof follows from \Cref{lemma:stack-depth-monotonicity-not-lr-pattern}.
Now, in the case of a stack overflow,
we know from \Cref{lemma:stack-depth-lr-pattern}
that the length of the stack $k$ is equal to the stack depth limit $d$,
which, in this case, we assume to be greater than $n$, the number of grammar rules.
Therefore, $\length{k} > n$.
We know from \Cref{lemma:coherent-stack}
that the stack only contains valid grammar rule indices.
That is, $\forall i \in k, i < n$.
We use these two observations and the pigeonhole principle to conclude
that the stack must have at least one repeated rule.
From \Cref{lemma:repeated-rule-in-stack},
we show that we can increase the stack depth limit arbitrarily,
and it will still result in a stack overflow.

Having defined the algorithm
that checks if a pattern is free of left recursion,
we now use this definition to create a function
that performs this check for a list of patterns.
\Cref{fig:lverifyrule-function} defines this function,
\begin{figure}
    \centering
    \begin{equation*}
    \begin{aligned}[t]
        & \lverifyrulecomp{g}{rs}{gas} := \\
        & \begin{aligned}[t]
            & \matchwith{rs} \\
            & \matchcase{nil}{\Some true} \\
            & \matchcase{r::rs'}{\begin{aligned}[t]
                & \letin{d}{\length{g}+1} \\
                & \matchwith{\verifyrulecomp{g}{r}{d}{false}{gas}} \\
                & \matchcase{\Some (\Some b, k)}{\lverifyrulecomp{g}{rs'}{gas}} \\
                & \matchcase{\Some (\None, k)}{\Some false} \\
                & \matchcase{\None}{\None} \\
                & \matchend{}
            \end{aligned}} \\
            & \matchend{}
        \end{aligned}
    \end{aligned}
\end{equation*}


% \begin{equation}
%     \lverifyrulecomp(g,[\ ],gas) = \Some(true)
% \end{equation}

% \begin{equation}
%     \lverifyrulecomp(g,r::rs,gas) = \begin{cases}
%         \text{where } ores := \verifyrulecomp(g,r,1+|g|,false,gas) \\
%         \lverifyrulecomp(g,rs,gas), \text{if } ores = \Some(\Some(b),k) \\
%         \Some(false), \text{if } ores = \Some(\None, k) \\
%         \None, \text{if } ores = \None
%     \end{cases}
% \end{equation}

    \caption{The left recursion detection function for lists of patterns.}
    \label{fig:lverifyrule-function}
\end{figure}
which receives a grammar, a list of patterns,
and a gas counter,
and returns an optional Boolean value
indicating whether all patterns in the grammar
are free of left recursion.

This new function provides values
for two of the parameters of the underlying function:
the stack depth limit $d$, initialized with $\length{g}+1$,
the lower bound from \Cref{lemma:stack-depth-eventual-constancy},
and the nullable accumulator $nb$, initialized with $false$.
We could have omitted the gas counter,
by providing the lower bound from \Cref{lemma:vr-termination},
but we decided to postpone
this omission to the top-most definition
of well-formedness in our formalization.

We provide a lower bound for the gas parameter,
for which the function returns some result.
Note that we're assuming that both the grammar $g$
and the list of rules $rs$ are coherent,
because they could be different.
In practice, however,
they will be the same.
In this case,
where $rs = g$,
the equation for the lower bound can be simplified
to $(\length{g} + 2) \cdot \size{g}$.
That is the origin of the lower bound of the
\textit{\verifygrammarname{}} function
as displayed in \Cref{fig:verifygrammargas}.

\begin{lemma}%[Verify Rules Termination]
    If $\lCoherent{g}{g}{true}$ and $\lCoherent{g}{rs}{true}$, \\
    then, $\forall gas \ge \size{rs} + (\length{g} + 1) \cdot \size{g}$,
    $\exists res\ \lverifyrulecomp{g}{rs}{gas} = \Some res$.
\end{lemma}

\begin{proof}
    The proof follows by induction on the list of rules $rs$,
    and from the gas lower bound for the function \textit{\verifyrulename{}}
    from \Cref{lemma:vr-termination},
    substituting the stack depth limit $d$ with $\length{g}+1$.
\end{proof}

We will use this function for
verifying that the grammar contains
no left-recursive rules,
since it's implemented as a list of rules.
Since we will also be using it in our proofs,
we will need an analogous inductive definition.
\Cref{fig:lverifyrule} defines this predicate,
\begin{figure}
    \centering
    \begin{mathpar}
    \namedinferrule{lvr-nil}
    { }
    {\lVerifyRule{g}{nil}{true}}

    \namedinferrule{lvr-cons-some}
    {\VerifyRule{g}{r}{d}{false}{\Some nb}{k} \\ \lVerifyRule{g}{rs}{b}}
    {\lVerifyRule{g}{r::rs}{b}}

    \namedinferrule{lvr-cons-none}
    {|g| < d \\ \VerifyRule{g}{r}{d}{false}{\None}{k}}
    {\lVerifyRule{g}{r::rs}{false}}
\end{mathpar}
    \caption{The left recursion detection predicate for lists of patterns.}
    \label{fig:lverifyrule}
\end{figure}
which also receives a grammar and a list of patterns,
and yields a Boolean value indicating
whether all patterns in the list are free of left recursion.

This predicate differs from the function in one aspect.
While the function provides an exact value for the
stack depth limit, the predicate allows any stack depth limit
to identify a pattern as either nullable or non-nullable.
That is because, according to \Cref{lemma:stack-depth-monotonicity-not-lr-pattern},
the returned label stays constant with increasing stack depth limits in such cases.
In the general case, however,
a stack depth limit greater than
the number of rules in the grammar
is necessary.

We prove some lemmas about this predicate.
\Cref{lemma:lverifyrule-determinism}
states that this predicate is deterministic,
and \Cref{lemma:lverifyrule-follows}
states that it follows the fixed-point definition.

\begin{lemma}
    If $\lVerifyRule{g}{rs}{b_1}$
    and $\lVerifyRule{g}{rs}{b_2}$,
    then $b_1 = b_2$.
    \label{lemma:lverifyrule-determinism}
\end{lemma}

\begin{lemma}
    If $\lverifyrulecomp{g}{rs}{gas} = \Some b$,
    then $\lVerifyRule{g}{rs}{b}$.
    \label{lemma:lverifyrule-follows}
\end{lemma}

\Cref{lemma:lverifyrule-safety} states that,
if a list of patterns passes the check,
then every pattern in the list
passes the individual check,
being either nullable or non-nullable.

\begin{lemma}%[Verify Rules Safety]
    If $\lVerifyRule{g}{rs}{true}$, \\
    then, $\forall r \in rs, \exists d\ \exists b\ \exists k\ \VerifyRule{g}{r}{d}{nb}{\Some b}{k}$.
    \label{lemma:lverifyrule-safety}
\end{lemma}

Before we end this section,
there is one final lemma we would like to present,
which uses all the predicates of the verification algorithm
we have defined up until now.
\Cref{lemma:no-lr-rule-in-grammar} shows that,
if a grammar is free of incoherent and left-recursive rules,
then any coherent pattern is either nullable or non-nullable.

\begin{lemma}%[No Left-Recursive Rule in Grammar]
    If $\Coherent{g}{p}{true}$,
    and $\lCoherent{g}{g}{true}$, \\
    and $\lVerifyRule{g}{g}{true}$,
    then $\exists d\ \exists b\ \exists k$,
    such that $\VerifyRule{g}{p}{d}{nb}{\Some b}{k}$.
    \label{lemma:no-lr-rule-in-grammar}
\end{lemma}

    \caption{The left recursion detection predicate.}
    \label{fig:verifyrule}
\end{figure}
defines such predicate,
denoted as $\VerifyRule{g}{p}{d}{nb}{res}{k}$.
It takes a grammar $g$,
a pattern $p$,
a stack depth limit $d$,
and a nullable accumulator $nb$,
and outputs a result $res$,
and a stack trace $k$.

In order to reach our final goal
of proving that the label returned
by the \textit{\verifyrulename{}} function
is monotonic and eventually constant
with respect to the stack depth limit,
we need to first prove some intermediary lemmas.
First, we need to relate the predicate and
the fixed-point definition together,
so that we can apply the proofs about the
former to the latter.

We begin with
\Cref{lemma:vr-determinism},
which states that,
for identical input,
the predicate yields the same output.
We can therefore state
that the predicate is deterministic.

\begin{lemma}
    If $\VerifyRule{g}{p}{d}{nb}{res_1}{k_1}$,
    and $\VerifyRule{g}{p}{d}{nb}{res_2}{k_2}$, \\
    then $res_1 = res_2$ and $k_1 = k_2$.
    \label{lemma:vr-determinism}
\end{lemma}

\Cref{lemma:vr-follows} shows that
every result returned by the fixed-point definition
can be inductively constructed using the predicate definition.

\begin{lemma}
    If $\verifyrulecomp{g}{p}{d}{nb}{gas} = \Some (res, k)$, \\
    then $\VerifyRule{g}{p}{d}{nb}{res}{k}$.
    \label{lemma:vr-follows}
\end{lemma}

\Cref{lemma:stack-depth-monotonicity-not-lr-pattern} shows that,
if a pattern evaluates to either nullable or non-nullable,
then increasing the stack depth limit
doesn't affect the result.
This is expected,
because, in both cases,
it doesn't surpass the limit,
and increasing it preserves this property
by transitivity.

\begin{lemma}%[Stack Depth Limit Increase without Overflow]
    If $\VerifyRule{g}{p}{d}{nb}{\Some nb'}{k}$, \\
    then $\forall d' \ge d, \VerifyRule{g}{p}{d'}{nb}{\Some nb'}{k}$.
    \label{lemma:stack-depth-monotonicity-not-lr-pattern}
\end{lemma}

\Cref{lemma:stack-depth-lr-pattern} shows that,
on stack overflow,
the output stack $k$ has length $d$.
That is expected,
because a stack overflow happens
when the stack is full
before trying to visit a rule.

\begin{lemma}%[Stack Length on Overflow]
    If $\VerifyRule{g}{p}{d}{nb}{\None}{k}$,
    then $\length{k} = d$.
    \label{lemma:stack-depth-lr-pattern}
\end{lemma}

\Cref{lemma:coherent-stack} states
that the output stack
only contains references to
existing rules in the grammar.
Since we are identifying rules
by their indices in a list,
we prove this by showing that these indices
are less than the number of rules in the grammar,
denoted as $\length{g}$.
This lemma may seem trivial,
but it is necessary for us to later prove,
using the pigeonhole principle,
that a stack with more rules than the grammar
will have at least one repeated rule.

\begin{lemma}%[Coherent Stack]
    If $\VerifyRule{g}{p}{d}{nb}{res}{k}$,
    then $\forall i \in k, i < \length{g}$.
    \label{lemma:coherent-stack}
\end{lemma}

% Next, we prove in \Cref{lemma:true-as-nb} that,
% when provided with $true$ as the value
% for the nullable accumulator $nb$,
% the algorithm outputs as a result
% either $\None$ (which means ``left-recursive'')
% or $\Some true$ (which means ``nullable'').

% \begin{lemma}%[True As Nullable Accumulator]
%     For a grammar $g$,
%     a pattern $p$,
%     a natural number $d$,
%     an optional Boolean value $res$,
%     and a list of natural numbers $k$,
%     if $\VerifyRule{g}{p}{d}{true}{res}{k}$,
%     then $res \in \{\None,\ \Some true\}$.
%     \label{lemma:true-as-nb}
% \end{lemma}

% We also state in \Cref{lemma:nb-change-not-lr-pattern} that,
% when a pattern is identified as either nullable or non-nullable,
% then changing the nullable accumulator
% might change the result to either nullable or non-nullable,
% but the output stack $k$ will remain the same.
% Note that we're saying that it will not make
% the pattern be marked as left-recursive.

% \begin{lemma}%[Nullable Accumulator Change When Pattern Is Not Left-Recursive]
%     For a grammar $g$,
%     a pattern $p$,
%     a natural number $d$,
%     Boolean values $nb$, $nb'$ and $b$,
%     and a list of natural numbers $k$,
%     if $\VerifyRule{g}{p}{d}{nb}{\Some b}{k}$,
%     then there exists some Boolean value $b'$,
%     such that $\VerifyRule{g}{p}{d}{nb'}{\Some b'}{k}$.
%     \label{lemma:nb-change-not-lr-pattern}
% \end{lemma}

% Alternatively,
% when a pattern is identified as left-recursive,
% changing the nullable accumulator
% will change neither the result nor the stack $k$.
% This is put forth in \Cref{lemma:nb-change-lr-pattern}.

% \begin{lemma}%[Nullable Accumulator Change When Pattern Is Left-Recursive]
%     For a grammar $g$,
%     a pattern $p$,
%     a natural number $d$,
%     Boolean values $nb$ and $nb'$,
%     and a list of natural numbers $k$,
%     if $\VerifyRule{g}{p}{d}{nb}{\None}{k}$,
%     then $\VerifyRule{g}{p}{d}{nb'}{\None}{k}$.
%     \label{lemma:nb-change-lr-pattern}
% \end{lemma}

% The following lemmas will start to deal with the output stack more closely.
% It is used by \lpeg{} to format error messages,
% but we will use it to help us formalize certain properties about the algorithm.
% \Cref{lemma:existential-ff} states that,
% for any evaluation with an output stack $k_1 \dplus i :: k_2$,
% there exists a stack depth limit and a nullable accumulator,
% for which the evaluation of the nonterminal pattern $\PNT{i}$
% yields the output stack $i :: k_2$.
% We call this operation ``fast-forwarding''.

% \begin{lemma}%[Existential Fast-forward]
%     $\forall gas\ \forall p\ \forall d\ \forall nb\ \forall res\ \forall k_2\ \forall k_2\ \forall i$, \\
%     if $\VerifyRule{g}{p}{d}{nb}{res}{k_1 \dplus i :: k_2}$,
%     then $\exists d'\ \exists nb'\ \exists res'$,
%     such that $\VerifyRule{g}{\PNT i}{d'}{nb'}{res'}{i :: k_2}$.
%     \label{lemma:existential-ff}
% \end{lemma}

% \Cref{lemma:existential-ff-for-lr-patterns}
% is a particular case of \Cref{lemma:existential-ff},
% in which the evaluation of pattern $p$ overflows the stack,
% and, as a result, so does the evaluation of nonterminal pattern $\PNT{i}$.

% \begin{lemma}%[Existential Fast-forward On Overflow]
%     $\forall gas\ \forall p\ \forall d\ \forall nb\ \forall k_2\ \forall k_2\ \forall i$, \\
%     if $\VerifyRule{g}{p}{d}{nb}{\None}{k_1 \dplus i :: k_2}$,
%     then $\exists d'\ \exists nb'$,
%     such that $\VerifyRule{g}{\PNT i}{d'}{nb'}{\None}{i :: k_2}$.
%     \label{lemma:existential-ff-for-lr-patterns}
% \end{lemma}

\Cref{lemma:ff-for-lr-patterns} states that,
for any evaluation that results in a stack overflow,
we can pick any rule $i$ from the output stack $k$,
and evaluate it with a certain stack depth limit,
so that it also results in a stack overflow,
and returns a suffix of the original stack $k$,
starting from $i$.

\begin{lemma}%[Fast-forward on Overflow]
    If $\VerifyRule{g}{p}{d}{nb}{\None}{k_1 \dplus i :: k_2}$, \\
    then $\VerifyRule{g}{\PNT i}{1+\length{k_2}}{nb'}{\None}{i :: k_2}$.
    \label{lemma:ff-for-lr-patterns}
\end{lemma}

Under the same assumptions,
\Cref{lemma:d-increase-lr} shows that,
if we evaluate a rule $i$ from the stack
with an increased stack depth limit
and it still results in a stack overflow
and returns a stack $i :: k_3$,
then we can increase the stack depth limit
of the original evaluation by the same amount,
it will also result in a stack overflow,
and return a stack that ends with $i :: k_3$.

\begin{lemma}%[Increase Overflown Stack Depth Limit]
    If $\VerifyRule{g}{p}{d}{nb}{\None}{k_1 \dplus i :: k_2}$, \\
    and $\VerifyRule{g}{\PNT i}{1+\length{k_3}}{nb'}{\None}{i :: k_3}$,
    and $\length{k_2} \le \length{k_3}$, \\
    then $\VerifyRule{g}{p}{1 + \length{k_1} + \length{k_3}}{nb}{\None}{k_1 \dplus i :: k_3}$.
    \label{lemma:d-increase-lr}
\end{lemma}

\Cref{lemma:repeated-rule-in-stack} shows that,
if an evaluation results in a stack overflow,
and a rule $i$ occurs more than once in the output stack,
then we can increase the stack depth limit by a certain amount,
and both conditions will still hold true.

\begin{lemma}%[Repeated Overflow Stack Section]
    If $\VerifyRule{g}{p}{d}{nb}{\None}{k_1 \dplus i :: k_2 \dplus i :: k_3}$, \\
    then $\exists d'$,
    such that $\VerifyRule{g}{p}{d'}{nb}{\None}{k_1 \dplus i :: k_2 \dplus i :: k_2 \dplus i :: k_3}$.
    \label{lemma:repeated-rule-in-stack}
\end{lemma}

Finally, we present the main lemma
that we wanted to prove.
\Cref{lemma:stack-depth-eventual-constancy}
shows that,
if an evaluation with a stack depth limit
greater than the number of grammar rules
yields a result,
then any evaluation with an even greater stack depth limit
yields the same result.
The stacks can be different,
but they are irrelevant
for our purpose of identifying
left-recursive rules.

\begin{lemma}%[Stack Depth Limit Increase]
    If $\VerifyRule{g}{p}{d}{nb}{res}{k}$, and $d > \length{g}$, \\
    then, for any $d' \ge d$,
    $\exists k'$,
    such that $\VerifyRule{g}{p}{d'}{nb}{res}{k'}$.
    \label{lemma:stack-depth-eventual-constancy}
\end{lemma}

We now explain the proof of this lemma.
For the cases in which the evaluation
does not result in a stack overflow,
the proof follows from \Cref{lemma:stack-depth-monotonicity-not-lr-pattern}.
Now, in the case of a stack overflow,
we know from \Cref{lemma:stack-depth-lr-pattern}
that the length of the stack $k$ is equal to the stack depth limit $d$,
which, in this case, we assume to be greater than $n$, the number of grammar rules.
Therefore, $\length{k} > n$.
We know from \Cref{lemma:coherent-stack}
that the stack only contains valid grammar rule indices.
That is, $\forall i \in k, i < n$.
We use these two observations and the pigeonhole principle to conclude
that the stack must have at least one repeated rule.
From \Cref{lemma:repeated-rule-in-stack},
we show that we can increase the stack depth limit arbitrarily,
and it will still result in a stack overflow.

Having defined the algorithm
that checks if a pattern is free of left recursion,
we now use this definition to create a function
that performs this check for a list of patterns.
\Cref{fig:lverifyrule-function} defines this function,
\begin{figure}
    \centering
    \begin{equation*}
    \begin{aligned}[t]
        & \lverifyrulecomp{g}{rs}{gas} := \\
        & \begin{aligned}[t]
            & \matchwith{rs} \\
            & \matchcase{nil}{\Some true} \\
            & \matchcase{r::rs'}{\begin{aligned}[t]
                & \letin{d}{\length{g}+1} \\
                & \matchwith{\verifyrulecomp{g}{r}{d}{false}{gas}} \\
                & \matchcase{\Some (\Some b, k)}{\lverifyrulecomp{g}{rs'}{gas}} \\
                & \matchcase{\Some (\None, k)}{\Some false} \\
                & \matchcase{\None}{\None} \\
                & \matchend{}
            \end{aligned}} \\
            & \matchend{}
        \end{aligned}
    \end{aligned}
\end{equation*}


% \begin{equation}
%     \lverifyrulecomp(g,[\ ],gas) = \Some(true)
% \end{equation}

% \begin{equation}
%     \lverifyrulecomp(g,r::rs,gas) = \begin{cases}
%         \text{where } ores := \verifyrulecomp(g,r,1+|g|,false,gas) \\
%         \lverifyrulecomp(g,rs,gas), \text{if } ores = \Some(\Some(b),k) \\
%         \Some(false), \text{if } ores = \Some(\None, k) \\
%         \None, \text{if } ores = \None
%     \end{cases}
% \end{equation}

    \caption{The left recursion detection function for lists of patterns.}
    \label{fig:lverifyrule-function}
\end{figure}
which receives a grammar, a list of patterns,
and a gas counter,
and returns an optional Boolean value
indicating whether all patterns in the grammar
are free of left recursion.

This new function provides values
for two of the parameters of the underlying function:
the stack depth limit $d$, initialized with $\length{g}+1$,
the lower bound from \Cref{lemma:stack-depth-eventual-constancy},
and the nullable accumulator $nb$, initialized with $false$.
We could have omitted the gas counter,
by providing the lower bound from \Cref{lemma:vr-termination},
but we decided to postpone
this omission to the top-most definition
of well-formedness in our formalization.

We provide a lower bound for the gas parameter,
for which the function returns some result.
Note that we're assuming that both the grammar $g$
and the list of rules $rs$ are coherent,
because they could be different.
In practice, however,
they will be the same.
In this case,
where $rs = g$,
the equation for the lower bound can be simplified
to $(\length{g} + 2) \cdot \size{g}$.
That is the origin of the lower bound of the
\textit{\verifygrammarname{}} function
as displayed in \Cref{fig:verifygrammargas}.

\begin{lemma}%[Verify Rules Termination]
    If $\lCoherent{g}{g}{true}$ and $\lCoherent{g}{rs}{true}$, \\
    then, $\forall gas \ge \size{rs} + (\length{g} + 1) \cdot \size{g}$,
    $\exists res\ \lverifyrulecomp{g}{rs}{gas} = \Some res$.
\end{lemma}

\begin{proof}
    The proof follows by induction on the list of rules $rs$,
    and from the gas lower bound for the function \textit{\verifyrulename{}}
    from \Cref{lemma:vr-termination},
    substituting the stack depth limit $d$ with $\length{g}+1$.
\end{proof}

We will use this function for
verifying that the grammar contains
no left-recursive rules,
since it's implemented as a list of rules.
Since we will also be using it in our proofs,
we will need an analogous inductive definition.
\Cref{fig:lverifyrule} defines this predicate,
\begin{figure}
    \centering
    \begin{mathpar}
    \namedinferrule{lvr-nil}
    { }
    {\lVerifyRule{g}{nil}{true}}

    \namedinferrule{lvr-cons-some}
    {\VerifyRule{g}{r}{d}{false}{\Some nb}{k} \\ \lVerifyRule{g}{rs}{b}}
    {\lVerifyRule{g}{r::rs}{b}}

    \namedinferrule{lvr-cons-none}
    {|g| < d \\ \VerifyRule{g}{r}{d}{false}{\None}{k}}
    {\lVerifyRule{g}{r::rs}{false}}
\end{mathpar}
    \caption{The left recursion detection predicate for lists of patterns.}
    \label{fig:lverifyrule}
\end{figure}
which also receives a grammar and a list of patterns,
and yields a Boolean value indicating
whether all patterns in the list are free of left recursion.

This predicate differs from the function in one aspect.
While the function provides an exact value for the
stack depth limit, the predicate allows any stack depth limit
to identify a pattern as either nullable or non-nullable.
That is because, according to \Cref{lemma:stack-depth-monotonicity-not-lr-pattern},
the returned label stays constant with increasing stack depth limits in such cases.
In the general case, however,
a stack depth limit greater than
the number of rules in the grammar
is necessary.

We prove some lemmas about this predicate.
\Cref{lemma:lverifyrule-determinism}
states that this predicate is deterministic,
and \Cref{lemma:lverifyrule-follows}
states that it follows the fixed-point definition.

\begin{lemma}
    If $\lVerifyRule{g}{rs}{b_1}$
    and $\lVerifyRule{g}{rs}{b_2}$,
    then $b_1 = b_2$.
    \label{lemma:lverifyrule-determinism}
\end{lemma}

\begin{lemma}
    If $\lverifyrulecomp{g}{rs}{gas} = \Some b$,
    then $\lVerifyRule{g}{rs}{b}$.
    \label{lemma:lverifyrule-follows}
\end{lemma}

\Cref{lemma:lverifyrule-safety} states that,
if a list of patterns passes the check,
then every pattern in the list
passes the individual check,
being either nullable or non-nullable.

\begin{lemma}%[Verify Rules Safety]
    If $\lVerifyRule{g}{rs}{true}$, \\
    then, $\forall r \in rs, \exists d\ \exists b\ \exists k\ \VerifyRule{g}{r}{d}{nb}{\Some b}{k}$.
    \label{lemma:lverifyrule-safety}
\end{lemma}

Before we end this section,
there is one final lemma we would like to present,
which uses all the predicates of the verification algorithm
we have defined up until now.
\Cref{lemma:no-lr-rule-in-grammar} shows that,
if a grammar is free of incoherent and left-recursive rules,
then any coherent pattern is either nullable or non-nullable.

\begin{lemma}%[No Left-Recursive Rule in Grammar]
    If $\Coherent{g}{p}{true}$,
    and $\lCoherent{g}{g}{true}$, \\
    and $\lVerifyRule{g}{g}{true}$,
    then $\exists d\ \exists b\ \exists k$,
    such that $\VerifyRule{g}{p}{d}{nb}{\Some b}{k}$.
    \label{lemma:no-lr-rule-in-grammar}
\end{lemma}

\section{A simpler algorithm for detecting nullable patterns}
\label{section:nullable}

Once we have made sure a grammar is free of left-recursive rules,
the next step is to check for degenerate loops,
which involves checking if certain patterns are nullable.
To this end,
we could use the algorithm we have just described in \Cref{section:lr-rules},
but \lpeg{} implements a simpler version,
which takes advantage of the fact that
the grammar contains no left-recursive rules.

The difference between the algorithm from \Cref{section:lr-rules}
and this simpler version can be seen,
for example, in the case of choice patterns $\PChoice{p_1}{p_2}$.
More specifically, if $p_1$ is nullable,
the function from \Cref{section:lr-rules}
would still need to evaluate $p_2$,
as it could potentially lead to left recursion.
Meanwhile, if we assume that $p_2$ cannot lead to left recursion,
then if $p_1$ is nullable,
we can state that $\PChoice{p_1}{p_2}$ is nullable,
without having to visit $p_2$.
We can also avoid visiting sub-patterns in the cases of
repetition patterns and predicate patterns,
because they are all nullable.

In \Cref{fig:nullable-function},
\begin{figure}[ht!]
    \centering
    \begin{align*}
    \begin{aligned}[t]
        & \nullablecomp{g}{p}{d}{0} = \None \\
        & \nullablecomp{g}{p}{d}{(1+gas)} = \\
        & \begin{aligned}[t]
            & \matchwith{p} \\
            & \matchcase{\PEmpty}{\Some \Some true} \\
            & \matchcase{\PSet{cs}}{\Some \Some false} \\
            & \matchcase{\PRepetition{p}}{\Some \Some true} \\
            & \matchcase{\PNot{p}}{\Some \Some true} \\
            & \matchcase{\PAnd{p}}{\Some \Some true} \\
            & \matchcase{\PNT{i}}{\begin{aligned}[t]
                & \matchwith{g[i]} \\
                & \matchcase{\Some p}{\begin{aligned}[t]
                    & \matchwith{d} \\
                    & \matchcase{0}{\Some \None} \\
                    & \matchcase{1+d'}{\nullablecomp{g}{p}{d'}{gas}} \\
                    & \matchend{}
                \end{aligned}} \\
                & \matchcase{\None}{\None} \\
                & \matchend{}
            \end{aligned}} \\
            & \matchcase{\PSequence{p_1}{p_2}}{\begin{aligned}[t]
                & \matchwith{\nullablecomp{g}{p_1}{d}{gas}} \\
                & \matchcase{\Some \Some true}{\nullablecomp{g}{p_2}{d}{gas}} \\
                & \matchcase{res}{res} \\
                & \matchend{}
            \end{aligned}} \\
            & \matchcase{\PChoice{p_1}{p_2}}{\begin{aligned}[t]
                & \matchwith{\nullablecomp{g}{p_1}{d}{gas}} \\
                & \matchcase{\Some \Some false}{\nullablecomp{g}{p_2}{d}{gas}} \\
                & \matchcase{res}{res} \\
                & \matchend{}
            \end{aligned}} \\
            & \matchend{}
        \end{aligned}
    \end{aligned}
\end{align*}
    \caption{The nullable function.}
    \label{fig:nullable-function}
\end{figure}
we define this simpler version as a fixed-point,
which takes a grammar,
a pattern, a stack depth limit, and a gas counter,
and returns an optional label.
Possible return values are, therefore,
$\None$ (out-of-gas), $\Some \None$ (stack overflow),
$\Some \Some true$ (nullable), and $\Some \Some false$ (non-nullable).

The gas parameter is still necessary to convince Coq that
the function terminates,
and because it could be called
with an incoherent pattern or
with a grammar that contains an incoherent rule.
The stack depth limit is also necessary
because the function could be called
with a grammar that contains a left-recursive rule.
For these two reasons,
this function
may still return $\None$ (out-of-gas)
or $\Some \None$ (stack overflow).

In reality, however,
this function should only be called after
the grammar and pattern are guaranteed to be
coherent and free of left-recursive rules.
For this reason,
implementations of this algorithm
such as the one in \lpeg{}
are able to safely drop both parameters,
and return just a Boolean value
indicating whether the pattern is nullable or not.

Also note that, unlike the function from \Cref{fig:vr-function},
this function does not receive a nullable accumulator as parameter
and does not return a stack.
The nullable accumulator is not necessary,
thanks to the assumption that
the grammar contains no left-recursive rules,
and the output stack was only used for proofs,
which we will be able to reuse from \Cref{section:lr-rules}.

Just as we did for the function from \Cref{section:lr-rules},
we prove that this simpler version is also
monotonic
with respect to the gas counter.

\begin{lemma}%[Nullable Gas Monotonicity]
    If $\nullablecomp{g}{p}{d}{gas} = \Some res$, \\
    then, $\forall gas' \ge gas$,
    $\nullablecomp{g}{p}{d}{gas'} = \Some res$.
    \label{lemma:nullable-gas-monotonicity}
\end{lemma}

Besides that, we also prove the function
is eventually constant with respect to the gas counter
by giving a gas lower bound for which the function
returns some result for any coherent pattern and grammar.

\begin{lemma}%[Nullable Termination]
    If $\Coherent{g}{p}{true}$,
    and $\lCoherent{g}{g}{true}$, \\
    then, $\forall gas \ge \size{p} + d \cdot \size{g}$,
    $\exists res\ \nullablecomp{g}{p}{d}{gas} = \Some res$.
    \label{lemma:nullable-termination}
\end{lemma}

Furthermore, we would also like to prove that
this function is eventually constant and monotonic
with respect to the stack depth limit.
However, in order to do that,
we found it better to first define an equivalent inductive predicate.
\Cref{fig:nullable-predicate} defines the predicate
\begin{figure}
    \centering
    \begin{mathpar}
    \namedinferrule{n-eps}
    { }
    {\Nullable{g}{\PEmpty}{d}{\Some true}}

    \namedinferrule{n-set}
    { }
    {\Nullable{g}{\PSet{cs}}{d}{\Some false}}

    \namedinferrule{n-rep}
    { }
    {\Nullable{g}{\PRepetition{p}}{d}{\Some true}}

    \namedinferrule{n-not}
    { }
    {\Nullable{g}{\PNot{p}}{d}{\Some true}}

    \namedinferrule{n-and}
    { }
    {\Nullable{g}{\PAnd{p}}{d}{\Some true}}

    \namedinferrule{n-nonterminal-zero}
    {g[i] = \Some p}
    {\Nullable{g}{\PNT{i}}{0}{\None}}
    
    \namedinferrule{n-nonterminal-succ}
    {g[i] = \Some p \\ \Nullable{g}{p}{d}{res}}
    {\Nullable{g}{\PNT{i}}{1+d}{res}}

    \namedinferrule{n-seq-none}
    {\Nullable{g}{p_1}{d}{\None}}
    {\Nullable{g}{\PSequence{p_1}{p_2}}{d}{\None}}
    
    \namedinferrule{n-seq-some-false}
    {\Nullable{g}{p_1}{d}{\Some false}}
    {\Nullable{g}{\PSequence{p_1}{p_2}}{d}{\Some false}}
    
    \namedinferrule{n-seq-some-true}
    {\Nullable{g}{p_1}{d}{\Some true} \\ \Nullable{g}{p_2}{d}{res}}
    {\Nullable{g}{\PSequence{p_1}{p_2}}{d}{res}}

    \namedinferrule{n-choice-none}
    {\Nullable{g}{p_1}{d}{\None}}
    {\Nullable{g}{\PChoice{p_1}{p_2}}{d}{\None}}
    
    \namedinferrule{n-choice-some-false}
    {\Nullable{g}{p_1}{d}{\Some false} \\ \Nullable{g}{p_2}{d}{res}}
    {\Nullable{g}{\PChoice{p_1}{p_2}}{d}{res}}
    
    \namedinferrule{n-choice-some-true}
    {\Nullable{g}{p_1}{d}{\Some true}}
    {\Nullable{g}{\PChoice{p_1}{p_2}}{d}{\Some true}}
\end{mathpar}
    \caption{The nullable predicate.}
    \label{fig:nullable-predicate}
\end{figure}
$\Nullable{g}{p}{d}{res}$ which takes a grammar $g$,
a pattern $p$, a stack depth limit $d$, and returns a label $res$.

About this predicate, we proved some basic lemmas.
First, we proved that it is deterministic.

\begin{lemma}
    If $\Nullable{g}{p}{d}{res_1}$,
    and $\Nullable{g}{p}{d}{res_2}$,
    then $res_1 = res_2$.
\end{lemma}

We also proved that it follows
the fixed-point definition.

\begin{lemma}
    If $\nullablecomp{g}{p}{d}{gas} = \Some res$,
    then $\Nullable{g}{p}{d}{res}$.
\end{lemma}

We also tied this predicate to the
one from \Cref{section:lr-rules},
showing how similar they are,
when the nullable accumulator is $false$,
and the pattern is either nullable or non-nullable.

\begin{lemma}%[Verify Rule Similar to Nullable]
    If $\VerifyRule{g}{p}{d}{false}{\Some b}{k}$,
    then $\Nullable{g}{p}{d}{\Some b}$.
\end{lemma}

We also proved that if a pattern
was identified as either nullable or non-nullable,
then increasing the stack depth does not impact the result.

\begin{lemma}%[Stack Depth Limit Increase Without Overflow]
    If $\Nullable{g}{p}{d}{\Some b}$,
    then $\forall d' \ge d, \Nullable{g}{p}{d'}{\Some b}$.
\end{lemma}

Maybe the most important lemma
about the nullable predicate
relates to the match predicate.
It states that a non-nullable pattern
never matches without consuming
some part of the input string.

\begin{lemma}%[Non-nullable pattern and Match]
    If $\Nullable{g}{p}{d}{\Some false}$,
    then $\nexists s$ such that $\Matches{g}{p}{s}{s}$.
\end{lemma}

This lemma is used to prove that,
when matching a non-nullable pattern,
the output string is a proper suffix of the input string.
We use the symbol ``$\ProperSuffix{}{}$''
to denote this relation.

\begin{lemma}%[Non-nullable pattern and Proper Suffix]
    \label{lemma:non-nullable-pattern-proper-suffix}
    If $\Nullable{g}{p}{d}{\Some false}$,
    and $\Matches{g}{p}{s}{s'}$,
    then $\ProperSuffix{s'}{s}$.
\end{lemma}

Finally, we show that the predicate is
eventually constant
with respect to the stack depth limit,
past a lower bound given by the number of rules in the grammar,
denoted as $\length{g}$.

\begin{lemma}%[Nullable Stack Depth Limit Eventual Constancy]
    If $\Coherent{g}{p}{true}$,
    and $\lCoherent{g}{g}{true}$,
    and $\lVerifyRule{g}{g}{true}$, \\
    and $\Nullable{g}{p}{d}{res}$,
    where $d > \length{g}$,
    then, $\forall d' \ge d$,
    $\Nullable{g}{p}{d'}{res}$.
\end{lemma}

\section{Degenerate loops}

After making sure that
all rules are coherent,
and that the grammar is free of left-recursive rules,
the next step is to look for degenerate loops,
which are repetition patterns $\PRepetition{p}$
where $p$ is nullable.
To detect nullable patterns,
we use the algorithm from
\Cref{section:nullable}.

\Cref{fig:checkloops-function}
defines this step of the verification process
as a fixed-point,
which takes a grammar,
a pattern,
a stack depth limit,
and a gas counter,
and returns an optional label.
Possible return values are
$\None$ (out-of-gas),
$\Some \None$ (stack overflow),
$\Some \Some true$ (degenerate),
and $\Some \Some false$ (non-degenerate).

This fixed-point does not visit rules
referenced by nonterminal patterns.
Instead, each rule is checked separately.
In this case, the stack depth limit parameter
is simply passed down on to the function that
checks whether a pattern is nullable or not.
However, as we've discussed in \Cref{section:nullable},
actual implementations can safely drop this parameter,
given that the grammar has been checked
for left-recursive rules already.

\begin{figure}
    \centering
    \begin{align*}
    \begin{aligned}[t]
        & \checkloopscomp{g}{p}{d}{0} = \None \\
        & \checkloopscomp{g}{p}{d}{(1+gas)} = \\
        & \begin{aligned}[t]
            & \matchwith{p} \\
            & \matchcase{\PEmpty}{\Some \Some false} \\
            & \matchcase{\PSet{cs}}{\Some \Some false} \\
            & \matchcase{\PNT{i}}{\Some \Some false} \\
            & \matchcase{\PNot{p}}{\checkloopscomp{g}{p}{d}{gas}} \\
            & \matchcase{\PAnd{p}}{\checkloopscomp{g}{p}{d}{gas}} \\
            & \matchcase{\PRepetition{p}}{\begin{aligned}[t]
                & \matchwith{\nullablecomp{g}{p}{d}{gas}} \\
                & \matchcase{\Some \Some false}{\checkloopscomp{g}{p}{d}{gas}} \\
                & \matchcase{res}{res} \\
                & \matchend{}
            \end{aligned}} \\
            & \matchcase{\PSequence{p_1}{p_2}}{\begin{aligned}[t]
                & \matchwith{\checkloopscomp{g}{p_1}{d}{gas}} \\
                & \matchcase{\Some \Some false}{\checkloopscomp{g}{p_2}{d}{gas}} \\
                & \matchcase{res}{res} \\
                & \matchend{}
            \end{aligned}} \\
            & \matchcase{\PChoice{p_1}{p_2}}{\begin{aligned}[t]
                & \matchwith{\checkloopscomp{g}{p_1}{d}{gas}} \\
                & \matchcase{\Some \Some false}{\checkloopscomp{g}{p_2}{d}{gas}} \\
                & \matchcase{res}{res} \\
                & \matchend{}
            \end{aligned}} \\
            & \matchend{}
        \end{aligned}
    \end{aligned}
\end{align*}
    \caption{The degenerate loop detection function.}
    \label{fig:checkloops-function}
\end{figure}

Just as with the other gas-based functions,
we would like to prove that this function
converges with respect to the gas counter.
\Cref{lemma:checkloops-gas-convergence}
states that, if this function returns some label,
then increasing the gas counter
won't change the label being returned.

\begin{lemma}%[Check Loops Gas Convergence]
    If $\checkloopscomp{g}{p}{d}{gas} = \Some res$, \\
    then, $\forall gas' \ge gas$,
    $\checkloopscomp{g}{p}{d}{gas'} = \Some res$.
    \label{lemma:checkloops-gas-convergence}
\end{lemma}

Moreover, \Cref{lemma:checkloops-termination}
states that, for any coherent pattern and grammar,
there exists a lower bound for the gas parameter,
for which the function returns some result.

\begin{lemma}%[Check Loops Termination]
    If $\Coherent{g}{p}{true}$,
    and $\lCoherent{g}{g}{true}$, \\
    and $gas \ge \size{p} + d \cdot \size{g}$,
    then $\exists res\ \checkloopscomp{g}{p}{d}{gas} = \Some res$.
    \label{lemma:checkloops-termination}
\end{lemma}

We would also like to prove that
this function converges with
respect to the stack depth limit.
However, in order to do that,
it is better to work with
an inductively-defined predicate.
\Cref{fig:checkloops} shows
the predicate we have defined.
It takes a grammar,
a pattern,
and a stack depth limit,
and returns an optional Boolean value.

\begin{figure}
    \centering
    \begin{mathpar}
    \namedinferrule{cl-eps}
    { }
    {\CheckLoops{g}{\PEmpty}{d}{\Some false}}

    \namedinferrule{cl-set}
    { }
    {\CheckLoops{g}{\PSet{cs}}{d}{\Some false}}

    \namedinferrule{cl-nonterminal}
    { }
    {\CheckLoops{g}{\PNT{i}}{d}{\Some false}}

    \namedinferrule{cl-not}
    {\CheckLoops{g}{p}{d}{res}}
    {\CheckLoops{g}{\PNot{p}}{d}{res}}

    \namedinferrule{cl-and}
    {\CheckLoops{g}{p}{d}{res}}
    {\CheckLoops{g}{\PAnd{p}}{d}{res}}

    \namedinferrule{cl-rep-lr}
    {\Nullable{g}{p}{d}{\None}}
    {\CheckLoops{g}{\PRepetition{p}}{d}{\None}}

    \namedinferrule{cl-rep-nullable}
    {\Nullable{g}{p}{d}{\Some true}}
    {\CheckLoops{g}{\PRepetition{p}}{d}{\Some true}}

    \namedinferrule{cl-rep-non-nullable}
    {\Nullable{g}{p}{d}{\Some false} \\ \CheckLoops{g}{p}{d}{res}}
    {\CheckLoops{g}{\PRepetition{p}}{d}{res}}

    \namedinferrule{cl-seq-none-1}
    {\CheckLoops{g}{p_1}{d}{\None}}
    {\CheckLoops{g}{\PSequence{p_1}{p_2}}{d}{\None}}

    \namedinferrule{cl-seq-none-2}
    {\CheckLoops{g}{p_2}{d}{\None}}
    {\CheckLoops{g}{\PSequence{p_1}{p_2}}{d}{\None}}

    \namedinferrule{cl-seq-some}
    {\CheckLoops{g}{p_1}{d}{\Some b_1} \\ \CheckLoops{g}{p_2}{d}{\Some b_2}}
    {\CheckLoops{g}{\PSequence{p_1}{p_2}}{d}{\Some (b_1 \vee b_2)}}

    \namedinferrule{cl-choice-none-1}
    {\CheckLoops{g}{p_1}{d}{\None}}
    {\CheckLoops{g}{\PChoice{p_1}{p_2}}{d}{\None}}

    \namedinferrule{cl-choice-none-2}
    {\CheckLoops{g}{p_2}{d}{\None}}
    {\CheckLoops{g}{\PChoice{p_1}{p_2}}{d}{\None}}

    \namedinferrule{cl-choice-some}
    {\CheckLoops{g}{p_1}{d}{\Some b_1} \\ \CheckLoops{g}{p_2}{d}{\Some b_2}}
    {\CheckLoops{g}{\PChoice{p_1}{p_2}}{d}{\Some (b_1 \vee b_2)}}
\end{mathpar}
    \caption{The degenerate loop detection predicate.}
    \label{fig:checkloops}
\end{figure}

As usual, we first prove
some basic lemmas about the predicate.
\Cref{lemma:checkloops-determinism}
states that the predicate
is deterministic,
meaning that, for the same input,
it yields the same output.

\begin{lemma}
    If $\CheckLoops{g}{p}{d}{res_1}$,
    and $\CheckLoops{g}{p}{d}{res_2}$,
    then $res_1 = res_2$.
    \label{lemma:checkloops-determinism}
\end{lemma}

\Cref{lemma:checkloops-follows}
states that every result returned by the function
can be constructed using the predicate.
We therefore say the predicate follows
the function.

\begin{lemma}
    If $\checkloopscomp{g}{p}{d}{gas} = \Some res$,
    then $\CheckLoops{g}{p}{d}{res}$.
    \label{lemma:checkloops-follows}
\end{lemma}

\Cref{lemma:checkloops-d-increase-no-overflow}
states that, if the predicate yields some result,
then increasing the stack depth limit
will not alter the result.

\begin{lemma}%[Stack Depth Limit Increase Without Overflow]
    If $\CheckLoops{g}{p}{d}{\Some res}$,
    then, $\forall d' \ge d$,
    $\CheckLoops{g}{p}{d'}{\Some res}$.
    \label{lemma:checkloops-d-increase-no-overflow}
\end{lemma}

Finally, \Cref{lemma:checkloops-d-convergence}
states that, for any coherent pattern and grammar
without left-recursive rules,
the label returned by the predicate converges
when the stack depth limit is greater than
the number of rules in the grammar.

\begin{lemma}%[Check Loops Stack Depth Limit Convergence]
    If $\Coherent{g}{p}{true}$,
    and $\lCoherent{g}{g}{true}$,
    and $\lVerifyRule{g}{g}{true}$, \\
    and $\CheckLoops{g}{p}{d}{res}$,
    where $d > \length{g}$,
    then, $\forall d' \ge d, \CheckLoops{g}{p}{d'}{res}$.
    \label{lemma:checkloops-d-convergence}
\end{lemma}

Having defined the algorithm
that checks if a pattern contains any degenerate loops,
we now define a function
that performs this check for a list of patterns.
Naturally, we will be using this
function to check all the rules of a grammar.
\Cref{fig:lcheckloops-function}
displays this function,
which takes a grammar, a list of patterns, and a gas counter,
and returns an optional Boolean value,
indicating whether it has found any degenerate loop.
We pass $\length{g}+1$ as
the stack depth limit to the underlying function.

\begin{figure}
    \centering
    \begin{equation*}
    \lcheckloopscomp{g}{rs}{gas} = \begin{aligned}[t]
        & \matchwith{rs} \\
        & \matchcase{nil}{\Some false} \\
        & \matchcase{r::rs'}{\begin{aligned}[t]
            & \letin{d}{\length{g}+1} \\
            & \matchwith{\checkloopscomp{g}{r}{d}{gas}} \\
            & \matchcase{\Some \Some false}{\lcheckloopscomp{g}{rs'}{gas}} \\
            & \matchcase{\Some \Some true}{\Some true} \\
            & \matchcase{res}{\None} \\
            & \matchend{}
        \end{aligned}} \\
        & \matchend{}
    \end{aligned}
\end{equation*}
    \caption{The degenerate loop detection function for lists of patterns.}
    \label{fig:lcheckloops-function}
\end{figure}

We prove that there is a lower bound for the gas counter
for which this function returns some result,
assuming the grammar
contains no incoherent or left-recursive rules,
and that the list of patterns only
contains coherent patterns.
In reality, we will be calling this
function while passing $g$ as the $rs$ parameter,
so, in our case,
it would suffice to state that
$g$ contains no incoherent or left-recursive rules.

\begin{lemma}%[Coherent Loops in List Termination]
    If $\lCoherent{g}{g}{true}$,
    and $\lCoherent{g}{rs}{true}$,
    and $\lVerifyRule{g}{g}{true}$, \\
    then, $\forall gas \ge \size{rs} + (\length{g} + 1) \cdot \size{g}$,
    $\exists b\ \lcheckloopscomp{g}{rs}{gas} = \Some b$.
\end{lemma}

In order to abstract away the gas counter,
and to help us in later induction proofs,
we also define an equivalent inductive predicate
for this list-based degenerate loop checker.
\Cref{fig:lcheckloops} displays this predicate,
which takes a grammar and a list of patterns,
and returns a Boolean value, indicating
whether none of the patterns in the list
contain a degenerate loop.

\begin{figure}
    \centering
    \begin{mathpar}
    \namedinferrule{lcl-nil}
    { }
    {\lCheckLoops{g}{nil}{false}}

    \namedinferrule{lcl-cons}
    {\CheckLoops{g}{r}{d}{\Some b_1} \\ \lCheckLoops{g}{rs}{b_2}}
    {\lCheckLoops{g}{r::rs}{b_1 \vee b_2}}
\end{mathpar}
    \caption{The degenerate loop detection predicate for lists of patterns.}
    \label{fig:lcheckloops}
\end{figure}

As usual, we prove some basic lemmas
about this predicate.
\Cref{lemma:lcheckloops-determinism}
states that it is
deterministic,
and \Cref{lemma:lcheckloops-follows}
states that it follows
the fixed-point definition.

\begin{lemma}
    \label{lemma:lcheckloops-determinism}
    If $\lCheckLoops{g}{rs}{b_1}$,
    and $\lCheckLoops{g}{rs}{b_2}$,
    then $b_1=b_2$.
\end{lemma}

\begin{lemma}
    \label{lemma:lcheckloops-follows}
    If $\lcheckloopscomp{g}{rs}{gas} = \Some b$,
    then $\lCheckLoops{g}{rs}{b}$.
\end{lemma}

We also prove that if a list of patterns
passes this list-based check,
then each pattern in this list also
passes the individual check.

\begin{lemma}%[Check Loops in List Safety]
    \label{lemma:lcheckloops-safety}
    If $\lCheckLoops{g}{rs}{false}$,
    then $\forall r \in rs, \exists d\ \CheckLoops{g}{r}{d}{\Some false}$.
\end{lemma}

\section{Correctness}

Having introduced each step of the well-formedness algorithm,
we can now present the definition of the \textit{\verifygrammarname{}} function,
which implements the algorithm step-by-step.
It starts by checking whether the grammar defines a first rule,
and whether every rule in the grammar is coherent.
It then makes sure the grammar contains no left-recurive rules,
and no degenerate loops, in this order.
\Cref{fig:verifygrammar-function} displays the function.
\begin{figure}
    \centering
    \begin{equation*}
    \begin{aligned}[t]
        & \verifygrammarcomp{g}{gas} = \\
        & \begin{aligned}[t]
            & \matchwith{\coherentfunc{g}{\PNT{0}}} \\
            & \matchcase{true}{\begin{aligned}[t]
                & \matchwith{\lcoherentfunc{g}{g}} \\
                & \matchcase{true}{\begin{aligned}[t]
                    & \matchwith{\lverifyrulecomp{g}{g}{gas}} \\
                    & \matchcase{\Some true}{\begin{aligned}[t]
                        & \matchwith{\lcheckloopscomp{g}{g}{gas}} \\
                        & \matchcase{\Some b}{\Some \neg b} \\
                        & \matchcase{\None}{\None} \\
                        & \matchend{}
                    \end{aligned}} \\
                    & \matchcase{res}{res} \\
                    & \matchend{}
                \end{aligned}} \\
                & \matchcase{false}{\Some false} \\
                & \matchend{}
            \end{aligned}} \\
            & \matchcase{false}{\Some false} \\
            & \matchend{}
        \end{aligned}
    \end{aligned}
\end{equation*}
    \caption{The well-formedness function with gas.}
    \label{fig:verifygrammar-function}
\end{figure}

Now, let us prove that the well-formedness check is correct.
To do so, we first define an
inductive predicate equivalent to
the \textit{\verifygrammarname{}} function
to help us in proofs by induction.
\Cref{fig:verifygrammar}
presents this predicate,
\begin{figure}
    \centering
    \begin{mathpar}
    \namedinferrule{vg1}
    {\Coherent{g}{\PNT{0}}{false}}
    {\VerifyGrammar{g}{false}}
    
    \namedinferrule{vg2}
    {\Coherent{g}{\PNT{0}}{true} \\ \lCoherent{g}{g}{false}}
    {\VerifyGrammar{g}{false}}

    \namedinferrule{vg3}
    {\Coherent{g}{\PNT{0}}{true} \\ \lCoherent{g}{g}{true} \\ \lVerifyRule{g}{g}{false}}
    {\VerifyGrammar{g}{false}}

    \namedinferrule{vg4}
    {\Coherent{g}{\PNT{0}}{true} \\ \lCoherent{g}{g}{true} \\ \lVerifyRule{g}{g}{true} \\ \lCheckLoops{g}{g}{true}}
    {\VerifyGrammar{g}{false}}

    \namedinferrule{vg5}
    {\Coherent{g}{\PNT{0}}{true} \\ \lCoherent{g}{g}{true} \\ \lVerifyRule{g}{g}{true} \\ \lCheckLoops{g}{g}{false}}
    {\VerifyGrammar{g}{true}}
\end{mathpar}
    \caption{The well-formedness predicate.}
    \label{fig:verifygrammar}
\end{figure}
which takes a grammar
and returns a Boolean value
indicating whether the grammar
passes all the checks.
It is based on the predicates
presented in the previous sections.
\Cref{lemma:verifygrammar-determinism}
states that the predicate is deterministic,
and \Cref{lemma:verifygrammar-follows}
states that it follows the fixed-point definition.

\begin{lemma}
    \label{lemma:verifygrammar-determinism}
    If $\VerifyGrammar{g}{b_1}$,
    and $\VerifyGrammar{g}{b_2}$,
    then $b_1 = b_2$.
\end{lemma}

\begin{lemma}
    \label{lemma:verifygrammar-follows}
    If $\verifygrammarcomp{g}{gas} = \Some b$,
    then $\VerifyGrammar{g}{b}$.
\end{lemma}

In order to prove correctness,
we need to generalize the pattern from $\PNT{0}$
to any pattern $p$,
and to break down the function \textit{wf}
into its separate steps.
We need to make this generalization
because the match predicate is defined
recursively on the current pattern.
The generalized theorem we need to prove is the following:
Given a grammar $g$ and a pattern $p$,
if $g$ only contains coherent rules
that do not lead to left recursion
and that do not have any degenerate loops,
and if $p$ is coherent
and does not have degenerate loops,
then ${\forall s, \exists res\ \Matches{g}{p}{s}{res}}$.

We begin the proof by doing a strong induction on $n$,
the length of the input string $s$,
which gives us the inductive hypothesis ``IHn''.
This hypothesis states that for any input string shorter than $s$
and any pattern, we can assume that the match yields some result.
From \Cref{lemma:no-lr-rule-in-grammar}
and the assumption that the grammar contains no left-recursive rules,
we can infer that the pattern also does not lead to left recursion.
This gives us the inductive predicate
$\VerifyRule{g}{p}{d}{nb}{\Some b}{k}$,
which tells us that $p$ is either nullable or non-nullable.
We do an induction on this predicate
and handle each case separately.

Let us start with the basic patterns.
The case of the empty pattern $\PEmpty{}$ is trivial,
as it matches any input string without consuming anything.
The case of the character set pattern $\PSet{cs}$ is also simple.
If $s$ is the empty string, the pattern fails to match.
Otherwise, the string may or may not begin with the character $a \in cs$.
If it does, then it matches while consuming $a$.
Otherwise, it fails to match.

The sequence pattern $\PSequence{p_1}{p_2}$ has two cases:
one in which $p_1$ is non-nullable and $p_2$ is not visited,
and another in which $p_1$ is nullable and $p_2$ is visited.
In both cases, we have an inductive hypothesis stating that
$p_1$ has a match result for input string $s$.
If this match result is a failure,
then the whole sequence $\PSequence{p_1}{p_2}$ also fails.
If the match result is a success,
then $p_1$ leaves a suffix string $s'$ unconsumed.
Let us handle this case for both scenarios separately.

If $p_1$ is non-nullable,
then we can use \Cref{lemma:non-nullable-pattern-proper-suffix}
to state that $s'$ is a proper suffix of $s$,
and therefore shorter than $s$.
This allows us to use IHn,
and state that $p_2$ has a match result for the input string $s'$.
This implies that the sequence has this same match result.

Otherwise, if $p_1$ is nullable, then we can use
\Cref{lemma:match-suffix} to state that
$s'$ is a suffix of $s$.
This means that $s'$ is either equal to $s$, or a proper suffix of $s$.
If $s$ is equal to $s'$,
then we can use the inductive hypothesis
to state that $p_2$ yields a match result for $s'=s$.
Otherwise, then we can use IHn in the same way as the previous case,
because $s'$ would be shorter than $s$.

Now let us consider the case of the choice pattern $\PChoice{p_1}{p_2}$.
Similar to the case of the sequence pattern,
we have induction hypotheses for $p_1$ and $p_2$
yielding a match result for the input string $s$.
This case is simpler because $s$ is the same input string for both choice options,
so these induction hypotheses are enough to prove this case.

The case of the repetition pattern $\PRepetition{p}$ is the most interesting one,
as we get to use the fact that $p$ must be non-nullable in the proof.
From the induction step, we have the inductive hypothesis that
$p$ yields a match result for the input string $s$.
If $p$ fails to match, then $\PRepetition{p}$ matches without consuming anything.
If, otherwise, $p$ matches, then it leaves a string $s'$ unconsumed.
Since $p$ is non-nullable, $s'$ must be a proper suffix of $s$.
Therefore, we can use IHn to state that $\PRepetition{p}$
yields a match result $res$ for $s'$, because it is shorter than $s$.
In this case, $\PRepetition{p}$ also yields $res$ for $s$.

The case of the predicates $\PNot{p}$ and $\PAnd{p}$ are pretty straightforward.
We first use the inductive hypothesis that $p$
yields a match result for the input string $s$.
If $p$ matches,
then $\PAnd{p}$ matches without consuming anything,
and $\PNot{p}$ fails to match.
Otherwise, if $p$ fails to match,
then so does $\PAnd{p}$,
and $\PNot{p}$ matches without consuming anything.

Finally, we prove the case of the non-terminal pattern $\PNT{i}$,
which is surprisingly simple.
From the induction step,
we are given $p$, the $i^{th}$ rule of the grammar.
From the initial hypotheses,
we know that $p$ is coherent and free of degenerate loops,
because it is a grammar rule.
We can then use the inductive hypothesis
from the $\verifyrulename{}$ predicate to
state that $p$ yields a match result for the input string $s$.
\begin{theorem}
    \label{theorem:wf-correctness-generalized}
    Given a grammar $g$ and a pattern $p$,
    if $g$ only contains coherent rules
    that do not lead to left recursion
    and that free of degenerate loops,
    and if $p$ is coherent
    and free of degenerate loops,
    then $\forall s, \exists res\ \Matches{g}{p}{s}{res}$.
\end{theorem}
Having proved \Cref{theorem:wf-correctness-generalized},
we can finally prove the original theorem,
which states that, for any grammar $g$
that satisfies $\wf{g} = true$,
and for any input string,
the non-terminal pattern $\PNT{0}$
yields a match result.

\Cref{lemma:verifygrammar-termination} states that
the function \textit{\verifygrammarname{}} returns
$\Some b$ when given a gas counter greater or equal
to $\verifygrammargas{g}$.
In the implementation of the function \textit{wf},
we pass $\verifygrammargas{g}$ as the gas counter
for the function \textit{\verifygrammarname{}}.
Therefore, if \textit{wf} returns $true$,
then it must be because \textit{\verifygrammarname{}}
returned $\Some true$.

If the function \textit{\verifygrammarname{}}
returns $\Some true$,
then,
according to \Cref{lemma:verifygrammar-follows},
we can construct the predicate $\VerifyGrammar{g}{true}$.
We can see that this only happens when
the input grammar has passed
all the checks.
From \Cref{fig:verifygrammar},
we can see that this implies in
several other predicates,
many of which are necessary to use
\Cref{theorem:wf-correctness-generalized}.
There are only two missing predicates:
$\Coherent{g}{\PNT{0}}{true}$,
which states that the initial pattern is coherent,
and $\CheckLoops{g}{\PNT{0}}{d}{false}$,
which states that it does not contain any degenerate loops.

We can derive both predicates
from the fact that $\PNT{0}$ is a rule,
which we have checked already.
We use \Cref{lemma:lcoherent-safety}
to prove that $\PNT{0}$ is coherent,
and \Cref{lemma:lcheckloops-safety}
to prove that it contains no degenerate loops.
With this, we are able to prove the original theorem.

\begin{theorem}
    For any grammar $g$,
    if $\wf{g} = true$,
    then $g$ is complete.
\end{theorem}
