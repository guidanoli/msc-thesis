\section{A simpler algorithm for detecting nullable patterns}
\label{section:nullable}

Once we have made sure a grammar is free of left-recursive rules,
the next step is to check for degenerate loops,
which involves checking if certain patterns are nullable.
To this end,
we could use the algorithm we have just described in \Cref{section:lr-rules},
but \lpeg{} implements a simpler version,
which takes advantage of the fact that
the grammar contains no left-recursive rules.

The difference between the algorithm from \Cref{section:lr-rules}
and this simpler version can be seen,
for example, in the case of choice patterns $\PChoice{p_1}{p_2}$.
More specifically, if $p_1$ is nullable,
the function from \Cref{section:lr-rules}
would still need to evaluate $p_2$,
as it could potentially lead to left recursion.
Meanwhile, if we assume that $p_2$ cannot lead to left recursion,
then if $p_1$ is nullable,
we can state that $\PChoice{p_1}{p_2}$ is nullable,
without having to visit $p_2$.
We can also avoid visiting sub-patterns in the cases of
repetition patterns and predicate patterns,
because they are all nullable.

In \Cref{fig:nullable-function},
\begin{figure}[ht!]
    \centering
    \begin{align*}
    \begin{aligned}[t]
        & \nullablecomp{g}{p}{d}{0} = \None \\
        & \nullablecomp{g}{p}{d}{(1+gas)} = \\
        & \begin{aligned}[t]
            & \matchwith{p} \\
            & \matchcase{\PEmpty}{\Some \Some true} \\
            & \matchcase{\PSet{cs}}{\Some \Some false} \\
            & \matchcase{\PRepetition{p}}{\Some \Some true} \\
            & \matchcase{\PNot{p}}{\Some \Some true} \\
            & \matchcase{\PAnd{p}}{\Some \Some true} \\
            & \matchcase{\PNT{i}}{\begin{aligned}[t]
                & \matchwith{g[i]} \\
                & \matchcase{\Some p}{\begin{aligned}[t]
                    & \matchwith{d} \\
                    & \matchcase{0}{\Some \None} \\
                    & \matchcase{1+d'}{\nullablecomp{g}{p}{d'}{gas}} \\
                    & \matchend{}
                \end{aligned}} \\
                & \matchcase{\None}{\None} \\
                & \matchend{}
            \end{aligned}} \\
            & \matchcase{\PSequence{p_1}{p_2}}{\begin{aligned}[t]
                & \matchwith{\nullablecomp{g}{p_1}{d}{gas}} \\
                & \matchcase{\Some \Some true}{\nullablecomp{g}{p_2}{d}{gas}} \\
                & \matchcase{res}{res} \\
                & \matchend{}
            \end{aligned}} \\
            & \matchcase{\PChoice{p_1}{p_2}}{\begin{aligned}[t]
                & \matchwith{\nullablecomp{g}{p_1}{d}{gas}} \\
                & \matchcase{\Some \Some false}{\nullablecomp{g}{p_2}{d}{gas}} \\
                & \matchcase{res}{res} \\
                & \matchend{}
            \end{aligned}} \\
            & \matchend{}
        \end{aligned}
    \end{aligned}
\end{align*}
    \caption{The nullable function.}
    \label{fig:nullable-function}
\end{figure}
we define this simpler version as a fixed-point,
which takes a grammar,
a pattern, a stack depth limit, and a gas counter,
and returns an optional label.
Possible return values are, therefore,
$\None$ (out-of-gas), $\Some \None$ (stack overflow),
$\Some \Some true$ (nullable), and $\Some \Some false$ (non-nullable).

The gas parameter is still necessary to convince Coq that
the function terminates,
and because it could be called
with an incoherent pattern or
with a grammar that contains an incoherent rule.
The stack depth limit is also necessary
because the function could be called
with a grammar that contains a left-recursive rule.
For these two reasons,
this function
may still return $\None$ (out-of-gas)
or $\Some \None$ (stack overflow).

In reality, however,
this function should only be called after
the grammar and pattern are guaranteed to be
coherent and free of left-recursive rules.
For this reason,
implementations of this algorithm
such as the one in \lpeg{}
are able to safely drop both parameters,
and return just a Boolean value
indicating whether the pattern is nullable or not.

Also note that, unlike the function from \Cref{fig:vr-function},
this function does not receive a nullable accumulator as parameter
and does not return a stack.
The nullable accumulator is not necessary,
thanks to the assumption that
the grammar contains no left-recursive rules,
and the output stack was only used for proofs,
which we will be able to reuse from \Cref{section:lr-rules}.

Just as we did for the function from \Cref{section:lr-rules},
we prove that this simpler version is also
monotonic
with respect to the gas counter.

\begin{lemma}%[Nullable Gas Monotonicity]
    If $\nullablecomp{g}{p}{d}{gas} = \Some res$, \\
    then, $\forall gas' \ge gas$,
    $\nullablecomp{g}{p}{d}{gas'} = \Some res$.
    \label{lemma:nullable-gas-monotonicity}
\end{lemma}

Besides that, we also prove the function
is eventually constant with respect to the gas counter
by giving a gas lower bound for which the function
returns some result for any coherent pattern and grammar.

\begin{lemma}%[Nullable Termination]
    If $\Coherent{g}{p}{true}$,
    and $\lCoherent{g}{g}{true}$, \\
    then, $\forall gas \ge \size{p} + d \cdot \size{g}$,
    $\exists res\ \nullablecomp{g}{p}{d}{gas} = \Some res$.
    \label{lemma:nullable-termination}
\end{lemma}

Furthermore, we would also like to prove that
this function is eventually constant and monotonic
with respect to the stack depth limit.
However, in order to do that,
we found it better to first define an equivalent inductive predicate.
\Cref{fig:nullable-predicate} defines the predicate
\begin{figure}
    \centering
    \begin{mathpar}
    \namedinferrule{n-eps}
    { }
    {\Nullable{g}{\PEmpty}{d}{\Some true}}

    \namedinferrule{n-set}
    { }
    {\Nullable{g}{\PSet{cs}}{d}{\Some false}}

    \namedinferrule{n-rep}
    { }
    {\Nullable{g}{\PRepetition{p}}{d}{\Some true}}

    \namedinferrule{n-not}
    { }
    {\Nullable{g}{\PNot{p}}{d}{\Some true}}

    \namedinferrule{n-and}
    { }
    {\Nullable{g}{\PAnd{p}}{d}{\Some true}}

    \namedinferrule{n-nonterminal-zero}
    {g[i] = \Some p}
    {\Nullable{g}{\PNT{i}}{0}{\None}}
    
    \namedinferrule{n-nonterminal-succ}
    {g[i] = \Some p \\ \Nullable{g}{p}{d}{res}}
    {\Nullable{g}{\PNT{i}}{1+d}{res}}

    \namedinferrule{n-seq-none}
    {\Nullable{g}{p_1}{d}{\None}}
    {\Nullable{g}{\PSequence{p_1}{p_2}}{d}{\None}}
    
    \namedinferrule{n-seq-some-false}
    {\Nullable{g}{p_1}{d}{\Some false}}
    {\Nullable{g}{\PSequence{p_1}{p_2}}{d}{\Some false}}
    
    \namedinferrule{n-seq-some-true}
    {\Nullable{g}{p_1}{d}{\Some true} \\ \Nullable{g}{p_2}{d}{res}}
    {\Nullable{g}{\PSequence{p_1}{p_2}}{d}{res}}

    \namedinferrule{n-choice-none}
    {\Nullable{g}{p_1}{d}{\None}}
    {\Nullable{g}{\PChoice{p_1}{p_2}}{d}{\None}}
    
    \namedinferrule{n-choice-some-false}
    {\Nullable{g}{p_1}{d}{\Some false} \\ \Nullable{g}{p_2}{d}{res}}
    {\Nullable{g}{\PChoice{p_1}{p_2}}{d}{res}}
    
    \namedinferrule{n-choice-some-true}
    {\Nullable{g}{p_1}{d}{\Some true}}
    {\Nullable{g}{\PChoice{p_1}{p_2}}{d}{\Some true}}
\end{mathpar}
    \caption{The nullable predicate.}
    \label{fig:nullable-predicate}
\end{figure}
$\Nullable{g}{p}{d}{res}$ which takes a grammar $g$,
a pattern $p$, a stack depth limit $d$, and returns a label $res$.

About this predicate, we proved some basic lemmas.
First, we proved that it is deterministic.

\begin{lemma}
    If $\Nullable{g}{p}{d}{res_1}$,
    and $\Nullable{g}{p}{d}{res_2}$,
    then $res_1 = res_2$.
\end{lemma}

We also proved that it follows
the fixed-point definition.

\begin{lemma}
    If $\nullablecomp{g}{p}{d}{gas} = \Some res$,
    then $\Nullable{g}{p}{d}{res}$.
\end{lemma}

We also tied this predicate to the
one from \Cref{section:lr-rules},
showing how similar they are,
when the nullable accumulator is $false$,
and the pattern is either nullable or non-nullable.

\begin{lemma}%[Verify Rule Similar to Nullable]
    If $\VerifyRule{g}{p}{d}{false}{\Some b}{k}$,
    then $\Nullable{g}{p}{d}{\Some b}$.
\end{lemma}

We also proved that if a pattern
was identified as either nullable or non-nullable,
then increasing the stack depth does not impact the result.

\begin{lemma}%[Stack Depth Limit Increase Without Overflow]
    If $\Nullable{g}{p}{d}{\Some b}$,
    then $\forall d' \ge d, \Nullable{g}{p}{d'}{\Some b}$.
\end{lemma}

Maybe the most important lemma
about the nullable predicate
relates to the match predicate.
It states that a non-nullable pattern
never matches without consuming
some part of the input string.

\begin{lemma}%[Non-nullable pattern and Match]
    If $\Nullable{g}{p}{d}{\Some false}$,
    then $\nexists s$ such that $\Matches{g}{p}{s}{s}$.
\end{lemma}

This lemma is used to prove that,
when matching a non-nullable pattern,
the output string is a proper suffix of the input string.
We use the symbol ``$\ProperSuffix{}{}$''
to denote this relation.

\begin{lemma}%[Non-nullable pattern and Proper Suffix]
    \label{lemma:non-nullable-pattern-proper-suffix}
    If $\Nullable{g}{p}{d}{\Some false}$,
    and $\Matches{g}{p}{s}{s'}$,
    then $\ProperSuffix{s'}{s}$.
\end{lemma}

Finally, we show that the predicate is
eventually constant
with respect to the stack depth limit,
past a lower bound given by the number of rules in the grammar,
denoted as $\length{g}$.

\begin{lemma}%[Nullable Stack Depth Limit Eventual Constancy]
    If $\Coherent{g}{p}{true}$,
    and $\lCoherent{g}{g}{true}$,
    and $\lVerifyRule{g}{g}{true}$, \\
    and $\Nullable{g}{p}{d}{res}$,
    where $d > \length{g}$,
    then, $\forall d' \ge d$,
    $\Nullable{g}{p}{d'}{res}$.
\end{lemma}
