\section{Matching the empty string}

Besides the first-set of a pattern,
the algorithm implemented in \lpeg{} also returns
a Boolean value, which indicates whether the
pattern may match the empty string,
a property we call \emph{emptiness}.
It is another conservative approximation:
the value $true$ has no meaning,
while the value $false$ indicates that the pattern
fails to match the empty string.
\lpeg{} needs this information
because it cannot use the first-set
to verify whether a pattern fails the empty string,
since the empty string has no first character.

Let us see how \lpeg{} computes
the emptiness of patterns.
The base cases are quite simple.
The empty pattern $\PEmpty$ and
the repetition pattern $\PRepetition{p}$
match every input string,
which includes the empty string.
So, for these patterns,
the function returns $true$.

The character class pattern $\PSet{cs}$,
as with any non-nullable pattern,
does not match the empty string.
Therefore, the value for this pattern is $false$.

For the not-predicate pattern $\PNot{p}$,
\lpeg{} is rather conservative,
always returning $true$.
However, we believe there is a way
to make the emptiness value
not a conservative approximation
but an exact representation of whether
the pattern matches the empty string:
The function could call itself recursively for $p$
and negate its emptiness value.
However, the cases of $\PNot{p}$
in which $p$ matches the empty string
are not useful in practice.

The and-predicate pattern $\PAnd{p}$
and the non-terminal pattern $\PNT{i}$
simply forward the Boolean value
from the underlying pattern,
because they fail if and only if
the underlying pattern fails.

The sequence pattern $\PSequence{p_1}{p_2}$
matches the empty string if
both $p_1$ and $p_2$ do.
Intuitively, this would mean that the emptiness value of the sequence
would be the Boolean \scand{} of the emptiness values of $p_1$ and $p_2$,
but that is not exactly what is implemented in the algorithm.
As we have discussed at the end of the previous section,
when $p_1$ is non-nullable, $p_2$ is not visited,
and, therefore, the emptiness value of $p_2$ is not calculated.
However, we don't need this value,
since the emptiness value of $p_1$ is $false$ in this case,
which allows for the short-circuit evaluation
of the Boolean \scand{} expression to $false$.
Meanwhile, when $p_1$ is nullable,
the emptiness value of both $p_1$ and $p_2$ are computed,
and their Boolean \scand{} is calculated normally.

Finally, the case of the choice pattern $\PChoice{p_1}{p_2}$
is similar to that of the sequence pattern,
but instead of a Boolean \scand{} operation,
it performs a Boolean \scor{}
of the emptiness values of $p_1$ and $p_2$.
That is because the choice matches the empty string
if one of the options does.