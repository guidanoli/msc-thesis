\section{Building intuition}

Before we formally define the algorithm,
we would like to first develop some intuition
for the function signature.
We start with an informal definition:
the function receives a pattern
and returns the set of first characters that
can be accepted by the pattern.
Let us try to informally define the first-set
of some simple patterns
using this initial signature.

\newcommand{\firsta}[1]{\firstname{}(#1)}

Starting with $\PEmpty$ and $\PRepetition{p}$,
we know both patterns accept every input string,
so it makes sense for this function to return
the full character set, which we represent as $\Sigma$.
As for the character set pattern $\PSet{cs}$,
we know that it only accepts strings that start with
a character in the set $\Set{cs}$.
\begin{align*}
    \firsta{\PEmpty} &= \Sigma \\
    \firsta{\PRepetition{p}} &= \Sigma \\
    \firsta{\PSet{cs}} &= \Set{cs}
\end{align*}

However, when we try to define the first-set of sequence patterns,
we run into some issues.
To better comprehend them,
it can be helpful to mentally compute the first-set of
some simple sequences,
using the pattern types for which
we have defined the function so far.
This may give us some intuition as to how the first-set
of sequences may be derived from its sub-patterns.
It is worth highlighting that these definitions
are informal and derived purely from intuition.
The actual formal definitions
are presented later in this chapter.
\begin{align*}
    \firsta{\PSequence{\PEmpty}{\PEmpty}} &= \Sigma \\
    \firsta{\PSequence{\PSet{cs_1}}{\PEmpty}} &= \Set{cs_1} \\
    \firsta{\PSequence{\PEmpty}{\PSet{cs_2}}} &= \Set{cs_2} \\
    \firsta{\PSequence{\PSet{cs_1}}{\PSet{cs_2}}} &= \Set{cs_1} \\
    \firsta{\PSequence{\PSet{cs_1}}{\PRepetition{\PSet{cs_2}}}} &= \Set{cs_1} \\
    \firsta{\PSequence{\PRepetition{\PSet{cs_1}}}{\PEmpty}} &= \Sigma \\
    \firsta{\PSequence{\PRepetition{\PSet{cs_1}}}{\PSet{cs_2}}} &=
    \SetUnion{\Set{cs_1}}{\Set{cs_2}} \\
    \firsta{\PSequence{\PRepetition{\PSet{cs_1}}}{\PRepetition{\PSet{cs_2}}}} &=
    \Sigma
\end{align*}

From the examples above,
we can observe that when $p_1$ is non-nullable,
the first-set of $p_2$ doesn't seem to be relevant
to the first-set of the sequence.
That is because $p_1$, being non-nullable,
is guaranteed to consume a character.
As a result, $p_2$ is given a proper suffix of the original string,
which doesn't contain its first character.
Based on this observation,
we can define the function
for sequences $\PSequence{p_1}{p_2}$
where $p_1$ is non-nullable
as the first-set of $p_1$.
\begin{align*}
    \firsta{\PSequence{p_1}{p_2}} &= \begin{cases}
        \firsta{p_1} & \text{if $p_1$ is non-nullable} \\
        \dots ? & \text{if $p_1$ is nullable}
    \end{cases}
\end{align*}

We still need to define the function
for the case where $p_1$ is nullable.
From the patterns we have defined so far,
only the empty pattern $\PEmpty$ and
the repetition pattern $\PRepetition{\PSet{cs_1}}$ are nullable.
Let us see their behavior in sequence with
the character set pattern $\PSet{cs_2}$.
\begin{align*}
    \firsta{\PSequence{\PEmpty}{\PSet{cs_2}}} &= \Set{cs_2} \\
    \firsta{\PSequence{\PRepetition{\PSet{cs_1}}}{\PSet{cs_2}}} &= \SetUnion{\Set{cs_1}}{\Set{cs_2}}
\end{align*}

Note that, when alone, their first-sets are equal.
However, when in sequence with the same pattern,
the resulting first-sets differ.
This discrepancy indicates that we cannot define
$\firsta{\PSequence{p_1}{p_2}}$ as a pure function
of $\firsta{p_1}$ and $\firsta{p_2}$.
Instead, we would have to break the definition down
for each case of $p_1$.
\begin{align*}
    \firsta{\PSequence{\PEmpty}{p_2}} &= \firsta{p_2} \\
    \firsta{\PSequence{\PRepetition{p}}{p_2}} &=
    \SetUnion{\firsta{p}}{\firsta{p_2}} \\
    & \dots
\end{align*}

With this signature,
we would have to define the function for
every nullable pattern twice:
one when alone, and another when followed by another pattern.
However, we would like to define the function recursively
only once for each pattern type.

\newcommand{\firstb}[2]{\firstname{}(#1, #2)}

To solve this issue, \lpeg{} adds an accumulator parameter
for the first-set of the following pattern in the sequence.
When the pattern is not followed by a pattern,
\lpeg{} uses the full character set as the accumulator.
We call this accumulator the \emph{follow-set}.
Let us adapt our definitions for this new signature:
For each basic pattern $p_1$,
we can derive the definition of $\firstb{p_1}{follow}$
from the previous definition of $\firsta{\PSequence{p_1}{p_2}}$,
and by replacing $\firsta{p_2}$ with the new parameter $follow$.
\begin{align*}
    \firstb{\PEmpty}{follow} &= follow \\
    \firstb{\PRepetition{p}}{follow} &= \SetUnion{\firstb{p}{follow}}{follow} \\
    \firstb{\PSet{cs}}{follow} &= \Set{cs}
\end{align*}

The first-set of the character set pattern $\PSet{cs}$
does not use the follow parameter,
because it doesn't depend on the first-set of the following pattern.
This is because the character set pattern is non-nullable.
We later prove this property for any non-nullable pattern.

The case of the sequence pattern is interesting,
as it demonstrates the follow-set working as an accumulator:
In the case where $p_1$ is nullable,
we use the first-set of $p_2$ as the follow-set of $p_1$.
As for the case in which $p_1$ is non-nullable,
we use the first-set of $p_1$ with any follow-set.
We choose the full character set as the follow-set
to demonstrate this independence from the accumulator,
but any character set could be used.
\begin{align*}
    \firstb{\PSequence{p_1}{p_2}}{follow} &= \begin{cases}
        \firstb{p_1}{\firstb{p_2}{follow}} & \text{if $p_1$ is nullable} \\
        \firstb{p_1}{\Sigma} & \text{if $p_1$ is non-nullable}
    \end{cases}
\end{align*}

Let us now define the first-set of the other pattern types.
Starting with the choice pattern $\PChoice{p_1}{p_2}$,
we intuitively define it as the union
of the first-sets of $p_1$ and $p_2$,
passing down the follow-set parameter to each sub-call.
\begin{align*}
    \firstb{\PChoice{p_1}{p_2}}{follow} &=
    \SetUnion{\firstb{p_1}{follow}}{\firstb{p_2}{follow}}
\end{align*}

The case of the not-predicate pattern $\PNot{p}$
highlights the conservative nature of first-sets.
From the first-set of $p$,
we can only infer the set of first characters that make $p$ fail,
and, therefore, make $\PNot{p}$ succeed.
This, however, gives us no information about
first characters that make $p$ succeed,
and, therefore, make $\PNot{p}$ fail.
Therefore, in general, we cannot use the first-set of $p$
to compute the first-set of $\PNot{p}$.

There is, however, information
that we can extract from the follow-set parameter.
When $\PNot{p}$ is followed by a pattern $p_2$,
the follow-set indicates which first characters make $p_2$ fail.
Given that $\PNot{p}$ doesn't consume any input,
these first characters also make the sequence $\PSequence{\PNot{p}}{p_2}$ fail,
so they should be part of the first-set of $\PNot{p}$.
Therefore, we could, in general, use the follow-set
as the first-set of any predicate.
However, for the specific case of $\PNot{\PSet{cs}}$,
\lpeg{} instead computes the first-set
of patterns as ${\SetMinus{\Sigma}{\Set{cs}}}$.
\begin{align*}
    \firstb{\PNot{p}}{follow} &= \begin{cases}
        \SetMinus{\Sigma}{\Set{cs}} & \text{if $p = \PSet{cs}$} \\
        follow & \text{otherwise}
    \end{cases}
\end{align*}

One topic for future research is
to investigate whether it would be possible to
replace $\Sigma$ with the provided follow-set parameter in \lpeg{}.

As for the and-predicate pattern $\PAnd{p}$,
we use both the first-set of $p$ and the follow-set.
That is because in order to match the sequence $\PSequence{\PAnd{p}}{p_2}$,
the input string must match both $p$ and $p_2$.
Conversely, if the string starts with a character that is not in the
first-set of either pattern, the sequence fails.
So, in this case,
we define the first-set
as the intersection of both sets.
\begin{align*}
    \firstb{\PAnd{p}}{follow} &= \SetIntersection{\firstb{p}{\Sigma}}{follow}
\end{align*}

One subtle difference between this definition
and the actual implementation of the algorithm in \lpeg{} is
the follow-set of $p$.
While \lpeg{} simply passes along the follow-set parameter,
we provide the full character set $\Sigma$.
This change was necessary for us to prove the
key property of first-sets in Coq.
Future research may check whether
passing along the follow-set parameter
is incorrect, equivalent, or even better
than passing $\Sigma$ as the follow-set.

\newcommand{\firstc}[3]{\firstname{}(#1, #2, #3)}

Finally, in order to define the case of the non-terminal pattern,
we need to add the grammar as a parameter,
so that we can look up the referenced grammar rule.
In all cases, this grammar parameter is simply
passed along to each recursive call.
\begin{align*}
    \firstc{g}{\PNT{i}}{follow} &= \firstc{g}{p}{follow} & \text{if $g[i] = \Some p$}
\end{align*}

This case brings up the topic of termination,
as it does not define the recursion on
the structure of the pattern,
like the other cases do.
Instead, termination in this case
relies on the assumption
that the input PEG is well-formed,
and, therefore, free of left-recursive rules.

On a deeper level, termination is derived from
the way in which the well-formedness and first-set algorithms
traverse patterns similarly.
The most interesting case
is that of the sequence pattern:
when $p_1$ is non-nullable,
both algorithms do not visit $p_2$.
In the case of the well-formedness check,
visiting $p_2$ is not necessary
because the whole sequence is non-nullable,
and any rules visited in $p_2$ would be matched
against a proper suffix of the input string $s$
(avoiding infinite loops).

Meanwhile, in the case of the first-set algorithm,
visiting $p_2$ can be avoided for two reasons:
If $p_1$ is non-nullable,
then, as we later prove, its emptiness value is $false$,
which means that it fails the empty string.
If $p_1$ fails the empty string,
then so does the sequence,
which allows the emptiness value of the sequence to also be $false$,
regardless of the emptiness value of $p_2$.
The second reason is that, if $p_1$ is non-nullable,
then its first-set is independent of the follow-set parameter,
which, in the general case, would be the first-set of $p_2$.
Therefore, when $p_1$ is non-nullable,
we can provide any follow-set parameter, such as $\Sigma$,
in order to avoid making a recursive call to $p_2$.