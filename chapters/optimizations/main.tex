\chapter{First-set Algorithm}
\label{chapter:first-set}

In \Cref{chapter:wf-algorithm},
we presented the well-formedness check
implemented in \lpeg{},
which ensures that the input PEG is complete.
However, this is not the only role of this algorithm:
It also ensures that other algorithms
implemented in \lpeg{} terminate,
as they traverse patterns in similar ways.
This chapter presents one such algorithm,
which we label as the first-set algorithm.

The first-set algorithm is responsible for
computing the set of first characters that
can possibly be accepted by a pattern,
and a Boolean value
that indicates whether the pattern may
accept the empty string.
We call this Boolean value the
\emph{emptiness} value of the pattern.
The key properties of the algorithm,
which we later prove,
are the following:
Any pattern fails any string that starts
with a character that is not in the first-set
of the pattern;
and the pattern fails the empty string
if its emptiness value is $false$.

Both are conservative approximations,
which means that a full first-set
and an emptiness value of $true$ are
the safest options, yet the least useful ones.
That is because \lpeg{} uses this algorithm and its properties
to optimize certain patterns in the code generation phase.
So, ideally, we would like the first-set to be as small as possible,
and the emptiness value of $false$ as common as possible,
so that \lpeg{} is able to optimize code more often.

One of the patterns that \lpeg{} tries to optimize
using this algorithm is the ordered choice.
As a base reference, we display below the code
that \lpeg{} generates for an ordered choice $\PChoice{p_1}{p_2}$
without any optimizations.
\begin{align*}
    & Choice(L_2) \\
    & p_1 \\
    & Commit(L_{end}) \\
    L_2:\ & p_2 \\
    L_{end}:\ &
\end{align*}

This unoptimized code starts with a choice instruction,
which creates a checkpoint for the initial match state,
so that, if $p_1$ fails,
it is able to restore this state
before running $p_2$.
If, instead, $p_1$ succeeds, it runs a commit instruction,
which deletes this checkpoint and jumps to $L_{end}$.

However,
if $p_1$ and $p_2$ have disjoint first-sets,
and $p_1$ has an emptiness value of $false$,
then \lpeg{} generates the following optimized code.
\begin{align*}
    & TestSet(first(p_1), L_2) \\
    & p_1 \\
    & Jump (L_{end}) \\
    L_2:\ & p_2 \\
    L_{end}:\ &
\end{align*}

This optimized code begins with an instruction (test set)
that checks whether the input string starts with
a character in the first-set of $p_1$,
and jumps to $L_2$ if not.
In the case where the input string is empty,
we know that $p_1$ would fail,
because $p_1$ has an emptiness value of $false$.
If, otherwise, the input string starts with a character
that is not in the first-set of $p_1$,
then we know that $p_1$ would also fail,
by the key property of first-sets.
So, in either case, we know that $p_1$ would fail,
making the choice pattern equivalent to $p_2$.
That is why, in the case of failure,
the test-set instruction jumps to the code of $p_2$,
without ever running the code of $p_1$.

Meanwhile, the test-set instruction succeeds
if the input string starts with a character
that \emph{is} in the first-set of $p_1$,
and, therefore, is non-empty.
In this case, because the first-sets of $p_1$ and $p_2$ are disjoint,
we know that this character is \emph{not} in the first-set of $p_2$.
And, from the key property of first-sets,
this means that $p_2$ would fail,
making the choice pattern equivalent to $p_1$.
So, in this case, \lpeg{} simply executes the code of $p_1$.

This optimization significantly improves
the performance \lpeg{} when matching some choice patterns.
This improvement is possible
by replacing the costly creation/restoration/deletion of checkpoints
with inexpensive jumps and inspections on the input string.

In this chapter, we discuss the first-set algorithm
and prove its key properties.

\section{Building intuition}

Before we formally define the algorithm,
we would like to first develop some intuition
for the function signature.
We start with an informal definition:
the function receives a pattern
and returns the set of first characters that
can be accepted by the pattern.
Let us try to informally define the first-set
of some simple patterns
using this initial signature.

\newcommand{\firsta}[1]{\firstname{}(#1)}

Starting with $\PEmpty$ and $\PRepetition{p}$,
we know both patterns accept every input string,
so it makes sense for this function to return
the full character set, which we represent as $\Sigma$.
As for the character set pattern $\PSet{cs}$,
we know that it only accepts strings that start with
a character in the set $\Set{cs}$.
\begin{align*}
    \firsta{\PEmpty} &= \Sigma \\
    \firsta{\PRepetition{p}} &= \Sigma \\
    \firsta{\PSet{cs}} &= \Set{cs}
\end{align*}

However, when we try to define the first-set of sequence patterns,
we run into some issues.
To better comprehend them,
it can be helpful to mentally compute the first-set of
some simple sequences,
using the pattern types for which
we have defined the function so far.
This may give us some intuition as to how the first-set
of sequences may be derived from its sub-patterns.
It is worth highlighting that these definitions
are informal and derived purely from intuition.
The actual formal definitions
are presented later in this chapter.
\begin{align*}
    \firsta{\PSequence{\PEmpty}{\PEmpty}} &= \Sigma \\
    \firsta{\PSequence{\PSet{cs_1}}{\PEmpty}} &= \Set{cs_1} \\
    \firsta{\PSequence{\PEmpty}{\PSet{cs_2}}} &= \Set{cs_2} \\
    \firsta{\PSequence{\PSet{cs_1}}{\PSet{cs_2}}} &= \Set{cs_1} \\
    \firsta{\PSequence{\PSet{cs_1}}{\PRepetition{\PSet{cs_2}}}} &= \Set{cs_1} \\
    \firsta{\PSequence{\PRepetition{\PSet{cs_1}}}{\PEmpty}} &= \Sigma \\
    \firsta{\PSequence{\PRepetition{\PSet{cs_1}}}{\PSet{cs_2}}} &=
    \SetUnion{\Set{cs_1}}{\Set{cs_2}} \\
    \firsta{\PSequence{\PRepetition{\PSet{cs_1}}}{\PRepetition{\PSet{cs_2}}}} &=
    \Sigma
\end{align*}

From the examples above,
we can observe that when $p_1$ is non-nullable,
the first-set of $p_2$ doesn't seem to be relevant
to the first-set of the sequence.
That is because $p_1$, being non-nullable,
is guaranteed to consume a character.
As a result, $p_2$ is given a proper suffix of the original string,
which doesn't contain its first character.
Based on this observation,
we can define the function
for sequences $\PSequence{p_1}{p_2}$
where $p_1$ is non-nullable
as the first-set of $p_1$.
\begin{align*}
    \firsta{\PSequence{p_1}{p_2}} &= \begin{cases}
        \firsta{p_1} & \text{if $p_1$ is non-nullable} \\
        \dots ? & \text{if $p_1$ is nullable}
    \end{cases}
\end{align*}

We still need to define the function
for the case where $p_1$ is nullable.
From the patterns we have defined so far,
only the empty pattern $\PEmpty$ and
the repetition pattern $\PRepetition{\PSet{cs_1}}$ are nullable.
Let us see their behavior in sequence with
the character set pattern $\PSet{cs_2}$.
\begin{align*}
    \firsta{\PSequence{\PEmpty}{\PSet{cs_2}}} &= \Set{cs_2} \\
    \firsta{\PSequence{\PRepetition{\PSet{cs_1}}}{\PSet{cs_2}}} &= \SetUnion{\Set{cs_1}}{\Set{cs_2}}
\end{align*}

Note that, when alone, their first-sets are equal.
However, when in sequence with the same pattern,
the resulting first-sets differ.
This discrepancy indicates that we cannot define
$\firsta{\PSequence{p_1}{p_2}}$ as a pure function
of $\firsta{p_1}$ and $\firsta{p_2}$.
Instead, we would have to break the definition down
for each case of $p_1$.
\begin{align*}
    \firsta{\PSequence{\PEmpty}{p_2}} &= \firsta{p_2} \\
    \firsta{\PSequence{\PRepetition{p}}{p_2}} &=
    \SetUnion{\firsta{p}}{\firsta{p_2}} \\
    & \dots
\end{align*}

With this signature,
we would have to define the function for
every nullable pattern twice:
one when alone, and another when followed by another pattern.
However, we would like to define the function recursively
only once for each pattern type.

\newcommand{\firstb}[2]{\firstname{}(#1, #2)}

To solve this issue, \lpeg{} adds an accumulator parameter
for the first-set of the following pattern in the sequence.
When the pattern is not followed by a pattern,
\lpeg{} uses the full character set as the accumulator.
We call this accumulator the \emph{follow-set}.
Let us adapt our definitions for this new signature:
For each basic pattern $p_1$,
we can derive the definition of $\firstb{p_1}{follow}$
from the previous definition of $\firsta{\PSequence{p_1}{p_2}}$,
and by replacing $\firsta{p_2}$ with the new parameter $follow$.
\begin{align*}
    \firstb{\PEmpty}{follow} &= follow \\
    \firstb{\PRepetition{p}}{follow} &= \SetUnion{\firstb{p}{follow}}{follow} \\
    \firstb{\PSet{cs}}{follow} &= \Set{cs}
\end{align*}

The first-set of the character set pattern $\PSet{cs}$
does not use the follow parameter,
because it doesn't depend on the first-set of the following pattern.
This is because the character set pattern is non-nullable.
We later prove this property for any non-nullable pattern.

The case of the sequence pattern is interesting,
as it demonstrates the follow-set working as an accumulator:
In the case where $p_1$ is nullable,
we use the first-set of $p_2$ as the follow-set of $p_1$.
As for the case in which $p_1$ is non-nullable,
we use the first-set of $p_1$ with any follow-set.
We choose the full character set as the follow-set
to demonstrate this independence from the accumulator,
but any character set could be used.
\begin{align*}
    \firstb{\PSequence{p_1}{p_2}}{follow} &= \begin{cases}
        \firstb{p_1}{\firstb{p_2}{follow}} & \text{if $p_1$ is nullable} \\
        \firstb{p_1}{\Sigma} & \text{if $p_1$ is non-nullable}
    \end{cases}
\end{align*}

Let us now define the first-set of the other pattern types.
Starting with the choice pattern $p_1/p_2$,
we intuitively define it as the union
of the first-sets of $p_1$ and $p_2$,
passing down the follow-set parameter to each sub-call.
\begin{align*}
    \firstb{p_1/p_2}{follow} &=
    \SetUnion{\firstb{p_1}{follow}}{\firstb{p_2}{follow}}
\end{align*}

The case of the not-predicate pattern $\PNot{p}$
highlights the conservative nature of first-sets.
From the first-set of $p$,
we can only infer the set of first characters that make $p$ fail,
and, therefore, make $\PNot{p}$ succeed.
This, however, gives us no information about
first characters that make $p$ succeed,
and, therefore, make $\PNot{p}$ fail.
Therefore, in general, we cannot use the first-set of $p$
to compute the first-set of $\PNot{p}$.

There is, however, information
that we can extract from the follow-set parameter.
When $\PNot{p}$ is followed by a pattern $p_2$,
the follow-set indicates which first characters make $p_2$ fail.
Given that $\PNot{p}$ doesn't consume any input,
these first characters also make the sequence $\PSequence{\PNot{p}}{p_2}$ fail,
so they should be part of the first-set of $\PNot{p}$.
Therefore, we could, in general, use the follow-set
as the first-set of any predicate.
However, for the specific case of $\PNot{\PSet{cs}}$,
\lpeg{} instead computes the first-set
of patterns as ${\SetMinus{\Sigma}{\Set{cs}}}$.
\begin{align*}
    \firstb{\PNot{p}}{follow} &= \begin{cases}
        \SetMinus{\Sigma}{\Set{cs}} & \text{if $p = \PSet{cs}$} \\
        follow & \text{otherwise}
    \end{cases}
\end{align*}

One topic for future research is
to investigate whether it would be possible to
replace $\Sigma$ with the provided follow-set parameter in \lpeg{}.

As for the and-predicate pattern $\PAnd{p}$,
we use both the first-set of $p$ and the follow-set.
That is because in order to match the sequence $\PSequence{\PAnd{p}}{p_2}$,
the input string must match both $p$ and $p_2$.
Conversely, if the string starts with a character that is not in the
first-set of either pattern, the sequence fails.
So, in this case,
we define the first-set
as the intersection of both sets.
\begin{align*}
    \firstb{\PAnd{p}}{follow} &= \SetIntersection{\firstb{p}{\Sigma}}{follow}
\end{align*}

One subtle difference between this definition
and the actual implementation of the algorithm in \lpeg{} is
the follow-set parameter used to calculate the first-set of $p$.
While \lpeg{} simply passes along the follow-set parameter,
we provide the full character set $\Sigma$.
This change was necessary for us to prove the
key property of first-sets in Coq,
which we will show later in this chapter.
Future research may check whether
passing along the follow-set parameter
is incorrect, equivalent, or even better
than passing $\Sigma$ as the follow-set.

\newcommand{\firstc}[3]{\firstname{}(#1, #2, #3)}

Finally, in order to define the case of the non-terminal pattern,
we need to add the grammar as a parameter,
so that we can look up the referenced grammar rule.
In all cases, this grammar parameter is simply
passed along to each recursive call.
\begin{align*}
    \firstc{g}{\PNT{i}}{follow} &= \firstc{g}{p}{follow} & \text{if $g[i] = \Some p$}
\end{align*}

This case brings up the topic of termination,
as it does not define the recursion on
the structure of the pattern,
like the other cases do.
Instead, termination in this case
relies on the assumption
that the input PEG is well-formed,
and, therefore, free of left-recursive rules.

On a deeper level, termination is derived from
the way in which the well-formedness and first-set algorithms
traverse patterns similarly.
The most interesting case
is that of the sequence pattern:
when $p_1$ is non-nullable,
both algorithms do not visit $p_2$.
In the case of the well-formedness check,
visiting $p_2$ is not necessary
because the whole sequence is non-nullable,
and any rules visited in $p_2$ would be matched
against a proper suffix of the input string $s$
(avoiding infinite loops).

Meanwhile, in the case of the first-set algorithm,
visiting $p_2$ can be avoided for two reasons:
If $p_1$ is non-nullable,
then, as we later prove, its emptiness value is $false$,
which means that it fails the empty string.
If $p_1$ fails the empty string,
then so does the sequence,
which allows the emptiness value of the sequence to also be $false$,
regardless of the emptiness value of $p_2$.
The second reason is that, if $p_1$ is non-nullable,
then its first-set is independent of the follow-set parameter,
which, in the general case, would be the first-set of $p_2$.
Therefore, when $p_1$ is non-nullable,
we can provide any follow-set parameter, such as $\Sigma$,
in order to avoid making a recursive call to $p_2$.

\section{Matching the empty string}

Besides the first-set of a pattern,
the algorithm implemented in \lpeg{} also returns
a Boolean value, which indicates whether the
pattern may match the empty string,
a property we call \emph{emptiness}.
It is another conservative approximation:
the value $true$ has no meaning,
while the value $false$ indicates that the pattern
fails to match the empty string.
\lpeg{} needs this information
because it cannot use the first-set
to verify whether a pattern fails the empty string,
since the empty string has no first character.

Let us see how \lpeg{} computes
the emptiness of patterns.
The base cases are quite simple.
The empty pattern $\PEmpty$ and
the repetition pattern $\PRepetition{p}$
match every input string,
which includes the empty string.
So, for these patterns,
the function returns $true$.

The character class pattern $\PSet{cs}$,
as with any non-nullable pattern,
does not match the empty string.
Therefore, the value for this pattern is $false$.

For the not-predicate pattern $\PNot{p}$,
\lpeg{} is rather conservative,
always returning $true$.
In fact, this seems to be
the only case making the emptiness value
a conservative approximation.
If instead this function were to call itself
recursively for $p$ and negate its emptiness value,
we would effectively compute whether
the pattern matches the empty string.
However, the cases of $\PNot{p}$
where $p$ matches the empty string
are not common nor useful in practice.

The and-predicate pattern $\PAnd{p}$
and the non-terminal pattern $\PNT{i}$
simply forward the Boolean value
from the underlying pattern,
because they fail if and only if
the underlying pattern fails.

The sequence pattern $\PSequence{p_1}{p_2}$
matches the empty string if
both $p_1$ and $p_2$ do.
Intuitively, this would mean that the emptiness value of the sequence
would be the Boolean \scand{} of the emptiness values of $p_1$ and $p_2$,
but that is not exactly what is implemented in the algorithm.
As we have discussed at the end of the previous section,
when $p_1$ is non-nullable, $p_2$ is not visited,
and, therefore, the emptiness value of $p_2$ is not calculated.
However, we don't need this value,
since the emptiness value of $p_1$ is $false$ in this case,
which allows for the short-circuit evaluation
of the Boolean \scand{} expression to $false$.
Meanwhile, when $p_1$ is nullable,
the emptiness value of both $p_1$ and $p_2$ are computed,
and their Boolean \scand{} is calculated normally.

Finally, the case of the choice pattern $p_1/p_2$
is similar to that of the sequence pattern,
but instead of a Boolean \scand{} operation,
it performs a Boolean \scor{}
of the emptiness values of $p_1$ and $p_2$.
That is because the choice matches the empty string
if one of the options does.

\section{Formal definition}

\Cref{fig:firstcomp} presents
the formal definition of the first-set algorithm.
It takes a grammar, a pattern, a follow-set,
and some gas, and returns an optional tuple.
The recursion is defined on the gas parameter,
so that, if it reaches zero,
the function returns $\None$.
Otherwise, the function returns $\Some (b, first)$,
where $b$ is the emptiness value,
and $first$ is the first-set.
If $b=false$,
then the pattern fails to match the empty string;
and if a string starts with a character
$x \notin first$,
then it is guaranteed to fail to match that string.
The follow-set parameter is an
accumulator that should be initialized with
the full character set $\Sigma$.
In order to improve the legibility
of the function for sequence and choice patterns,
we also define the auxiliary functions $\otimes$ and $\oplus$,
respectively.
\begin{figure}
    \centering
    \begin{align*}
    \begin{aligned}[t]
        & \firstcomp{g}{p}{follow}{0} = \None \\
        & \firstcomp{g}{p}{follow}{(1+gas)} = \\
        & \begin{aligned}[t]
            & \matchwith{p} \\
            & \matchcase{\PEmpty}{\Some (true, follow)} \\
            & \matchcase{\PSet{cs}}{\Some (false, \Set{cs})} \\
            & \matchcase{\PRepetition{p}}{\begin{aligned}[t]
                & \matchwith{\firstcomp{g}{p}{follow}{gas}} \\
                & \matchcase{\Some (b, first)}{\Some (true, \SetUnion{first}{follow})} \\
                & \matchcase{\None}{\None} \\
                & \matchend{}
            \end{aligned}} \\
            & \matchcase{\PNot{p}}{\begin{aligned}[t]
                & \matchwith{p} \\
                & \matchcase{\PSet{cs}}{\Some (true, \SetMinus{\Sigma}{\Set{cs})}} \\
                & \matchcase{otherwise}{\Some (true, follow)} \\
                & \matchend{}
            \end{aligned}} \\
            & \matchcase{\PAnd{p}}{\begin{aligned}[t]
                & \matchwith{\firstcomp{g}{p}{\Sigma}{gas}} \\
                & \matchcase{\Some (b, first)}{\Some (b, \SetIntersection{first}{follow})} \\
                & \matchcase{\None}{\None} \\
                & \matchend{}
            \end{aligned}} \\
            & \matchcase{\PNT{i}}{\begin{aligned}[t]
                & \matchwith{g[i]} \\
                & \matchcase{\Some p}{\firstcomp{g}{p}{follow}{gas}} \\
                & \matchcase{\None}{\None} \\
                & \matchend{}
            \end{aligned}} \\
            & \matchcase{\PSequence{p_1}{p_2}}{\begin{aligned}[t]
                & \matchwith{\nullablecomp{g}{p_1}{gas}} \\
                & \matchcase{\Some false}{\firstcomp{g}{p_1}{\Sigma}{gas}} \\
                & \matchcase{\Some true}{\begin{aligned}[t]
                    & \matchwith{\firstcomp{g}{p_2}{follow}{gas}} \\
                    & \matchcase{\Some (b_2, first_2)}{b_2 \otimes (\firstcomp{g}{p_1}{first_2}{gas})} \\
                    & \matchcase{\None}{\None} \\
                    & \matchend{}
                \end{aligned}} \\
                & \matchcase{\None}{\None} \\
                & \matchend{}
            \end{aligned}} \\
            & \matchcase{\PChoice{p_1}{p_2}}{(\firstcomp{g}{p_1}{follow}{gas}) \oplus (\firstcomp{g}{p_2}{follow}{gas})} \\
            & \matchend{}
        \end{aligned}
    \end{aligned}
\end{align*}
    \caption{The first-set function.}
    \label{fig:firstcomp}
\end{figure}
\begin{figure}
    \centering
    \begin{align*}
    \begin{aligned}[t]
        & b \otimes res = \\
        & \matchwith{res} \\
        & \matchcase{\Some (b', first')}{\Some (b \wedge b', first')} \\
        & \matchcase{\None}{\None} \\
        & \matchend{}
    \end{aligned}
\end{align*}
    \begin{align*}
    \begin{aligned}[t]
        & res_1 \oplus res_2 = \\
        & \matchwith{res_1, res_2} \\
        & \matchcase{\Some (b_1, first_1), \Some (b_2, first_2)}
                    {\Some (b_1 \vee b_2, \SetUnion{first_1}{first_2})} \\
        & \matchcase{otherwise}{\None} \\
        & \matchend{}
    \end{aligned}
\end{align*}
    \caption{The auxiliary $\otimes$ and $\oplus$ functions.}
    \label{fig:firstaux}
\end{figure}

Having formally defined the first-set algorithm,
we now prove its key properties.
We begin by proving that if the function returns
some result for some gas amount,
it will return the same result if you provide
a higher gas amount.
In some sense, this means the function is stable
when you increase the gas amount.

\begin{lemma}
If $\firstcomp{g}{p}{follow}{gas} = \Some res$, \\
then, $\forall gas' \ge gas, \firstcomp{g}{p}{follow}{gas'} = \Some res$.
\end{lemma}

One natural consequence of this lemma is that,
for the same grammar, pattern, and follow-set,
the function cannot return contradicting results.

\begin{lemma}
If $\firstcomp{g}{p}{follow}{gas} = \Some res$, \\
and $\firstcomp{g}{p}{follow}{gas'} = \Some res'$, \\
then $res = res'$.
\end{lemma}

The previous two lemmas show how consistent
the return of the function is,
but both assume the existence of a gas amount
for which the function returns some result.
However, we know this is not always the case.
In fact, for ill-formed grammars, it may return $\None$
for any gas amount.
So, it is important to prove that,
for well-formed PEGs,
there exists a lower bound for the gas amount,
for which the function returns some result.
This lower bound effectively shows that
the algorithm terminates even without the gas parameter,
as it is implemented in \lpeg{}.

\begin{lemma}
If $\wf{g} = true$, \\
then $\forall gas \ge \size{p} + (1 + \length{g}) \cdot \size{g}$, \\
$\exists res, \firstcomp{g}{p}{follow}{gas} = \Some res$.
\end{lemma}

Having proved termination,
let us now focus on the key properties
of the first-set algorithm,
starting with the emptiness value.
We prove that,
for well-formed PEGs,
if $b=false$,
then the pattern fails the empty string
(denoted as $nil$).
Note that, in this case,
the follow-set parameter is irrelevant.

\begin{lemma}
If $\wf{g} = true$, \\
and $\firstcomp{g}{p}{follow}{gas} = \Some (false, first)$, \\
then $\Matches{g}{p}{nil}{\Failure}$.
\end{lemma}

Now, we prove lemmas about the
relation between the follow-set parameter
and the first-set return value.
These lemmas are necessary to prove a
more important lemma later in this section.
We start by proving that if the follow-set parameter
is incremented by an extra set (through a set union operation),
then the first-set return value is incremented by a subset
of this extra set.
Note that the emptiness value stays the same
with this follow-set increment.

\begin{lemma}
If $\firstcomp{g}{p}{follow}{gas} = \Some (b, first)$, \\
then $\forall extra, \exists extra' \subseteq extra$, \\
such that $\firstcomp{g}{p}{(\SetUnion{follow}{extra})}{gas} = \Some (b, (\SetUnion{first}{extra'}))$.
\end{lemma}

A particular case
is when this extra set is the first-set itself,
as if it were fed back into the function
through the follow-set parameter.
In this case, the first-set output by the function is the same,
since $\SetUnion{first}{extra'} \equiv first$ when $extra' \subseteq first$.
This particular lemma
is the one we actually use
to prove the more important lemma.

\begin{lemma}
If $\firstcomp{g}{p}{follow}{gas} = \Some (b, first)$, \\
then $\firstcomp{g}{p}{(\SetUnion{follow}{first})}{gas} = \Some (b, first)$.
\end{lemma}

The following lemma is the cornerstone of the key property of first-sets:
If $p$ matches some string $s$, leaving a suffix $s'$ unconsumed,
then $s$ must be either empty or start with a character
that \emph{is} in the first-set of $p$.
We also assume that the input PEG is well-formed,
and that $s'$ is either empty or starts with a character in the follow-set.
This last assumption is necessary to prove the lemma
in the case of the sequence pattern.

\begin{lemma}
If $\wf{g} = true$, \\
and $\firstcomp{g}{p}{follow}{gas} = \Some (b, first)$, \\
and $\Matches{g}{p}{s}{s'}$, \\
and $s'$ either is empty or starts with $x \in follow$, \\
then $s$ either is empty or starts with $y \in first$.
\end{lemma}

In the case of the and-predicate pattern $\PAnd{p}$,
we noticed that it would be easier to prove this lemma
if we passed $\Sigma$ as the follow-set of $p$.
That is because $\PAnd{p}$ matches when $p$ matches,
but $p$ leaves an unconsumed suffix $s'$ that is
discarded and whose starting character (if non-empty)
we know nothing about. Ultimately, we cannot say that $s'$
is either empty or starts with a character in
an arbitrary follow-set. Instead, we use $\Sigma$
as the follow-set of $p$, as this turns
this hypothesis into a tautology.

Finally, we prove the key property of first-sets:
For a well-formed PEG,
if the emptiness value is $false$,
then the pattern fails for any string
that does not start with a character in its first-set.
Note that we use the full character set $\Sigma$ as the follow-set.

\begin{lemma}
If $\wf{g} = true$, \\
and $\firstcomp{g}{p}{\Sigma}{gas} = \Some (false, first)$, \\
and $s$ either is empty or starts with $x \notin first$, \\
then $\Matches{g}{p}{s}{\Failure}$.
\end{lemma}

Besides this main property,
we also prove that, for non-nullable patterns,
the follow-set parameter does not influence the result.

\begin{lemma}
If $\nullablecomp{g}{p}{gas_n} = \Some false$, \\
and $\firstcomp{g}{p}{follow_1}{gas_1} = \Some res_1$, \\
and $\firstcomp{g}{p}{follow_2}{gas_2} = \Some res_2$, \\
then $res_1 = res_2$.
\end{lemma}

This lemma explains why, in the cases of character set patterns
and sequence patterns with non-nullable first patterns,
the follow-set parameter can be completely ignored.
We can also observe that, in the case of repetitions $\PRepetition{p}$,
\lpeg{} passes along the follow-set parameter to $p$,
but any follow-set could be provided,
given that $p$ is non-nullable from the well-formedness property.

Another fact about non-nullable patterns
is that their emptiness value is always $false$.
From the key property of emptiness values,
this indicates that non-nullable pattern
fail to match the empty string,
which we know is true.

\begin{lemma}
If $\nullablecomp{g}{p}{gas_n} = \Some false$, \\
and $\firstcomp{g}{p}{follow}{gas} = \Some (b, first)$, \\
then $b = false$.
\end{lemma}



% Nova seção


\section{Application in \lpeg{}}

Having proved the key properties of the first-set algorithm,
we would like to formalize its application in \lpeg{}.
As discussed at the beginning of this chapter,
\lpeg{} uses this algorithm
when generating code for choice patterns,
making use of test-set instructions.
Despite this optimization occurring at the virtual machine code level,
we would like to formalize it at the syntactic level.

\newcommand{\ifthenelsepat}[3]{\PChoice{\PSequence{\PAnd{#1}}{#2}}{\PSequence{\PNot{#1}}{#3}}}

The test-set instruction basically checks
whether the input string
starts with a character in a given set $\Set{cs}$,
jumping to a given label if it does not.
We can check the first character of the input string
through the character set pattern $\PSet{cs}$,
and emulate the logic of
``if $p_{cond}$ matches, then try $p_1$, otherwise try $p_2$''
through the following pattern construction.
\begin{align*}
    \ifthenelsepat{p_{cond}}{p_1}{p_2}
\end{align*}

In the optimized code of the choice pattern,
the test-instruction checks if the input string
starts with a character in the first-set of $p_1$,
and jumps to the code of $p_2$ if it does not.
This instruction is followed by the code of $p_1$,
which is executed if the check succeeds.
We can represent this optimized code
as the following pattern.
Let $\Set{first_1}$ denote the first-set of $p_1$.
\begin{align*}
    \ifthenelsepat{\PSet{first_1}}{p_1}{p_2}
\end{align*}

We now prove the correctness of this optimization.
Assuming the grammar $g$ and patterns $p_1$ and $p_2$ are well-formed,
and that the first-sets of $p_1$ and $p_2$ are disjoint,
and that the emptiness value of $p_1$ is $false$,
we first prove that if the original choice pattern matches a string $s$,
the optimized choice also matches $s$,
yielding the same unconsumed suffix $s'$.
We also need to assume that $s'$
either is empty or starts with a character in the
follow-set of $p_2$.

\begin{lemma}
If $g$, $p_1$ and $p_2$ are well-formed, \\
and $s'$ either is empty or starts with $x \in follow$, \\
and $\firstcomp{g}{p_1}{\Sigma}{gas_1} = \Some (false, first_1)$, \\
and $\firstcomp{g}{p_2}{follow}{gas_2} = \Some (b, first_2)$, \\
and $\SetIntersection{first_1}{first_2} = \EmptySet{}$, \\
and $\Matches{g}{\PChoice{p_1}{p_2}}{s}{s'}$, \\
then $\Matches{g}{\ifthenelsepat{\PSet{first_1}}{p_1}{p_2}}{s}{s'}$.
\end{lemma}

As for the case in which the choice fails,
we also show the optimized choice fails as well.
The proof follows from two facts:
The choice fails either because of $p_1$ or $p_2$;
And, for any input string,
either $\PAnd{\PSet{first_1}}$ matches
and $\PNot{\PSet{first_1}}$ fails,
or the other way around.

\begin{lemma}
If $\Matches{g}{\PChoice{p_1}{p_2}}{s}{\Failure}$, \\
then $\Matches{g}{\ifthenelsepat{\PSet{first_1}}{p_1}{p_2}}{s}{\Failure}$.
\end{lemma}