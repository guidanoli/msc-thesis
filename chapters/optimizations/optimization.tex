\section{Application in \lpeg{}}

Having proved the key properties of the first-set algorithm,
we would like to formalize its application in \lpeg{}.
As discussed at the beginning of this chapter,
\lpeg{} uses this algorithm
when generating code for choice patterns,
making use of test-set instructions.
Despite this optimization occurring at the virtual machine code level,
we would like to formalize it at the syntactic level.

\newcommand{\ifthenelsepat}[3]{\PChoice{\PSequence{\PAnd{#1}}{#2}}{\PSequence{\PNot{#1}}{#3}}}

The test-set instruction basically checks
whether the input string
starts with a character in a given set $\Set{cs}$,
jumping to a given label if it does not.
We can check the first character of the input string
through the character set pattern $\PSet{cs}$,
and emulate the logic of
``if $p_{cond}$ matches, then try $p_1$, otherwise try $p_2$''
through the following pattern construction.
\begin{align*}
    \ifthenelsepat{p_{cond}}{p_1}{p_2}
\end{align*}

In the optimized code of the choice pattern,
the test-instruction checks if the input string
starts with a character in the first-set of $p_1$,
and jumps to the code of $p_2$ if it does not.
This instruction is followed by the code of $p_1$,
which is executed if the check succeeds.
We can represent this optimized code
as the following pattern.
Let $\Set{first_1}$ denote the first-set of $p_1$.
\begin{align*}
    \ifthenelsepat{\PSet{first_1}}{p_1}{p_2}
\end{align*}

We now prove the correctness of this optimization.
Assuming the grammar $g$ and patterns $p_1$ and $p_2$ are well-formed,
and that the first-sets of $p_1$ and $p_2$ are disjoint,
and that the emptiness value of $p_1$ is $false$,
we first prove that if the original choice pattern matches a string $s$,
the optimized choice also matches $s$,
yielding the same unconsumed suffix $s'$.
We also need to assume that $s'$
either is empty or starts with a character in the
follow-set of $p_2$.

\begin{lemma}
If $g$, $p_1$ and $p_2$ are well-formed, \\
and $s'$ either is empty or starts with $x \in follow$, \\
and $\firstcomp{g}{p_1}{\Sigma}{gas_1} = \Some (false, first_1)$, \\
and $\firstcomp{g}{p_2}{follow}{gas_2} = \Some (b, first_2)$, \\
and $\SetIntersection{first_1}{first_2} = \EmptySet{}$, \\
and $\Matches{g}{\PChoice{p_1}{p_2}}{s}{s'}$, \\
then $\Matches{g}{\ifthenelsepat{\PSet{first_1}}{p_1}{p_2}}{s}{s'}$.
\end{lemma}

As for the case in which the choice fails,
we also show the optimized choice fails as well.
The proof follows from two facts:
The choice fails either because of $p_1$ or $p_2$;
And, for any input string,
either $\PAnd{\PSet{first_1}}$ matches
and $\PNot{\PSet{first_1}}$ fails,
or the other way around.

\begin{lemma}
If $\Matches{g}{\PChoice{p_1}{p_2}}{s}{\Failure}$, \\
then $\Matches{g}{\ifthenelsepat{\PSet{first_1}}{p_1}{p_2}}{s}{\Failure}$.
\end{lemma}