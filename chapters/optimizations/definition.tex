\section{Formal definition}

\Cref{fig:firstcomp} presents
\begin{figure}
    \centering
    \begin{align*}
    \begin{aligned}[t]
        & \firstcomp{g}{p}{follow}{0} = \None \\
        & \firstcomp{g}{p}{follow}{(1+gas)} = \\
        & \begin{aligned}[t]
            & \matchwith{p} \\
            & \matchcase{\PEmpty}{\Some (true, follow)} \\
            & \matchcase{\PSet{cs}}{\Some (false, \Set{cs})} \\
            & \matchcase{\PRepetition{p}}{\begin{aligned}[t]
                & \matchwith{\firstcomp{g}{p}{follow}{gas}} \\
                & \matchcase{\Some (b, first)}{\Some (true, \SetUnion{first}{follow})} \\
                & \matchcase{\None}{\None} \\
                & \matchend{}
            \end{aligned}} \\
            & \matchcase{\PNot{p}}{\begin{aligned}[t]
                & \matchwith{p} \\
                & \matchcase{\PSet{cs}}{\Some (true, \SetMinus{\Sigma}{\Set{cs})}} \\
                & \matchcase{otherwise}{\Some (true, follow)} \\
                & \matchend{}
            \end{aligned}} \\
            & \matchcase{\PAnd{p}}{\begin{aligned}[t]
                & \matchwith{\firstcomp{g}{p}{\Sigma}{gas}} \\
                & \matchcase{\Some (b, first)}{\Some (b, \SetIntersection{first}{follow})} \\
                & \matchcase{\None}{\None} \\
                & \matchend{}
            \end{aligned}} \\
            & \matchcase{\PNT{i}}{\begin{aligned}[t]
                & \matchwith{g[i]} \\
                & \matchcase{\Some p}{\firstcomp{g}{p}{follow}{gas}} \\
                & \matchcase{\None}{\None} \\
                & \matchend{}
            \end{aligned}} \\
            & \matchcase{\PSequence{p_1}{p_2}}{\begin{aligned}[t]
                & \matchwith{\nullablecomp{g}{p_1}{gas}} \\
                & \matchcase{\Some false}{\firstcomp{g}{p_1}{\Sigma}{gas}} \\
                & \matchcase{\Some true}{\begin{aligned}[t]
                    & \matchwith{\firstcomp{g}{p_2}{follow}{gas}} \\
                    & \matchcase{\Some (b_2, first_2)}{b_2 \otimes (\firstcomp{g}{p_1}{first_2}{gas})} \\
                    & \matchcase{\None}{\None} \\
                    & \matchend{}
                \end{aligned}} \\
                & \matchcase{\None}{\None} \\
                & \matchend{}
            \end{aligned}} \\
            & \matchcase{\PChoice{p_1}{p_2}}{(\firstcomp{g}{p_1}{follow}{gas}) \oplus (\firstcomp{g}{p_2}{follow}{gas})} \\
            & \matchend{}
        \end{aligned}
    \end{aligned}
\end{align*}
    \caption{The first-set function.}
    \label{fig:firstcomp}
\end{figure}
the formal definition of the first-set algorithm.
It takes a grammar, a pattern, a follow-set,
and some gas, and returns an optional tuple.
The recursion is defined on the gas parameter,
so that, if it reaches zero,
the function returns $\None$.
Otherwise, the function returns $\Some (b, first)$,
where $b$ is the emptiness value,
and $first$ is the first-set.
If $b=false$,
then the pattern fails to match the empty string;
and if a string starts with a character
$x \notin first$,
then it is guaranteed to fail to match that string.
The follow-set parameter is an
accumulator that should be initialized with
the full character set $\Sigma$.
In order to improve the legibility
of the function for sequence and choice patterns,
we also define
in \Cref{fig:firstaux}
\begin{figure}
    \centering
    \begin{align*}
    \begin{aligned}[t]
        & b \otimes res = \\
        & \matchwith{res} \\
        & \matchcase{\Some (b', first')}{\Some (b \wedge b', first')} \\
        & \matchcase{\None}{\None} \\
        & \matchend{}
    \end{aligned}
\end{align*}
    \begin{align*}
    \begin{aligned}[t]
        & res_1 \oplus res_2 = \\
        & \matchwith{res_1, res_2} \\
        & \matchcase{\Some (b_1, first_1), \Some (b_2, first_2)}
                    {\Some (b_1 \vee b_2, \SetUnion{first_1}{first_2})} \\
        & \matchcase{otherwise}{\None} \\
        & \matchend{}
    \end{aligned}
\end{align*}
    \caption{The auxiliary $\otimes$ and $\oplus$ functions.}
    \label{fig:firstaux}
\end{figure}
the auxiliary functions $\otimes$ and $\oplus$,
respectively.

Having formally defined the first-set algorithm,
we now prove its key properties.
We begin by proving that if the function returns
some result for some gas amount,
it will return the same result if you provide
a higher gas amount.
In some sense, this means the function is stable
when you increase the gas amount.

\begin{lemma}
If $\firstcomp{g}{p}{follow}{gas} = \Some res$, \\
then, $\forall gas' \ge gas, \firstcomp{g}{p}{follow}{gas'} = \Some res$.
\end{lemma}

One natural consequence of this lemma is that,
for the same grammar, pattern, and follow-set,
the function cannot return contradicting results.

\begin{lemma}
If $\firstcomp{g}{p}{follow}{gas} = \Some res$, \\
and $\firstcomp{g}{p}{follow}{gas'} = \Some res'$, \\
then $res = res'$.
\end{lemma}

The previous two lemmas show how consistent
the return of the function is,
but both assume the existence of a gas amount
for which the function returns some result.
However, we know this is not always the case.
In fact, for ill-formed grammars, it may return $\None$
for any gas amount.
So, it is important to prove that,
for well-formed PEGs,
there exists a lower bound for the gas amount,
for which the function returns some result.
This lower bound effectively shows that
the algorithm terminates even without the gas parameter,
as it is implemented in \lpeg{}.

\begin{lemma}
If $\wf{g} = true$, \\
then $\forall gas \ge \size{p} + (1 + \length{g}) \cdot \size{g}$, \\
$\exists res, \firstcomp{g}{p}{follow}{gas} = \Some res$.
\end{lemma}

Having proved termination,
let us now focus on the key properties
of the first-set algorithm,
starting with the emptiness value.
We prove that,
for well-formed PEGs,
if $b=false$,
then the pattern fails the empty string
(denoted as $nil$).
Note that, in this case,
the follow-set parameter is irrelevant.

\begin{lemma}
If $\wf{g} = true$, \\
and $\firstcomp{g}{p}{follow}{gas} = \Some (false, first)$, \\
then $\Matches{g}{p}{nil}{\Failure}$.
\end{lemma}

Now, we prove lemmas about the
relation between the follow-set parameter
and the first-set return value.
These lemmas are necessary to prove a
more important lemma later in this section.
We start by proving that if the follow-set parameter
is incremented by an extra set (through a set union operation),
then the first-set return value is incremented by a subset
of this extra set.
Note that the emptiness value stays the same
with this follow-set increment.

\begin{lemma}
If $\firstcomp{g}{p}{follow}{gas} = \Some (b, first)$, \\
then $\forall extra, \exists extra' \subseteq extra$, \\
such that $\firstcomp{g}{p}{(\SetUnion{follow}{extra})}{gas} = \Some (b, (\SetUnion{first}{extra'}))$.
\end{lemma}

A particular case
is when this extra set is the first-set itself,
as if it were fed back into the function
through the follow-set parameter.
In this case, the first-set output by the function is the same,
since $\SetUnion{first}{extra'} \equiv first$ when $extra' \subseteq first$.
This particular lemma
is the one we actually use
to prove the more important lemma.

\begin{lemma}
If $\firstcomp{g}{p}{follow}{gas} = \Some (b, first)$, \\
then $\firstcomp{g}{p}{(\SetUnion{follow}{first})}{gas} = \Some (b, first)$.
\end{lemma}

The following lemma is the cornerstone of the key property of first-sets:
If $p$ matches some string $s$, leaving a suffix $s'$ unconsumed,
then $s$ must be either empty or start with a character
that \emph{is} in the first-set of $p$.
We also assume that the input PEG is well-formed,
and that $s'$ is either empty or starts with a character in the follow-set.
This last assumption is necessary to prove the lemma
in the case of the sequence pattern.

\begin{lemma}
If $\wf{g} = true$, \\
and $\firstcomp{g}{p}{follow}{gas} = \Some (b, first)$, \\
and $\Matches{g}{p}{s}{s'}$, \\
and $s'$ either is empty or starts with $x \in follow$, \\
then $s$ either is empty or starts with $y \in first$.
\end{lemma}

In the case of the and-predicate pattern $\PAnd{p}$,
we noticed that it would be easier to prove this lemma
if we passed $\Sigma$ as the follow-set of $p$.
That is because $\PAnd{p}$ matches when $p$ matches,
but $p$ leaves an unconsumed suffix $s'$ that is
discarded and whose starting character (if non-empty)
we know nothing about. Ultimately, we cannot say that $s'$
is either empty or starts with a character in
an arbitrary follow-set. Instead, we use $\Sigma$
as the follow-set of $p$, as this turns
this hypothesis into a tautology.

Finally, we prove the key property of first-sets:
For a well-formed PEG,
if the emptiness value is $false$,
then the pattern fails for any string
that does not start with a character in its first-set.
Note that we use the full character set $\Sigma$ as the follow-set.

\begin{lemma}
If $\wf{g} = true$, \\
and $\firstcomp{g}{p}{\Sigma}{gas} = \Some (false, first)$, \\
and $s$ either is empty or starts with $x \notin first$, \\
then $\Matches{g}{p}{s}{\Failure}$.
\end{lemma}

Besides this main property,
we also prove that, for non-nullable patterns,
the follow-set parameter does not influence the result.

\begin{lemma}
If $\nullablecomp{g}{p}{gas_n} = \Some false$, \\
and $\firstcomp{g}{p}{follow_1}{gas_1} = \Some res_1$, \\
and $\firstcomp{g}{p}{follow_2}{gas_2} = \Some res_2$, \\
then $res_1 = res_2$.
\end{lemma}

This lemma explains why, in the cases of character set patterns
and sequence patterns with non-nullable first patterns,
the follow-set parameter can be completely ignored.
We can also observe that, in the case of repetitions $\PRepetition{p}$,
\lpeg{} passes along the follow-set parameter to $p$,
but any follow-set could be provided,
given that $p$ is non-nullable from the well-formedness property.

Another fact about non-nullable patterns
is that their emptiness value is always $false$.
From the key property of emptiness values,
this indicates that non-nullable pattern
fail to match the empty string,
which we know is true.

\begin{lemma}
If $\nullablecomp{g}{p}{gas_n} = \Some false$, \\
and $\firstcomp{g}{p}{follow}{gas} = \Some (b, first)$, \\
then $b = false$.
\end{lemma}



% Nova seção

